
\documentclass[12pt,reqno]{amsart}

\usepackage{amsmath}
\usepackage{amsfonts}
\usepackage{amssymb}

\usepackage{graphicx}
%\usepackage{epstopdf}
\usepackage[colorlinks=true]{hyperref}
\usepackage[left=1in,right=1in,top=0.9in,bottom=0.9in]{geometry}
\usepackage{multirow}
\usepackage{verbatim}
\usepackage{fancyhdr}
\usepackage{mdframed}
\usepackage{natbib}
\usepackage{multicol}
\renewcommand{\cite}{\citet}

%\usepackage[small,compact]{titlesec} 

%\usepackage{pxfonts}
%\usepackage{isomath}
\usepackage{mathpazo}
%\usepackage{arev} %     (Arev/Vera Sans)
%\usepackage{eulervm} %_   (Euler Math)
%\usepackage{fixmath} %  (Computer Modern)
%\usepackage{hvmath} %_   (HV-Math/Helvetica)
%\usepackage{tmmath} %_   (TM-Math/Times)
%\usepackage{cmbright}
%\usepackage{ccfonts} \usepackage[T1]{fontenc}
%\usepackage[garamond]{mathdesign}
\usepackage{xcolor}
\usepackage[normalem]{ulem}

\newtheorem{theorem}{Theorem}[section]
\newtheorem{conjecture}{Conjecture}[section]
\newtheorem{corollary}{Corollary}[section]
\newtheorem{lemma}{Lemma}[section]
\newtheorem{proposition}{Proposition}[section]
\theoremstyle{definition}
\newtheorem{assumption}{}[section]
%\renewcommand{\theassumption}{C\arabic{assumption}}
\newtheorem{definition}{Definition}[section]

\newtheorem{step}{Step}[section]
\newtheorem{rem}{Comment}[section]
\newtheorem{ex}{Example}[section]
\newtheorem{hist}{History}[section]
\newtheorem*{ex*}{Example}
\newtheorem{exer}{Exercise}[section]


\newenvironment{example}
  {\begin{mdframed}\begin{ex}}
  {\end{ex}\end{mdframed}}

\newenvironment{history}
  {\begin{mdframed}\begin{hist}}
  {\end{hist}\end{mdframed}}

\newenvironment{exercise}
  {\begin{mdframed}\begin{exer}}
  {\end{exer}\end{mdframed}}

\newenvironment{remark}
  {\begin{mdframed}\begin{rem}}
  {\end{rem}\end{mdframed}}



\linespread{1.1}

\pagestyle{fancy}
%\renewcommand{\sectionmark}[1]{\markright{#1}{}}
\fancyhead{}
\fancyfoot{} 
%\fancyhead[LE,LO]{\tiny{\thepage}}
\fancyhead[CE,CO]{\tiny{\rightmark}}
\fancyfoot[C]{\small{\thepage}}
\renewcommand{\headrulewidth}{0pt}
\renewcommand{\footrulewidth}{0pt}

\fancypagestyle{plain}{%
\fancyhf{} % clear all header and footer fields
\fancyfoot[C]{\small{\thepage}} % except the center
\renewcommand{\headrulewidth}{0pt}
\renewcommand{\footrulewidth}{0pt}}

\makeatletter
\renewcommand{\@maketitle}{
  \null 
  \begin{center}%
    \rule{\linewidth}{1pt} 
    {\Large \textbf{\textsc{\@title}}} \par
    {\normalsize \textsc{Paul Schrimpf}} \par
    {\normalsize \textsc{\@date}} \par
    {\small \textsc{University of British Columbia}} \par
    {\small \textsc{Economics 526}} \par
    \rule{\linewidth}{1pt} 
  \end{center}%
  \par \vskip 0.9em
}
\makeatother

\definecolor{ubcblue}{cmyk}{1.00,.62,.0,.52}
\definecolor{ubcgrey}{cmyk}{.42,.08,.0,.40}

\hypersetup{
  linkcolor  = ubcblue,
  citecolor  = ubcgrey,
  urlcolor   = ubcblue,
  colorlinks = true
}

\usepackage{mathtools}
\newcommand{\verteq}{\rotatebox{90}{$\,=$}}
\newcommand{\equalto}[2]{\underset{\scriptstyle\overset{\mkern4mu\verteq}{#2}}{#1}}
\newcommand{\nulls}{\mathrm{null}}
\newcommand{\range}{\mathrm{range}}

\newcommand{\argmax}{\operatornamewithlimits{arg\,max}}
\newcommand{\argmin}{\operatornamewithlimits{arg\,min}}
\def\inprobLOW{\rightarrow_p}
\def\inprobHIGH{\,{\buildrel p \over \rightarrow}\,} 
\def\inprob{\,{\inprobHIGH}\,} 
\def\indist{\,{\buildrel d \over \rightarrow}\,} 
\def\F{\mathbb{F}}
\def\R{\mathbb{R}}
\def\Er{\mathrm{E}}
\def\Pr{\mathrm{P}}
\def\En{\mathbb{E}_n}
\def\Pn{\mathbb{P}_n}
\newcommand{\gmatrix}[1]{\begin{pmatrix} {#1}_{11} & \cdots &
    {#1}_{1n} \\ \vdots & \ddots & \vdots \\ {#1}_{m1} & \cdots &
    {#1}_{mn} \end{pmatrix}}
\newcommand{\iprod}[2]{\left\langle {#1} , {#2} \right\rangle}
\newcommand{\norm}[1]{\left\Vert {#1} \right\Vert}
\newcommand{\abs}[1]{\left\vert {#1} \right\vert}
\renewcommand{\det}{\mathrm{det}}
\newcommand{\rank}{\mathrm{rank}}
\newcommand{\spn}{\mathrm{span}}
\newcommand{\row}{\mathrm{Row}}
\newcommand{\col}{\mathrm{Col}}
\renewcommand{\dim}{\mathrm{dim}}
\newcommand{\prefeq}{\succeq}
\newcommand{\pref}{\succ}
\newcommand{\seq}[1]{\{{#1}_n \}_{n=1}^\infty }
\renewcommand{\to}{{\rightarrow}}
\newcommand{\corres}{\overrightarrow{\rightarrow}}
\def\rel{\,{\buildrel R \over \sim}\,}

 

\title{Dynamic Programming}
\date{\today}

\begin{document}

\maketitle

\begin{quotation}
  ``[Dynamic] also has a very interesting property as an adjective, and
  that is it’s impossible to use the word, dynamic, in a pejorative
  sense. Try thinking of some combination that will possibly give it a
  pejorative meaning.  It’s impossible. Thus, I thought dynamic
  programming was a good name.'' - Richard Bellman
\end{quotation}

Most of our applications of optimal control involved choosing
something as a function of time. Dynamic programming is another
approach to solving optimization problems that involve time. Dynamic
programming can be especially useful for problems that involve
uncertainty. Dynamic programming has the advantage
that it lets us focus on one period at a time, which can often be
easier to think about than the whole sequence. Because it only
requires maximizing over a few variables at a time, dynamic
programming can be a much more efficient way to calculate
solutions. The computational advantage of dynamic programming is
especially pronounced when some of the variables being maximized over
are discrete. 

\section{References}

These notes are about dynamic programming.  References from our text
books are chapter 11 of \cite{dixit1990}, chapter 12 of
\cite{fuente2000} and parts of \cite{carter2001}. Two other useful
references are \cite{adda2003} and
\cite{acemoglu2008}. \cite{adda2003} is very nice and available online
from UBC library.  \cite{acemoglu2008} focuses on economic growth, but
includes two very nice chapters on dynamic programming and optimal
control.

\cite{stokey1989} is the classic economics reference for dynamic
programming, but is more advanced than what we will cover.
\textit{Applied dynamic programming} by \cite{bellman1962} and
\textit{Dynamic programming and the calculus of variations} by
\cite{dreyfus1965} provide a good introduction to the main idea of
dynamic programming, and are especially useful for contrasting the
dynamic programming and optimal control approaches.
\cite{dreyfus2002} has some amusing anecdotes from Bellman about the
initial development of dynamic programming.  \cite{bertsekas1976} is
the classic reference for dynamic programming with uncertainty.

\section{Introduction}

Dynamic programming deals with similar problems as optimal control. To
begin with consider a discrete time version of a generic optimal
control problem. 
\begin{align}
  \max_{x_t,y_t} & \sum_{t=0}^T f(x_t,y_t,t)  \label{eq:generic} \\
  \text{s.t.} & y_{t+1} - y_t = g(y_t,x_t,t) \notag \\
  & h(x_t,y_t,t) \leq \notag 0 \\
  & y_0 \text{ given}
\end{align}
Dynamic programming can also be used for continuous time problems, but
we will focus on discrete time. 
\begin{example}\label{ex:consav}[Consumption-savings]
  An infinite horizon consumption-savings problem, 
  \[ \max_{\{c_t\}_{t=0}^\infty,\{s_t\}_{t=1}^\infty} \sum_{t=0}^\infty
  \beta^t u(c_t) \text{ s.t. } s_{t+1} = (1+r_t)(s_t - c_t), \]  
  involves maximizing over a countably infinite sequence of $c_t$ and
  $s_t$. The interpretation of this problem is that $u(c)$ is the
  per-period utility from consumption. $c_t$ is consumption at time
  $t$. $s_t$ is the savings you have at time $t$. $r_t$ is the return to
  savings at time $t$ in period $t+1$, and $\beta$ is the discount
  factor. 
\end{example}
Problems like (\ref{eq:generic}) and example \ref{ex:consav} can be
solved using the Lagrangian. This approach focuses on characterizing
$x^*$ and $y^*$ through the first order conditions. That is, optimal
control focus on characterizing the maximizer. An alternative approach
is to focus on the value of the maximized function. This value will
depend on the entire problem, but in particular it depends on the
initial condition $y_0$. Thus, we can think of the value as function
of the initial state. Dynamic programming focuses on characterizing
the value function.

The basic idea of dynamic programming can be illustrated in a familiar
finite dimensional optimization problem. Consider a finite horizon
discrete time consumption savings choice. 
\[ \max_{c_t,s_t} \sum_{t=0}^T \beta^t u(c_t) \text{ s.t. } s_{t+1} =
(1+r_t) (s_t - c_t) \] with $s_0$ given, and the constraint that
$s_{T+1} = 0$. We could just write down the first order conditions and
try to solve them for $c_t$. However, if $T$ is large, this might be
very difficult. It can be especially difficult to calculate a solution
numerically. The easiest maximization problems to solve numerically
are ones where the objective function is linear or quadratic. In
either of these cases, the amount work needed is proportional to the
number of variables cubed. If $T$ is large, $T^3$ can be so large that
computing a solution takes prohibitively long.  

We can divide this $T$ dimensional problem to a
series of smaller ones by first thinking about what happens at time
$T$. At time $T$ we have some savings $s_T$ and want to choose $c_T$
to solve
\[ \max_{c_T} u(c_T) \text{ s.t. } s_{T+1} = (1+r_T)(s_T -c_T) = 0 \]
As long as $u$ is increasing, it must be that $c_T^*(s_T) = s_T$. If
we define the value of savings at time $T$ as
\[ V_T(s) = u(s), \]
then at time $T-1$ given $s_{T-1}$, we can choose $c_{T-1}$ to solve
\[ \max_{c_{T-1},s'} u(c_{T-1}) + \beta V_T(s') \text{ s.t.} s' =
  (1+r_{T-1})(s_{T-1}-c_{T-1}). \]
This is a relatively simple maximization problem with just two
variables, so we can solve it without too much difficulty. Repeating
in this way, for each $t$ we can define the value of savings at time
$t$ as
\begin{align} 
  V_t(s) = \max_{c_t,s'} u(c_{t}) + \beta V_{t+1}(s') \text{ s.t.} s' =
  (1+r_{t})(s - c_{t}). \label{e:bell}
\end{align}
This approach to sequential optimization was first proposed by Richard
Bellman, so (\ref{e:bell}) is called a Bellman equation. Notice that
if $(c_t^*(s_t),s_{t+1}^*(s_t)$ is a maximizer of (\ref{e:bell}) for
each $t$, then 
the sequence of $c_0^*(s_0),s_1^*(s_0), c_1^*(s_1), ..., c_T^*$ must
be a maximizer of the original problem. Bellman called this
observation the \textbf{principle of optimality}. He described it as,
``An optimal policy has the property that whatever the initial state
and initial decision are, the remaining decisions must constitute an
optimal policy with regard to the state resulting from the first
decision.'' (\cite{bellman1962})

\subsection{Infinite horizon}
In finite horizon problems, it easy to see that the
Bellman equations will exist. However, if we have an infinite horizon
problem, 
\[ \max_{c_t,s_t} \sum_{t=0}^\infty \beta^t u(c_t) \text{ s.t. } s_{t+1} =
(1+r_t) (s_t - c_t) \]
then we cannot start from the last period to define the value
function. However, if the problem is stationary, that is if the
problem at time $t$ and at time $t+1$ look the same, then it seems
reasonable to think that the value function would not depend on $t$
and we could just write 
\[ V(s) = \max_{c,s'} u(c) + \beta V(s') \text{ s.t. } s' =
(1+r)(s-c). \]
\cite{stokey1989} provide a fairly comprehensive analysis of
various conditions when this is possible. We will just look at one
case. 

Consider a problem that is slightly more general than the consumption
savings choice problem with fixed interest rate.
\begin{align*}
  \max_{c_t,s_t} & \sum_{t=0}^\infty \beta^t u(c_t,s_t) \\
  \text{s.t.} & g_0(c_t,s_t) \leq s_{t+1} \leq g_1(c_t,s_t),
  & \underline{c} \leq c_t \leq \overline{c}
\end{align*}
where $c \in \R$, $s \in \R$, $0<\beta<1$, and $u,g:\R^2 \to \R$.
We want to show that the value function exists. Suppose we start with
some guess at the value function $v_0:\R\to \R$. Then we refine that guess by
setting 
\begin{align*} 
  v_1(s) = \max_{c,s'} & u(c,s) + \beta v_0(s') \\
           \text{s.t.} & g_0(c,s) \leq s' \leq g_1(c,s),\\
                       & \underline{c} \leq c \leq \overline{c}
\end{align*}]
We could do this repeatedly. Hopefully eventually the value function guesses
would stop changing and we would have the value function. Iteratively 
refining the value function guess in this manner is called value
function iteration. It does converge to the true value function under
fairly general conditions.
\begin{theorem}
  Consider 
  \begin{align*}
    \max_{c_t,s_t} & \sum_{t=0}^\infty \beta^t u(c_t,s_t) \\
    \text{s.t.} & g_0(c_t,s_t) \leq s_{t+1} \leq g_1(c_t,s_t), \\
    & \underline{c} \leq c_t \leq \overline{c}
  \end{align*}
  Assume 
  \begin{enumerate}
  \item $u$ is continuous and bounded (i.e. $u(c,s) \leq M$ for all
    $c$ and $s$)
  \item $g_0$ and $g_1$ are continuous, $\underline{c}$ and
    $\overline{c}$ are finite
  \item $0 \leq \beta < 1$
  \end{enumerate}
  Then there exists a unique value function, $v$ such that
  \begin{align*} v(s) = \max_{c,s'} & u(c,s) + \beta v(s') \\
    \text{s.t.} & g_0(c,s) \leq s' \leq g_1(c,s),\\
    & \underline{c} \leq c \leq \overline{c}
  \end{align*}
  Moreover, $v$ is the limit of a sequence of value function
  iteration.
\end{theorem}
\begin{proof}
  Let $T$ be the operator defined by value function iteration. 
  \begin{align*}
    T(v)(s) = & \max_{c,s'} u(c,s) + \beta v(s') \\
    \text{s.t.} & \; g_0(c,s) \leq s' \leq g_1(c,s), \\
    & \underline{c} \leq c \leq \overline{c}
  \end{align*}
  In order for $T$ to be well defined, the above maximum must exist. 
  Our assumptions that $u$ is continuous, and that $c$ and $s'$ have
  upper and lower bounds ensures that the maximum exists. We will
  prove this later. These assumptions could be replaced by others that
  ensure the maximum exists. 
  
  Value function iteration generates a sequence with $v_{i+1} =
  T(v_i)$.  Recall that for sequences of real numbers, $\{x_i\}$, $x_i
  \to x$ means that for any $\epsilon >0$ there exists an $N$ such
  that for all $i \geq N$, $|x_i - x| < \epsilon$. Similarly, for
  functions $v_i$, $v_i$ converges to $v$ if for all $\epsilon>0$
  there exists an $N$ such that for all $i \geq N$, $\sup_{s} |v_i(s)
  - v(s)| < \epsilon$. In other words, $v_i \to v$ means that as $i$
  increases the maximum of the absolute value of the difference
  between $v_i$ and $v$ shrinks to $0$. 
  
  Consider
  \begin{align*}
    T(v_0)(s)-T(v_1)(s) =  \left(u(c_0,s) + \beta v_0(s_0')  \right) -
    \left(u(c_1,s) + \beta v_1(s_1')  \right)
  \end{align*} 
  where $c_i,s_i'$ is the maximizer to 
  \begin{align*}
    \max_{c_i,s_i'} & u(c_i,s) + \beta v_i(s_i') \\
    \text{s.t.} & \; g_0(c_i,s) \leq s_i' \leq g_1(c_i,s), \\
                & \underline{c} \leq c_i \leq \overline{c}
  \end{align*}
  Notice that
  \[ T v_0 (s) = u(c_0,s) + \beta v_0(s_0')  \geq u(c_1,s) + \beta
  v_0(s_1'). \]
  Therefore,
  \[ T(v_0)(s) - T(v_1)(s) \geq \beta (v_0(s_1') - v_1(s_1'). \]
  Similarly,
  \[ T(v_0)(s)-T(v_1)(s) \leq \beta (v_0(s_0') - v_1(s_0'). \]
  It follows that
  \begin{align*}
    \sup_{s}
    \left\vert T(v_0)(s)-T(v_1)(s) \right\vert 
    \leq & \sup_s \left\vert \beta(v_0(s) -v_1(s)) \right\vert 
  \end{align*}
  We assumed that $\beta < 1$. Thus, we have shown that $T(v_0)$ and
  $T(v_1)$ are closer together than $v_0$ and $v_1$. A mapping with
  this property is called a contraction mapping. To show that the
  value function is unique, suppose $v = T(v)$ and $\tilde{v} =
  T(\tilde{v})$. Then, 
  \begin{align*}
    \sup_s \left\vert v(s) -\tilde{v}(s) \right\vert = \sup_{s}
    \left\vert T(v)(s)-T(\tilde{v})(s) \right\vert \leq & \beta \sup_s
    \left\vert v(s) -\tilde{v}(s) \right\vert
  \end{align*}
  which is only possible if $\sup_s \left\vert v(s) -\tilde{v}(s)
  \right\vert = 0$, i.e. $v = \tilde{v}$.
 
  To show that $v$ exists we must show that the sequence of $v_i$
  converges.  Since $T$ contracts and makes the $v_i$ closer and
  closer together, it makes sense that $v_i$ should converge. Showing
  this formally is tedious.  \footnote{If you are familiar with the fact
    that Cauchy sequences converge in complete metric spaces, then it
    is easier. We have not yet covered Cauchy sequences or metric
    spaces, so we will not use this fact.}

  A defining assumption about real numbers is that every set of
  numbers bounded above has a least upper bound. See the notes on sets
  for more information. An immediate consequence of this assumption is
  that any increasing sequence of real numbers that is bounded above,
  must converge. The argument below will use this fact to show that
  the sequence of $v_i$ converge. 
  
  Choose a sequence $\{\epsilon_i\}$ with $\epsilon_i \to 0$. Let
  $N_i$ be such that 
  \[ \sup_s \left\vert \beta(v_{n}(s) -v_{N_i}(s))  \right \vert <
  \epsilon_i \]
  for all $n \geq N_i$. Such an $N_i$ exists because $v_n =
  \underbrace{T(T (
    ... T(}_{n \text{ times }} v_0) ... ) = T^n v_0$, and by repeating
  the same argument as above,
  \begin{align*}
    \sup_{s} \left\vert T^N (T^{n-N}( v_0))(s)- T^N(v_0) (s) \right\vert = 
    \leq & \beta^N \sup_s \left\vert (T^{n-N}v_0(s) -v_0(s)) \right\vert.
  \end{align*}
  Also, since $u$ is bounded by say $M$, $|v_i(s)|$ is at most
  $M/(1-\beta)$, so $\sup_s \left\vert (T^{n-N}v_0(s) -v_0(s))
  \right\vert \leq 2 M /(1-\beta)$. Thus,
  \[ 
    \sup_{s} \left\vert T^N (T^{n-N}( v_0))(s)- T^N(v_0) (s) \right\vert
    \leq \beta^N 2M/(1-\beta)  
  \]
  so we can always choose e.g. $N_i = \log(\epsilon_i
  (1-\beta)/(2M))/\log(\beta) + 1$.
  
  Define a new sequence of functions $U_i$ with 
  \[ U_1(s) = v_{N_1}(s) + \epsilon_1 \] and 
  \[ U_{i+1}(s) = \max\{
  U_{i}(s), v_{N_{i+1}}(s) + \epsilon_{i+1} \}.\] Similarly define $L_i$
  but subtracting $\epsilon_i$ instead of adding it. 
  Note that by definition for all $n \geq N_i$ and all
  $s$, $L_i(s) \leq v_n(s) \leq U_i(s)$. Also, $\sup_{s} |L_i(s) -
  U_i(s) | \leq 2\epsilon_i$, and $U_i(s)$ decreases and $L_i(s)$
  increases with $i$ for each $s$. 

  For each $s$, $U_i(s)$ is decreasing sequence of numbers bounded
  below by $L_1(s)$. Therefore, it must converge. Similarly, $L_i(s)$
  is an increasing sequence bounded above, so it must
  converge.\footnote{We will prove that bounded monotonic sequences
    converge later.} Also, since $|L_i(s) - U_i(s) | \leq 2\epsilon_i
  \to 0$, it must be that $\lim_{i\to \infty} L_i(s) = \lim_{i\to
    \infty } U_i(s)$. Define $v(s)$ as this limit. Then, for $n \geq N_i$,
  \begin{align*}
    \sup_{s} \left\vert v_n(s) - v(s) \right\vert & 
    \leq \sup_s \left\vert v_n(s) - L_i(s) \right\vert + \sup_s
    \left\vert L_i(s) - v(s) \right\vert \\
    \leq & \epsilon_i + 2\epsilon_i
  \end{align*}
  We can then conclude that $v_n \to v$.   
\end{proof}   
 
\section{Solving dynamic programs} 

There are three basic ways to solve a dynamic program. They are:
\begin{enumerate}
\item Guess and verify the form of the value function
\item Iterate the Bellman equation analytically
\item Iterate (or use some other alogirthm to solve)  the Bellman equation numerically
\end{enumerate}
If you guess correctly, the first method is fairly
straightforward. However, guessing correctly is difficult and often is
not possible at all. The second method will always work, but may not
lead to a closed form expression, and can be tedious. The third method
is the main way dynamic programs are solved in practice, but we will
not go into the details.

\begin{example}[Optimal growth by guessing and verifying]
  Consider an economy with a single infinitely lived representative
  consumer with per-period log utility from consumption and a discount
  factor of $\delta$. The economy's production function is
  Cobb-Douglas with capital as the only input. Anything not consumed
  at time $t$ becomes capital at time $t+1$. The optimal growth
  problem is
  \begin{align*}
    \max_{\{c_t\}_{t=0}^\infty} & \sum_{t=0}^\infty \delta^t
    \log(c_t) \\
    \text{s.t. } & c_t + k_{t+1} = k_t^\alpha.
  \end{align*}
  If we use the constraint to solve for $c_t$ and substitute into the
  objective, then we have
  \begin{align*}
    \max_{\{k_t\}_{t=1}^\infty} & \sum_{t=0}^\infty \delta^t
    \log(k_t^\alpha - k_{t+1}) \\
    \text{s.t.} & 0 \leq k_{t+1} \leq k_t^\alpha 
  \end{align*}
 The Bellman equation for this problem is
 \begin{align*}
   v(k) = & \max_{k' \in [0,k^\alpha]} \log(k^\alpha - k') + \delta
   v(k')
 \end{align*}
 Now, we guess the functional form of $v$. Since the per-period
 utility function is logarithmic and production is Cobb-Douglas, it is
 sensible to guess that $v(k) = c_0 + c_1 \log(k^a)$ where $c_0$,
 $c_1$, and $a$ are each constant for which we solve. Now, since $c_1
 \log(k^a) = c_1 a \log(k)$, $a$ and $c_1$ are redundant, so we can
 get rid of $a$, and just guess that $v(k) = c_0 + c_1 \log(k)$. 
 
 We now use the Bellman equation to solve for $c_0$ and $c_1$. First
 we solve for the optimal $k'$ for a given $c_0$ and $c_1$. The
 Bellman equation is:
 \begin{align*}
   c_0 + c_1 \log k = & \max_{k' \in [0,k^\alpha]} \log(k^\alpha-k') +
   \delta\left(c_0 + c_1 \log k' \right).
 \end{align*}   
 We could write the Lagrangean with the constraints that $k'\geq 0$
 and $k'\leq k^\alpha$. If we were not sure whether these constraints
 would bind we would include them in the Lagrangean and check the
 complementary slackness conditions. However, it is slightly easier to
 just notice that these constraints cannot bind because utility
 approaches $-\infty$ as $k$ approaches $k^\alpha$ and the next
 period's value approaches $-\infty$ as $k'$ approaches $0$, so
 neither constraint will bind.  Without the constraints, the first
 order condition is:
 \begin{align*}
   -\frac{1}{k^\alpha - k'}  + \delta c_1 \frac{1}{k'} = & 0 \\
   -k' + \delta c_1 (k^\alpha - k') & = 0 \\
   k' = & \frac{\delta c_1}{1+\delta c_1} k^\alpha
 \end{align*}
 Now, we plug this back into the Bellman equation and solve for $c_0$
 and $c_1$ by varying $k$. 
 \begin{align*}
   c_0 + c_1 \log k = & \max_{k' \in [0,k^\alpha]} \log(k^\alpha-k') +
   \delta\left(c_0 + c_1 \log k' \right) \\
   = & \log\left(k^\alpha - \frac{\delta c_1}{1+\delta c_1}
     k^\alpha\right) + \delta \left(c_0 + c_1 \log \left(\frac{\delta
         c_1}{1+\delta c_1} k^\alpha\right) \right) \\
   = & \log\left(\frac{1}{1+\delta c_1}\right) + \alpha \log k +
   \delta\left(c_0 + c_1 \log\left(\frac{\delta
         c_1}{1+\delta c_1} \right) + \alpha \log k \right) \\
   = & \underbrace{\log\left(\frac{1}{1+\delta c_1}\right) + \delta
     c_1 \log\left(\frac{\delta c_1}{1+\delta c_1} \right)}_{=c_0} +
   \underbrace{(\alpha + \delta c_1 \alpha )}_{= c_1} \log k
 \end{align*}
 Both the left and right sides of this equation are affince function
 of $\log k$. They can only be equal for all $k$ if the coefficients
 are equal. Thus,
 \begin{align*} 
   c_1 = & \alpha + \delta c_1 \alpha \\
   c_1 = & \frac{\alpha}{1-\delta \alpha}
 \end{align*}
 and 
 \begin{align*}
   c_0 = & \log\left(\frac{1}{1+\delta c_1}\right) + \delta c_1
   \log\left(\frac{\delta c_1}{1+\delta c_1} \right) \\
   = & -\left(1 + \delta \frac{\alpha}{1-\delta \alpha}\right) 
   \log\left(1+\delta  \frac{\alpha}{1-\delta \alpha}\right) + 
   \delta \frac{\alpha}{1-\delta \alpha} 
   \log\left(\delta \frac{\alpha}{1-\delta \alpha} \right) \\
   = & \log(1-\delta \alpha) + \frac{\delta \alpha}{1-\delta \alpha}
   \log(\delta \alpha).
 \end{align*}
 Finally, we should make sure that this solution doesn't violate the
 constraint. We have
 \[ k' = \frac{\delta c_1}{1+\delta c_1} k^\alpha = \delta \alpha
 k^\alpha, \]
 so the constraints are satisfied as long  as $\delta \alpha \in
 (0,1)$.  
\end{example}

If we cannot guess the form of the value function, we can try to find
it by repeatedly applying the Bellman operator. The Bellman operator
is the $T$ operator we defined above,
\[ T(v)(s) = \max_{c,s'} u(c,s) + \beta v_0(s') \text{ s.t. } s'=g(c,s). \]
We already showed that $T$ is a contraction (provided $u$ is bounded
and $\abs{\beta} < 1$). Among other things, this means that if we
start with an arbitrary guess of the value function, $v_0$, and then
construct a sequence by repeatedly applying $T$, i.e., 
\[ v_i = T(v_{i-1}), \]
then the sequence $v_i$ will converge to a unique fixed point, $v$,
that satisfies the Bellman equation. 
\begin{example}[Optimal growth by iterating]
  The same optimal growth problem as in the previous example can also
  be solved by iterating the Bellman operator. We start with
  \emph{any} guess of the value function for $v_0$. A common choice is
  the zero function, $v_0(k) = 0$ for all $k$. Then we find $v_1$ by
  solving 
  \begin{align*}
    v_1(k) = & \max_{k' \in [0,k^\alpha]} \log(k^\alpha - k') + \delta v_0(k)
    \\
    = & \max_{k' \in [0,k^\alpha]} \log(k^\alpha - k') \\
    = & \alpha \log k.
  \end{align*}
  Then, we repeat to find $v_2$.
  \begin{align*}
    v_2(k) = & \max_{k' \in [0,k^\alpha]} \log(k^\alpha - k') + \delta v_1(k)
    \\
    = & \max_{k' \in [0,k^\alpha]} \log(k^\alpha - k') + \delta
    \alpha \log(k')  \\
    = & c_2 + (\alpha + \delta \alpha^2) \log k,
  \end{align*}
  where $c_2$ is some constant that involves $\delta$, $\alpha$, and
  their logs. The third equality comes from writing the first order
  condition, solving for $k'$, and subsituting back into the
  objective.  We can explicitly solve for $c_2$, but it doesn't matter
  for the first order condition for $v_3$, so we don't need to know it
  exactly. We repeat again to get $v_3$
  \begin{align*}
    v_3(k) = & \max_{k' \in [0,k^\alpha]} \log(k^\alpha - k') + \delta v_1(k)
    \\
    = & \max_{k' \in [0,k^\alpha]} \log(k^\alpha - k') + \delta
    \alpha \log(k')  \\
    = & c_3 + (\alpha + \delta \alpha^2 + \delta^2 \alpha^3) \log k.
  \end{align*}    
  We could repeat again to get 
  \begin{align*}
    v_4(k) = & c_4 + (\alpha + \delta \alpha^2 + \delta^2 \alpha^3 +
    \delta^3 \alpha^4) \log k \\
    v_5(k) = & c_5 + (\alpha + \delta \alpha^2 + \delta^2 \alpha^3 +
    \delta^3 \alpha^4 + \delta^4 \alpha^5) \log k \\
    \vdots
  \end{align*}
  etc. Eventually, we hopefully notice a pattern. The more obvious
  pattern is that each $v_i$ and will always be of the form $v_i(k) =
  c_i + m_i \log(k)$. Thus, we know that $v(k)$ will have that same
  form and we can go back to the guess and verify method. Better yet,
  we could notice that
  \[ v_i(k) = c_i + \alpha \sum_{j = 0}^i (\alpha \delta)^j \log(k),\]
  so 
  \begin{align*}
    v(k) = C + \frac{\alpha}{1-\alpha \delta} \log k.
  \end{align*}
  If we care about $C$, we could find it by either explicitly writing
  $c_i$ in terms of $\delta$ and $\alpha$ and taking the limit; or
  using the guess verify method just for $C$. 
\end{example}

Solving for the value function, whether by guessing and verifying or
iterating can be a bit tedious. Even worse, for most specifications of
the per-period payoff $u$ and constraints $g$, there will be no closed
form solution for $v$. That makes it impossible to guess the form, and
iterating the Bellman equation will not lead to a discernible pattern
(although it will still give a convergent sequence). Using a computer to
solve for the value function avoids both these problems. A computer
does not care that Bellman operator iteration is tedious, and it can
numerically compute $v(k)$ even if it has no closed form. 

\begin{example}[Optimal growth numerically] 
  We can also solve for the value function numerically. The simplest
  (although usually not the fastest) method is to iterate the Bellman
  operator. This can be done as follows:
  \begin{itemize}
  \item Choose a grid of capital values $k_g$ for $g=1,..., G$
  \item Set $\hat{v}_0(k) = 0$ for all $k$ (or anything else)
  \item Repeatedly:
    \begin{enumerate}
    \item Maximize the Bellman equation for each $k_g$
      \[ v_{g,i} = \max_{c,k'} u(c) + \delta \hat{v}_{i-1}(k') \text{ s.t. } k' = f(k_g)
      - c \]
    \item Set $\hat{v}_{i} (k) = $ linear interpolation of $\{k_g,
      v_{g,i} \}$
    \end{enumerate}
    Stop when the value function stops changing, e.g. when 
    \[ \max_{g} \abs{v_{g,i} - v_{g,i-1} } < \epsilon \]
  \end{itemize}
  
  Figure \ref{fig:value} shows the resulting value function
  approximations for various $i$.  Figure \ref{fig:policy} shows the
  optimal policies for $c$ and $k'$ given $k$. 
  
  These figures were created using
  \href{https://bitbucket.org/paulschrimpf/econ526/src/master/03-dynamicProgramming/dp.R?at=master}
  {this R code.}
  \cite{sargent2013} has a similar examples in Python and Julia.   
\end{example}

\begin{figure}\caption{Value function iterations \label{fig:value}}
  \begin{centering}
    \includegraphics{value}
  \end{centering}
\end{figure}

\begin{figure}\caption{Policy function \label{fig:policy}}
  \begin{centering}
    \includegraphics{policy}
  \end{centering}
\end{figure}

\clearpage
\subsection{Discrete control}
  
Another situation where dynamic programs can be solved analytically is
when the control variable is discrete. For example, a person could be
choosing to work or not each period, or a firm could be choosing to
enter or exit a market. 

\begin{example}[Investment option]
  Each period an investor has an opportunity to invest in a
  project. Once the investment is made, the model ends. The investment
  costs $I$. At time $t$ the project pays $z_t$. $z_t$ is drawn
  independently from some distribution with pdf $f$, cdf $F$, and
  support $[0, B]$.
  The investor is risk neutral with discount rate $\beta$. At
  time $t$, the investor can choose to invest now or wait. Investing
  now pays $z_t - I$. Let $V(z)$ denote the value of this choice given the
  current $z$. The Bellman equation is 
  \[ V(z) = \max\{ z- I , \beta \int V(z') f(z') dz' \}. \]
  The value of waiting, $\beta \int V(z') f(z') dz'$, does not depend
  on $z$, so the optimal decision is to invest if $z \geq z^*$ for
  some threshold $z^*$. At $z^*$, the investor is indifferent between
  waiting and investing, so
  \begin{align*}
    z^* - I = & \beta \int_0^B V(z') f(z') dz' \\
    = & \beta \int_0^{z^*} \underbrace{\left[\beta \int_0^B V(z')
        f(z') dz'\right]}_{=z^* - I}
    f(\tilde{z}) d\tilde{z} + \beta \int_{z^*}^B (z' - I) f(z') dz' \\
    = & \beta \int_0^{z^*} (z^* - I) f(\tilde{z}) d\tilde{z} + \beta
    \int_{z^*}^B (z' - I) f(z') dz' \\
    & \text{(add and subtract $\int_{z^*}^B z^* f(z')dz'$ )} \\
    = & \beta \int_0^{B} (z^* - I) f(\tilde{z}) d\tilde{z} + \beta
    \int_{z^*}^B (z' - z^*) f(z') dz' \\
    z^* - I = & \frac{\beta}{1-\beta} \int_{z^*}^B (z' - z^*) f(z') dz'
  \end{align*}
  From this, we can see that $z^* > I$. The investor waits for a
  project with a strictly positive return. Also, we can see that as
  $\beta$ increases, $z^*$ increases, i.e.\ the investor will wait for
  a higher return. The probability of waiting $t$ periods if
  $F(z^*)^{t-1} (1-F(z^*))$. The expected waiting time is then
  $\frac{1}{1-F(z^*)}$. This too increases with $z^*$. 

  We can also think about what happens as the riskiness of the
  investment changes. Changing the riskiness of investment is some
  change to the distribution of $z$.   
  Not all changes in $f$ are increases in riskiness. One way to
  consider a change in $f$ that is an increase in risk, is to look at
  change that holds $E[z]$ constant but increases the variance of
  $z$. Such a change is called a mean-preserving spread. We can always
  create a mean-preserving spread by replacing $z$ with $\tilde{z} = z
  + \epsilon$ where $\Er[\epsilon|z] = 0$ and $V(\epsilon) \geq 0$. A
  function $g$ is concave if for all $z$ there exists $d(z)$ such that
  $g(z + \epsilon) \geq g(z) + d(z) \epsilon$ for all $\epsilon$. If
  $g$ is differentiable, then $d$ is just the derivative of $g$. For
  any concave function if $\tilde{z}$ is a mean preserving spread of
  $z$, then 
  \begin{align*}
    \Er[g(\tilde{z})] = \Er[g(z+\epsilon)] \geq & \Er[g(z) + d(z) \epsilon] \\
    \geq & \Er[g(z) + d(z) \Er[\epsilon|z]] \\
    \Er[g(\tilde{z})] \geq & \Er[g(z)].
  \end{align*}\footnote{More generally, we say that the distribution
    of $\tilde{z}$ second-order stochastic dominates the distribution
    of $z$ if this inequality holds for all concave functions $g$.}
  Now, let 
  \[ g_{z^*}(z) = \max\{0, z - z^* \}. \] This function is concave and
  decreasing in $z^*$. Also, from the above,
  \begin{align*}
    z^* - I = & \frac{\beta}{1-\beta} \Er[g_{z^*}(z)] 
  \end{align*}
  Thus, if we add a mean-preserving spread to $z$, then $\Er[g_{z^*}(z)]$
  will increase, and $z^*$ must increase to compensate. 
\end{example}

\subsection{Uncertainty}

Dynamic programming is often especially useful for dealing with models
with uncertainty. The investment option model above is one
example. Let's look at another.
\begin{example}[Growth with random productivity]
  Let's consider the same growth model as before, but now production
  is $A_t k_t^\alpha$, where $A_t$ is productivity. We will assume
  that productivity follows a Markov process. That is, the
  distribution of $A_{t+1}$ connditional on all information at time
  $t$, only depends on $A_t$. The problem is to choose $c_t$ given the
  current $k_t$ and $A_t$. The Bellman equation is
  \begin{align*}
    V(k_t, A_t) = & \max_{c_t, k_{t+1}} \log (c_t) + \delta
                    \Er[V(k_{t+1}, A_{t+1}) | A_t] \\
                  & \text{ s.t. } c_t + k_{t+1} = A_t k_t^\alpha 
  \end{align*}
  Let's guess and verify that the value function will again be
  log-linear with a constant that depends on $A_t$, i.e.
  \[ V(k,A) = b_1 \log(k) + b_0(A) \]
  Putting in this guess for $V$, and substituting in the constraint to
  eliminate $c$, the problem becomes
  \[ V(k,A) =  \max_{k'} \log (Ak^\alpha - k') + \delta
    \Er[b_1\log(k') + b_0(A') | A] \]
  Solving the first order condition gives
  \[ k' = k^\alpha \frac{A \delta b_1}{1+\delta b_1} \]
  Putting this $k'$ back into our guess for $V$ and the Bellman
  equation gives
  \begin{align*}
    b_1 \log(k) + b_0(A) = & \log \left(k^\alpha \frac{A}{1+\delta
                             b_1} \right) + \delta \Er\left[
                             b_1 \log\left(A k^\alpha \frac{A \delta
                             b_1}{1+\delta b_1}\right) + b_0(A') | A
                             \right] \\
    b_1 \log(k) + b_0(A) = & \alpha \log(k) (1+\delta b_1)  + (1+\delta b_1)\log(A)
                              + \delta \Er[b_0(A') | A] + D
  \end{align*}
  where $D = \delta b_1 \log(\delta b_1) -(1+\delta b_1)\log(1+\delta
  b_1)$. For this equation to hold for any $k$, it must be that
  \[ b_1 = \alpha (1+\delta b_1) \]
  so
  \[ b_1 = \frac{\alpha}{1-\alpha \delta}. \]
  We must also have
  \[ b_0(A) = (1+\delta b_1)\log(A) + \delta \Er[b_0(A') | A] + D. \]
  The exact value of $b_0(A)$ depends on the distribution of $A'$
  given $A$. The simplest case is if $A$ is independent over
  time. Then $\Er[b_0(A')|A] = \Er[b_0(A')]$. In that case, we get
  \begin{align*}
    \Er[b_0(A)] & = \Er\left[(1+\delta b_1) \log(A) + \delta \Er[b_0(A)]
                  + D \right] \\
    \Er[b_0(A)] & = \frac{1}{1-\delta} \left( (1+\delta b_1)
                  \Er[\log(A)] + D \right) 
  \end{align*}
  Hence,
  \[ b_0(A) = \frac{\log(A)}{1-\alpha \delta} +
    \frac{\delta}{1-\delta} \frac{\Er[\log(A')]}{1-\alpha \delta} +
    \frac{D}{1-\delta} \]
  and the value function is
  \begin{align*}
    V(k,A) = \frac{\alpha}{1-\alpha \delta} \log k + \frac{\log(A)}{1-\alpha \delta} +
    \frac{\delta}{1-\delta} \frac{\Er[\log(A')]}{1-\alpha \delta} +
    \frac{D}{1-\delta} 
  \end{align*}
\end{example}


\clearpage
\bibliographystyle{jpe}
\bibliography{../526}

\end{document}
