
\documentclass[12pt,reqno]{amsart}

\usepackage{amsmath}
\usepackage{amsfonts}
\usepackage{amssymb}

\usepackage{graphicx}
%\usepackage{epstopdf}
\usepackage[colorlinks=true]{hyperref}
\usepackage[left=1in,right=1in,top=0.9in,bottom=0.9in]{geometry}
\usepackage{multirow}
\usepackage{verbatim}
\usepackage{fancyhdr}
\usepackage{mdframed}
\usepackage{natbib}
\usepackage{multicol}
\renewcommand{\cite}{\citet}

%\usepackage[small,compact]{titlesec} 

%\usepackage{pxfonts}
%\usepackage{isomath}
\usepackage{mathpazo}
%\usepackage{arev} %     (Arev/Vera Sans)
%\usepackage{eulervm} %_   (Euler Math)
%\usepackage{fixmath} %  (Computer Modern)
%\usepackage{hvmath} %_   (HV-Math/Helvetica)
%\usepackage{tmmath} %_   (TM-Math/Times)
%\usepackage{cmbright}
%\usepackage{ccfonts} \usepackage[T1]{fontenc}
%\usepackage[garamond]{mathdesign}
\usepackage{xcolor}
\usepackage[normalem]{ulem}

\newtheorem{theorem}{Theorem}[section]
\newtheorem{conjecture}{Conjecture}[section]
\newtheorem{corollary}{Corollary}[section]
\newtheorem{lemma}{Lemma}[section]
\newtheorem{proposition}{Proposition}[section]
\theoremstyle{definition}
\newtheorem{assumption}{}[section]
%\renewcommand{\theassumption}{C\arabic{assumption}}
\newtheorem{definition}{Definition}[section]

\newtheorem{step}{Step}[section]
\newtheorem{rem}{Comment}[section]
\newtheorem{ex}{Example}[section]
\newtheorem{hist}{History}[section]
\newtheorem*{ex*}{Example}
\newtheorem{exer}{Exercise}[section]


\newenvironment{example}
  {\begin{mdframed}\begin{ex}}
  {\end{ex}\end{mdframed}}

\newenvironment{history}
  {\begin{mdframed}\begin{hist}}
  {\end{hist}\end{mdframed}}

\newenvironment{exercise}
  {\begin{mdframed}\begin{exer}}
  {\end{exer}\end{mdframed}}

\newenvironment{remark}
  {\begin{mdframed}\begin{rem}}
  {\end{rem}\end{mdframed}}



\linespread{1.1}

\pagestyle{fancy}
%\renewcommand{\sectionmark}[1]{\markright{#1}{}}
\fancyhead{}
\fancyfoot{} 
%\fancyhead[LE,LO]{\tiny{\thepage}}
\fancyhead[CE,CO]{\tiny{\rightmark}}
\fancyfoot[C]{\small{\thepage}}
\renewcommand{\headrulewidth}{0pt}
\renewcommand{\footrulewidth}{0pt}

\fancypagestyle{plain}{%
\fancyhf{} % clear all header and footer fields
\fancyfoot[C]{\small{\thepage}} % except the center
\renewcommand{\headrulewidth}{0pt}
\renewcommand{\footrulewidth}{0pt}}

\makeatletter
\renewcommand{\@maketitle}{
  \null 
  \begin{center}%
    \rule{\linewidth}{1pt} 
    {\Large \textbf{\textsc{\@title}}} \par
    {\normalsize \textsc{Paul Schrimpf}} \par
    {\normalsize \textsc{\@date}} \par
    {\small \textsc{University of British Columbia}} \par
    {\small \textsc{Economics 526}} \par
    \rule{\linewidth}{1pt} 
  \end{center}%
  \par \vskip 0.9em
}
\makeatother

\definecolor{ubcblue}{cmyk}{1.00,.62,.0,.52}
\definecolor{ubcgrey}{cmyk}{.42,.08,.0,.40}

\hypersetup{
  linkcolor  = ubcblue,
  citecolor  = ubcgrey,
  urlcolor   = ubcblue,
  colorlinks = true
}

\usepackage{mathtools}
\newcommand{\verteq}{\rotatebox{90}{$\,=$}}
\newcommand{\equalto}[2]{\underset{\scriptstyle\overset{\mkern4mu\verteq}{#2}}{#1}}
\newcommand{\nulls}{\mathrm{null}}
\newcommand{\range}{\mathrm{range}}

\newcommand{\argmax}{\operatornamewithlimits{arg\,max}}
\newcommand{\argmin}{\operatornamewithlimits{arg\,min}}
\def\inprobLOW{\rightarrow_p}
\def\inprobHIGH{\,{\buildrel p \over \rightarrow}\,} 
\def\inprob{\,{\inprobHIGH}\,} 
\def\indist{\,{\buildrel d \over \rightarrow}\,} 
\def\F{\mathbb{F}}
\def\R{\mathbb{R}}
\def\Er{\mathrm{E}}
\def\Pr{\mathrm{P}}
\def\En{\mathbb{E}_n}
\def\Pn{\mathbb{P}_n}
\newcommand{\gmatrix}[1]{\begin{pmatrix} {#1}_{11} & \cdots &
    {#1}_{1n} \\ \vdots & \ddots & \vdots \\ {#1}_{m1} & \cdots &
    {#1}_{mn} \end{pmatrix}}
\newcommand{\iprod}[2]{\left\langle {#1} , {#2} \right\rangle}
\newcommand{\norm}[1]{\left\Vert {#1} \right\Vert}
\newcommand{\abs}[1]{\left\vert {#1} \right\vert}
\renewcommand{\det}{\mathrm{det}}
\newcommand{\rank}{\mathrm{rank}}
\newcommand{\spn}{\mathrm{span}}
\newcommand{\row}{\mathrm{Row}}
\newcommand{\col}{\mathrm{Col}}
\renewcommand{\dim}{\mathrm{dim}}
\newcommand{\prefeq}{\succeq}
\newcommand{\pref}{\succ}
\newcommand{\seq}[1]{\{{#1}_n \}_{n=1}^\infty }
\renewcommand{\to}{{\rightarrow}}
\newcommand{\corres}{\overrightarrow{\rightarrow}}
\def\rel{\,{\buildrel R \over \sim}\,}

 


\title{Linearity}
\date{\today}

\begin{document}

\maketitle

These notes are about linear algebra. References from the primary
texts are chapters 10, 11, and 27 of \cite{sb1994} and section 1.4 and
portions of chapter 3 of \cite{carter2001}. There are many mathematics
texts on linear algebra.  \cite{axler1997} is excellent. Many people
like
\href{http://ocw.mit.edu/courses/mathematics/18-06-linear-algebra-spring-2010/video-lectures/}
{Gilbert Strang's video lectures} (and his textbook). These notes were
originally based on the material in \cite{sb1994}, but they are now
closer to the approach of \cite{axler1997} for finite dimensional
spaces, and some mix of \cite{luenberger1969} and \cite{clarke2013}
for infinite dimensional spaces. 

We will study linear algebra with two goals in mind. First, we will
finally carefully prove that the Lagrangian works. Recall that for a
constrained optimization problem, 
\[ \max_x f(x) \text{ s.t.  } h(x) = c, \]
we argued that $x^*$ is a local max if for all $v$, 
\[ Dh_{x^*} v = 0 \implies Df_{x^*} v = 0. \] 
We then made a heuristic argument that this is equivalent to the
existence of Lagrange multipliers such that 
\[ Df_{x^*} + \lambda^T Dh_{x^*} = 0. \]
This result will be a consequence of the separating hyperplane theorem
(or the geometric form of the Hahn-Banach theorem for infinite
dimensional spaces).

The second main result that we will build toward are the first and
second welfare theorems. The first welfare theorem states that a
competitive equilibrium is Pareto efficient. The second welfare
theorems states that every Pareto efficient allocation can be achieved
by some competitive equilibrium. The second welfare theorem is also a
consequence of the separating hyperplane theorem. The welfare theorems
involve preferences (the subject of the notes on sets), vector spaces
(the topic of these notes), and some continuity (the subject of the
previous set of notes).
%%%%%%%%%%%%%%%%%%%%%%%%%%%%%%%%%%%%%%%%%%%%%%%%%%%%%%%%%%%%%%%%%%%%%% 

\section{Vector spaces}

A vector space is a set whose elements can be added and scaled. Vector
spaces appear quite often in economics because many economic
quantities can be added and scaled. For example, if firm $A$ produces
quantities $y_1^A$ and $y_2^A$ of goods $1$ and $2$, while firm $B$
produces $(y_1^B,y_2^B)$, then total production is $(y_1^A+y_1^B,
y_2^A+y_2^B)$. If firm $A$ becomes 10\% more productive, then it will
produce $(1.1 y_1^A, 1.1 y_2^A)$.

We have been working with $\R^n$, which is the most common vector
space. There are three ways of approaching vector spaces. The first is
geometrically --- introduce vectors as directed arrows. This works
well in $\R^2$ and $\R^3$ but is difficult in higher dimensions.  The
second is analytically --- by treating vectors as n-tuples of numbers
$(x_1, ..., x_n)$. The third approach is axiomatically --- vectors are
elements of a set that has some special properties. You likely already
have some familiarity with the first two approaches. Here, we are
going to take the third approach. This approach is more abstract, but
this abstraction will allow us to generalize what we might know about
$\R^n$ to other more exotic vector spaces. Also, some theorems and
proofs become shorter and more elegant.

\begin{definition}
  A \textbf{vector space} is a set $V$  with
  two operations, addition $+$, which takes two elements of $V$ and
  produces another element in $V$, and scalar multiplication $\cdot$,
  which takes an element in $V$ and an element in $\R$ and
  produces an element in $V$, such that
  \begin{enumerate}
  \item $(V, +)$ is a commutative group, i.e.
    \begin{enumerate}
    \item Closure: $\forall v_1 \in V$ and $v_2 \in V$ we have $v_1
      + v_2 \in V$. 
    \item Associativity: $\forall v_1, v_2, v_3 \in V$ we have $v_1
      + (v_2 + v_3 ) = (v_1 + v_2) + v_3 $. 
    \item Identity exists: $\exists 0 \in V$ such that $\forall v \in
      V$, we have $v + 0 = v$
    \item Invertibility: $\forall v \in V$ $\exists -v \in V$ such
      that $v + (-v) = 0$
    \item Commutativity: $\forall v_1, v_2 \in V$ we have $v_1+v_2 =
      v_2 + v_1$
    \end{enumerate}
  \item Scalar multiplication has the following properties:
    \begin{enumerate}
    \item Closure: $\forall v \in V$ and $\alpha \in \R$ we have
      $\alpha v \in V$
    \item Distributivity: $\forall v_1 , v_2 \in V$ and $\alpha_1, \alpha_2 \in
      \R$
      \begin{align*}
        \alpha_1 (v_1 + v_2) = \alpha_1 v_1 + \alpha_1 v_2 
      \end{align*}
      and 
      \begin{align*}
        (\alpha_1 + \alpha_2)v_1 = \alpha_1 v_1 + \alpha_2 v_1
      \end{align*}
    \item Consistent with field multiplication: $\forall v \in V$ and
      $\alpha_1, \alpha_2 \in V$ we have
      \begin{align*}
        1 v = v
      \end{align*}
      and 
      \begin{align*}
        (\alpha_1 \alpha_2) v =\alpha_1 (\alpha_2 v)
      \end{align*}
    \end{enumerate}
  \end{enumerate}
\end{definition}
A vector space is also called a linear space. Like \citet{carter2001}
says, ``This long list of requirements does not mean that a linear
space is complicated. On the contrary, linear spaces are beautifully
simple and possess one of the most complete and satisfying theories in
mathematics.  Linear spaces are also immensely useful providing one of
the principal foundations of mathematical economics. The most
important examples of linear spaces are $\R$ and $\R^n$. Indeed, the
abstract notion of linear space generalizes the algebraic behavior of
$\R$ and $\R^n$.''
One way of looking at vector spaces is that they are a way of trying
to generalize the things that we know about two and three dimensional
space to other contexts. 

\subsubsection{Examples}
We now give some examples of vector spaces. 
\begin{example} \label{ex:Rn}
  $\R^n$ is a vector space. You are likely already
  familiar with this space. Vector addition and multiplication are
  defined in the usual way. If $\mathbf{x}_1 = (x_{11}, ..., x_{n1})$
  and $\mathbf{x}_2 = (x_{12}, ..., x_{n2})$, then vector addition is
  defined as
  \[ \mathbf{x}_1 + \mathbf{x}_2 = (x_{11}+x_{12}, ... , x_{n1} +
  x_{n2}). \]
  The fact that $(\R^n,+)$ is a commutative group follows from the
  fact that $(\R,+)$ is a commutative group. Scalar multiplication is
  defined as
  \[ a \mathbf{x} = (a x_1, ..., ax_n) \] for $a \in \R$ and
  $\mathbf{x} \in R^n$. You should verify that the three properties in
  the definition of vector space hold.  The vector space $(\R^n, \R,
  +, \cdot)$ is so common that it is called \textbf{Euclidean
    space}\footnote{To be more accurate, Euclidean space refers to
    $\R^n$ as an inner product space, which is a special kind of
    vector space that will be defined below.} We will often just refer
  to this space as $\R^n$, and it will be clear from context that we
  mean the vector space $(\R^n, \R, + , \cdot)$. In fact, we will
  often just write $V$ instead of $(V,\R,+,\cdot)$ when referring to a
  vector space.
\end{example}
%\begin{example}
%  Any linear subspace of $\R^n$ along with the field
%  $\R$ and the usual vector addition and scalar multiplication is a
%  vector space. Linear subspaces are closed under $+$ and $\cdot$ by
%  definition. Linear subspaces inherit all the other properties
%  required by the definition of a vector space from $\R^n$.
%\end{example}
\begin{example}
  The set of all solutions to a homogenous system of linear equation
  with the right hand size equal to $0$,
  i.e., $(x_1, ..., x_n) \in \R^n$ such that 
  \begin{align*}
    a_{11} x_1 + a_{12} x_2 + ... + a_{1n} x_n = & 0 \\
    a_{21} x_1 + a_{22} x_2 + ... + a_{2n} x_n = & 0 \\
    \vdots & \vdots \\
    a_{m1} x_1 + a_{m2} x_2 + ... + a_{mn} x_n = & 0 ,
  \end{align*}
\end{example}  
Most of the time, the two operations on a vector space are the usual
addition and multiplication. However, they can appear different, as
the following example illustrates.
\begin{example}
  Take $V = \R^+$. Define ``addition'' as $x \oplus y = xy$ and define
  ``scalar multiplication'' as $\alpha \odot x = x^\alpha$. Then
  $(\R^+,\R, \oplus, \odot)$ is a vector space with identity element
  $1$.   
\end{example}

The previous few examples are each finite dimensional vector
spaces. There are also infinite dimensional vector spaces.
\begin{example}
  Let $V = \{$ all sequences of real numbers $\}$. For two sequences
  $\mathbf{x} = \{x_1, x_2, ... \}$ and $\mathbf{y} = \{y_1, y_2,
  ... \}$, define $\mathbf{x} + \mathbf{y} = \{ x_1 + y_1 , x_2 + y_2,
  ... \}$ and defie scalar multiplication as $\alpha \mathbf{x} = \{
  \alpha x_1 , \alpha x_2, ... \}$. Then this is a vector space.
\end{example}
We encounter vector spaces of sequences in economics when we study
infinite horizon discrete time optimization problems. 

Spaces of functions are often vector spaces. In economic
theory, we might want to work with a set of functions because we want
to prove something for all functions in the set. That is, we prove
something for all utility functions or for all production
functions. In non-parametric econometrics, we try to estimate an
unknown function instead of an unknown finite dimensional
parameter. For example, instead of linear regression $y = x\beta +
\epsilon$ where want to estimate the unknown vector $\beta$, we might
say $y = f(x) + \epsilon$ and try to estimate the unknown function
$f$. 

Here are some examples of vector spaces of functions. It would be a good
exercise to verify that these examples have all the properties listed
in the definition of a vector space. 
\begin{example} \label{ex:funcSpace}
  Let $V = $ all functions from $[0,1]$ to $\R$. For $f, g \in V$,
  define $f + g$ by $(f+g)(x) = f(x) + g(x)$. Define scalar
  multiplication as $(\alpha f)(x) = \alpha f(x)$. Then this is a
  vector space. 
\end{example}
Sets of functions with certain properties also form vector spaces. 
\begin{example}
  The set of all continuous functions with addition and scalar
  multiplication defined as in \ref{ex:funcSpace}.
\end{example}
\begin{example}
  The set of all $k$ times continuously differentiable functions with
  addition and scalar multiplication defined as in \ref{ex:funcSpace}.
\end{example}
\begin{example}
  The set of all polynomials with addition and scalar
  multiplication defined as in \ref{ex:funcSpace}.
\end{example}
\begin{example} 
  The set of all polynomials of degree at most $d$ with addition and scalar
  multiplication defined as in \ref{ex:funcSpace}.
\end{example}

Generally, the vector space with which we are most interested is
Euclidean space, $\R^n$. In fact, a good way to think about other
vector spaces is that they are just variations of $\R^n$. The whole
reason for defining and studying abstract vector spaces is to take our
intuitive understanding of two and three dimensional Euclidean space
and apply it to other contexts. If you find the discussion of abstract
vector spaces and their variations to be confusing, you can often
ignore it and think of two or three dimensional Euclidean space
instead.

Vector spaces often contain other vector spaces. For example, either
axis (or more generally any line passing through the origin) in $\R^2$
is itself a vector space.  
\begin{definition}
  A set $S \subseteq V$ is called a \textbf{linear
    subspace} if it is closed under (i) scalar multiplication and (ii)
  addition in other words, if 
  \begin{itemize}
  \item[(i)] for every $\mathbf{x} \in S$ and $\alpha \in \mathbb{R}$,
    we have $\alpha \mathbf{x} \in S$, and
  \item[(ii)] for every $\mathbf{x} \in S$ and $\mathbf{y} \in
    S$, we have
    $\mathbf{x} + \mathbf{y} \in S$
  \end{itemize}
\end{definition}
This two requirements are sometimes written more succintly as $\forall
\mathbf{x}, \mathbf{y} \in S$ and $\alpha \in \R$, $\alpha \mathbf{x}
+ \mathbf{y} \in S$. When the ``linear'' is clear from context, linear
subspaces are often simply called subspaces. 
\begin{exercise}
  Show that if $S$ is a linear subspace, then $0 \in S$. 
\end{exercise}
\begin{example}
  For any vector space $V$, the set of containing only the $0$
  element, $\{0\}$, is a linear subspace.
\end{example}
\begin{example}
  In $\R^2$, any line passing through the origin is a linear subspace.
  In $\R^3$, any line passing through the origin, and any plane
  passing through the origin is a linear subspace.
\end{example}
\begin{example}
  The set of all continuous functions from $\R \to \R$ such that
  $f(294) = 0$ is a linear subspace of the vector space of continuous
  functions. 
\end{example}
\begin{example}
  The set of all polynomials of degree at most $d$ is a linear
  subspace of the space of all polynomials.  
\end{example}

The intersection of two linear subspaces is also a linear subspace.
\begin{example}
  If $S$ and $W$ are linear subspaces of $V$, then so is $S \cap W$.
\end{example}
However, the union of two linear subspaces is not necessarily a
subspace. For example in $\R^2$, the two axes are each subspaces, but
their union is not because, $(1,0)$ and $(0,1)$ are both in the union,
but $(1,0) + (0,1) = (1,1)$ is not. 
\begin{definition}
  Let $S_1, ..., S_k$ be linear subspaces of $V$. The \textbf{sum} of
  them is
  \[ S_1 + \cdots + S_k = \{ \mathbf{x}_1 + \cdots + \mathbf{x}_k :
  \mathbf{x}_j \in S_j \} \]
\end{definition}
$S_1 + \cdots + S_k$ is another linear subspace of $V$. 

%%%%%%%%%%%%%%%%%%%%%%%%%%%%%%%%%%%%%%%%%%%%%%%%%%%%%%%%%%%%%%%%%%%%%%%%%%%%%%%%
\subsection{Linear combinations}

\begin{definition}
  Let $V$ be a vector space and $\mathbf{x}_1,..., \mathbf{x}_k \in
  V$. A \textbf{linear combination} of $\mathbf{x}_1,...,
  \mathbf{x}_k$ is any vector
  \[c_1 \mathbf{x}_1 + ... + c_k \mathbf{x}_k \]
  where $c_1, ..., c_k \in \R$. 
\end{definition}
Note that by the definition of a vector space (in particular the
requirement that vector spaces are closed under addition and
multiplication), it must be that $c_1 \mathbf{x}_1 + ... + c_k \mathbf{x}_k \in V$. 

If we take all possible linear combinations of, $\{c_1 x_1 + c_2
x_2 : c_1 \in \R, c_2 \in \R\}$, then the set will contain $0$, and it
will be a linear subspace. This motivates the following definition.
\begin{definition}
  Let $V$ be a vector space and $W \subseteq V$. The
  \textbf{span} of $W$ is the set
  of all finite linear combinations of elements of $W$
\end{definition}
When $W$ is finite, say $W = \{\mathbf{x}_1, ..., \mathbf{x}_k\}$, the span of $W$ is
the set 
\[ \{ c_1 \mathbf{x}_1 + ... + c_k \mathbf{x}_k : c_1, ..., c_k \in \R
\}. \] 
When $W$ is infinite, the span of $W$ is the set of all finite
weighted sums of elements of $W$.
\begin{lemma}
  The \textbf{span} of any  $W \subseteq V$ is a linear subspace.
\end{lemma}
\begin{proof}
  Left as an exercise.
\end{proof}

\begin{example}
  Let $V$ be the vector space of all functions from $[0,1]$ to $\R$ as
  in example \ref{ex:funcSpace}. The span of $\{1, x, ..., x^n\}$ is
  the set of all polynomials of degree less than or equal $n$.
\end{example}

%%%%%%%%%%%%%%%%%%%%%%%%%%%%%%%%%%%%%%%%%%%%%%%%%%%%%%%%%%%%%%%%%%%%%%%%%%%%%%%%
\subsection{Dimension, linear independence, and basis} 

You are probably familiar with the idea that $\R^n$ is
$n$-dimensional. Roughly speaking if a vector space is $n$
dimensional, then we should be able to describe any vector in it by
listing $n$ scalars or coordinates.  In this section we formally
define dimension.
\begin{definition}
  A set of vectors $W \subseteq V$, is \textbf{linearly
    independent} if the only solution to
  \begin{align*}
    \sum_{j=1}^k c_j \mathbf{x}_j = 0 
  \end{align*}
  is $c_1 = c_2 = ... = c_k = 0$ for any $k$ and $\mathbf{x}_1, ...,
  \mathbf{x}_k \in W$. If $W$ is not linearly independent, then it is
  \textbf{linearly dependent},
\end{definition}
\begin{example}
  In $\R^2$, $\{(1,0), (0,1)\}$ is linearly independent. $\{(0,0)\}$
  is linearly dependent. Any set of three or more vectors in linearly
  dependent. 
\end{example}
\begin{definition}
  The \textbf{dimension} of a vector space, $V$, is the cardinality of
  the largest set of linearly independent elements in $V$.
\end{definition} 
How large can the largest set of linearly independent elements be? The
following theorem is a starting point. It tells us that a linearly
independent set is smaller than any set that spans. 
\begin{remark}
  The above definitions for linear independence and dimension work for
  any dimension, finite or infinite. However, some of the theorems
  that follow are awkward to prove for infinite dimension, so some of
  them will assume finite dimension. Exercise \ref{ex:basis}, sketches
  how to prove the existence of a basis for infinite dimensional
  spaces. 
\end{remark}
% The above
% definitions for linear independence and dimension work for any
% dimension, finite or infinite. However, some of the theorems that
% follow are awkward to state and prove for infinite dimension, so most
% of them will assume finite dimension. Most of the important results
% also apply to infinite dimensional spaces, with the important
% exception of theorem \ref{thm:fund}.
\begin{theorem}\label{thm:spanLin}
  Suppose $\mathbf{v}_1, ...., \mathbf{v}_n$ span $V$, and 
  $\mathbf{u}_1, ..., \mathbf{u}_m$ are linearly independent, then  $n
  \geq m$. 
\end{theorem}
\begin{proof}
  Since $\mathbf{v}_1, ...., \mathbf{v}_n$ span $V$, it must be that 
  $\mathbf{u}_1, \mathbf{v}_1, ... , \mathbf{v}_n$ are linearly
  dependent. Therefore, $\exists \alpha_j \in \R$ and not all $0$ such that 
  \[ \alpha_1 \mathbf{u}_1 + \sum_{j=1}^n \alpha_{j+1} \mathbf{v}_j =
  0 \]
  Since the $\mathbf{u}$'s are linearly independent, $\mathbf{u}_1
  \neq 0$. Therefore, for some $j \geq 2$, $\alpha_j \neq 0$. Let
  $\ell$ be the largest such $j$. We can then rearrange to write
  \[ \mathbf{v}_\ell = -\frac{\alpha_1}{\alpha_\ell} \mathbf{u}_1 -
  \sum_{j=1}^{\ell-1} \frac{\alpha_{j+1}}{\alpha_\ell} \mathbf{v}_j = 0. \]
  It follows that we can remove $\mathbf{v}_\ell$ and the remaining
  $\mathbf{v}$'s along with $\mathbf{u}_1$ will still span $V$. 

  We can then repeat the above argument.  Since $\mathbf{u}_1, ...,
  \mathbf{u}_{i-1}, \mathbf{v}_1, ...., \mathbf{v}_{n-i+1}$ span $V$,
  it must be that $\mathbf{u}_1, ...,
  \mathbf{u}_{i}, \mathbf{v}_1, ...., \mathbf{v}_{n-i+1}$ are
  linearly dependent. Therefore, $\exists \alpha_j \in \R$ and not all
  $0$ such that
  \[ \sum_{k=1}^i \alpha_i \mathbf{u}_i + \sum_{j=1}^n \alpha_{j+1}
  \mathbf{v}_j = 0 \] 
  Since the $\mathbf{u}$'s are linearly
  independent, for some $j \geq i+1$, $\alpha_j \neq 0$. Let $\ell$ be
  the largest such $j$. We can then rearrange to write
  \[ \mathbf{v}_\ell = -\sum_{k=1}^i \frac{\alpha_i}{\alpha_\ell} \mathbf{u}_i -
  \sum_{j=1}^{\ell-1} \frac{\alpha_{j+i+1}}{\alpha_\ell} \mathbf{v}_j = 0. \]
  It follows that we can remove $\mathbf{v}_\ell$ and the remaining
  $\mathbf{v}$'s along with $\mathbf{u}_1, ... , \mathbf{u}_i$ will
  still span $V$. 

  If there are fewer $\mathbf{v}$'s than $\mathbf{u}$'s, then the
  above induction contradicts the assumption that the $\mathbf{u}$'s
  are linearly independent.
\end{proof}
\begin{remark}
  A more general version of theorem \ref{thm:spanLin} is that if $W
  \subseteq V$ spans $V$ and $U \subseteq V$ is linearly independent,
  then $\abs{W} \geq \abs{U}$. 

  The argument we used to prove theorem \ref{thm:spanLin} does not
  suffice when $U$ is infinite because the iterative replacement of
  elements of $W$ with elements of $U$ would never terminate. Exercise
  \ref{ex:basis} sketches how to prove this theorem with infinite
  $U$. 
\end{remark}
This theorem implies that if $B$ is linearly independent and spans
$V$, then any other linearly independent set must have smaller
cardinality. Hence, the dimension of $V$ must equal the cardinality of
$B$. Sets that are linearly independent and span a vector space are
very useful, so they have a name.
\begin{definition}
  A \textbf{basis} of a vector space $V$ is any set of linearly
  independent vectors $B$ such that the span of $B$ is $V$.
\end{definition}
If $V$ has a basis with $k$ elements, then the dimension of $V$ must
be at least $k$. In fact, the previous theorem implies that dimension
of $V$ must be exactly $k$. Another consequence is that any two bases must have
the same cardinality. 
\begin{corollary}
  Any two bases for a vector space have the same cardinality.
\end{corollary}
\begin{proof}
  Let $B_1$ and $B_2$ be bases for a vector space $V$. Since $B_1$
  spans and $B_2$ is linearly independent, by theorem
  \ref{thm:spanLin}, $|B_1| \geq |B_2|$. Conversely, since $B_2$ spans
  and $B_1$ is linearly independent, $|B_2| \geq |B_1|$. Hence $|B_1|
  = |B_2|$.
\end{proof}
\begin{example}
  A basis for $\R^n$ is $e_1 = (1, 0, ..., 0 )$, $e_2 = (0, 1, 0, ...,
  0)$, $...$, $e_n = (0, ... , 0 , 1)$. This basis is called the
  standard basis of $\R^n$. 

  The standard basis is not the only basis for $\R^n$. In fact, there
  are infinite different bases. Can you give some examples?
\end{example}
\begin{exercise} 
  What is the dimension of each of the examples of vector spaces
  above? Can you find a basis for them? 
  
  Note that an important requirement for a basis is that every
  $\mathbf{x} \in V$ can be written as a \textbf{finite} sum of basis
  elements. Therefore, for example, in
  $\ell^\infty = \{(x_1, x_2, ...): x_i \in \R, \sup_{1\leq i\leq
    \infty} \abs{x_i} < \infty\}$,
  consider the set $E = \{e_i\}_{i=1}^\infty$, where $e_i$ is an
  element of all $0$'s, except for a $1$ in the $i$th position. $E$ is
  linearly independent, but $E$ does not span $\ell^\infty$ because
  e.g. you cannot write $(1,1,1,1,...)$ as a finite sum of the
  elements of $E$.
\end{exercise}
Given a set that spans a vector space, it is always possible to remove
elements until the set is also linearly independent, and hence a
basis. Doing this will be useful in various proofs, so we state it is
a lemma.
\begin{lemma}\label{lem:spanToBasis}
  Suppose $\mathbf{v}_1, ..., \mathbf{v}_n$ span $V$. Then there is a
  subset of the $\mathbf{v}$'s that is a basis for $V$.
\end{lemma}
\begin{proof}
  We proceed by induction. If $\mathbf{v}_1 = 0$, then remove
  it. Otherwise, keep it. For $j=2,..., n$, if $\mathbf{v}_j \in
  \text{span}(\mathbf{v}_1, ..., \mathbf{v}_{j-1})$, then delete it,
  otherwise keep it. At every step, the remaining $\mathbf{v}$ still
  span $V$. Furthermore, each step ensures that the non-deleted
  $\mathbf{v}_1, ..., \mathbf{v}_j$ are linearly independent. 
\end{proof}
Conversely, every linearly independent set can be expanded to a
basis. 
\begin{lemma}\label{lem:linToBasis}
  Suppose $V$ is finite dimensional and $\mathbf{v}_1, ...,
  \mathbf{v}_n$ are linearly independent.  Then $\exists$
  $\mathbf{v}_{n+1}, ... , \mathbf{v}_m$ such that $\mathbf{v}_1, ...,
  \mathbf{v}_m$ is a basis for $V$.
\end{lemma}
\begin{proof}
  By assumption $V$ is finite dimensional, so it has a basis, say
  $\mathbf{u}_1, ..., \mathbf{u}_m$. Following the same argument as in
  the proof of theorem \ref{thm:spanLin}, the $\mathbf{u}$'s can be
  replaced by the $\mathbf{v}$'s to get another basis consisting of
  all the $\mathbf{v}$'s and $m-n$ of the $\mathbf{u}$'s.
\end{proof}
This lemma implies that if $V$ is finite dimensional, then a basis for
$V$ exists. Even if $V$ is infinite dimensional, then a basis
exists. The argument is outlined in the following exercise.
\begin{exercise}[Difficult]\label{ex:basis}
  A partial order is a relation that is reflexive, transitive, and
  antisymmetric. A set with a partial order is called a partially
  ordered set. Not all elements of a partially ordered set are
  comparable (a partial order need not be complete). A \textbf{chain}
  is any subset of a partially order set where all elements are
  comparable to one another. If $Y$ is a partially ordered set with
  partial order $\succeq$, and $A \subseteq Y$, then an upper bound
  for $A$ is a $y \in Y$ such that $y \succeq z$ for all $z \in A$.
  $y \in Y$ is a maximal element if there does not exist any $z \in Y$
  such that $z \succ y$.
  
  Zorn's lemma\footnote{Zorn's lemma is equivalent to the axiom of
    choice, which is a basic assumption of set theory. The axiom of
    choice says that given a collection of non-empty sets,
    $\{S_i\}_{i \in \mathcal{I}}$, we can choose an element from each
    set $\{x_i\}_{i \in \mathcal{I}}$.} says that if every chain in a
  partially ordered set has an upper bound, then the partially ordered
  set has a maximal element.
  \begin{enumerate}
  \item Let $X$ be a set $\mathcal{P}(X)$ be a the power set of
    $X$. Show that $\subseteq$ is a partial order on $\mathcal{P}(X)$.
  \item Let $V$ be a vector space (possibly of infinite dimension) and
    let $L \subseteq V$ be linearly independent. Define 
    \[ \mathcal{P} = \{S \subseteq V: L \subseteq S \text{ and } S
    \text{ linearly independent}\} \]
    Show that $(\mathcal{P},\subseteq)$ is a partially order set (this is
    short).
  \item Let $\mathcal{C} \subseteq \mathcal{P}$ be a chain. Show that
    $\cup_{C \in \mathcal{C}} C$ is an upper bound for
    $\mathcal{C}$. [Hint: you need to show $\cup_{C \in \mathcal{C}} C
    \in \mathcal{P}$.]
  \item Argue that $\mathcal{P}$ has a maximal element, $B$. [Hint:
    use Zorn's lemma.]
  \item Show that $\spn(B) = V$. [Hint: if $x \not\in \spn(B)$, then
    $\{x\} \cup B \in \mathcal{P}$.]
  \item Conclude $B$ is a basis for $V$.
  \end{enumerate}
  For finite dimensional spaces, we could explicitly extend a linearly
  independent set to a basis by adding elements. This exercise shows a
  similar result for infinite dimensional spaces.
\end{exercise}

The elements of a vector space can always be written uniquely in terms
of a basis.
\begin{lemma} \label{lem:uniqueRep}
  Let $B$ be a basis for a vector space $V$. Then
  $\forall \mathbf{x} \in V$ there exists a unique $x_1, ..., x_k \in \R$ and
  $b_1, ..., b_k \in B$
  such that $\mathbf{x} = \sum_{i=1}^k x_i b_i$  
\end{lemma}
\begin{proof}
  By the definition of a basis, $B$ spans $V$, so such
  $(x_1, ..., x_k)$ must exist. Now suppose there exists another such
  $(x_1', ..., x_j')$ and associated $b_i'$. The $\{b_1, ..., b_k\}$ and
  $\{b_1', ..., b_j'\}$ might not be the same collection of elements
  of $B$. Let $\{\tilde{b}_1, ..., \tilde{b}_n \} =  \{b_1, ...,
  b_k\} \cup \{b_1', ..., b_j'\}$. Define $\tilde{x}_i = x_j$ if
  $\tilde{b}_i = b_j$, else $0$. Similarly define $\tilde{x}_i'$. With
  this new notation we have
  \begin{align*}
    v = \sum_{i=1}^n \tilde{x}_i \tilde{b}_i = & \sum_{i=1} \tilde{x}_i' \tilde{b}_i \\
    \sum_{i=1}^n (\tilde{x}_i - \tilde{x}_i')\tilde{b}_i = & 0 \\
  \end{align*}
  However, if $B$ is a basis, its elements must be linearly
  independent so $\tilde{x}_i = \tilde{x}_i'$ for all $i$, so the
  original $x_1, ..., x_k$ must be unique.
\end{proof}

\subsection{$\R^n$ as the  only finite dimensional vector spaces}

$\R^n$ is the only $n$-dimesion vector space in the sense that any
other finite dimensional vector space can be viewed as a simple change
of basis.

Suppose $V$ is an $n$-dimension vector space. By the definition of dimension,
there must be a set of $n$ linearly independent elements that span
$V$. These elements form a basis. Call them $b_1, ..., b_n$. For each
$\mathbf{x} \in V$, there are unique $x_1, ..., x_n \in \R$ such that 
\[ \mathbf{x} = \sum_{i=1}^n x_i b_i. \]
Thus we can construct a function, say $\mathcal{I}: V \to \R^n$ defined by 
\[ \mathcal{I}(\mathbf{x}) = (x_1, ..., x_n). \]
By lemma \ref{lem:uniqueRep}, $\mathcal{I}$ must be one-to-one. $\mathcal{I}$ must also be
onto since by definition of a vector space, for any $(x_1, ..., x_n)
\in \R^n$, the linear combination, $\sum_{i=1}^n x_i b_i$ is in
$V$. Moreover, $\mathcal{I}$ preserves addition in that for
any $\mathbf{x}^1, \mathbf{x}^2 \in V$,
\begin{align*}
  \mathcal{I}(\mathbf{x}^1 + \mathbf{x}^2) = & (x_1^1 + x_1^2, ..., x_n^1 + x_n^2) \\
  = & (x_1^1, ..., x_n^1) + (x_1^2 + ... + x_n^2) \\
  = & \mathcal{I}(\mathbf{x}^1) + \mathcal{I}(\mathbf{x}^2).
\end{align*}
Similarly, $\mathcal{I}$ preserves scalar multiplication in that for all $\mathbf{x} \in
V$, $\alpha \in \R$
\[ \mathcal{I}(\alpha \mathbf{x}) = \alpha \mathcal{I}(\mathbf{x}). \]
Thus, $V$ and $\R^n$ are essentially the same in that there is a
one-to-one and onto mapping between them that preserves all the
properties that make them vector spaces. 
\begin{definition}
  Let $V$ and $W$ be vector spaces over the field $\F$. $V$ and $W$ are
  \textbf{isomorphic} if there exists a one-to-one and onto function,
  $\mathcal{I}:V \to W$ such that 
  \[ \mathcal{I}(\mathbf{x}^1 + \mathbf{x}^2) =
  \mathcal{I}(\mathbf{x}^1) + \mathcal{I}(\mathbf{x}^2) \] for all
  $\mathbf{x}^1, \mathbf{x}^2 \in V$, and
  \[ \mathcal{I}(\alpha \mathbf{x}) = \alpha \mathcal{I}(\mathbf{x}) \]
  for all $\mathbf{x} \in V$, $\alpha \in \F$.
  Such an $\mathcal{I}$ is called an \textbf{isomorphism}.
\end{definition}
The discussion preceeding this definition showed that all
$n$-dimensional real\footnote{Here ``real'' refers to the fact that
  the scalars for the vector space are real numbers. In this course,
  all vector spaces will be real. However, you can define vector
  spaces with scalars from other fields, such as the complex numbers.}
vector spaces are isomorphic to $\R^n$. 
%%%%%%%%%%%%%%%%%%%%%%%%%%%%%%%%%%%%%%%%%%%%%%%%%%%%%%%%%%%%%%%%%%%%%%%%%%%%%%%%

\section{Normed vector spaces}

One property of two and three dimensional Euclidean space is that
vectors have lengths. Our definition of vector spaces does not
guarantee that we have a way of measuring length, so let's define a
special type of vector space where we can measure length. 
\begin{definition}
  A \textbf{normed vector space}, $(V,+,\cdot, \norm{\cdot})$, is a
  vector space with a function, called the \textbf{norm}, from $V$ to
  $\R$ and denoted by $\norm{v}$ with the following properties:
  \begin{enumerate}
  \item (Positive definite) $\norm{v} \geq 0$ and $\norm{v} = 0$ iff $v = 0$,
  \item (Homogenous) $\norm{\alpha v} = |\alpha|\norm{v}$ for all $\alpha \in \F$,
  \item The triangle inequality holds:
    \[ \norm{v_1+v_2} \leq \norm{v_1} + \norm{v_2} \]
    for all $v_1, v_2 \in V$.
  \end{enumerate}
\end{definition}
As above, when the addition, multiplication,
and norm are clear from context, we will just write $V$ instead of
$(V,+,\cdot, \norm{\cdot})$ to denote a normed vector space. 
Like length, a norm is always non-negative and only zero for the zero
vector. Also, similar to length, if we multiply a vector by a scalar,
the norm also gets multiplied by the scalar. The triangular inequality
means that norm obeys the idea that the shortest distance between two
points is a straight line. If you go directly from $x$ to $y$ you
``travel'' $\norm{x - y}$. If you stop at point $z$ in between, you travel
$\norm{x - z} + \norm{z - y}$. The triangle inequality guarantees that 
\[ \norm{x-y} \leq \norm{x - z} + \norm{z - y}. \]

\subsection{Examples}
\begin{example}
  $\R^3$ is a normed vector space with norm
  \[ \norm{x} = \sqrt{x_1^2 + x_2^2 + x_3^2}. \] 
  This norm is exactly
  how we usually measure distance. For this reason, it is called the
  Euclidean norm.

  More generally, for any $n$, $\R^n$, is a normed vector space with
  norm 
  \[ \norm{x} = \sqrt{\sum_{i=1}^n x_i^2 }. \]
\end{example}
The Euclidean norm is the most natural way of measuring distance in
$\R^n$, but it is not the only one.  A vector space can often be given
more than one norm, as the following example shows.
\begin{example}
  $\R^n$ with the norm 
  \[ \norm{x}_p = \left( \sum_{i=1}^n |x_i|^p \right)^{1/p} \]
  for $p \in [1,\infty]$\footnote{Where $\norm{x}_\infty = \max_{1\leq
      i \leq n} |x_i| $} is a normed vector space. This norm is called
  the p-norm. 
\end{example}
For nearly all practical purposes, $\R^n$ with any p-norm is
essentially the same as $\R^n$ with any other p-norm. $\R^n$ is the
same collection of elements regardless of the choice of p-norm, and
the choice of p-norm does not affect the topology (i.e. which sets are
open and closed) of $\R^n$ or the definition of derivatives. However,
there are normed vector spaces where the choice of norm makes a
difference.
\begin{example}
  Let $\ell_p =$ infinite sequences such that 
  \[ \norm{x}_p = \left( \sum_{i=1}^\infty |x_i|^p \right)^{1/p} \]
  is finite. Then $\ell_p$ with $\norm{\cdot}_p$ is a normed vector
  space. 
\end{example}
\begin{example}
  Define 
  \[ \norm{f}_p = \left(\int_0^1 |f(x)|^p dx\right)^{1/p}. \]
  Let $\mathcal{F}_p = \{f: (0,1) \to \R \text{ such that }
  \norm{f}_p<\infty\}$. 
  Define an equivalence relation between functions as
  \[ f \sim_p \tilde{f} \iff \norm{f - \tilde{f}}_p = 0 \]
  The space  $\mathcal{L}^p(0,1) = \{\text{equivalence classes of
  }\mathcal{F}_p\}$ with norm $\norm{\cdot}_p$ is a
  normed vector space. The space is defined as the set of equivalences
  classes because if $f$ and $\tilde{f}$ differ at only a finite
  collection of points, then
  \[ \norm{f-\tilde{f}}_p = \left(\int_0^1 |f(x) - \tilde{f}(x) |^p dx\right)^{1/p} =
  0. \]
  If such $f$ and $\tilde{f}$ were considered different vectors, then
  the norm would not be positive definite. 
  
  Moreover, $\mathcal{L}^p(0,1)$ is a different
  space for different $p$. For example, $\frac{1}{x^{1/p}} \not\in
  \mathcal{L}^p(0,1)$, but $\frac{1}{x^{1/p}} \in \mathcal{L}^q(0,1)$
  for $q < p$. 
\end{example}
Having a norm allows us to consider limits, continuity, and
derivatives. When working with limits we will typically need to be
sure that Cauchy sequences converge. In other words, we will want to
work in complete normed vector spaces. Complete normed vector spaces
are called \textbf{Banach} spaces. 

%%%%%%%%%%%%%%%%%%%%%%%%%%%%%%%%%%%%%%%%%%%%%%%%%%%%%%%%%%%%%%%%%%%%%%%%%%%%%%%%

\section{Linear transformations}

An isomorphism is a one-to-one and onto (bijective) functions that
preserves addition and scalar multiplication. Can a function between
vector spaces preserve addition and multiplication without being
bijective?  Let's try to construct an example. We know from above that
all finite dimensional vector spaces are isomorphic to $\R^n$, so we
might as well work with $\R^n$. To keep everything as simple as
possible, let's just work with $\R^1$. Consider $f:\R \to \R$ defined
by $f(x) = 0$ for all $x \in \R$. Clearly, $f$ is not bijective. $f$
preserves addition since
\[ f(x) + f(y) = 0 + 0 = 0 = f(x+y). \]
$f$ also preserves multiplication because 
\[ \alpha f(x) = \alpha 0 = 0 = f(\alpha x). \]
Thus, we know there are functions that preserve addition and scalar
multiplication but are not necessarily isomorphisms. Let's give such
functions a name. 
\begin{definition}
  A \textbf{linear transformation} (aka linear function) is a
  function, $A$, from a vector space $(V,\R,+,\cdot)$ to a vector
  space $(W,\R,+,\cdot)$ such that $\forall v_1, v_2 \in V$,
  \begin{align*}
    A (v_1 + v_2) = A v_1 + A v_2 
  \end{align*}
  and 
  \begin{align*}
    A (\alpha v_1) = \alpha A v_1
  \end{align*}
  for all scalars $\alpha \in \R$.   

  A linear transformation from $V$ to $V$ is called a \textbf{linear
    operator} on $V$. A linear transformation from $V$ to $\R$ is
  called a \textbf{linear functional} on $V$.
\end{definition}
Any isomorphism between vector spaces is a linear
transformation. 
\begin{example}
  Define $f: \R^2 \to \R$ by $f( (x_1, x_2) ) = x_1$, that is $f(x)$ is
  the first coordinate of $x$. Then,
  \[ f(\alpha x + y) = \alpha x_1 + y_1 = \alpha f(x) + f(y) \]
  so $f$ is a linear transformation.
\end{example}

In general we can construct linear transformations between finite
dimensional vector spaces as follows. Let 
\[ A = \gmatrix{a} \]
be a matrix. As usual let 
\[ A\mathbf{x} = 
\begin{pmatrix} 
  \sum_{j=1}^n a_{1j} x_j \\
  \vdots \\
  \sum_{j=1}^n a_{mj} x_j 
\end{pmatrix}, \] for $\mathbf{x} = (x_1, ..., x_n) \in \R^n$. Then
$A$ is a linear transformation from $\R^n$ to $\R^m$. You may want to
verify that $A(\alpha \mathbf{x}_1 + \mathbf{x}_2 ) = \alpha A
\mathbf{x}_1 + A \mathbf{x}_2$ for scalars $\alpha \in \R$ and
vectors $\mathbf{x}_1, \mathbf{x}_2 \in \R^n$. 

Conversely let $A$ be a linear transformation from $V$ to $W$ (if it
is helpful, you can let $V=\R^n$ and $W=\R^m$), and let $b_1, b_2,
..., b_n$ be a basis for $V$. By the definition of a basis, any $v \in
V$ can be written $v = \sum_{j=1}^n v_j b_j$ for some $v_j
\in \R$. By the definition of a linear transformation, we have
\begin{align*}
  A v = \sum_{j=1}^n v_j A b_j. 
\end{align*}
Thus, a linear transformation is completely determined by its action
on a basis. Also, if $d_1, ..., d_m$ is a basis for $W$ then for each
$A b_j$ we must be able to write $A b_j$ as a sum of the basis
elements $d_1, ..., d_m$, i.e.\
\begin{align*}
  A b_j = \sum_{i=1}^m a_{ij} d_i.
\end{align*}
Substituting this equation into the previous one, we can write $Av$ as
\begin{align*}
  Av = & \sum_{j=1}^n v_j A b_j \\
  = & \sum_{j=1}^n v_j \sum_{i=1}^m a_{ij} d_i \\
  = & \sum_{i=1}^m d_i \left( \sum_{j=1}^n a_{ij} v_j \right) 
\end{align*}
Thus, associated with a linear transformation there is an array of
$a_{ij} \in \R$ determined by the linear transformation (and choice of
basis for $V$ and $W$). In the previous paragraph, we saw that
conversely, if we have an array of $a_{ij} \in \R$ we can construct a
linear transformation. This leads us to the following result.
\begin{theorem}
  For any linear transformation, $A$, from $\R^n$ to $\R^m$ there is an
  associated $m$ by $n$ matrix,
  \[ 
  \gmatrix{a}
  \]
  where $a_{ij}$ is defined by $A e_j = \sum_{i=1}^m a_{ij}
  e_i$. Conversely, for any $m$ by $n$ matrix, there is an associated
  linear transformation from $\R^n$ to $\R^m$ defined by $A e_j =
  \sum_{i=1}^n a_{ij} e_i$.
\end{theorem}
Thus, we see that matrices and linear transformations from $\R^m$ to
$\R^n$ are the same thing. This fact will help us make sense of many
of the properties of matrices that we will go through in the next
section. Also, it will turn out that most of the properties of
matrices are properties of linear transformations. There are linear
transformations that cannot be represented by matrices, yet many of
the results and definitions that are typically stated for matrices
will apply to these sorts of linear transformations as well.

Two examples of linear transformations that cannot be represented by
matrices are integral and differential operators,
\begin{example}[Integral operator]
  Let $k(x,y)$ be a function from $(0,1)$ to $(0,1)$ such that
  $\int_0^1 \int_0^1 k(x,y)^2 dx dy$ is finite.  Define
  $K:\mathcal{L}^2(0,1) \rightarrow \mathcal{L}^2(0,1)$ by
  \begin{align*}
    (K f) (x) = \int_0^1 k(x,y) f(y) dy
  \end{align*}
  Then $K$ is a linear transformation because
  \begin{align*}
    (K ( \alpha f + g) ) (x) = & \int_0^1 k(x,y) (\alpha f(y) + g(y))dy
    \\
    = & \alpha \int_0^1 k(x,y) f(y) dy + \int_0^1 k(x,y) g(y)dy \\
    = & \alpha (K f) (x) + (K g)(x) 
  \end{align*}      
\end{example}

\begin{example}[Conditional expectation\label{ex:condex}]
  One special type of an integral operator that appears often in
  economics is the conditional expectation operator. Suppose $X$ and
  $Y$ are real valued random variables with joint pdf $f_{xy}(x,y)$
  and marginal pdfs $f_x(x) = \int_\R f(x,y) dy$ and $f_y(y) = \int_\R
  f(x,y) dx$. Consider the vector spaces 
  \[ 
  V = \mathcal{L}^2(\R,f_y) = \{g:
  \R \to \R \text{ such that } \int_{\R} f_y(y) g(y)^2 dy <
  \infty \} \]
  and 
  \[ 
  W = \mathcal{L}^2(\R,f_x) = \{g:
  \R \to \R \text{ such that } \int_{\R} f_x(x) g(x)^2 dx <
  \infty \} 
  \]  
  $V$ is the space of all functions of $Y$ such that the variance of
  $g(Y)$ is finite. Similarly, $W$ is the space of all functions of
  $X$ such that the variance of $g(X)$ is finite. 
  The conditional expectation operator is $\mathcal{E}: V \to W$
  defined by 
  \begin{align*}
    (\mathcal{E} g)(x) = E[g(Y) | X = x] = & \int_\R
    \frac{f_{xy}(x,y)}{f_x(x) f_y(y)} g(y) f_y(y) dy.
  \end{align*}
  The conditional expectation operator is an integral operator, so it
  is a linear transformation. 
\end{example}

\begin{example}[Differential operator]
  Let $C^\infty(0,1)$ be the set of all infinitely differentiable
  functions from $(0,1)$ to $\R$. $C^\infty(0,1)$ is a vector space.
  Let $D:C^\infty(0,1) \rightarrow C^\infty(0,1)$ be defined by
  \[ (D f) (x) = \frac{d f}{dx}(x) \]
  Then $D$ is a linear transformation.
\end{example}
Integral and differential operators are very important when studying
differential equations.  They are also useful in many areas of
econometrics and in dynamic programming. We already encountered some linear
transformations on infinite dimensional spaces when studying optimal
control. 

\section{Matrix operations and properties}

Let $A$ and $B$ be linear transformations from $\R^m$ to $\R^n$ and
let $\gmatrix{a}$ and $\gmatrix{b}$ be the associated matrices.  Since
the linear transformation $A$ and the matrix $\gmatrix{a}$ represent
the same object, we will use $A$ to denote both.  From the previous
section, we know that $A$ and $B$ are characterized by their action on
the standard basis vectors in $\R^n$. In particular, $A e_j =
\sum_{i=1}^m a_{ij} e_i$ and $B e_j = \sum_{i=1}^m b_{ij} e_i$. 

\subsection{Addition} 
To define matrix addition, it makes sense to
require $(A+B)x = Ax + Bx$. Then,
\begin{align*}
  (A + B) e_j = & A e_i + B e_j \\
  = & \sum_{j=1}^m a_{ij} e_i + \sum_{j=1}^m b_{ij} e_i\\
  = & \sum_{j=1}^m (a_{ij} + b_{ij}) e_i,
\end{align*}
so the only way sensible way to define matrix addition is 
\begin{align*}
  A + B = \begin{pmatrix} a_{11} + b_{11} & \cdots &
    a_{1n} + b_{1n}  \\ \vdots & \ddots & \vdots \\ a_{m1} + b_{m1} & \cdots &
    a_{mn}+b_{mn} \end{pmatrix}
\end{align*}
As an exercise, you might want to verify that matrix addition has the
following properties:
\begin{enumerate}
\item Associative: $A+(B + C) = (A+B) + C$,
\item Commutative: $A + B = B + A$ ,
\item Identity: $A + \mathbf{0} = A$, where $\mathbf{0}$ is an $m$ by
  $n$ matrix of zeros, and
\item Invertible $A + (-A) = \mathbf{0}$ where $-A = \gmatrix{-a}$.
\end{enumerate}

\subsection{Scalar multiplication}
The definition of linear transformations requires that $A \alpha x =
\alpha A x$ where $\alpha \in \mathbb{R}$ and $x \in V$. To be
consistent with this, for matrices we must define
\begin{align*}
  \alpha A = \begin{pmatrix} \alpha a_{11} & \cdots &
    \alpha a_{1n} \\ \vdots & \ddots & \vdots \\ \alpha a_{m1} & \cdots &
    \alpha a_{mn} \end{pmatrix}
\end{align*}
We have now defined addition and scalar multiplication for
matrices. It should be no surprise that the set of all $m$ by $n$
matrices along with these two operations and the field $\R$ forms a
vector space. 
\begin{example}
  The set of all $m$ by $n$ matrices is a vector space. 
\end{example}
In fact, the above is not only true of the set of all
$m$ by $n$ matrices, but of any set of all linear transformations
between two vector spaces.
\begin{example}
  Let $L(V,W)$ be the set of all linear transformations from $V$ to
  $W$. Define addition and scalar multiplication as above. Then 
  $L(V,W)$ is a vector space.
\end{example}
$L(\R^n, \R^m)$ is the set of all linear transformations from $\R^n
\to \R^m$, i.e.\ all $m$ by $n$ matrices. 

\subsection{Matrix multiplication}
Matrix multiplication is really the composition of two linear
transformations. Let $A$ be a linear transformation from $\R^n$ to
$\R^m$ and $B$ be a linear transformation from $\R^p$ to $\R^n$. Now,
we defined matrices by looking at how a linear tranformation acts on a
basis vectors, so to define multiplication, we should look at $A(B
e_k)$
\begin{align*}
  A(B e_k) = & A (\sum_{j=1}^n b_{jk} e_j) & \text{definition of $Be_k$}\\
  = & \sum_{j=1}^n b_{jk} A e_j & \text{Definition of linear
    transformtion} \\
  = & \sum_{j=1}^n b_{jk} \left(\sum_{i=1}^m a_{ij} e_i\right)  &
  \text{definition of $Ae_j$} \\
  = & \sum_{i=1}^m \left(\sum_{j=1}^n a_{ij} b_{jk} \right) e_i \\
  = & \begin{pmatrix} 
    \sum_{j=1}^n a_{1j} b_{j1} & \cdots & \sum_{j=1}^n a_{1j} b_{jp} \\
    \vdots & \ddots & \vdots \\
    \sum_{j=1}^n a_{mj} b_{j1} & \cdots & \sum_{j=1}^n a_{mj} b_{jp}
  \end{pmatrix} e_k  \\
  = & (AB)e_k .
\end{align*}
The indexing in the above equations is unpleasant and could be
confusing. The important thing to remember is that matrix
multiplication is the composition of linear transformations. It then
makes sense that if $A$ is $m$ by $n$ (a transformation from $\R^n$ to
$\R^m$)and $B$ is $k$ by $l$ (a transformation from $R^l$ to $\R^k$),
we can only multiply $A$ times $B$ if $k = m$. Matrix multiplication
has the following properties:
\begin{enumerate}
\item Associative: $A(BC) = (AB) C$
\item Distributive: $A(B+C) = AB + AC$ and $(A+B)C = AC + BC$. 
\item Identity: $AI_n = A$ where $A$ is $m$ by $n$ and $I_n$ is the linear
  transformation from $\R^n$ to $\R^n$ such that $I_nx = x \forall x \in
  \R^n$.
\end{enumerate}
Matrix multiplication is not commutative. 

\section{Null Spaces and Ranges}

We are often interested in solving linear equations of the form $A x =
b$, where $x \in V$, $b \in W$, and $A \in L(V,W)$. For example, $V$
could be $\R^n$, and $W$ could be $\R^m$ and then $A$ would be an $m
\times n$ matrix. The null space and range of a linear transformation
are two subspaces that can describe when the solution to $Ax=b$ is
unique and when it the solution exists.

\begin{definition}
  Let $A \in L(V,W)$. The set of solutions to the homogeneous
  equation $Ax = 0$ is the \textbf{null space} (or kernel) of
  $A$, denoted by $\mathcal{N}(A)$ (or $\nulls A$),
  \[ \mathcal{N}(A) = \{x \in V: Ax = 0 \} \]
\end{definition}
As its name suggests, the null space of a linear transformation is a
subspace.
\begin{exercise}
  Show that $\mathcal{N}(A)$ is a linear subspace.
\end{exercise}
Null spaces are important for studying linear equations because if
$z \in \mathcal{N}(A)$, then $A (x + z) = Ax + A z = Ax$. In other
words, if $Ax = b$ for some $x$, then $A(x+z) = b$ for all
$z \in \mathcal{N}(A)$. Note that $0 \in \mathcal{N}(A)$ always. From
this discussion, we see that if $A x = b$ has at least one solution,
then it will have multiple solutions if $\mathcal{N}(A) \neq \{0 \}$.
\begin{definition}
  Let $A \in L(V,W)$. $A$ is \textbf{one-to-one} (or
  \textbf{injective}) if $Ax = A v \implies x = v$.
\end{definition}
Note that $A$ is injective if and only if $\mathcal{N}(A) = 0$. Thus,
we can also say that if $A x = b$ has one solution, then it will have
multiple solutions if $A$ is not injective.

\begin{definition}
  Let $A \in L(V,W)$. The \textbf{range} of $A$ is the subset of $W$
  consisting of $A x$ for some $x \in V$, i.e.
  \[ \range A = \{ Ax: x \in V \} \subseteq W \]
\end{definition}
When $A$ is a matrix, its range is called its column space. 
\begin{exercise}
  Show that the range of a linear transformation is a linear subspace.
\end{exercise}
In terms of linear equations, $Ax = b$ has a solution if and only if
$b \in \range A$. 
\begin{definition}
  Let $A \in L(V,W)$, $A$ is \textbf{onto} (or \textbf{surjective}) if
  $\range A = W$.
\end{definition}

The dimensions of the range and null spaces of
a linear transformation are related.
\begin{theorem}[Rank-Nullity theorem \label{thm:rankNull}]
  If $V$ is finite dimensional and $A \in L(V,W)$, then 
  \[ \dim(V) = \dim(\nulls A) + \dim(\range A). \]
\end{theorem}
\begin{proof}
  Since $V$ is finite dimensional and $\nulls A \subseteq V$,
  $\nulls A$ is also finite dimensional. Let $u_1, ..., u_n$ be
  a basis for $\nulls A$. By lemma \ref{lem:linToBasis} we can
  expand this to a basis for $V$. Let, $u_1, ...,
  u_n$, $e_1, ..., e_m$ be a basis for $V$, so $\dim(V) = n + m$. 
  Let $v \in V$, then $\exists \alpha$'s and $\beta$'s $\in \R$ such
  that 
  \[ v = \sum_{i=1}^n \alpha_i u_i  + \sum_{i=1}^m \beta_i e_i. \]
  Since $A$ is linear, 
  \[ A v = \sum_{i=1}^n \alpha_i A u_i  + \sum_{i=1}^m \beta_i A
  e_i. \]
  Since $u_i \in \nulls A$, 
  \[ A v =  \sum_{i=1}^m \beta_i A
  e_i. \]
  Therefore, $A e_1, ..., A e_m$ span $\range A$. 
  
  We now show that $A e_1, ..., A e_m$ must also be linearly
  independent. Suppose
  \[ \sum_{i=1}^m c_i A e_i = 0 \]
  then 
  \[ A \left( \sum_{i=1}^m c_i e_i \right) = 0\]
  so $\left( \sum_{i=1}^m c_i e_i \right) \in \nulls
  A$. However, the $u$'s span $ \nulls A$, so $\exists d$'s $\in
  \R$ such that
  \[ \sum_{i=1}^n d_i u_i = \sum_{i=1}^m c_i e_i. \]
  Finally, since $u_1, ..., u_n, e_1, ..., e_m$ are linearly
  independent, the previous equation can only hold if the $c$'s and
  $d$'s are all $0$. 
\end{proof}
This theorem has some important implications for when a linear
transformation can be one-to-one and onto. 
\begin{corollary}
  If $V$ is finite dimensional and $\dim(V) > \dim(W)$, then no linear
  transformation from $V$ to $W$ is one-to-one.
\end{corollary}
\begin{proof}
  Let $A \in L(V,W)$. $A$ is one-to-one iff $\nulls A = \{0\}$,
  i.e. iff $\dim(\text{null A}) = 0$. From the theorem, this is
  impossible since
  \begin{align*}
    \dim(\nulls A) = \dim(V) - \dim(\range A) \geq
    \dim(V) - \dim(W) > 0 
  \end{align*}
\end{proof}
\begin{corollary}
  If $W$ is finite dimensional and $\dim(V) < \dim(W)$, then no linear
  transformation from $V$ to $W$ is onto.
\end{corollary}
\begin{proof}
  Left as an exercise. 
\end{proof}

If $V$ and $W$ are normed vector spaces, then the space of linear
transformations can also be given a norm.
\begin{definition}
  A linear transformation $A:V \to W$ is bounded if there exists $M
  \in \R$ such that $\norm{Ax}_W \leq M \norm{x}_V$ for all $x \in V$.
\end{definition}
\begin{lemma}
  A linear transformation is bounded if and only if it is continuous. 
\end{lemma}
\begin{proof}
  On problem set 5.
\end{proof}
\begin{lemma}
  If $V$ and $W$ are finite dimensional, and $A \in L(V,W)$, then $A$
  is bounded.
\end{lemma}
\begin{proof}
  On problem set 5.
\end{proof}
In infinite dimensional spaces, there are discontinuous linear
transformations. Let $B(V,W)$ denote the set of all bounded linear
transformations from $V$ to $W$. 
\begin{exercise}
  Show $B(V,W)$ is a linear subspace of $L(V,W)$.
\end{exercise}
$B(V,W)$ can be given norm: 
\[ \norm{A}_{B(V,W)} = \sup_{x \neq 0, x\in V}
\frac{\norm{Ax}_{W}}{\norm{x}_V}. \]

\section{Transpose and dual spaces}

Even more can be said about linear equations after we have defined the
transpose of a linear transformation.  Defining the transpose requires
first introducing dual spaces. Dual spaces are of independent interest
because they are needed to define differentiation on vector spaces,
which we will want to do later. Also, in optimization problems, Lagrange
multipliers are elements of dual spaces. 
\begin{definition}
  Let $V$ be a vector space. The \textbf{dual space} of $V$, denote
  $V^\ast$ is the set of all (continuous)\footnote{All linear
    functionals on finite dimensional spaces are continuous. Some
    linear functionals on infinite dimensional spaces are not
    continuous. The definition of dual space does not always require
    continuity. Often the dual space is defined as the set of all
    linear functionals, and the topological dual space is the set of
    all continuous linear functionals. We will ignore this
    distinction.}  linear functionals, $v^\ast: V \to \R$.
\end{definition}
\begin{example}\label{ex:rdual}
  The dual space of $\R^n$ is the set of $1 \times n$ matrices. In
  fact, for any finite dimensional vector space, the dual space is the
  set of row vectors from that space. 
\end{example}
In fact, since any $n$ dimensional vector space is isomorphic to
$\R^n$, the dual space of any $n$ dimensional space is the space
itself. Dual spaces are especially important in economics because
prices are in dual spaces.
\begin{example}[Prices as elements of a dual space]
  Suppose we have an economy with bundles of represented by vectors in
  some vector space, $V$. There could be $n$ goods, and $V$ could be
  $\R^n$. We could also think of the bundles of good as something like
  consumption at every instance of time and every state of the
  world. In that case, $V$ would be a vector space of sequences (if
  time and states of the world are discrete) or functions (if time and
  states of the world are continuous). The dual space of $V$, $V^\ast$
  is the set of linear transformations from $V$ to $\R$. If $p \in
  V^\ast$, then $pv$ is the total cost of purchasing $v$. $p$ is the
  price vector.
\end{example}

Finite dimensional spaces are self-dual in the sense of example
\ref{ex:rdual}, i.e. $V$ and $V^*$ are isomorphic. Infinite
dimensional spaces are often not self-dual, as in the following
example.
\begin{example}\label{ex:lpdual}
  The space $\ell_p$ for $1 \leq p \leq \infty$ is the set of
  sequences of real numbers $\mathbf{x}=(x_1, x_2, ...)$ such that
  $\sum_{i=1}^\infty |x_i|^p < \infty$. (When $p = \infty$, $\ell_\infty = \{ 
  (x_1, x_2, ...) : \max_{i \in \mathbb{N}} |x_i| < \infty \}$). Such
  spaces appear in economics in discrete time, infinite horizon
  optimization problems. 

  Let's consider the dual space of $\ell_\infty$. In macro models, we
  rule out everlasting bubbles and ponzi schemes by requiring
  consumption divided by productivity to be in $\ell_\infty$. Every
  sequence, $\mathbf{p} = (p_1, p_2, ...) \in \ell_1$ gives rise to a linear
  functional on $\ell_\infty$ defined by
  \begin{align*}
    \mathbf{p}^\ast \mathbf{x} = \sum_{i=1}^\infty p_i x_i \leq
    \left(\sum_{i=1}^\infty |p_i| \right) \left(\max_{i \in
        \mathbb{N}} |x_i| < \infty\right). 
  \end{align*}
  We can conclude that $\ell_1 \subseteq \ell_\infty^\ast$. 

  As a (difficult) exercise, you could try to show whether or not
  $\ell_1 = \ell_\infty^\ast$. Exercise 3.46 of Carter is very
  related. 
\end{example}
It is always however always the case that $V \subseteq (V^*)^*$. 

\begin{example}
  What is the dual space of $V = \mathcal{L}^2(\R,f_x) = \{g:
  \R \to \R \text{ such that } \int_{\R} f_x(x) g(x)^2 dx <
  \infty\}$? Let $h \in \mathcal{L}^2(\R,f_x)$. Define 
  \[ h^\ast(g) = \int_\R f_x(x) g(x) h(x) dx. \]
  Assuming $h^\ast(g)$ exists, $h^\ast$ is an integral operator from
  $V$ to $\R$, so it is linear. To show that $h^\ast \in V^\ast$ all
  we need to do is establish that $h^\ast(g)$ exists (is finite) for
  all $g \in V$.  H\"{o}lder's inequality\footnote{See e.g.
    \href{http://en.wikipedia.org/wiki/H\%C3\%B6lder\%27s_inequality}
    {Wikipedia for a proof and more information.}} , which we have not
  studied but is good to be aware of, says that
  \[ \int_\R f_x(x) |g(x) h(x)| dx \leq \sqrt{\int f_x(x) g(x)^2 dx}
  \sqrt{\int f_x(x) h(x)^2 dx}. \]
  Since $h$ and $g \in V$, the right hand side must be finite, so
  $h^\ast(g)$ is finite as well. Thus all such $h^\ast$ is a subset of
  $V^\ast$.   In fact, all such $h^\ast$ is equal to
  $V^\ast$.\footnote{This is a consequence of the Riesz representation
    theorem.}
 
  We were actually working with $V^\ast$ and similar dual spaces when
  we studied optimal control.
\end{example}
\begin{definition}
  If $A: V \to W$ is a linear transformation, then the
  \textbf{transpose} (or adjoint) of $A$ is $A^T: W^\ast \to V^\ast$
  defined by $(A^Tw^\ast)v = w^\ast(Av)$.
\end{definition}
To parse this definition, note that $A^T w^\ast$ is an element of
$V^\ast$, so it is a linear transformation from $V$ to $\R$. Thus,
$(A^T w^\ast) v \in \R$. Similarly, $Av \in W$, and $w^\ast: W \to
\R$, so $w^\ast (A v) \in \R$. 
\begin{example}
  Let $A \in L(\R^n, \R^m)$, so that $A$ can be represented by an $m
  \times n$ matrix, $\gmatrix{a}$. Also, $A^T \in L(\R^m, \R^n)$ can
  be represented by an $n \times m$ matrix, $\gmatrix{\tilde{a}}$. Let
  $v = e_k$ be the $k$th standard basis vector, and $w^\ast =
  e_j^\ast$. Then the definition of the transpose says that
  \begin{align*}
    (A^T e_j^\ast) e_k = & e_j^\ast A e_k  \\
    \left(\begin{pmatrix} \tilde{a}_{11} & \cdots & \tilde{a}_{1m} \\
        \vdots & & \vdots \\
        \tilde{a}_{n1} & \cdots & \tilde{a}_{nm} \end{pmatrix}  e_j\right)^T e_k = & e_j^T \gmatrix{a}
    e_k \\
    \begin{pmatrix} \tilde{a}_{1j} & \cdots &  \tilde{a}_{nj}
    \end{pmatrix} e_k  = & e_j^T \begin{pmatrix} a_{1k} \\ \vdots \\  a_{mk}
    \end{pmatrix} \\
    \tilde{a}_{kj} = & a_{jk}
  \end{align*}
  In other words, the definition is to simply swap rows and columns.  
\end{example}
\begin{exercise}(Difficult) What is the transpose of the conditional
  expectation operator from example \ref{ex:condex}?
\end{exercise}
Since the dual space of $V$ is defined at $V^* = B(V,\R)$, a norm on
$V^*$ can be defined in the same way, 
\[ \norm{v^*}_{V^*} = \sup_{v \neq 0, v \in V}
\frac{\abs{v^*v}}{\norm{v}_V}. \]

%%%%%%%%%%%%%%%%%%%%%%%%%%%%%%%%%%%%%%%%%%%%%%%%%%%%%%%%%%%%%%%%%%%%%%%%%%%%%%%%
\section{Separating hyperplane theorem}

A line in $\R^2$ splits $\R^2$ into two pieces. A plane in $\R^3$
splits it into two pieces. More generally, an $n-1$ dimensional
affine space splits $\R^n$ into two pieces.
\begin{definition}
  A \textbf{hyperplane} in $\R^n$ is an $n-1$ dimensional affine
  subspace. Equivalently, a hyperplane is the set of solutions to a
  single equation with $n$ variables.
\end{definition}
Any hyperplane can be written in the form:
\[ H_{\xi,c} = \{x: \xi^Tx = c \} \]
where $c \in \R$ and $\xi \in (\R^n)^*=\R^n$. Based on this, we define
a hyperplane in an arbitrary vector space as follows:
\begin{definition}
  A \textbf{hyperplane} in $V$ is any set that can be written as
  \[ H_{\xi,c} = \{x \in V: \xi x = c\} \]
  for some $c \in \R$ and $\xi \in V^*, \xi \neq 0$
\end{definition}
Hyperplanes play an important role in optimization.  There is one
theorem that is especially useful.  We will use this theorem to prove
the existence of Lagrange multipliers and the second welfare
theorem. First, a definition.
\begin{definition}
  A set $S \subseteq V$ is \textbf{convex} if $\forall x_1, x_2 \in
  S$ and $\lambda \in (0,1)$, we have $x_1 \lambda + x_2(1-\lambda)
  \in S$.
\end{definition}
If a set is convex, when we draw a line segment between any two points
in the set, the line segment remains entirely within the set.  In
$\R^2$, convex sets include things shaped like triangles, squares,
pentagons, circles, ellipses, etc. Some non-convex shapes are stars,
horseshoes, rings, etc.
\begin{theorem}[Separating hyperplane theorem] \label{thm:sht} If
  $S_1$ and $S_2 \subseteq V$ are convex and
  $S_1 \cap S_2 = \emptyset$ and either $V$ is finite dimensional or
  the interior of $S_1$ or $S_2$ is not empty. Then there exists a
  hyperplane, $H_{\xi c} = \{ x: \xi'x = c \}$ such that
  \[ \xi s_1 \leq c \leq \xi s_2 \]
  for all $s_1 \in S_1$ and $s_2 \in S_2$. We say that $H_{\xi,c}$
  separates $S_1$ and $S_2$. 
\end{theorem}
Visually, this theorem says that we can draw a hyperplane, $H$,
between $S_1$ and $S_2$. $H$ is orthogonal to the line passing through
$\xi$ and $0$. The projection of $S_1$ on $\xi$ is disjoint from
the projection of $S_2$ on $\xi$. See figure \ref{fig:sht} for an
illustration in $\R^2$.

% Recall that the projection of $S_1$ on $H$ is the set
% \[ P_H S_1 = \{ \iprod{s_1}{\xi}\xi : s_1 \in S_1. \] The projections
% are disjoint or almost disjoint)\footnote{Really this argument only
%   shows disjointness if $S_1$ and $S_2$ are topologically
%   closed, something that we will discuss later.} because if
% $\iprod{s_1}{\xi} < \iprod{s_2}{\xi}$ $\forall s_1 \in S_1, s_2 \in
% S_2$, we can never have $\iprod{s_1}{\xi}\xi = \iprod{s_2}{\xi}\xi $.
\begin{figure}\caption{Separating hyperplane \label{fig:sht}}
  \begin{centering}
    \includegraphics[width=0.8\textwidth]{separatingHyperplane}
  \end{centering}
\end{figure}

Exercises 3.182-3.186 of \cite{carter2001} guide you through a proof
of the separating hyperplane theorem in $\R^n$. A proof of the general
version can be found in appendix \ref{app:hyperplane}, or any text on
functional analysis such as \cite{luenberger1969}, \cite{clarke2013},
or \cite{holmes1975}. 


%%%%%%%%%%%%%%%%%%%%%%%%%%%%%%%%%%%%%%%%%%%%%%%%%%%%%%%%%%%%%%%%%%%%%%%%%%%%%%%%
\subsection{Existence of Lagrange multipliers}

In studying constrained maximization problems, we showed that if $x^*$
is a local maximizer for 
\[ \max f(x) \text{ s.t. } h(x) = c \]
then it must be that for all $v$ such that $Dh_{x^*} v = 0$ we also have
$Df_{x^*} v = 0$. We then made some heuristic arguments that this is
equivalent to $Df_x = \mu^T Dh_x$ for some Lagrange multipliers
$\mu$. The separating hyperplane theorem let's us prove this
fact. Notice that $Df_{x^*}$ is a $1 \times n$ matrix, i.e. a linear
transformation from $\R^n \to \R$, and $Dh_{x^*}$ is an $m \times n$
matrix, i.e. a linear transformation from $R^n \to \R^m$. 
\begin{theorem}\label{thm:multipliers}
  Let $V$ and $W$ be normed vector spaces, $A \in B(V,\R)$ and
  $C \in B(V,W)$. Assume that $\range C$ is closed. Then
  $\nulls A \supseteq \nulls C$ if and only if $A = \mu C$ for some
  $\mu \in W^*$.
\end{theorem}
\begin{proof}
  Assume $\nulls A \supseteq \nulls C$. Consider $D:V \to \R \times W$
  defined by $Dv = (Av, -Cv)$. $D$ is linear and bounded since $A$ and
  $C$ are.  Therefore $\range D$ is a subspace of $\R \times W$. Also
  $(1,0_W) \not\in \range D$ because $Cv = 0_w$ only if
  $v \in \nulls C$, but if $v \in \nulls C$, then $v \in \nulls A$ by
  assumption, and $Av = 0 \neq 1$. Since $\range C$ is closed and
  $\range A$ is finite dimensional and therefore closed, $\range D$ is
  closed. Therefore, there exists an open neighborhood of $(1,0_W)$
  that does not intersect $\range D$. Call it $N$. This neighborhood
  is convex, and so is $\range D$. Therefore, by the separating
  hyperplane theorem, there exists $\xi \in (\R \times W)^*$ and
  $c \in \R$ such that
  \[ \xi x \leq c < \xi y \]
  for all $x \in \range D$ and $y \in N$.  It must be that $\xi x = 0$
  for all $x \in \range D$. If not, say $\xi x = d \neq >0$, then
  since $\range D$ is a subspace, $2c/d x \in \range D$, and since
  $\xi$ is linear, $\xi(2c/d x) = 2c > c$.  Let
  $\xi = (\xi_1, -\tilde{\mu})$, where $\xi_1 \in \R^* = \R$ and
  $\mu \in W^*$. Since $(1,0_w) \in N$, it must be that $\xi_1 >
  0$. Therefore, for all $v \in V$,
  \begin{align*}
    0 = & \xi Dv \\
    0 = & \xi_1 Av - \tilde{\mu} Cv \\
    \mu C v = & Av 
  \end{align*}
  i.e. $\mu C = A$.

  Conversely, suppose $\mu C = A$. Let $v \in \nulls C$. Then, 
  $ 0 = C v $, so $A v = \mu C v = \mu 0 = 0$. Therefore $v \in \nulls A$.
\end{proof}
If we take $A = Df_x$ and $C = Dh_x$, then this theorem shows that
existence of Lagrange multipliers such that $Df_x = \mu^T Dh_x$ is
equivalent to $Dh_{x^*} v = 0$ implying that $Df_{x^*} v = 0$. 

%%%%%%%%%%%%%%%%%%%%%%%%%%%%%%%%%%%%%%%%%%%%%%%%%%%%%%%%%%%%%%%%%%%%%%%%%%%%%%%%
\section{Welfare theorems}
A second major use of the separating hyperplane theorem (and vector
spaces more generally) is in the proof of the first and second welfare
theorems. The first welfare theorem says that every competitive
equilibrium is Pareto efficient. The second welfare theorem says that every
Pareto efficient allocation can be achieved by some competitive
equilibrium. 

We have some set of commodities, $S$, which we will assume is a normed
vector space. For example, in a world with $n$ goods, $S$ could be
$\R^n$ and for each $s = (s_1,...,s_n) \in S$, $s_j$ represents the
quantity of the $j$th good. These goods include everything that is
bought or sold, including things like food or clothing that we usually
think of as goods, and things like labor and land. There are $I$
consumers, indexed by $i$. Each consumer chooses goods from a feasible
set $X_i \subseteq S$. These $X_i$ are feasible consumption sets, not
budget sets. It is supposed to represent the physical constraints of
the world. For example if there are three goods: food, clothing, and labor
measured in days of labor per day, then $X_i$ might be $[0, \infty)
\times [0,\infty) \times [0, 1]$.  Each consumer has preferences over
$X_i$ represented by a preference relation, $\prefeq_i$ that as in the
previous lecture the following properties:
\begin{enumerate}
\item (complete) $\forall x, z \in X_i$, either $x \prefeq_i z$ or $z
  \prefeq_i x$ or both,
\item (transitive) $\forall x, w, z \in X_i$, if $x \prefeq_i w$ and $w
  \prefeq_i z$ then $x \prefeq_i z$,
\item (reflexive) $\forall x \in X_i$, $x \prefeq_i x$.
\end{enumerate}
In words, $x \prefeq_i z$ means person $i$ likes the bundle of goods $x$
as much as or more than the bundle of goods $z$. If you wish, you can
think of the preference relation coming from a utility function,
$u_i(x) : X_i \rightarrow \R$ and $x \prefeq_i z$ means $u_i(x) \geq
u_i(z)$.  If $x \prefeq_i z$ but $z \not\prefeq_i x$, then we say that
$x$ is strictly preferred to $z$ and write $x \pref_i z$. If $x
\prefeq_i z$ and $z \prefeq_i x$ we say that person $i$ is indifferent
between $x$ and $z$ and write $x \simeq_i z$.

There are also $J$ firms indexed by $j$. Each firm $j$ chooses
production $y_j$ from production possibility set $Y_j \subseteq
S$. The firm will produce positive quantities of its outputs and
negative quantities of its inputs. Continuing with the example of
three goods, if the firm produces $F^f(l)$ units of food from $l$
units of labor and $F^c(l)$ units of clothing, then production
possibility set could be written: 
\[ Y_j = \{ (f,c,l) \in S: l \leq 0 \wedge f \leq F^f(\alpha |l|)
\wedge c \leq F^c((1-\alpha) |l|) \text{ for some } \alpha \in
[0,1]\}. \] 
Firms produce goods and consumers consume goods. For the market to
\textbf{clear} we must have sum of production equal to the sum of
consumption, 
i.e. 
\begin{align*}
  \sum_{i=1}^I x_i = \sum_{j=1}^J y_j \label{eq:mc}
\end{align*}
We call the $I+J$-tuple of all $x_i$ and $y_j$, $\left( (x_1,..., x_I)
  , (y_1, ..., y_J) \right)$ (which we will sometimes shorted by just
writing $((x_i),(y_j))$) an \textbf{allocation}. An allocation is
\textbf{feasible} if $x_i \in X_i \forall i$, $y_j \in Y_j \forall j$,
and $ \sum_{i=1}^I x_i = \sum_{j=1}^J y_j$.
\begin{definition}
  An allocation, $((x_i^0),(y_j^0))$, is \textbf{Pareto efficient} (or
  Pareto optimal) if it is a feasible and there is no other feasible
  allocation, $((x_i),(y_j))$, such that $x_i \prefeq_i x_i^0$ for all
  $i$ and $x_i \pref_i x_i^0$ for some $i$.
\end{definition}
This definition is just a mathematical way of stating the usual verbal
definition of Pareto efficient. An allocation is Pareto efficient if
there is no other allocation that makes at least one person better off
and no one worse off. 

We are going to be comparing competitive equilibria to Pareto
efficient allocations. To do that we must first define a competitive
equilibrium. A price system is a continuous linear transformation,
$p:S \rightarrow \R$, i.e. $p \in S^\ast$. In the case where $S =
\R^n$, a price system is just a $1 \times n$ matrix. The entries in
this price matrix are the prices of each of the $n$ goods. $px$ for $x
\in S$ represents the total expenditure needed to purchase the bundle
of goods $x$.
\begin{definition}
  An allocation, $((x_i^0),(y_j^0))$, along with a price system, $p$,
  is a \textbf{competitive equilibrium} if 
  \renewcommand{\theenumi}{C\arabic{enumi}}
  \begin{enumerate}
  \item\label{c1} The allocation is feasible
  \item\label{c2} For each $i$ and $x \in X_i$ if $px \leq px_i^0$
    then $x_i^0 \prefeq_i x$,
  \item\label{c3} For each $j$ if $y \in Y_j$ then $p y \leq p y_j^0$
  \end{enumerate}
  \renewcommand{\theenumi}{\roman{enumi}}
\end{definition}
Condition \ref{c2} says that each consumer must be choosing the most
preferred bundle of goods that he or she can afford. If the preference
relation comes from a utility function, \ref{c2} says that consumers
maximize their utility given prices. Similarly, condition \ref{c3}
says that producers maximize profits.

The first welfare theorem requires one additional condition on
preferences.
\begin{definition}
  Preference relation $\pref_i$ has the \textbf{local non-satiation
    condition} if for each $x \in X_i$ and $\epsilon > 0$ $\exists x'
  \in X_i$ such that $\norm{x - x'} \leq \epsilon $ and $x' \pref_i
  x$.
\end{definition}
This condition says that given any bundle of goods you can find
another bundle very close by that is preferred. If the preference
relation comes from utility function, the utility function having a
non-zero derivative everywhere implies local non-satiation. The
intuition for why the first welfare theorem requires local
non-satiation is that local non-satiation rules out the following
scenario. Suppose person $i$ does not care about clothing at all. Then 
you take clothes away from person $i$, making person $i$ no worse off,
and give them to someone else, making that person better off. However,
there is nothing in the definition of a competitive equilibrium that
prevents person $i$ from having clothes.

\subsection{First welfare theorem}
\begin{theorem}[First welfare theorem]
  If $((x_i^0),(y^0_j))$ and $p$ is a competitive equilibrium and all
  consumers' preferences have the local non-satiation condition, then
  $((x^0_i),(y^0_i))$ is Pareto efficient.
\end{theorem}
\begin{proof}
  We will prove it by contradiction. Suppose that a competitive
  equilibrium is not Pareto efficient. Then there exists another
  feasible allocation\footnote{This sort of allocation is called a
    Pareto improvement.}, $((x_i),(y_j))$, such that there is at least
  one $x_{i^*} \pref_{i^*} x_{i^*}^0$. The contrapositive of condition
  \ref{c2} in the definition of competitive equilibrium implies that
  then $p x_{i^*} > p x_{i^*}^0$.  For all other $i \neq i^*$ it must
  be that $x_i \prefeq_i x_i^0$. When $x_i \pref_i x_i^0$, by the same
  argument as above, $p x_i > p x_i^0$. When $x_i \simeq_i x_i^0$,
  then we will show that local non-satiation implies $px_i \geq
  px_i^0$. If not and $px_i < p x_i^0$, then by continuity\footnote{We
    have not yet defined continuity, so do not worry if you find this
    part of the proof confusing. A function $f: V\rightarrow W$ where
    $V$ and $W$ are normed vector spaces is continuous if $\forall
    \epsilon > 0$ $\exists \delta > 0$ such that whenever $\norm{x -
      y}<\delta$ we also have $\norm{f(x) - f(y)} < \epsilon$. In this
    proof, $V$ is $S$ , $W$ is $\R$, $f$ is $p$, and $\epsilon$ is
    $|px_i - px_i^0|$. } of $p$ there exists some $\delta > 0$ such
  that for all $x'$ with $\norm{x_i - x'} < \delta$, we have
  \[ | p x_i - p x' | < | px_i - p x_i^0| \]
  and in particular, 
  \[ p x' < px_i^0. \]
  Additionally since preferences are locally non-satiated, there
  exists some $\tilde{x}$ with $\norm{x_i - \tilde{x}}<\delta$ and
  $\tilde{x} \pref_i x_i \simeq_i x_i^0$. However, then we also have
  $\tilde{x} \pref_i x_i^0$ and $p \tilde{x} < p x_i^0$, which contradicts
  $x_i^0$  and $p$ being part of a competitive equilibrium. Thus, we
  can conclude that $p x_i \geq p x_i^0$.

  At this point we have shown that if $((x_i^0),(y_j^0))$ is a
  competitive equilibrium that is not Pareto efficient, then there is
  some other allocation $((x_i),(y_j))$ that is feasible and has $x_i
  \prefeq_i x_i^0$, which implies that $p x_i \geq p x_i^0$. Each
  consumer spends (weakly) more in this alternative, Pareto improving
  allocation. Now we will show that each consumer spending at least as
  much contradicts profit maximization. The total expenditure of
  consumers in the alternate allocation must be greater than in the
  competitive equilibrium because there is one consumer who is
  spending strictly more. That is,
  \begin{align}
    \sum_{i=1}^I p x_i > \sum_{i=1}^I p x_i^0 \label{ieq:ex}
  \end{align}
  The price system is a linear transformation, so
  \[ \sum_{i=1}^I p x_i = p \left(\sum_{i=1}^I x_I \right) \]
  Both allocations are feasible, and, in particular, market clearing
  so
  \begin{align*}
    \sum_{i=1}^I x_i & = \sum_{j=1}^J y_j \\
  \end{align*}
  Applying $p$ to both sides, 
  \begin{align*}
    p\left( \sum_{i=1}^I x_i\right) & = p\left( \sum_{j=1}^J y_j
    \right) \\ 
    = & \sum_{j=1}^J p y_j.
  \end{align*}
  Identical reasoning would show that 
  \[ \sum_{i=1}^I p x_i^0 = \sum_{j=1}^J p y_j^0. \]
  Substituting into (\ref{ieq:ex}) we get
  \begin{align}
    \sum_{j=1}^J p y_j > \sum_{j=1}^J p y_j^0. \label{ieq:prof}
  \end{align}
  But this contradicts profit maximization (\ref{c3}) since $y_j \in
  Y_j$ and we cannot have (\ref{ieq:prof}) if $p y_j \leq p y_j^0$. 
  Therefore, we conclude that there can be no Pareto improvement from
  a competitive equilibrium, i.e.\ any competitive equilibrium is
  Pareto efficient.
\end{proof}

\subsection{Second welfare theorem}
The second welfare theorem is the converse of the first welfare
theorem. The second welfare theorem says that any Pareto efficient
allocation can be achieved by some competitive equilibrium. The second
welfare theorem does not hold quite as generally as the first welfare
theorem. 
\begin{definition}
  A preference relation, $\prefeq_i$, is \textbf{convex} if whenever $x
  \prefeq_i z$ and $y \prefeq_i z$, then $\lambda x + (1-\lambda) y
  \prefeq_i z$ for all $\lambda \in [0,1]$. 
\end{definition}
Alternatively, a preference relation is convex if the set $\{x\in X_i: x
\prefeq_i z\}$ is convex for each $z$. Whenever you have seen convex
indifference curves, the associated preference relation is convex. If
the preference relation is generated by a concave (more generally
quasi-concave) utility function, then the preference relation is
convex.
\begin{definition}
  A preference relation, $\prefeq_i$, is \textbf{continuous} if
  for any  $x \pref_i z$ there exists a $\delta >
  0$ such that for all $x'$ with $\norm{x - x'}<\delta$ we have $x'
  \pref_i z$.
\end{definition}
A continuous preference relation can be generated by a continuous
utility function.

\begin{theorem}[Second welfare theorem]
  Assume the preferences of each consumer are convex, locally
  non-satiated, and continuous, and that $X_i$ is convex and
  non-empty.  
  Also assume that $Y_j$ is convex and non-empty for each
  firm $j$. 

  Suppose $((x_i^e), (y_j^e))$ is a Pareto efficient allocation such
  that for any price system, $p$, there is always a cheaper bundle of
  goods, i.e.\ $\exists x_i \in X_i$ s.t. $p x_i < p x_i^e$ for each
  $i$. Then there exists a price system, $p^e$ such that $((x_i^e),
  (y_j^e))$ and $p^e$ is a competitive equilibrium.
\end{theorem}
\begin{proof}
  We are going to construct the price system by applying the
  separating hyperplane theorem. Let $V_i = \{ x \in X_i : x \pref_i
  x_i^e \}$ be the set of $x$ strictly preferred by person $i$. Let 
  \begin{align*}
    V = \{ \chi \in S: \chi = \sum_{i=1}^I x_i \text{ for some } x_i
    \in V_i \}  
  \end{align*}
  be the set of sums of elements from each $V_i$. The convexity of
  $X_i$ and the preference relation implies that $V_i$ is convex for
  each $i$. That, in turn, implies that $V$ is convex.\footnote{It
    might be a good exercise to prove these claims. } 
  Similarly, if 
  \begin{align*}
    Y = \{ \psi \in S: \psi = \sum_{i=j}^J y_j \text{
      for some } y_j \in Y_j \}  
  \end{align*}
  is the sum of each firms' production possibility set, then $Y$ is
  convex. 

  We have two convex sets. Now, we just need to show that they are
  disjoint, and then we can apply the separating hyperplane
  theorem. Suppose $\chi \in Y \cap V$. Then $\exists x_i \in V_i$ and
  $y_j \in Y_j$ such that $\chi = \sum_{i=1}^I x_i = \sum_{j=1}^J$. This
  is feasible allocation, and $x_i \pref_i x_i^e$ by
  construction. This contradicts $((x_i^e),(y_i^e))$ being Pareto
  efficient. Therefore, $Y \cap V = \emptyset$. o

  Now, by the separating hyperplane theorem, $\exists p$ such
  that\footnote{In the notation of theorem \ref{thm:sht}, $p$ is $\xi$.}
  \begin{align}
    p \chi \geq p \psi \label{ieq:p}
  \end{align}
  for all $\chi \in V$ and $\psi \in Y$. Now we need to verify
  that $((x_i^e),(y_j^e))$ with $p$ is a competitive equilibrium. It
  is feasible because $((x_i^e),(y_j^e))$ is Pareto efficient, and
  feasible by definition.
  
  We now show that (\ref{ieq:p}) holds with equality at $\chi^e =
  \sum_{i=1}^I x_i^e$ and $\psi^e = \sum_{j=1}^J y_j^e$. On the
  one hand, $\chi^e = \psi^e \in Y$, so we must have 
  \begin{align*}
    p \chi \geq p \chi^e 
  \end{align*}
  for all $\chi \in V$. On the other hand, for any$\delta > 0$, by
  local non-satiation, we can find $x_i$ such that $x_i \pref_i x_i^e$
  and $\norm{x_i - x_i^e}<\delta/I$. It follows that
  $\norm{\sum_{i=1}^I x_i - \sum_{i=1}^T x_i^e } < \delta$. $p$ is
  continuous, so
  \[
  \left\vert p\left( \sum_{i=1}^e x_i \right) - p\left(\sum_{i=1}^I
      x_i\right) \right\vert < \epsilon,
  \]
  and we can choose $\epsilon>0$ to be as small as we want. 
  Then, for any $\epsilon >0$, 
  \begin{align*}
    p \chi^e + \epsilon \geq p \psi
  \end{align*}
  for all $y \in Y$. Since this is true for any $\epsilon$, it must be
  that $p \chi^e \geq p \psi$. Therefore, we have now shown
  that
  \begin{align}
    p\chi \geq p \chi^e = p \psi^e \geq p \psi
  \end{align}  
  for all $\chi \in V$ and $\psi \in Y$. 

  It must be then also be that $p x_i \geq p x_i^e $ for each $i$ and
  all $x_i \in V_i$. If not, then there is an $\epsilon>0$ such that
  $p x_i + \epsilon < p x_i^e$, and then using local non-satiation we
  can choose $x_k$ for $k\neq i$ such that $x_k \in V_k$ and 
  \[\left| \sum_{
      k \neq i } px_k - \sum_{k \neq i } p x_k^e \right| < \epsilon/2
  \]
  and then 
  \[ \sum_{k=1}^I p x_k + \epsilon/2 < \sum_{k=1}^I p x_k^e. \]
  Similarly, we must have $p y_j^e \geq p y_j$ for all $y_j \in Y_j$,
  which proves that profit maximization, (\ref{c3}), holds. 

  We have nearly shown that utility maximization, (\ref{c2}), also
  holds. We have shown that for each $i$ if $x_i \pref_i x_i^e$ then
  $p x_i \geq p x_i^e$. To strengthen it to the form in the
  definition, we need to show that $px_i > p x_i^e$. We will use the
  continuity of preferences and the cheaper good condition. Suppose $p
  x_i = p x_i^e$ and $\exists x'_i \in X_i$ such that $p x_i' < p
  x_i^e$. Then for any $\lambda \in (0,1)$, $p(\lambda x_i'
  +(1-\lambda) x_i') < p x_i^e$. Also, by the continuity of
  preferences, for $\lambda$ close enough to $0$, $\lambda x_i +
  (1-\lambda)x_i' \pref_i x_i^e$. However, then $\lambda x_i +
  (1-\lambda) x_i' \in V_i$ contradicting $p(\lambda x_i
  +(1-\lambda) x_i') < p x_i^e$. Therefore, if the cheaper good
  exists, we must have $p x_i < p x_i^e$.   
\end{proof}

\clearpage
\appendix
\section{Proof of the separating hyperplane theorem \label{app:hyperplane}}

This section is based on \cite{holmes1975}.  The following lemma is a
first step to proving the theorem.
\begin{lemma}[Stone]\label{lem:stone}
  Let $V$ be a vector space and $A$ and $B$ be disjoint convex subsets
  of $V$, then there exist convex sets $C$ and $D$ such that $A
  \subseteq C$, $B \subseteq D$, $C \cap D = \emptyset$ and $C \cup D
  = V$.
\end{lemma}
\begin{proof}
  We employ a similar technique as used to show the existence of a
  basis. Let $\mathcal{C}$ be the collection of all convex sets
  containing $A$ and disjoint from $B$. $\mathcal{C}$ is non-empty
  because $A \in \mathcal{C}$.  As in the exercise showing existence
  of a basis, $(\mathcal{C}, \subseteq)$ is a partially ordered
  set. Also, for any chain $\mathcal{H} \subseteq \mathcal{C}$,
  $\cup_{E \in \mathcal{H}} E$ is an upper bound. Therefore, by Zorn's
  lemma, $\mathcal{C}$ has a maximal element. Let $C$ be a maximal
  element. Define $D = V \setminus C$.  By construction
  $C \cap D = \emptyset$, $B \subseteq D$, and $C \cup D = V$.

  To complete the proof, we just need to show that $D$ is
  convex. Suppose $D$ is not convex, then there exists $x,z \in D$,
  $\lambda \in (0,1)$ such that $y = \lambda x + (1-\lambda) z \not\in
  D$. Then $y \in C$. Furthermore, observe that for any $d \in D$, there
  must be a $c \in C$ and $\lambda \in [0,1]$ such that $b = \lambda d
  + (1-\lambda) c \in B$. If there were not, then the set 
  \[ \tilde{C} = \{\lambda d + (1-\lambda) c:  \lambda \in [0,1], c
  \in C\} \]
  would be a convex set containing $A$ and disjoint from $B$,
  contradicting the fact that $C$ is a maximal such set. 
  Then can find $p,q \in C$ and $u,v \in B$ such $u$ is a convex
  combination of $p$ and $x$, and $v$ is a convex combination of $q$
  and $z$, i.e. there are $\lambda_u, \lambda_v \in (0,1)$ such that:
  \begin{align*}
    u = & \lambda_u x + (1-\lambda_u) p \\
    v = & \lambda_v z + (1-\lambda_v) q 
  \end{align*}
  However, then there would be some convex combination of $u$ and $v$
  that can also be written as a convex combination of $p$, $q$, and
  $y$. Since $B$ is convex and $u,v\in B$, this convex combination
  would also be in $B$. Since $C$ is convex and $p,q,y \in C$, the
  convex combination would also be in $C$. In other words $B \cap C$
  would not be empty, but this contradicts the way $C$ is
  defined. Therefore, it must be that $D$ is convex. 
\end{proof}
For any hyperplane, $H_{\xi,c}$, the sets $\{v \in V: \xi v \geq c\}$
and $\{v \in V: \xi v < c\}$ are disjoint and convex. The next step is
to show that disjoint convex sets from lemma \ref{lem:stone} take this
form. First, we need some definitions of the boundary and interior of
subsets of a vector space (without a norm). For convex sets in finite
dimensional normed vector spaces, these definitions are the same as
the definitions using open and closed sets. In general, the
definitions differ, but they are capturing similar ideas. 
\begin{definition}
  Let $V$ be a vector space and $A \subseteq V$. The \textbf{algebraic
    interior} of $A$ is set of all $a \in A$ such that for every $v
  \in V$ there exists $\bar{\lambda} \in (0,1)$ such that 
  \[ a(1-\lambda) + \lambda v \in A \]
  for all $\lambda \in [0,\bar{\lambda}]$. Denote the algebraic
  interior of $A$ as $\mathrm{int}(A)$.
\end{definition}
At interior points, we can move slighty from $a$ toward any other
point $v$ and remain inside $A$. This algebraic interior is different
than the topological interior we defined using open sets earlier. For
example, in $\ell^\infty$ let $A = \{(x_1, x_2, ...) : \abs{x_n} <
1/n \}$. $A$ is equal to its algebraic interior, but the topological
interior of $A$ is empty. 
\begin{definition}
  Let $V$ be a vector space and $A \subseteq V$. The \textbf{algebraic
    closure}\footnote{This is non-standard terminology,
    \cite{holmes1975} calls this lin(A). Holmes's notation for my
    $\mathrm{int}(A)$ is $\mathrm{cor}(A)$.} of $A$ is set
  of all $v \in V$ such that $\exists a \in A$ such that
  $\forall \lambda \in [0,1)$,
  \[ a(1-\lambda) + \lambda v \in A \]
  Denote the algebraic closure of $A$ as $\overline{A}$.
\end{definition}
Finally, we need to define affine sets.
\begin{definition}
  $A \subseteq V$ is \textbf{affine} if $\forall x, y \in A$ and $\lambda \in
  \R$, $\lambda x + (1-\lambda)y \in A$.
\end{definition}
The difference between affine and convex is that affine sets allow
$\lambda <0$ and $\lambda >1$. Affine sets contain the line passing
through any two vectors in the set. Convex sets only contain the line
segment between any two points. Affine sets are like linear subspaces
in that they are lines, planes, etc, except that affine sets need not
contain $0$.  We also need the following lemma about hyperplanes.
\begin{lemma}
  An affine set $A$ is a hyperplane if and only if it is a maximal
  proper affine set with respect to inclusion.
\end{lemma}
\begin{proof}
  Left as an exercise.
\end{proof}
\begin{lemma}\label{lem:compconvex}
  Let $C$ and $D$ be non-empty convex sets in a vector space
  $V$ with $C \cap D = \emptyset$ and $C \cup D = V$. Let
  $M = \overline{C} \cap \overline{D}$, then either $M = V$ or $M$ is
  a hyperplane in $V$.
\end{lemma}
\begin{proof}
  Since $C$ and $D$ are convex, so are $\overline{C}$ and
  $\overline{D}$. For any convex sets, their intersection is also
  convex, so $M$ is convex. Moreover, $M$ is affine. To see this let
  $x,y \in M$ and $z = \lambda x + (1-\lambda) y $ for $\lambda \in
  \R$. If $\lambda \in (0,1)$, we known $M$ is convex, so $z \in
  M$. If $z \not\in M$, then $z \in \mathrm{int}(C) \cup
  \mathrm{int}(D)$. To be concrete, suppose $z \in \mathrm{int}(C)$ and $\lambda<1$
  (the other cases are dealt with similarly). Then $y = z/(1-\lambda)
  - \lambda/(1-\lambda) x$ is a convex combination of $x$ and
  $z$. Notice that $x \in \overline{C}$ and $z \in \mathrm{int}(C)$ implies
  $y \in \mathrm{int}(C)$, but this contradicts the assumption that $y
  \in M$. Therefore, it must be that $z \in M$. 

  Suppose $M \neq V$. Pick any $m \in M$. Then
  $\exists (m+p) \in V \setminus M = \mathrm{int}(C) \cup
  \mathrm{int}(D)$.
  Without loss of generality, assume $(m+p) \in \mathrm{int}(C)$.  If
  $(m-p) \in \mathrm{int}(C)$, then
  $0.5(m+p) + 0.5(m-p) = m \in \mathrm{int}(C)$ by convexity, but we
  know $m \not\in \mathrm{int}(C)$, so it must be that
  $(m-p) \in \mathrm{int}(D)$. To conclude, we show that the span of
  $\{p\} \cup M = V$, so $M$ is a maximal affine set, i.e. a
  hyperplane. Let $x \in C$. By definition of $\overline{C}$ and
  $\overline{D}$, there exists $\lambda \in (0,1)$ such that
  $\lambda x + (1-\lambda) (m-p) \in M$, so
  $x \in \spn(M \cup \{p\})$. Similarly if $x \in D$,
  $x \in \spn(M \cup \{p\})$.
\end{proof}
Finally, we can prove the separating hyperplane theorem.
\begin{proof}[Proof of theorem \ref{thm:sht}]
  By Stone's separation lemma \ref{lem:stone}, there exists convex
  sets $C$ and $D$ such that $S_1 \subseteq C$, $S_2 \subseteq D$,
  $C \cap D = \emptyset$, and $C \cup D = V$. By lemma
  \ref{lem:compconvex}, $H = \overline{C} \cap \overline{D}$ is either
  $V$ or the separating hyperplane that we want. $H$ is not a
  hyperplane only if $\overline{C} = \overline{D} = V$. If $V$ is
  finite dimensional, and $C$ and $D$ are not empty, then this is
  impossible. Proving this is left as an exercise. Regardless of the
  dimension of $V$, if $C$ (or $D$) has a non-empty interior, then
  lemma \ref{lem:ubiq} implies that $\overline{C}$ (or $\overline{D}$)
  cannot be all of $V$, since we know that $C \neq V$.
\end{proof}
\begin{lemma}\label{lem:ubiq}
  If $C \subseteq V$ is convex, $\mathrm{int}(C)$ is not empty, and
  $\overline{C} = V$, then $C = V$. 
\end{lemma}
\begin{proof}
  Let $x \in V$, and $y \in \mathrm{int}(C)$.  Without loss of
  generality assume $y = 0$. Since $\overline{C} = V$ so $2x \in \overline{C}$,
  $\exists z \in C$ such that
  \[ (1-\lambda)z + \lambda 2x \in C \]
  for all $\lambda \in [0,1)$. 
  Since $y = 0 \in \mathrm{int}(C)$ there is $\delta>0$ such that 
  $-\delta z \in C$. Since $C$ in convex, for any $t \in [0,1]$ and
  $\lambda \in [0,1)$,
  \[ [(1-\lambda)z + \lambda 2x] t - (1-t) \delta z \in C \]
  Setting $\lambda = \frac{1+\delta}{1+2\delta}$ and $t =
  \frac{1+2\delta}{2(1+\delta)}$, we can conclude that $x \in C$, so
  $C = V$.  
\end{proof}

 %%%%%%%%%%%%%%%%%%%%%%%%%%%%%%%%%%%%%%%%%%%%%%%%%%%%%%%%%%%%%%%%%%%%%%%%%%%%%%%%
\section{Fundamental theorem of linear algebra}
I've covered this in past years, but we will not get to it this year.

\begin{definition}
  If $S_1 + \cdots + S_k = V$ and $\forall \mathbf{v} \in V$, there
  are unique $\mathbf{x}_j \in S_j$ such that 
  \[ \mathbf{v} = \mathbf{x}_1 + \cdots  + \mathbf{x}_k \]
  then $V$ is the \textbf{direct sum} of $S_1, ..., S_k$, written 
  \[ V = S_1 \oplus \cdots \oplus S_k \]
\end{definition}
Representing a vector space as a direct sum will be important for some
of the results below. The following lemma will be useful.
\begin{lemma}\label{lem:dirSum}
  Suppose $S_1$ and $S_2$ are linear subspaces of $V$. Then $V = S_1
  \oplus S_2$ iff $V = S_1 + S_2$ and $S_1 \cap S_2 = \{0\}$. 
\end{lemma}
\begin{proof}
  Suppose $V = S_1 \oplus S_2$. Then by definition $V = S_1 +
  S_2$. Also, if $\mathbf{x} \in S_1 \cap S_2$, then $0 = \mathbf{x} + (-\mathbf{x})$. The
  definition of direct sum requires this representation to be unique,
  so $\mathbf{x} = 0$ must be the only element of $S_1 \cap S_2$.

  Suppose $V = S_1 + S_2$ and $S_1 \cap S_2 = \{0\}$. Let $\mathbf{v} \in
  V$. Since $V = S_1 + S_2$, $\exists \mathbf{x}_1 \in S_1$ and $\mathbf{x}_2 \in S_2$
  such that $\mathbf{v} = \mathbf{x}_1 + \mathbf{x}_2$. Suppose that
  $\mathbf{y}_1 \in S_1$ and $\mathbf{y}_2 \in S_2$ and $\mathbf{v} =
  \mathbf{y}_1 + \mathbf{y}_2$. Subtracting, 
  \[ 0 = \underbrace{(\mathbf{x}_1 - \mathbf{y}_1)}_{\in S_1} + \underbrace{(\mathbf{x}_2 -
    \mathbf{y}_2)}_{\in S_2} \]
  so $(\mathbf{x}_1 - \mathbf{y}_1) = -(\mathbf{x}_2 -
  \mathbf{y}_2)$. By the definition of subspaces, $S_1$ and $S_2$ are
  closed under scalar multiplication. Therefore, $(\mathbf{x}_i -
  \mathbf{y}_i) \in S_1 \cap S_2$ for $i = 1, 2$. By assumption, this
  intersection only contains $0$, so we can conclude that
  $\mathbf{x}_i = \mathbf{y}_i$. The representation of $\mathbf{v} =
  \mathbf{x}_1 + \mathbf{x}_2$ is unique.
\end{proof}

The next three lemmas will be used later when proving theorem
\ref{thm:fund}. You may want to skip ahead and only come back to these
lemmas when they're needed.
\begin{lemma}\label{lem:directSum}
  Suppose $V$ is finite dimensional and $S$ is a subspace of $V$. Then
  $\exists$ another subspace, $W$, of $V$ such that $V = S \oplus W$,
  and $\dim(V) = \dim(S) + \dim(W)$.
\end{lemma}
\begin{proof}
  Construct a basis for $S$ as follows. Set the basis $B = \{\}$. If
  $S = \text{span}(B)$, then stop. Otherwise, choose $\mathbf{b}_j \in
  S \setminus \text{span}(B)$ and add it to $B$. Since $V$ is finite
  dimensional, this process must stop after at most $\dim(V)$
  steps. This gives a basis $B$ for $S$. 

  We will now construct a basis for $W$. Set $E = \{\}$. If
  $\text{span}(B \cup E) = V$, then stop. Otherwise choose $\mathbf{e}_j \in V
  \setminus \text{span}(B \cup E)$ and add it to $E$. Again, this
  process must stop because $V$ is finite dimensional. Let $W =
  \text{span}(E)$. By construction $\text{span}(B \cup E) = S + W =
  V$. Also, lemma \ref{lem:uniqueRep} implies that each $\mathbf{v}
  \in V$ can be uniquely written as a linear combination of elements
  of $B \cup E$, each $\mathbf{s} \in S$ can be uniquely written as a
  linear combination of $B$, and each $\mathbf{w} \in W$ can be
  uniquely written as a linear combination of $E$. It follows that for
  each $\mathbf{v} \in V$ there are unique $\mathbf{s} \in S$ and
  $\mathbf{w} \in W$ such that $\mathbf{v} = \mathbf{s} +
  \mathbf{w}$. 
\end{proof}

\begin{lemma}\label{lem:dimisect}
  Suppose $V$ is finite dimensional and $S$ and $T$ are subspaces of
  $V$. If $S \cap T = 0$ and $\dim(S) + \dim(T) = \dim(V)$, then $V =
  S \oplus T$. 
\end{lemma}
\begin{proof}
  Let $b_1, ..., b_n$ be a basis for $S$ and $e_1, ..., e_m$ be a
  basis for $T$. Suppose 
  \[ \sum_{i=1}^n \alpha_i b_i +  \sum_{j=1}^m \beta_j e_j = 0 \]
  Then let
  \[ \mathbf{x} = \sum_{i=1}^n \alpha_i b_i = -\sum_{j=1}^m \beta_j
  e_j \]
  $\mathbf{x} \in S$, and $\mathbf{x} \in T$. We assume $S \cap T =
  0$, so then $\mathbf{x} = 0$. Since $b_1, ..., b_n$ are linearly
  independent, we must have $\alpha_1 = \cdots = \alpha_n =
  0$. Similarly all $\beta_j = 0$. Therefore, $b_1, ..., b_n$, $e_1,
  ..., e_m$ are linearly independent. This is a linearly independent
  set of $\dim(S) + \dim(T) = \dim(V)$ elements, so it is a basis for
  $V$. Hence, $V = S + T$. 
\end{proof}
\begin{lemma}\label{lem:dimSum}
  Let $S$ and $T$ be subspaces of a finite dimensional space $V$. Then
  \[ \dim(S + T) = \dim(S) + \dim(T) - \dim(S \cap T). \]
\end{lemma}
\begin{proof}
  We will be brief for this proof. You may want to add the details as
  an exercise. 
  
  Let $B$ be basis for $S \cap T$. Extend $U$ to a basis for $S$, call
  it $B \cup B_S$. Similarly extend $U$ to a basis for $T$, $B\cup
  B_T$. Then $\spn(B \cup B_S \cup B_T) = S + T$. Also, you can show
  that $B \cup B_S \cup B_T$ is linearly independent, and hence a
  basis, for $S+T$. Finally, note that since $B$, $B_S$, and $B_T$ must
  be disjoint, 
  \[ \dim(S+T) = | B \cup B_S \cup B_T | = | B \cup B_S | + | B \cup
  B_T | - | B |. \]
\end{proof}

There is an interesting relationship among the null spaces and ranges
of a linear transformation and its transpose. Let $A \in L(V,W)$ and
suppose $V$ and $W$ are finite dimensional. Finite dimension ensures
that $V^\ast = V$ and $W^\ast = W$. Then $\nulls A
\subseteq V$ and $\range A^T \subseteq V$. How are these
subspaces related?  
\begin{theorem}[Fundamental theorem of linear algebra]\label{thm:fund}
  Let $A \in L(V,W)$, where $V$ and $W$ are finite dimensional. Then 
  \[ V = \nulls A \oplus \range A^T \]
  and 
  \[ W = \nulls A^T \oplus \range A. \] 
  Also, $\dim(\range A) = \dim(\range  A^T)$ and,
  \[ \dim(V) = \dim(\nulls A) + \dim(\range A) \]
  and 
  \[ \dim(W) = \dim(\nulls A^T) + \dim(\range A). \]
\end{theorem}
\begin{proof}
  Suppose $x$ is in the null space of $A$. Then $A x
  = 0$. By the definition of the transpose,
  \[ w^T A x = (A^T w)^T x = 0 \]
  The set $\{v : v = A^T w\}$ is the range of $A^T$. If $x \in
  \nulls A \cap \range A^T$, then there is a $w$ such that
  $A^T w = x$. We know from the previous equation that $(A^T w)^T x = 0
  = x^T x$. Therefore, it must be that $x=0$, so $\nulls A \cap
  \range A^T = \{ 0 \}$. 
  
  Let $U$ be such that $\nulls A \oplus U = V$ (lemma
  \ref{lem:directSum}), and $\dim(\nulls A) + \dim(U) =
  \dim(V)$. Theorem \ref{thm:rankNull} then implies that, 
  $\dim(U) = \dim(\range A)$. Additionally, lemma \ref{lem:dimSum}
  implies that $\dim(\nulls A + \range A^T) = \dim(\nulls A) +
  \dim(\range A^T) \leq \dim (V)$, so we can conclude that
  $\dim(\range A^T) \leq \dim(\range A)$.
  
  Identical reasoning shows the opposite inequality. Therefore
  $\dim(\range A) = \dim(\range A^T)$. Finally, since
  $\dim(\nulls A) + \dim(\range A^T) = \dim(V)$ and 
  $\nulls A \cap \range A^T = \{ 0 \}$, using lemma
  \ref{lem:dimisect}, we can conclude that
  $V = \nulls A \oplus \range A^T$. 
  Identical reasoning shows that $W = \nulls A^T \oplus \range A$. 
\end{proof}
\cite{strang1993} has a nice discussion of this theorem. 

The requirement of finite dimension is really essential in this
theorem. In infinite dimension, $V^\ast$ need not be the same as
$V$. If $V$ and $V^\ast$ are different, it does not make sense to talk
about $\nulls A + \range A^T$, because $\nulls A \subseteq V$ and
$\range A^T \subseteq V^\ast$. Some people object to calling this
theorem ``fundamental'' because it does not extend to infinite
dimensional spaces. 

This theorem has some nice implications for systems of linear
equations. If $A = \gmatrix{a}$ is a matrix, then the range of $A$ is 
\begin{align*}
  \{Ax : x \in \R^n\} = \left\lbrace \begin{pmatrix} \sum_{j=1}^n
      a_{1j} x_j \\ 
      \vdots
      \sum_{j=1}^n a_{mj} x_j 
    \end{pmatrix} : x_j \in \R \right\rbrace,
\end{align*}
the set of linear combinations of the columns of $A$. 
The \textbf{column space} of $A$,
denoted $\col(A)$, is the space spanned by the column vectors of
$A$. The column space of $A$ is the same as the range of $A$.

Similar reasoning shows that the space of linear combinations of the
rows of $A$ is the range of $A^T$. The \textbf{row space} of $A$,
denoted $\row (A)$, is the space spanned by the row vectors of
$A$. The fundamental theorem of linear algebra shows that the
dimensions of the column and row spaces are equal. The rank of a
matrix is the dimension of its row and column spaces.
\begin{definition}
  The \textbf{rank} of a linear transformation is the dimension of its
  range.
\end{definition}

\begin{example}
  Let $X$ be an $n \times k$ matrix. Define:
  \[ P_x = X (X^T X)^{-1} X^T \]
  and 
  \[ M_x = I - X (X^T X)^{-1} X^T. \]
  Both $P_x$ and $M_x$ are linear transformations from $\R^n$ to
  $\R^n$. Also, both $P_x = P_x^T$ and $M_x = M_x^T$. Therefore, from
  the fundamental theorem of linear algebra, 
  \[ \R^n = \nulls P_x \oplus \range  P_x \]
  and
  \[ \R^n = \range M_x \oplus \nulls  M_x. \]
  
  Suppose $w \in \range  P_x$, then $\exists y \in \R^n$ such
  that $P_x y = w$. Notice that 
  \begin{align*}
    M_x w = & M_x P_x y \\
    = & (P_x - P_x)y = 0, 
  \end{align*}
  so $w \in \nulls  M_x$.  Similarly, if $w \in \range 
  M_x$, then $w \in \nulls  P_x$. Also, if $w \in \nulls 
  P_x$, then 
  \[ M_x w = w - P_x w = w \]
  so $w \in \range  M_x$. We can conclude that null $P_x = $
  range $M_x$ and null $M_x = $ range $P_x$. Hence,
  \[
  \begin{array}{cccc}
    \R^n = & \nulls  P_x & \oplus & \range  P_x \\
                & \verteq &            & \verteq \\
    \R^n = & \range M_x & \oplus & \nulls  M_x.
  \end{array}
  \]  
\end{example}

In studying constrained optimization  problems, 
\[ \max_x f(x) \text{ s.t.  } h(x) = c, \]
we used the fact that
\begin{equation}
  D h_x v = 0 \implies D f_x v = 0 \label{eq:dhdf}
\end{equation}
is equivalent to there existing $\lambda^T$ such that
\[ Df_x + \lambda^T Dh_x = 0. \]
Equation (\ref{eq:dhdf}) is just another way of saying $\nulls Dh_x
\subseteq \nulls Df_x$. An immediate consequence of theorem
\ref{thm:fund} is the following. 
\begin{corollary}\label{cor:nullrange}
  Let $A \in L(V,W)$ and $B \in L(V,Z)$. Then $\nulls A \subseteq
  \nulls B$ iff $\range A^T \supseteq \range B^T$.
\end{corollary}
\begin{proof}
  From theorem \ref{thm:fund}, we know that $V = \nulls A \oplus
  \range A^T = \nulls B \oplus \range B^T$. Suppose $\nulls A
  \subseteq \nulls B$. Let $v \in \range B^T$. Then either $v =
  0$, in which case $v \in \range A^T$, or $v \not\in \nulls B$. If $v
  \not\in \nulls B$, $v \not\in \nulls A$ as well. Therefore, $v \in \range
  A^T$. 

  If $\range A^T \supseteq \range B^T$, an identical argument shows
  $\nulls A \subseteq \nulls B$.
\end{proof}
Letting $A = Dh_x$ and $B=Df_x$, we know that $\nulls Dh_x \subseteq
Df_x$ iff $\range Dh_x^T \supseteq \range Df_x^T$. This means that for
every $w^* \in W^*$, $\exists \lambda_w \in Z^*$ such that
\[ Df_x^T w^* = Dh_x^T \lambda_w. \]
For optimization problems, $W=\R$, so it suffices to just consider
$w^* = 1$, 
\[ Df_x^T = Dh_x^T \lambda. \]
 If $V$ and $Z$ are finite dimensional, then we can substract and
 transpose to get
\[ Df_x - \lambda^T Dh_x = 0. \]

Corollary (\ref{cor:nullrange}) is also true in infinite dimensional
spaces under some additional conditions.\footnote{The ranges of $A$
  and $B$ must be closed. See e.g.\ section 6.6 of
  \cite{luenberger1969}.} In fact, theorem \ref{thm:multipliers} is
equivalent to this corollary.  When working with optimal control
problems, we were using this result and making some implicit
assumptions about how elements from the duals spaces can be written as
integrals. Appropriately defining $V$ and $Z$ ensures that these
assumptions hold. The details are tedious, so we will not go into
them.

%%%%%%%%%%%%%%%%%%%%%%%%%%%%%%%%%%%%%%%%%%%%%%%%%%%%%%%%%%%%%%%%%%%%%%%%%%%%%%%%
% \section{Inner product spaces}

% In $\R^n$ we can measure angles and talk about vectors being
% orthogonal to one another. The following definition generalizes this
% idea to other vector spaces.
% \begin{definition}
%   An \textbf{inner product space} is a vector space with an additional
%   operation called the inner product that is 
%   function from $V \times V$ to $\mathbb{R}$. We denote the inner
%   product of $v_1, v_2 \in V$ by $\iprod{v_1}{v_2}$. It has the
%   following properties:
%   \begin{enumerate}
%   \item Symmetry: $\iprod{v_1}{v_2} = \iprod{v_2}{v_1}$
%   \item Bilinear: $\iprod{a v_1 + b v_2}{v_3} = a \iprod{v_1}{v_3} + b
%     \iprod{v_2}{v_3}$ for $a, b \in \R$
%   \item Positive definite: $\iprod{v}{v} \geq 0$ and equals $0$ iff
%     $v=0$. 
%   \end{enumerate}  
% \end{definition}
% Any inner product space is also a normed vector space with norm
% \[ \norm{x} = \sqrt{\iprod{x}{x}}. \]
% \begin{example}
%   $\R^n$ with the \textbf{dot product}, $x \cdot y = \sum_{i=1}^n x_i
%   y_i$, is an inner product space. 
% \end{example}
% \begin{example}
%   $\ell^p$ with $\iprod{x}{y} = \sum_{i=1}^\infty x_i y_i$ is
%   an inner product space.
% \end{example}
% \begin{example}
%   $\mathcal{L}^2(0,1)$ with $\iprod{f}{g}
%   \equiv \int_0^1 f(x) g(x) dx$ is an inner product space. 
% \end{example}

% Suppose we have an inner product space then:
% \begin{align*}
%   \norm{x + y}^2 = & \iprod{x+y}{x+y} \\
%   = & \iprod{x}{x} + 2\iprod{x}{y} + \iprod{y}{y}
% \end{align*}
% In $\R^n$ with the Euclidean norm when $x$ and $y$ are at right angles
% to one other, $\iprod{x}{y} = 0$, and we have the Pythagorean theorem:
% \[ \norm{x + y}^2 = \norm{x}^2 + \norm{y}^2. \]
% This motivates the following definition
% \begin{definition}
%   Let $x,y \in V$, an inner product space. $x$ and $y$ are
%   \textbf{orthogonal} iff $\iprod{x}{y} = 0$. 
% \end{definition}

% \subsection{Useful inequalities}
% When we work with limits, we often need to prove that the norm of
% something is small. There are a number of inequalities that we will
% repeatedly use. The most common is the triangle inequality, which was
% part of our definition of norms.  The triangle inequality has many
% implications, some of which are not obvious. These implications are
% often useful in proofs. The most common is what is known as the
% reverse triangle inequality.
% \begin{theorem}[Reverse triangle inequality]
%   Let $V$ be a normed vector space and $x,y \in V$. Then
%   \[ \left| \norm{x} - \norm{y} \right| \leq \norm{x-y}. \]
% \end{theorem}
% \begin{proof}
%   By the usual triangle inequality,
%   \begin{align*}
%     \norm{x} + \norm{x-y} \geq & \norm{y} \\
%     \norm{x-y} \geq \norm{y} - \norm{x}
%   \end{align*}
%   and
%   \begin{align*}
%     \norm{y} + \norm{y-x} \geq & \norm{x} \\
%     \norm{y-x} \geq \norm{x} - \norm{y}.
%   \end{align*}
%   Also, $\norm{y-x} = \norm{(-1)(x-y)} = |-1| \norm{x-y} =
%   \norm{x-y}$ is greater than both $\norm{x}-\norm{y}$ and
%   $\norm{y}-\norm{x}$ and 
%   \[ \norm{x-y} \geq \left| \norm{x} - \norm{y} \right|. \]
% \end{proof}

% After the triangle inequality, arguably the most important inequality
% in mathematics is the Cauchy-Schwarz inequality. 
% \begin{theorem}[Cauchy-Schwarz inequality \label{thm:cauchy-schwarz}]
%   Let $V$ be an inner product space and let $u,v\in V$. Then,
%   \[ \left\vert \iprod{u}{v} \right\vert \leq \norm{u}\norm{v}. \]
% \end{theorem}
% \begin{proof}
%   The idea of this proof can be illustrated in $\R^2$ by forming a right
%   angle triangle with vertices at $0$, $u$, and $tv$ and sides of
%   lengths $\norm{tv}$, $\norm{u-tv}$, and $\norm{v}$, where $t$ is
%   chosen such that $v$ and $u-tv$ are orthogonal, as shown in the
%   following diagram.
%   \begin{center}
%     \includegraphics[width=0.5\linewidth]{cauchy-schwarz.png}
%   \end{center}
%   We are choosing $t$ so that $\iprod{v}{u-tv} = 0$, so solving for $t$,
%   \begin{align*}
%     \iprod{v}{u-tv} = & 0 \\
%     \iprod{v}{v} = & t \iprod{u}{v} \\
%     t = & \frac{\iprod{u}{v}}{\norm{v}^2}.
%   \end{align*}
%   We can choose such a $t$ in any inner product space, not just
%   $\R^2$.
  
%   Now, let $z = u-tv$. By construction, $\iprod{z}{v} = 0$ and $u = tv
%   + z$. Hence, by the Pythagorean theorem,
%   \begin{align*}
%     \norm{u}^2 = & \norm{tv}^2 + \norm{z}^2 \\
%     = & \frac{\iprod{u}{v}^2}{\norm{v}^2} + \norm{z}^2 
%   \end{align*}
%   $\norm{z}^2\geq 0$, so
%   \begin{align*}
%     \norm{u}^2 \geq & \frac{\iprod{u}{v}^2}{\norm{v}^2} \\
%     \norm{u}\norm{v} \geq |\iprod{u}{v}|.
%   \end{align*}
% \end{proof}
% Notice that in the proof, we also saw that  $\norm{u}\norm{v} =
% |\iprod{u}{v}|$ if and only if $\norm{z} = 0$. $\norm{z}$ is zero
% whenever $u$ and $v$ are linearly dependent i.e. $u = \alpha v$ where
% $\alpha \in \R$. 

% For normed vector spaces without an inner-product, there is a version
% of the Cauchy-Schwarz inequality involving dual spaces.
% \begin{corollary}[Cauchy-Schwarz inequality for dual spaces]
%   Let $V$ be a normed vector space and let $v\in V$, and $u \in
%   V^*$. Then,
%   \[ \left\vert uv \right\vert \leq \norm{u}_{V^*}\norm{v}_V. \]
% \end{corollary}


% \subsection{Projections}
% The mapping from $u$ to $tv$ that we saw in the proof of theorem
% \ref{thm:cauchy-schwarz} is so common that it has
% a name. It is the projection of $u$ onto $v$. 
% \begin{definition}
%   Let $V$ be an inner product space and $x,y \in V$. The (orthogonal) 
%   \textbf{projection} of $y$ onto $x$ is 
%   \[ P_x y = \frac{\iprod{y}{x}}{\norm{x}^2} x. \]
% \end{definition}
% Note that if $z \in \spn(\{x\})$, then 
% \begin{align*}
%   \norm{z - y}^2 = & \norm{ (z - P_x y) + (P_x y - y)}^2 \\
%   = & \norm{ z - P_x y }^2+ 2\iprod{z-P_x y}{P_x y - y} + \norm{P_x y
%     - y}^2 \\
%   = & \norm{ z - P_x y }^2+ \norm{P_x y
%     - y}^2 \geq \norm{P_x y - y}^2.
% \end{align*}
% In other words, $P_x y$ minimizes $\norm{\tilde{x} - y}$ for
% $\tilde{x} \in \spn(x)$. This motivates the following definition of a
% projection.
% \begin{definition}
%   Let $U \subseteq V$ be a subspace of the inner-product space
%   $V$. Then the \textbf{projection} of $x$ onto $U$ is the unique $P_U x
%   \in U$ such that $\norm{x - P_U x} \leq \norm{x - u}$ for all $u \in
%   U$. 
% \end{definition}
% To ensure that this definition makes sense, we should verify that $P_U
% x$ exists and is unique. Uniqueness follows from the following.
% \begin{lemma} \label{lem:orthoProj}
%   $\iprod{x-P_U x}{u} = 0$ for all $u \in U$.
% \end{lemma}
% \begin{proof}
%   Let $u_1 = P_U x + \iprod{x-P_U x}{u} u$. Then
%   \begin{align*}
%     \norm{x - u_1}^2 = & \norm{x - P_U x -  \iprod{x-P_U x}{u} u}^2 \\
%     = & \norm{x - P_u x}^2 - \iprod{x-P_U x}{u}^2 
%   \end{align*}
%   This must be greater than or equal to $\norm{x - P_U x}^2$, so it must
%   be that $\iprod{x - P_U x}{u} = 0$.
% \end{proof}
% \begin{lemma}
%   $P_U x$ is unique.
% \end{lemma}
% \begin{proof}
%   Let $z \in U$
%   \begin{align*}
%     \norm{z - x}^2 = & \norm{ (z - P_U x) + (P_U x - x)}^2 \\
%     = & \norm{ z - P_U x }^2+ 2\iprod{z-P_U x}{P_U x - x} + \norm{P_U x
%       - x}^2 \\
%     = & \norm{ z - P_U x }^2+ \norm{P_U x
%       - x}^2 \geq \norm{P_U x - x}^2.
%   \end{align*}
%   Also, $  \norm{z - x}^2 = \norm{P_U x - x}^2$ only if $\norm{P_u x -
%     z}=0$, i.e. only if $z = P_U x$.
% \end{proof}
% In finite dimensional spaces, $P_U x$ can be constructed from the
% basis. First, another definition.
% \begin{definition}
%   $E \subseteq V$ are \textbf{orthonormal} if for all $e \in E$,
%   $\norm{e} = 1$ and for all $e \neq b \in E$, $\iprod{e}{b} = 0$.
% \end{definition}
% Projections can be written in terms of an orthonormal basis for
% subspaces. 
% \begin{lemma}
%   If $\{e_1, ..., e_n\}$ are an orthonormal basis for $U \subseteq V$,
%   then $P_U x = \sum_{i=1}^n \iprod{e_i}{x} e_i.$
% \end{lemma}
% \begin{proof}
%   ...
% \end{proof}
% The existence of an orthonormal basis is shown by the Gram-Schmidt process.
% \begin{theorem}[Gram-Schmidt]
%   Let $\{v_i\}$ be a countable or finite set of linearly independent
%   vectors. Then there exists an orthnormal set $\{e_i\}$ such that
%   $\spn(v_1, ... , v_n) = \spn(e_1, ..., e_n)$ for all $n$.
% \end{theorem}
% \begin{proof}
%   Let $e_1 = \frac{v_1}{\norm{v_1}}$. Define the remaining $e_i$ by
%   induction.
%   Let $z_i = v_i - \sum_{j=1}^{i-1} \iprod{v_i}{e_j}e_j$
%   and $e_i = \frac{z_i}{\norm{z_i}}$.
% \end{proof}
% In infinite dimensional spaces an additional condition called
% completeness is required to show $P_U x$ exists. We may define
% completeness later.
% \begin{lemma} 
%   A projection is a linear transformation. 
% \end{lemma}
% \begin{proof}
%   ...
% \end{proof}

% Projections are related to subspaces as follows.
% \begin{definition}
%   Let $U \subseteq V$. The \textbf{orthogonal complement} of $U$ is
%   \[ U^\perp = \{v \in V: \iprod{v}{w} = 0 \forall w \in U \}. \]
% \end{definition}
% The bilinearity of the inner product implies that $U^\perp$ is a
% linear subspace. Also, if $U$ is itself is a subspace, then
% \begin{lemma}
%   If $U \subseteq V$ is a subspace,  then 
%   \[ V = U \oplus U^\perp. \]
% \end{lemma}
% \begin{proof}
%   We will use lemma \ref{lem:dirSum}. The positive definiteness of the
%   inner product implies $U \cap U^\perp = \{0\}$. Also, for any $x \in
%   V$, $x = P_U x + (x - P_U x)$. $P_U x \in U$ by definition, and $x -
%   P_U x \in U^\perp$ by lemma \ref{lem:orthoProj}. Therefore $V = U +
%   U^\perp.$ 
% \end{proof}


\bibliographystyle{jpe}
\bibliography{../526}



\end{document}
