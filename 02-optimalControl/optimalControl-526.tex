\documentclass[12pt,reqno]{amsart}

\usepackage{amsmath}
\usepackage{amsfonts}
\usepackage{graphicx}
%\usepackage{epstopdf}
\usepackage{hyperref}
\usepackage[left=1in,right=1in,top=0.9in,bottom=0.9in]{geometry}
\usepackage{multirow}
\usepackage{verbatim}
\usepackage{fancyhdr}
\usepackage{mdframed}
\usepackage{natbib}
\renewcommand{\cite}{\citet}

%\usepackage[small,compact]{titlesec} 

%\usepackage{pxfonts}
%\usepackage{isomath}
\usepackage{mathpazo}
%\usepackage{arev} %     (Arev/Vera Sans)
%\usepackage{eulervm} %_   (Euler Math)
%\usepackage{fixmath} %  (Computer Modern)
%\usepackage{hvmath} %_   (HV-Math/Helvetica)
%\usepackage{tmmath} %_   (TM-Math/Times)
%\usepackage{cmbright}
%\usepackage{ccfonts} \usepackage[T1]{fontenc}
%\usepackage[garamond]{mathdesign}
\usepackage{color}
\usepackage[normalem]{ulem}

\newtheorem{theorem}{Theorem}[section]
\newtheorem{conjecture}{Conjecture}[section]
\newtheorem{corollary}{Corollary}[section]
\newtheorem{lemma}{Lemma}[section]
\newtheorem{proposition}{Proposition}[section]
\theoremstyle{definition}
\newtheorem{assumption}{}[section]
%\renewcommand{\theassumption}{C\arabic{assumption}}
\newtheorem{definition}{Definition}[section]

\newtheorem{step}{Step}[section]
\newtheorem{remark}{Comment}[section]
\newtheorem{ex}{Example}[section]
\newtheorem*{ex*}{Example}
\newtheorem{exer}{Exercise}[section]


\newenvironment{example}
  {\begin{mdframed}\begin{ex}}
  {\end{ex}\end{mdframed}}

\newenvironment{exercise}
  {\begin{mdframed}\begin{exer}}
  {\end{exer}\end{mdframed}}

\linespread{1.1}

\pagestyle{fancy}
%\renewcommand{\sectionmark}[1]{\markright{#1}{}}
\fancyhead{}
\fancyfoot{} 
%\fancyhead[LE,LO]{\tiny{\thepage}}
\fancyhead[CE,CO]{\tiny{\rightmark}}
\fancyfoot[C]{\small{\thepage}}
\renewcommand{\headrulewidth}{0pt}
\renewcommand{\footrulewidth}{0pt}

\fancypagestyle{plain}{%
\fancyhf{} % clear all header and footer fields
\fancyfoot[C]{\small{\thepage}} % except the center
\renewcommand{\headrulewidth}{0pt}
\renewcommand{\footrulewidth}{0pt}}

\makeatletter
\renewcommand{\@maketitle}{
  \null 
  \begin{center}%
    \rule{\linewidth}{1pt} 
    {\Large \textbf{\textsc{\@title}}} \par
    {\normalsize \textsc{Paul Schrimpf}} \par
    {\normalsize \textsc{\@date}} \par
    {\small \textsc{University of British Columbia}} \par
    {\small \textsc{Economics 526}} \par
    \rule{\linewidth}{1pt} 
  \end{center}%
  \par \vskip 0.9em
}
\makeatother

\newcommand{\argmax}{\operatornamewithlimits{arg\,max}}
\newcommand{\argmin}{\operatornamewithlimits{arg\,min}}
\def\inprobLOW{\rightarrow_p}
\def\inprobHIGH{\,{\buildrel p \over \rightarrow}\,} 
\def\inprob{\,{\inprobHIGH}\,} 
\def\indist{\,{\buildrel d \over \rightarrow}\,} 
\def\F{\mathbb{F}}
\def\R{\mathbb{R}}
\def\Er{\mathrm{E}}
\def\Pr{\mathrm{P}}
\def\En{\mathbb{E}_n}
\def\Pn{\mathbb{P}_n}
\newcommand{\gmatrix}[1]{\begin{pmatrix} {#1}_{11} & \cdots &
    {#1}_{1n} \\ \vdots & \ddots & \vdots \\ {#1}_{m1} & \cdots &
    {#1}_{mn} \end{pmatrix}}
\newcommand{\iprod}[2]{\left\langle {#1} , {#2} \right\rangle}
\newcommand{\norm}[1]{\left\Vert {#1} \right\Vert}
\newcommand{\abs}[1]{\left\vert {#1} \right\vert}
\renewcommand{\det}{\mathrm{det}}
\newcommand{\rank}{\mathrm{rank}}
\newcommand{\spn}{\mathrm{span}}
\newcommand{\row}{\mathrm{Row}}
\newcommand{\col}{\mathrm{Col}}
\renewcommand{\dim}{\mathrm{dim}}
\newcommand{\prefeq}{\succeq}
\newcommand{\pref}{\succ}
\newcommand{\seq}[1]{\{{#1}_n \}_{n=1}^\infty }
\renewcommand{\to}{{\rightarrow}}
 

\title{Optimal Control}
\date{\today}

\begin{document}

\maketitle

\section{References}

These notes are about optimal control.  References from our text books
are chapter 10 of \cite{dixit1990} and chapter 20 \cite{cw2005}.
\cite{dorfman1969} derives the maximum principle in a model of capital
accumulation. \cite{intriligator1975} has some economic
examples. There are two textbooks about optimal control in economics
available online through UBC libraries. \cite{sethi2000} focuses on
examples. \cite{caputo2005} also has many examples, but goes into a
bit more mathematical detail.

Although more advanced than what these notes cover,
\cite{luenberger1969} is the classic mathematics text on optimal
control and is excellent. \cite{clarke2013} is available online
through UBC libraries and covers similar material as
\cite{luenberger1969}, but at an even more advanced level. 


\section{Introduction}

In the past few lectures we have focused on optimization problems of the form 
\[ \max_{x} f(x) \text{ s.t. } h(x) = c \]
where $x \in\R^n$. The variable that we are optimizing over,
$x$, is a finite dimensional vector. There are interesting
optimization problems in economics that involve an infinite
dimensional choice variable. The most common example are models where
something is chosen at every instant of time.
\begin{example}\label{ex:grow}[Optimal growth]
  
\end{example}
\begin{example}\label{ex:inv}[Investment costs]
  
\end{example}
The previous two examples involved making a decision at every instant
in time. Problems that involve making a decision for each of a
continuum of types also lead to infinite choice variables.
\begin{example}\label{ex:contract}[Contracting with a continuum of
  types]
  On problem set 1, we studied a problem where a price discriminating
  monopolist was selling a good to two types of consumers. We could
  also imagine a similar situation where there is a continuous
  distribution of types. A consumer of type $\theta$ gets $0$ utility
  from not buying the good, and $\theta \nu(q) - T$ from buying $q$
  units of the good at cost $T$. Let the types be indexed by $\theta \in
  [\theta_l,\theta_h]$ and suppose the density of $\theta$ is $f_\theta$. The
  seller does not observe consumers' types. However, the seller can
  offer a menu of contracts $(q(\theta),T(\theta))$ such that type
  $\theta$ will choose contract $(q(\theta),T(\theta))$. The seller
  chooses the contracts to maximize profits subject to the requirement
  that each type chooses its contract.
  \begin{align}
    \max_{q(\theta),T(\theta)} & \int_{\theta_l}^{\theta_h} 
    \left[T(\theta) - cq(\theta)\right]
    f_\theta(\theta) d\theta \notag \\
    & \text{s.t.} \notag \\
    &\theta \nu\left(q(\theta)\right) - T(\theta) \geq 0  \forall
    \theta \label{pc} \\
    &\theta \nu\left(q(\theta)\right) - T(\theta) \geq
    \max_{\tilde{\theta}} \theta \nu\left(q(\tilde{\theta}) \right) -
    T(\tilde{\theta}) \forall \theta \label{ic} 
  \end{align}
  The first constraint (\ref{pc}) is referred to as the participation
  (or individual rationality)
  constraint. It says that each type $\theta$ must prefer buying the
  good to not. The second constraint  (\ref{ic}) is referred to as the
  incentive compatibility constraint. It says that type $\theta$ must
  prefer buying the $\theta$ contract instead of pretending to be
  some other type $\tilde{\theta}$. This approach to contracting with
  asymmetric information---that you can setup contracts such that each
  type chooses a contract designed for it---is called the revelation
  principle because the choice of contract makes the consumers reveal
  their types. 
  
  Note that this setup does not just apply to price discriminating
  monopolists. There are many other applications. For example, if you
  consider a government that needs to raise a certain amount of
  revenue by taxing workers with heterogeneous productivity, then you
  end up with essentially the same problem. $\theta$ would be worker
  productivity. $q(\theta)$ is the labor supplied by type $\theta$,
  $\theta \nu(q(\theta))$ is the output of type $\theta$, and
  $T(\theta)$ is the tax. The government maximizes total output
  subject to a revenue constraint and the constraints above.  
\end{example}
\begin{example}[Optimal income taxation \citet{mirrlees1971}]
  There are a continuum of workers with productivity $w \in [w_l,w_h]$
  with density $f$. Workers' wages equal their productivity. If a
  worker supplies $\ell$ units of labor, she produces $w \ell$ units
  of consumption. Workers' utility from consumption, $c$, and labor
  $\ell$ is given by $u(c,\ell)$ with $\frac{\partial u}{\partial c} >
  0$ and $\frac{\partial u}{\partial \ell} < 0$. The government
  chooses taxes $t(w\ell)$ to maximize social welfare, which is given
  by $\int G(u(c(w),\ell(w))) f(w) dw$ subject to some required
  spending of $g$. The government does
  not observe $w$. Using the revelation principle as in the previous
  example, we can write the government's problem as
  \begin{align*}
    \max_{ell,t} &  \int_{w_l}^{w_h} G\left(u(w\ell(w) - t(w),\ell(w))
    \right) f(w) dw \\
    & \text{ s.t. } \\
    & \int_{w_l}^{w_h} t(w) f(w) dw \geq g
    & \ell(w) \in \argmax_{\tilde{\ell}} u(w\ell - t(w\ell), \ell )
  \end{align*}  
  Under standard conditions on $u$, we can replace the last constraint
  with a first order condition,
  \[ \frac{\partial u}{\partial c} w\left(1 - t'(w\ell)\right) +
  \frac{\partial u}{\partial \ell} = 0. \]
  As in the first two examples, this last constraint now involves the
  derivative of one the variables we are maximizing over, $t'$.
\end{example}

All of these examples have a common structure. They each have the
following form:
\begin{align*}
  \max_{x(t),y(t)} & \int_0^T F(x(t),y(t),t) dt \\
  & \text{ s.t.} \\
  & \frac{d y}{dt} = g(x(t),y(t),t) \forall t \in [0,T] \\ 
  & y(0) = y_0
\end{align*}
This is a generic continuous time optimal control problem. 
$x$ is called a control variable, and $y$ is called a state
variable. The choice of $x$ controls the evolution of $y$ through the
first constraint. We will now derive some necessary conditions for a
maximum. 

\section{The maximum principle}

Let's approximate our continuous time problem with a discrete time
problem. This will be useful because we already know how to solve
optimization problems with a finite dimensional choice variable. 

The integral can be approximated by a sum. If divide $[0,T]$ into $J$
intervals of length $\Delta$, we have
\begin{align*}
  \int_0^T F(x(t),y(t),t) dt \approx \sum_{j=1}^T \Delta f(x(\Delta
  j), y(\Delta j) , \Delta j).
\end{align*}
Similarly, we can approximate $\frac{dy}{dt} \approx \frac{y(\Delta j)
  - y(\Delta(j-1))}{\Delta}$. 
Thus, we can approximate the whole problem by:
\begin{align*}
  \max_{x_1, ..., x_J, y_1, ..., y_J} & \sum_{j=1}^J \Delta F(x_j,
  y_j,\Delta j) 
  & \text{ s.t.} \\
  & y_j - y_{j-1} = \Delta g(x_j,y_j,\Delta j) \text{ for} j = 1,...,J
\end{align*}
where we are letting $x_j = x(\Delta j)$, $y_j = y(\Delta j)$. This is
just a usual optimization problem. The first order conditions are
\begin{align*}
  [x_j]: &&  \Delta \frac{\partial F}{\partial x} + \lambda_j \Delta
  \frac{\partial g}{\partial x} = & 0 \\
  [y_j]: &&  \frac{\partial F}{\partial y} - \lambda_j + \lambda_{j+1}
  + \lambda_j \Delta
  \frac{\partial g}{\partial y} = & 0 \\
  [\lambda_j]: && y_j - y_{j-1} - \Delta g(x_j,y_j,\Delta j) = & 0 .
\end{align*}
Each of these equations hold for $j = 1,..., J$. Also, since there is
no $J+1$ constraint, we set $\lambda_{J+1}=0$. Rearranging these
slightly gives
\begin{align*}
  [x_j]: &&  \frac{\partial F}{\partial x} + \lambda_j \frac{\partial
    g}{\partial x} = & 0 \\ 
  [y_j]: &&  \frac{\partial F}{\partial y} + \lambda_j 
  \frac{\partial g}{\partial y} = & -\frac{\lambda_{j+1} -
    \lambda_j}{\Delta} \\
  [\lambda_j]: && g(x_j,y_j,\Delta j) = & \frac{y_j - y_{j-1}}{\Delta}.
\end{align*}
Using our approximation to go back to continuous time, these become,
\begin{align*}
  [x_j]: &&  \frac{\partial F}{\partial x} + \lambda(t) frac{\partial
    g}{\partial x} = & 0 \\ 
  [y_j]: &&  \frac{\partial F}{\partial y} + \lambda(t)
  \frac{\partial g}{\partial y} = & -\frac{d\lambda}{dt} \\ 
  [\lambda_j]: && g(x_j,y_j,\Delta j) = & \frac{dy}{dt}. 
\end{align*}
Any optimal $x(t)$, $y(t)$, and $\lambda(t)$ must satisfy these
equations. This result is known as Pontryagin's maximum principle. It
is often stated by defining the Hamiltonian, 
\[ H(x,y,\lambda,t) = F(x(t),y(t),t) + \lambda(t) g(x(t),y(t),t). \]
Using the Hamiltonian, the three equations can be written as
\begin{align*}
  [x]: && 0 = & \frac{\partial H}{\partial x}(x^*,y^*,\lambda^*,t)
  \\
  [y]: && -\frac{d\lambda}{dt}(t) = & \frac{\partial H}{\partial y}(x^*,y^*,\lambda^*,t) \\
  [\lambda]: && \frac{dy}{dt}(t) = & \frac{\partial H}{\partial
    \lambda}(x^*,y^*,\lambda^*,t).
\end{align*}
The $[y]$ equation is called the co-state equation, and $\lambda$ are
called co-state variables ($\lambda$ is still also called the Lagrange
multiplier).

Pontryagin's maximum principle can also be derived by starting with a
continuous time Lagrangian. Recall that our problem is:
\begin{align*}
  \max_{x(t),y(t)} & \int_0^T F(x(t),y(t),t) dt \\
  & \text{ s.t.} \\
  & \frac{d y}{dt} = g(x(t),y(t),t) \forall t \in [0,T] \\ 
  & y(0) = y_0.
\end{align*}
The Lagrangian is the objective function minus the sum of the
multiplies times the constraints. Here, we have a continuum of
constraints on $dy/dt$, so instead of a sum we must use an
integral. The Lagrangian is then 
\begin{align*}
  L(x,y,\lambda,\mu_0) = \int_0^T F(x(t),y(t),t) dt - \int_0^T
  \lambda(t)\left( \frac{dy}{dt} - g(x(t),y(t),t) \right) dt - \mu_0
  (y(0) - y_0) 
\end{align*}
It is somewhat difficult to think about the derivative of $L$ with
respect to $y$ because $L$ involves both $y$ and $dy/dt$. We can get
around this by eliminating $dy/dt$ through integration by
parts.\footnote{Integration by parts says that $\int_a^b u(x) v'(x) dx
  = u(b)v(b) - u(a)v(a) - \int_a^b v(x) u'(x) dx$.} Integrating by
parts gives
\begin{align*}
  L(x,y,\lambda,\mu_0) = \int_0^T F(x(t),y(t),t) dt +\int_0^T  \left(  
    \lambda(t) g(x(t),y(t),t) + y(t) \frac{d\lambda}{dt}\right)dt -
  \lambda(T) y(T) + \lambda(0) y(0) - \mu_0
  (y(0) - y_0).
\end{align*} 
We can then differentiate with respect to $x(t)$ and $y(t)$ to get the
first order conditions
\begin{align*}
  [x]: && \frac{\partial F}{\partial x} + \lambda(t) \frac{\partial
    g}{\partial x} = &  0 \\
  [y]: && \frac{\partial F}{\partial y} + \lambda(t) \frac{\partial
    g}{\partial y} = &  -\frac{d\lambda}{dt}  
\end{align*}
These, along with the constraint, are once again the Maximum
principle. 

\subsection{Transversality conditions}

The three equations of the maximum principle are not necessarily
enough to determine $x$, $y$, and $\lambda$. The reason is that they
tell us about $dy/dt$ and $d\lambda/dt$ instead of $y$ and
$\lambda$. For any constant, $c$, $d(y+c)/dt = dy/dt$, so the three
equations only determine $y$ and $\lambda$ up to a constant. To pin
down the constant for $y$, the problem tells us what $y(0)$ must
be. To pin down the constant for $\lambda$, there is an extra
condition on $\lambda(T)$. This condition is called a transversality
condition. Unfortunately, the heuristic derivations we did above
obscure the transversality condition somewhat. In the discrete time
approach, we set $\lambda_{J+1} =0$, so we also should impose
$\lambda(T) = 0$. 

The same condition appears in the Lagrangian approach, if we more
careful about taking derivatives. Taking the derivative of $\int_0^T
F(x(t),y(t),t) dt$ with respect to $x(\tau)$ holding $x(\cdot)$
constant for all other periods does not really make sense because
changing $x$ at a single point will not change the integral at
all. Instead, we need to consider directional derivatives. Let $v$ be
another function of $t$. Then the derivative of $L$ with respect to
$x$ in direction $v$ is
\[ d_xL(x,y,\lambda,\mu_0;v) = \lim_{\alpha \to 0} \frac{L(x + \alpha
  v, y, \lambda, \mu_0) - L(x,y,\lambda,\mu_0)}{\alpha} =
\frac{d}{d\alpha} L(x+\alpha v, y, \lambda, \mu_0). \]
This is exactly the same as our previous definition of a directional
derivative, except now the direction is a function. The first order
conditions are that the directional derivatives are zero for all
directions (functions) $v$. Assuming that $F$ is well-behaved so that
we can interchange integration and differentiation,\footnote{i.e. the
  dominated convergence theorem applies}, the first order conditions
are then [FIXME: show more details]
\begin{align*}
  [x]: && \int_0^T \frac{\partial F}{\partial x}v(t) + \lambda(t) \frac{\partial
    g}{\partial x}v(t) dt = &  0 \\
  [y]: && \int_0^T \left(\frac{\partial F}{\partial y}v(t) + \lambda(t) \frac{\partial
      g}{\partial y}-\frac{d\lambda}{dt} \right)v(t) dt - \lambda(T) 
  v(T) + \lambda(0) v(0) - \mu_0 v(0)= & 0
\end{align*}
for all functions $v$. If we take $v(t) = \begin{cases} 0 \text{ if
  }t<T\\
1 & \text{ if } t=T \end{cases}$, then the second equation requires
$\lambda(T)=0$. 

We can also deduce the transversality condition by thinking about the
interpretation of the multiplier as the shadow price of the relaxing
the constraint. Relaxing the constraint affects the objective function
by changing future $y$. At time $T$, there is no future $y$, so the
value of relaxing the constraint must be $0=\lambda(T)$. 

It is important to understand where the transversality condition comes
from, because the transversality condition can change depending on
whether $T$ is finite or infinite and whether or not there are
constraint on $y(T)$ or $x(T)$. 

%%%%%%%%%%%%%%%%%%%%%%%%%%%%%%%%%%%%%%%%%%%%%%%%%%%%%%%%%%%%%%%%%%%%%%%%%%%%%%%% 

\section{Applications}

\subsection{Optimal contracting with a continuum of types} 
Let's solve example \ref{ex:contract}. 
\begin{align}
  \max_{q(\theta),T(\theta)} & \int_{\theta_l}^{\theta_h} 
  \left[T(\theta) - cq(\theta)\right]
  f_\theta(\theta) d\theta \notag \\
  & \text{s.t.} \notag \\
  &\theta \nu\left(q(\theta)\right) - T(\theta) \geq 0  \forall
  \theta \label{pc2} \\
  &\theta \nu\left(q(\theta)\right) - T(\theta) \geq
  \max_{\tilde{\theta}} \theta \nu\left(q(\tilde{\theta}) \right) -
  T(\tilde{\theta}) \forall \theta \label{ic2} 
\end{align}
First, notice that if the participation constraint (\ref{pc2}) holds for type
$\theta_l$, and (\ref{ic2}) holds for $\theta$, then the participation
constraint must also hold for $\theta$. 

We can show that the incentive compatibility constraint (\ref{ic2}) is
equivalent to the following local incentive 
compatibility constraint and monotonicity constraint.
\begin{align}
  \theta \nu'(q(\theta))q'(\theta) - T'(\theta) = & 0 \label{lic} \\
  \frac{dq(\theta)}{d\theta} \geq & 0 \label{mon}
\end{align}
Consider the incentive compatibility constraint
(\ref{ic2}). The first order condition for the maximization is
\[ \theta \nu'(q(\tilde{\theta})) q'(\tilde{\theta}) =
T'(\tilde{\theta}). \]
This is the same as the local incentive compatibility constraint with
$\theta = \tilde{\theta}$.

The second order condition for (\ref{ic2}) is 
\[ \theta \nu''(q(\tilde{\theta}))q'(\tilde{\theta})^2 +
\tilde{\theta} \nu'(q(\tilde{\theta})) q''(\tilde{\theta}) -
T''(\tilde{\theta}) \leq 0 \]   
On the other hand if we differentiate the local incentive
compatibility constraint we get 
\begin{align*}
  \nu'(q(\theta))q'(\theta) + \theta \nu''(q(\theta)) q'(\theta)^2 + \theta
  \nu'(q(\theta)) q''(\theta) - T''(\theta) = & 0 \\
  \theta \nu''(q(\theta)) q'(\theta)^2 + \theta
  \nu'(q(\theta)) q''(\theta) - T''(\theta) = & -\nu'(q(\theta))q'(\theta)  
\end{align*}
We assume that $\nu'>0$, and the monotonicity constraint says that
$q'\geq 0$. Hence, this equation implies the second order
condition. Therefore, we have shown that the local incentive
compatibility constraint and monotonicity constraint are equivalent
incentive compatibility constraint.

Now, we can write the seller's problem as
\begin{align}
  \max_{q(\theta),T(\theta)} & \int_{\theta_l}^{\theta_h} 
  \left[T(\theta) - cq(\theta)\right]
  f_\theta(\theta) d\theta \notag \\
  & \text{s.t.} \notag \\
  &\theta_l \nu\left(q(\theta_l)\right) - T(\theta_l) \geq 0
  \label{pcl} \\
  & \theta \nu'(q(\theta))q'(\theta) - T'(\theta) =  0 \label{lic2} \\
  & \frac{dq(\theta)}{d\theta} \geq  0 \label{mon2}
\end{align}
The first order condition for $T$ is for any $x:[\theta_l,\theta_h] \to \R$, 
\begin{align*}
  0 = &\int_{\theta_l}^{\theta_h} x(\theta) f_\theta(\theta)
  d\theta - \int_{\theta_l}^{\theta_h} \mu(\theta)
  \frac{dx}{d\theta}(\theta) d\theta + \mu_0 x(\theta_l)  \\
  0 = & \int_{\theta_l}^{\theta_h} x(\theta)\left( f_\theta(\theta) +
    \frac{d\mu}{d \theta}(\theta)\right) - \mu(\theta_h)x(\theta_h) +
  \mu(\theta_l)x(\theta_l) + \mu_0 x(\theta_l).
\end{align*}
From this we see that $\mu'(\theta) = -f_\theta(\theta)$,
$\mu(\theta_l) = -\mu_0$, and $\mu(\theta_h) = 0$. Given $\mu'$ and
$\mu(\theta_h)$, it must be that
\begin{align*}
  \mu(\theta) = & \int_{\theta_h}^\theta -f_\theta(\hat{\theta}) d\hat{\theta} \\
  = & 1 - \int_{\theta_l}^\theta f_{\theta}(\hat{\theta})d\hat{\theta}
  \\
  = & 1 - F_\theta(\theta) 
\end{align*}
where $F_\theta$ is the cdf of $f_\theta$. 
The first order condition for $q$ is
\begin{align*}
  0 = & \int_{\theta_l}^{\theta_h} c x(\theta) f_\theta(\theta)
  d\theta - \int_{\theta_l}^{\theta_h} \mu(\theta)
  \left( \theta \nu''(q(\theta))q'(\theta) x(\theta) + \theta
    \nu'(q(\theta)) x'(\theta) \right)d\theta - \\
  & - \mu_0 \theta_l
  \nu'(q(\theta_l))x'(\theta_l) - \int_{\theta_l}^{\theta_h}
  \lambda(\theta) x'(\theta) d\theta \\
  0 = & \int_{\theta_l}^{\theta_h} x(\theta)\left( c f_\theta(\theta)
    - \mu(\theta) \left( \theta \nu''(q(\theta))q'(\theta) x(\theta) -
      \nu'(q(\theta)) - \theta \nu''(q(\theta))q'(\theta) \right) \right)
  d\theta + \\
  & + \int_{\theta_l}^{\theta_h} 
  \mu'(\theta) \theta \nu'(\theta) d\theta 
  - \mu(\theta_h)\theta_h \nu'(q(\theta_h))x(\theta_h) +
  \mu(\theta_l)\theta_l \nu'(q(\theta_l))x(\theta_l) - \\
  & - \mu_0 \theta_l \nu'(q(\theta_l))x'(\theta_l) 
  - \int_{\theta_l}^{\theta_h} \lambda(\theta) x'(\theta) d\theta \\
  0 = & \int_{\theta_l}^{\theta_h} x(\theta)\left( c f_\theta(\theta)
    + \mu(\theta) \nu'(q(\theta)) + 
    \mu'(\theta) \theta \nu'(\theta)\right) d\theta  - \\
    & - \mu(\theta_h)\theta_h \nu'(q(\theta_h))x(\theta_h) +
    \mu(\theta_l)\theta_l \nu'(q(\theta_l))x(\theta_l) - \\
    & - \mu_0 \theta_l \nu'(q(\theta_l))x'(\theta_l) 
    - \int_{\theta_l}^{\theta_h} \lambda(\theta) x'(\theta) d\theta 
\end{align*}
If we assume that the monotonicy constraint does not bind, so
$\lambda(\theta) = 0$, we see that
\begin{align*}
  0 = &c f_\theta(\theta) + \mu(\theta) \nu'(q(\theta)) + \mu'(\theta)
   \theta \nu'(q(\theta))  \\
   0 = & cf_\theta(\theta) + (1-F_\theta(\theta)) \nu'(q(\theta)) -
   f_{\theta}(\theta) \theta \nu'(q(\theta)) \\
   \theta \nu'(q(\theta)) = & c +
   \frac{1-F_\theta(\theta)}{f_\theta(\theta)} \nu'(q(\theta)) \\
   \left(\theta - \frac{1-F_\theta(\theta)}{f_\theta(\theta)}
   \right)\nu'(q(\theta)) = & c 
 \end{align*}
You may recall from problem set 1 that with symmetric information,
$\theta \nu'(q(\theta)) = c$. Since $\nu'$ is decreasing in $q$, this
implies that $q(\theta)$ is less than what it would be in the first
best symmetric information case for all $\theta < \theta_h$. The
highest type, $\theta_h$ gets the optimal level of consumption since
$F_\theta(\theta_h) = 1$. 
 




\clearpage
\bibliographystyle{jpe}
\bibliography{../526}

\end{document}
