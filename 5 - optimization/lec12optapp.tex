\documentclass[12pt,reqno]{amsart}

\usepackage{amsmath}
\usepackage{amsfonts}
\usepackage{graphicx}
%\usepackage{epstopdf}
\usepackage{hyperref}
\usepackage[left=1in,right=1in,top=0.9in,bottom=0.9in]{geometry}
\usepackage{multirow}
\usepackage{verbatim}
\usepackage{fancyhdr}
%\usepackage[small,compact]{titlesec} 

%\usepackage{pxfonts}
%\usepackage{isomath}
\usepackage{mathpazo}
%\usepackage{arev} %     (Arev/Vera Sans)
%\usepackage{eulervm} %_   (Euler Math)
%\usepackage{fixmath} %  (Computer Modern)
%\usepackage{hvmath} %_   (HV-Math/Helvetica)
%\usepackage{tmmath} %_   (TM-Math/Times)
%\usepackage{cmbright}
%\usepackage{ccfonts} \usepackage[T1]{fontenc}
%\usepackage[garamond]{mathdesign}
\usepackage{color}
\usepackage[normalem]{ulem}

\newtheorem{theorem}{Theorem}[section]
\newtheorem{conjecture}{Conjecture}[section]
\newtheorem{corollary}{Corollary}[section]
\newtheorem{lemma}{Lemma}[section]
\newtheorem{proposition}{Proposition}[section]
\theoremstyle{definition}
\newtheorem{assumption}{}[section]
%\renewcommand{\theassumption}{C\arabic{assumption}}
\newtheorem{definition}{Definition}[section]

\newtheorem{step}{Step}[section]
\newtheorem{remark}{Comment}[section]
\newtheorem{example}{Example}[section]
\newtheorem*{example*}{Example}

\linespread{1.1}

\pagestyle{fancy}
%\renewcommand{\sectionmark}[1]{\markright{#1}{}}
\fancyhead{}
\fancyfoot{} 
%\fancyhead[LE,LO]{\tiny{\thepage}}
\fancyhead[CE,CO]{\tiny{\rightmark}}
\fancyfoot[C]{\small{\thepage}}
\renewcommand{\headrulewidth}{0pt}
\renewcommand{\footrulewidth}{0pt}

\fancypagestyle{plain}{%
\fancyhf{} % clear all header and footer fields
\fancyfoot[C]{\small{\thepage}} % except the center
\renewcommand{\headrulewidth}{0pt}
\renewcommand{\footrulewidth}{0pt}}

\makeatletter
\renewcommand{\@maketitle}{
  \null 
  \begin{center}%
    \rule{\linewidth}{1pt} 
    {\Large \textbf{\textsc{\@title}}} \par
    {\normalsize \textsc{Paul Schrimpf}} \par
    {\normalsize \textsc{\@date}} \par
    {\small \textsc{University of British Columbia}} \par
    {\small \textsc{Economics 526}} \par
    \rule{\linewidth}{1pt} 
  \end{center}%
  \par \vskip 0.9em
}
\makeatother

\newcommand{\argmax}{\operatornamewithlimits{arg\,max}}
\newcommand{\argmin}{\operatornamewithlimits{arg\,min}}
\def\inprobLOW{\rightarrow_p}
\def\inprobHIGH{\,{\buildrel p \over \rightarrow}\,} 
\def\inprob{\,{\inprobHIGH}\,} 
\def\indist{\,{\buildrel d \over \rightarrow}\,} 
\def\F{\mathbb{F}}
\def\R{\mathbb{R}}
\newcommand{\gmatrix}[1]{\begin{pmatrix} {#1}_{11} & \cdots &
    {#1}_{1n} \\ \vdots & \ddots & \vdots \\ {#1}_{m1} & \cdots &
    {#1}_{mn} \end{pmatrix}}
\newcommand{\iprod}[2]{\left\langle {#1} , {#2} \right\rangle}
\newcommand{\norm}[1]{\left\Vert {#1} \right\Vert}
\newcommand{\abs}[1]{\left\vert {#1} \right\vert}
\renewcommand{\det}{\mathrm{det}}
\newcommand{\rank}{\mathrm{rank}}
\newcommand{\spn}{\mathrm{span}}
\newcommand{\row}{\mathrm{Row}}
\newcommand{\col}{\mathrm{Col}}
\renewcommand{\dim}{\mathrm{dim}}
\newcommand{\prefeq}{\succeq}
\newcommand{\pref}{\succ}
\newcommand{\seq}[1]{\{{#1}_n \}_{n=1}^\infty }
\renewcommand{\to}{{\rightarrow}}
\providecommand{\En}{\mathbb{E}_n}
\providecommand{\Er}{{\mathrm{E}}}
\renewcommand{\Pr}{{\mathrm{P}}}
\providecommand{\set}[1]{\left\{#1\right\}}
\providecommand{\plim}{\operatornamewithlimits{plim}}
\newcommand\indep{\protect\mathpalette{\protect\independenT}{\perp}}
\def\independenT#1#2{\mathrel{\setbox0\hbox{$#1#2$}%
    \copy0\kern-\wd0\mkern4mu\box0}} 



\title{Applications of optimization}
\date{\today}

\begin{document}

\maketitle

The last two lectures have studied unconstrained and constrained
optimization in some detail. Today, we will go over some examples of
the use of the results from those two lectures.

\section{Extremum estimators}

Most estimators is econometrics are solutions to optimizations
problems. For example, ordinary least squares solves
\[ \hat{\beta} = \argmin_{\beta \in \R^k} \frac{1}{n} \sum_{i=1}^n
(y_i - x_i^T \beta)^2  \] 
where we have data on some outcome $y_i \in \R$ and some covariates,
$x_i \in \R^k$. 

Maximum likelihood estimators are the maximizer of the
likelihood of $y$ given $x$. That is, if we have some model that says
the density of $y$ given $x$ is $f(y|x;\theta)$, where $\theta \in
\Theta \subseteq \R^k$ are
some unknown parameters, then the likelihood function is
\[ \mathcal{L}_n(\theta) = \frac{1}{n} \sum_{i=1}^n f(y|x;\theta). \]
The maximum likelihood estimate of $\theta$ is
\[ \hat{\theta} = \argmax_{\theta \in \Theta}
\mathcal{L}_n(\theta). \] 

A third class of estimators that can be written as the solution to
optimization problems are generalized method of moments (GMM)
estimators. Suppose we have an economic model that perhaps does not
fully specify the density of $y$ given $x$, but does tell us that some
moments of functions of $y$ and $x$ are zero. That is,
\[ \Er[m(y,x;\theta)] = 0 \]
where $m(y,x;\theta) \in \R^m$. Then if $m \geq k$, we could estimate
$\theta$ by solving these moment equations for $\theta$. However, we
do not know the true expectation $\Er[m(y,x;\theta)]$, and we must
replace it with the empirical expectation,
\[ \En[m(y,x;\theta)] \equiv \frac{1}{n} \sum_{i=1}^n
m(y_i,x_i;\theta). \]
Since $\En$ will differ from $\Er$ there may not be any $\theta$ that
exactly solves 
\begin{align} \En[m(y,x;\theta)]=0 \label{e:emn} \end{align}
even if there is some a solution with the true expectation. For this
reason, we estimate $\theta$ by finding $\hat{\theta}$ that
approximately solves (\ref{e:emn}). We want the best approximately
solution, so one way to choose the best would be to set
\[ \hat{\theta} = \argmin \En[m(y,x;\theta)]^T \En[m(y,x;\theta)]. \]
More generally, we could say
\[ \hat{\theta} = \argmin \En[m(y,x;\theta)]^T W
\En[m(y,x;\theta)], \]
where $W$ is some positive definite weighting matrix. 

Each of the three examples above have the common feature that they can
be written as 
\[ \hat{\theta}_n = \argmin_{\theta \in \Theta} Q_n(\theta) \]
where $Q_n$ is some data dependent objective function. Estimators with
this form are called extremum estimators. Given an extremum estimator,
we would like to be know what $\hat{\theta}$ is estimating (i.e. what
it converges to), and the asymptotic distribution of $\hat{\theta}$ so
that we can perform inference (hypothesis tests, confidence intervals,
etc). Suppose $Q_n(\theta)$ converges in probability to some fixed
limit $Q_0(\theta)$. This means that as the sample size $n \to
\infty$, $Q_n(\theta)$ gets closer and closer to $Q_0(\theta)$. The
data is random, so $Q_n(\theta)$ is random as well, so the definition
of convergence of limits that we have been using isn't
appropriate. Instead, convergence in probability means that for all
$\epsilon>0$ 
\[ \lim_{n\to \infty} \Pr(|Q_n(\theta) - Q_0(\theta)|>\epsilon) = 0. \]
I would guess that you have already learned about convergence in
probability in you econometrics course, so we will not focus on it
here. 

Anyway, if 
\[ \theta_0 = \argmin_{\theta \in \Theta} Q_0(\theta) \]
then with some additional assumptions we can show that $\hat{\theta}_n
\inprob \theta_0$. In particular,
\begin{theorem}[Consistency for extremum estimators]
  If (i) $Q(\theta)$ is uniquely minimized at the true parameter value
  $\theta_0$, (ii) $\Theta$ is compact, (iii) $Q(\cdot)$ is
  continuous, and (iv) $\sup_{\theta \in \Theta} |Q_n(\theta) -
  Q(\theta)| \inprob 0$, then $\hat{\theta}_n \inprob \theta_0$.
\end{theorem}
\begin{proof}
  See Newey and McFadden (1994). 
\end{proof}
We will not go over the proof of this theorem, nor do you need to
remember it for this course. 

From your econometrics course, you probably know that for OLS
$\hat{\beta}$ is asymptotically, meaning that the distribution of
$\sqrt{n}(\hat{\beta} - \beta_0)$ converges to a normal distribution
as $n$ gets large. We denote this as
\[ \sqrt{n}(\hat{\beta} - \beta_0) \indist N(0,V). \]
We can develop an analogous result for extremum estimators. Suppose
$Q_n$ is twice continuously differentiable in $\Theta$ and
$\hat{\theta}$ is in the interior of $\Theta$. $\hat{\theta}$ must
satisfy the first order condition because it is in the interior and
$Q_n$ is differentiable. 
\begin{align*}
  0 = & D {Q_n}_{\hat{\theta}} 
\end{align*}
We want to end up with something like $\sqrt{n}(\hat{\theta} -
\theta_0)$, so let's take a mean value expansion of $DQ_n$ around
$\theta_0$, then
\begin{align*}
  0 = & D {Q_n}_{\hat{\theta}} \\
  = & D{Q_n}_{\theta_0} + D^2
  {Q_n}_{\bar{\theta}} (\hat{\theta} - \theta_0) \\
\end{align*}
If $D^2Q_n{\bar{\theta}}$ is invertible then,
\[ (\hat{\theta} - \theta_0) =  -(D^2{Q_n}_{\bar{\theta}} )^{-1}
D{Q_n}_{\theta_0} \]   
Generally, $D^2 Q_n \in D^2 Q$ and $DQ_n \inprob DQ$, but this is not
what we want. We want to get a distribution for
$\hat{\theta}-\theta_0$, not a point limit. Therefore, we should
multiple by some $a_n$ to make $a_n(\hat{\theta}-\theta_0)$ converge
to a non-degenerate distribution. The right choice of $a_n$ depends
somewhat on the details of the problem, but the vast majority of the
time the right choice is $a_n = \sqrt{n}$. This is the right choice
because, you can often show that
\begin{align} 
  \sqrt{n} {DQ_n}_{\theta_0} \indist N(0,V). \label{e:normd} 
\end{align}
This fact follows from a central limit theorem. For example, it is
often the case that 
\[ D{Q_n}_{\theta_0} = \frac{1}{n} \sum_{i=1}^n D_\theta
f(y_i,x_i,\theta_0). \] 
Additionally, if observations are independent
and identically distributed\footnote{This can be relaxed and the
  finite variance condition strengthened, and the conclusion still
  holds.}, $\Er[D_\theta f(y_i,x_i,\theta)] = 0$, and
theorem says that (\ref{e:normd}) holds. Regardless of why, if we just
$\Er[f(y,x,\theta_0)^2]$ is finite then the classical central limit
theorem says that (\ref{e:normd}) holds. Regardless of why, if we just
assume (\ref{e:normd}), then
\begin{align*}
  \sqrt{n}(\hat{\theta} - \theta_0) = & D^2 {Q_n}_{\bar{\theta}}
  (\sqrt{n} D {Q_n}_{\theta_0} ) \\
  \indist & N\left(0, (D^2 {Q_n}_{\theta_0})^{-1} V (D^2
    {Q_n}_{\theta_0})^{-1} \right).
\end{align*}

\subsection{With constraints}
Sometimes we have an extremum estimator with constraints. Suppose we
have a constrained extremum estimate of the form
\[ \hat{\theta} = \argmin_{\theta \in \Theta} Q_n(\theta) \text{
  s.t. } h_n(\theta) = 0. \]
As above, we will assume that $Q_n \inprob Q$ and $h_n \inprob h$. We
will let $\theta_0$ be the solution to 
\[ \theta_0 = \argmin_{\theta \in \Theta} Q(\theta) \text{
  s.t. } h(\theta) = 0. \]
We will assume that $\theta_0$ is a unique strict minimizer in the
interior of $\Theta$. As above, $\hat{\theta}$ solve the first order
condition:
\begin{align*}
  0 = & \begin{pmatrix} D{Q_n}_{\hat{\theta}} - \hat{\mu}^T
    D{h_n}_{\hat{\theta}} \\
    h_n(\hat{\theta}) 
  \end{pmatrix}
\end{align*}
If we expand around $\theta_0$ and $\mu_0$ we get
\begin{align*}
  0 = & \begin{pmatrix} D{Q_n}_{\theta_0} - \mu_0^T
    D{h_n}_{\theta_0} \\
    h_n(\theta_0) 
  \end{pmatrix} + 
  \begin{pmatrix} 
    D^2 {Q_n}_{\bar{\theta}} - \sum_{j=1}^m \bar{\mu}_j D^2
    {h_{jn}}_{\bar{\theta}} & D{h_n}_{\bar{\theta}}^T \\
    D{h_n}_{\bar{\theta}} & 0 
  \end{pmatrix} 
  \begin{pmatrix} 
    \hat{\theta} - \theta_0 \\
    \hat{\mu} - \mu_0 
  \end{pmatrix}
\end{align*}
Using the same sort of reasoning as in the previous section, we could
show that
\begin{align*}
  \sqrt{n}   \begin{pmatrix} 
    \hat{\theta} - \theta_0 \\
    \hat{\mu} - \mu_0 
  \end{pmatrix}
  \indist N(0,V)
\end{align*}
for some $V$. 

\section{A model of pollution}

Let's analyze a simple model of pollution. Suppose there is a
representative firm that takes one input, $\ell$, and produces two
outputs, a consumption good $c$, and pollution $z$. The firm allocates
its input to two activities: production and pollution reduction. If it
devotes $\alpha$ portion of $\ell$ to production, then it produces
$c=f_c(\alpha \ell)$ and $z = f_z(\alpha \ell) - g((1-\alpha)\ell)$ We will assume that
$f_c'>0$, $f_z'>0$ and $g'>0$. 

There is also a representative consumer with an
endowment of $L$ units of $\ell$ and preferences over consumption and
pollution represented by a utility function $u:\R^2 \to \R$. We will
assume that $\frac{\partial u}{\partial c}>0$ and $\frac{\partial
  u}{\partial z} < 0$. 

\subsection{Optimal allocation}

First let's solve the social planner's problem and find the optimal
allocation in this economy. The social planner maximizes utility
subject to the production function.
\begin{align*}
  \max_{c,\alpha,\ell,z} u(c,z) \text{ s.t. } & \ell = L \\
  & c = f_c(\alpha \ell) \\
  & z = f_z(\alpha \ell) - g((1-\alpha)\ell)  \\
  & 0 \leq \alpha \leq 1
\end{align*}
Now we could just substitute in the constraints and have an
unconstrained maximization over $\alpha$, but the multipliers are
going to be useful later, so we will work with them. The Lagrangian is
\begin{align*}
  L(c,a,\ell,z,\mu) = & u(c,z) - \mu_\ell (\ell -L) - \mu_c\left(c-
f_c(\alpha \ell) \right) - \\
& - \mu_z \left(f_z(\alpha \ell) - g((1-\alpha)
\ell) \right) - \mu_{\alpha_1}(\alpha-1) + \mu_{\alpha_0} \alpha 
\end{align*}
The first order conditions are
\begin{align}
  [c]: && \frac{\partial u}{\partial c} = &\mu_c \label{sc}\\
  [z]: && \frac{\partial u}{\partial z} = &\mu_z \label{sz}\\
  [\ell]: && -\mu_\ell + \mu_c f_c'(\alpha \ell) \alpha + \mu_z\left(
    f_z'(\alpha\ell) \alpha - g'((1-\alpha)\ell)(1-\alpha) \right) =& 0
  \label{sl} \\
  [\alpha]: &&  \mu_c f_c'(\alpha \ell) \ell + \mu_z
  \left(f_z'(\alpha\ell) \ell + g'((1-\alpha)\ell) \ell\right) -
  \mu_{\alpha_1} + \mu_{\alpha_0} = & 0 \label{sfa}
\end{align}
As well as the first three constraints. We will assume the last
constraint does not bind. If we combine (\ref{sc}),
(\ref{sz}), and (\ref{sfa}) we get
\begin{align}
  \frac{\partial u}{\partial c} f_c'(\alpha \ell)  + \frac{\partial
    u}{\partial z} \left(f_z'(\alpha\ell) + g'((1-\alpha) \ell)\right) = &
    0 \label{smarg}
  \end{align}
This characterizes the optimal choice of $\alpha^*$. Under reasonable
assumptions, it will be true that $\ell^* = L$.  Then, once we know
$\alpha^*$ we can find $c^*$ and $z^*$ from the
constraints. (\ref{smarg}) says that the marginal benefit of devoting
more resource to producing $c$ to be equal to the marginal disutility
of devoting fewer resources to preventing pollution. 

\subsection{Competitive equilibrium}

Let the price of $c$ be $p$ and $\ell$ be $w$. Pollution, $z$ is not
traded. The firm's problem is
\begin{align*}
  \max_{c,\alpha,\ell,z} p f_c(\alpha \ell) - w \ell \text{ s.t. } & 0
  \leq \alpha \leq 1 
\end{align*}
Let the multipliers be $\lambda_0$ and $\lambda_1$. The first order conditions are: 
\begin{align*} 
  [\ell]: & & p f_c'(\alpha \ell) \alpha - w = & 0 \\
  [\alpha]: & & p f_c'(\alpha \ell) \ell + \lambda_{0} -
  \lambda_{1} = & 0 \\ 
  [\lambda_{1}]: && \lambda_{1}(\alpha -1) = & 0 \\
  [\lambda_{0}]: && \lambda_{0}\alpha = & 0 
\end{align*}
Since $f_c'>0$ and $\alpha$ only enters the objective function, the
maximum must have $\alpha = 1$ and $\lambda_0 = 0$. 

The consumer's problem is
\begin{align*}
  \max_{c,\ell} u(c,z) \text{ s.t. } & pc \leq w \ell \\
  & \ell \leq L 
\end{align*}
The Lagrangian is
\[ L(c,\ell,\psi) = u(c,z) - \psi_c(pc - w\ell) - \psi_\ell (\ell -L) \]
The first order conditions are
\begin{align*}
  [c]&& \frac{\partial u}{\partial c} - p \psi_c = & 0 \\
  [\ell]&&  w \psi_c - \psi_\ell = & 0 \\
  [\psi_c]&& \psi_c(pc-w\ell) = & 0 \\
  [\psi_\ell] && \psi_\ell(\ell-L) = & 0.
\end{align*}
Since $u$ is increasing in $c$ and if $p$ and $w$ are positive, then
the first constraint must bind, so $\psi_c>0$. From the envelope
theorem, $\psi_c$ is the marginal value of increase $w\ell$. If this
is positive, then the consumer will want $\ell$ as large as possible,
so $\ell = L$. Then from the budget constraint, $\frac{p}{w} =
\frac{L}{c}$. Also, given the firm's problem has $\alpha=1$, it must
be that $c = f_c(L)$. Also, $z = f_z(L) - g(0)$. Given our assumption
that $f_z'>0$ and $g'>0$, $z$ is greater in this competitive
equilibrium than in the optimal allocation.

\subsection{Competitive equilibrium with taxes}

Suppose the production of pollution is taxed at rate $\tau$. Also,
assume that the government gives all tax revenue to the consumer as a
lump sum transfer. 
In this case, the firm's problem becomes
\begin{align*}
  \max_{c,\alpha,\ell,z} p f_c(\alpha \ell) - w \ell -
  \tau\left(f_z(\alpha \ell) - g((1-\alpha)\ell) \right) \text{ s.t. } & 0
  \leq \alpha \leq 1 
\end{align*}
Let the multipliers be $\lambda_0$ and $\lambda_1$. The first order
conditions are:  
\begin{align*} 
  [\ell]: & & p f_c'(\alpha \ell) \alpha - w - \tau \left( f_z'(\alpha
    \ell)\alpha - g((1-\alpha)\ell)(1-\alpha) \right) = &
  0  \\
  [\alpha]: & & p f_c'(\alpha \ell) \ell - \tau \left( f_z'(\alpha\ell)
    \ell + g'((1-\alpha)\ell) \ell \right) + \lambda_{0} - \lambda_{1} =
  & 0  \\ 
  [\lambda_{1}]: && \lambda_{1}(\alpha -1) = & 0 \\
  [\lambda_{0}]: && \lambda_{0}\alpha = & 0 
\end{align*}
Notice that (\ref{ftl}) is nearly the same as the social planner's
first order condition for $\ell$ (\ref{sl}), except the former has
$p$, $w$, and $\tau$ whereas the later has $\mu_c$, $\mu_\ell$, and
$\mu_z$. Suppose the government sets the tax rate to 
\[ \tau = -\mu_z = -
\frac{\partial u}{\partial z}(c^*,z^*) \].
If $p = \mu_c$ and $w=\mu_\ell$, then $c^*$ and $c^*$ solve the firm's
first order condition with taxes. To show this is an equilibrium, we
just need to verify that the consumer's first order conditions hold.
The consumer's problem is now 
\begin{align*}
  \max_{c,\ell} u(c,z) \text{ s.t. } & pc \leq w \ell + T \\
  & \ell \leq L 
\end{align*}
The first order conditions are
\begin{align}
  [c]&& \frac{\partial u}{\partial c} - p \psi_c = & 0 \label{cc} \\
  [\ell]&&  w \psi_c - \psi_\ell = & 0 \label{cl} \\
  [\psi_c]&& \psi_c(pc-w\ell - T) = & 0 \\
  [\psi_\ell] && \psi_\ell(\ell-L) = & 0.
\end{align}
From (\ref{sc}), $\frac{\partial u}{\partial c} = \mu_c = p $, so it
must be that $\psi_c = 1$.  Then from (\ref{cl}), $\psi_\ell = 1$ as
well. Finally, integrating the firm's first order condition for $\ell$
with respect to $\ell$, and using the fact that $\tau z = T$, we see
that the consumer's budget constraint must hold. 

Thus, by appropriately setting the pollution tax, we could achieve the
optimal allocation. This sort of result--that certain taxes can
achieve first (or sometimes second) best outcomes---appears quite
often in public finance. Also, this approach---finding the social
planner's solution, and showing that setting prices and taxes equal to
certain multipliers give a competitive equilibrium with taxes---is the
standard way of showing it.

This has gotten long enough, but a useful exercise might be to
consider a cap and trade program as well. Suppose the consumers are
given tradeable permits that allow up to $Z$ units of pollution. You
can show that if $Z = z^*$, then the price of the permits will end up
being $\mu_z$, and $p=\mu_c$ and $w=\mu_\ell$\footnote{Actually only
  relative prices are determined, and this is one possible set of
  prices. Nonetheless all possible prices led to the same
  allocation.}as above, so this is another way of decentralizing the
efficient allocation.  

\section{Demand theory}

Marshallian demand, Hicksian demand, Roy's identity, Slutsky equation,
etc. See Simon and Blume chapter 22. 




\end{document}
