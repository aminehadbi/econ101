\documentclass[compress]{beamer} 
\usepackage{amsmath}
\usepackage{amsfonts}
\usepackage{graphicx}
%\usepackage{epstopdf}
\usepackage{hyperref}
\usepackage{multirow}
\usepackage{verbatim}
%\usepackage[small,compact]{titlesec} 
%\usecolortheme{beaver}
\usetheme{Hannover}
%\usecolortheme{whale}

%\usepackage{pxfonts}
%\usepackage{isomath}
%\usepackage{mathpazo}
%\usepackage{arev} %     (Arev/Vera Sans)
%\usepackage{eulervm} %_   (Euler Math)
%\usepackage{fixmath} %  (Computer Modern)
%\usepackage{hvmath} %_   (HV-Math/Helvetica)
%\usepackage{tmmath} %_   (TM-Math/Times)
%\usepackage{tgheros}
%\usepackage{cmbright}
%\usepackage{ccfonts} \usepackage[T1]{fontenc}
%\usepackage[garamond]{mathdesign}
\usepackage{color}
\usepackage{ulem}

\setbeamertemplate{navigation symbols}{}
\AtBeginSection[] % Do nothing for \section*
{ \frame{\sectionpage} }

\newcommand{\argmax}{\operatornamewithlimits{arg\,max}}
\newcommand{\argmin}{\operatornamewithlimits{arg\,min}}
\def\inprobLOW{\rightarrow_p}
\def\inprobHIGH{\,{\buildrel p \over \rightarrow}\,} 
\def\inprob{\,{\inprobHIGH}\,} 
\def\indist{\,{\buildrel d \over \rightarrow}\,} 
\def\F{\mathbb{F}}
\def\R{\mathbb{R}}
\newcommand{\gmatrix}[1]{\begin{pmatrix} {#1}_{11} & \cdots &
    {#1}_{1n} \\ \vdots & \ddots & \vdots \\ {#1}_{m1} & \cdots &
    {#1}_{mn} \end{pmatrix}}
\newcommand{\iprod}[2]{\left\langle {#1} , {#2} \right\rangle}
\newcommand{\norm}[1]{\left\Vert {#1} \right\Vert}
\newcommand{\abs}[1]{\left\vert {#1} \right\vert}
\renewcommand{\det}{\mathrm{det}}
\newcommand{\rank}{\mathrm{rank}}
\newcommand{\spn}{\mathrm{span}}
\newcommand{\row}{\mathrm{Row}}
\newcommand{\col}{\mathrm{Col}}
\renewcommand{\dim}{\mathrm{dim}}
\newcommand{\prefeq}{\succeq}
\newcommand{\pref}{\succ}
\newcommand{\seq}[1]{\{{#1}_n \}_{n=1}^\infty }
\renewcommand{\to}{{\rightarrow}}


\title{Unconstrained optimization}
\author{Paul Schrimpf}
\institute{UBC \\ Economics 526}
\date{\today}

\begin{document}

\frame{\titlepage}
%\setcounter{tocdepth}{2}

\begin{frame}
  \tableofcontents  
\end{frame}

\section{Notation and definitions}

\begin{frame}
  \frametitle{Notation and definitions}
  \begin{itemize} 
  \item Unconstrained optimization problem
    \[ \max_{x\in U} F(x) \]
    \begin{itemize}
    \item $x \in U \subseteq \R^n$ 
    \item $F: U \to \R$.
    \end{itemize}
  \end{itemize}
  \begin{definition}\label{d:max}
    $F^* = \max_{x \in U} F(x)$ is the \textbf{maximum} of $F$ on $U$ if
    $F(x) \leq F^*$ for all $x \in U$ and $F(x^*) = F^*$ for some $x^*
    \in U$
  \end{definition}
  \begin{definition} \label{d:maxer}
    $x^* \in U$ is a \textbf{maximizer} of $F$ on $U$ if $F(x*)
    = \max_{x \in U} 
    F(x)$. The set of all maximizers is denoted $\argmax_{x \in U}
    F(x)$.
  \end{definition}
\end{frame}

\begin{frame}
  \begin{definition}\label{d:smaxer}
    $x^* \in U$ is a \textbf{strict maximizer} of $F$ on $U$ if $F(x*) > F(x)
    $ for all $x \in U$ with $x \neq x*$. 
  \end{definition}
  \begin{definition}\label{d:lmax}
    $F$ has a \textbf{local maximum} at $x$ if $\exists \delta > 0$ such
    that $F(y) \leq F(x)$ for all $y \in  N_\delta(x) \cap U$. Each such
    $x$ is called a \textbf{local maximizer} of $F$. If $F(y) < F(x)$
    for all $y \neq x$, $y \in N_\delta(x) \cap U$, then we say $F$ has
    a \textbf{strict local maximum} at $x$.
  \end{definition}
\end{frame}

\begin{frame}
  \begin{example}
    Here are some examples of functions from $\R \to \R$ and their
    maxima and minima.
    \begin{enumerate}
    \item $F(x) = x^2$ is minimized at $x = 0$ with minimum $0$. 
    \item $F(x) = c$ has minimum and maximum $c$. Any $x$ is a
      maximizer.
    \item $F(x) = \cos(x)$ has maximum $1$ and minimum $-1$. $2\pi n$
      for $n \in \mathbb{Z}$ is a maximizer. 
    \item $F(x) = \cos(x) + x/2$ has no global maximizer or minimizer,
      but has many local ones.
    \end{enumerate}
  \end{example}
\end{frame}

\begin{frame}\frametitle{Supremum vs maximum}
  \begin{itemize}
  \item The supremum of $F$ on $U$ is $\sup_{x \in U} F(x) = S$ if $S
    \geq F(x)$ $\forall x \in U$, and for any $S'<S$, $\exists x$
    s.t. $F(x) > S'$. 
  \item When $\max_{x \in U} F(x)$ exists, $\sup_{x \in U} F(x) =
    \max_{x \in U} F(x)$
  \item Sometimes maximum does not exist, but supremum does e.g.\
    $F(x) = x$, $U = (0,1)$
  \item The infinum is to the minimum as the supremum is to the
    maximum 
  \end{itemize}
\end{frame}

\section{First order conditions}
\begin{frame}\frametitle{First order conditions}
  \begin{theorem}
    Let $U \subseteq \R^n$, $F: U \to \R$,  and suppose $F$ has a local
    maximum or minimum at $x$, $F$ is differentiable at $x$, and $x \in
    \mathrm{interior}(U)$ . Then $DF_x = 0$.  
  \end{theorem}
  
  \begin{definition}
    Any point such that $DF_x = 0$ is call a \textbf{critical point} of
    $F$. 
  \end{definition}
  \begin{itemize}
  \item $F$ cannot have local minima or maxima (=local extrema) at
    interior non-critical points. 
  \item $F$ might have a local extrema its critical
    points, but it does not have to. 
  \item e.g.\ $F(x) = x^3$,  $F(x) = x_1^2 - x_2^2$
  \end{itemize}
\end{frame}

\section{Second order conditions}

\begin{frame}\frametitle{Second order conditions}
  \begin{itemize}
  \item $F:\R^n \to \R$, $x^*$ is a critical
    point
  \item $DF_{x^*} = 0$
  \item To check if $x^*$ is a local maximum,
    need to look at $F(x^* + h)$ for small $h$
  \item Taylor expansion:
    \[ F(x^*+h) = F(x^*) + DF_{x^*} h + h^T D^2 F_{x*} h + r(x^*,h) \]
  \item $D^2 F_{x^*}$ is the matrix of second derivatives, called the
    \textbf{Hessian} of $F$.    
  \item $r$ is small, so ignore it, then $F(x^*+h) < F(x^*)$ if
    $h^T D^2 F_{x^*} h < 0$ for all $h$
  \end{itemize}
\end{frame}

\begin{frame}
  \begin{definition}
    Let $A$ be a symmetric matrix, then $A$ is
    \begin{itemize}
    \item \textbf{Negative definite} if $x^T A x < 0$ for all $x \neq 0$
    \item \textbf{Negative semi-definite} if $x^T A x \leq 0$ for all $x \neq 0$
    \item \textbf{Positive definite} if $x^T A x > 0$ for all $x \neq 0$
    \item \textbf{Positive semi-definite} if $x^T A x \geq 0$ for all $x
      \neq 0$
    \item \textbf{Indefinite} if $\exists$ $x_1$ s.t. $x_1^T A x_1 > 0$ and
      some other $x_2$ such that $x_2^T A x_2 < 0$.
    \end{itemize}  
  \end{definition}
\end{frame}

\begin{frame}
  \frametitle{Second order condition}
  \begin{theorem}\label{thm:soc}
    Let $F: U \to \R$ be twice continuously differentiable on $U$ and
    let $x^*$ be a critical point in the interior of $U$. If
    \begin{enumerate}
    \item\label{c:smax} The Hessian, $D^2 F_{x^*}$ is negative definite,
      then $x^*$ is a strict local maximizer.  
    \item The Hessian, $D^2 F_{x^*}$ is positive definite, then $x^*$ is
      a strict local minimizer. 
    \item The Hessian, $D^2 F_{x^*}$ is indefinite, $x^*$ is neither a
      local min nor a local max.
    \item\label{c:ambig} The Hessian is positive or negative
      semi-definite, then $x^*$ could be a local maximum, minimum, or
      neither. 
    \end{enumerate}
  \end{theorem}
\end{frame}


\begin{proof}
  \begin{align*}
    F(x^*+h) - F(x^*) = & h^T D^2 F_{x^*} h + r(x^*,h) \\
    = & h^T D^2 F_{x^*} h \pm h^T h \frac{r(x^*,h)}{\norm{h}^2} \\
    = & 
    h^T \left( D^2 F_{x^*}\pm \frac{r(x^*,h)}{\norm{h}^2} \right) h \\
    \leq & h^T \left( D^2 F_{x^*}\pm \epsilon \right) h \\
    \leq & t^2 (q^T D^2 F_{x*} q) + \epsilon t^2 \text{ where} tq=h, \norm{q}=1
  \end{align*}
  Pick $\epsilon < \inf_{\norm{q} =1} |q^T D^2  F_{x*} q|$. The set
  $\{q: \norm{q} = 1\}$ is compact so there is 
  some $q$ that achieves this minimum, and $q^T D^2 F_{x^*} q < 0$, so
  $\epsilon >0$.
  \begin{align*}
    F(x^*+h) - F(x^*) < & t^2\left((q^T D^2 F_{x*} q) + |(q^T D^2 F_{x*}
      q)|\right) \\
    < & 0
  \end{align*}
\end{proof}

\begin{frame}\begin{example}
    $F:\R \to \R$, $F(x) = x^4$. The first order condition is $4x^3 =
    0$, so $x^* = 0$ is the only critical point. The Hessian is $F''(x)
    = 12x^2 = 0$ at $x^*$. However, $x^4$ has a strict local minimum at
    $0$.
  \end{example}
\end{frame}  

\begin{frame}
  \begin{example}
    $F: \R^2 \to \R$, $F(x_1,x_2) = -x_1^2$. The first order condition is
    $DF_x = (-2x_1,0) = 0$, so the $x_1^*=0$, $x_2^* \in \R$ are all
    critical points. The Hessian is
    \[ D^2 F_x = \begin{pmatrix} -2 & 0 \\
      0 & 0 \end{pmatrix} \]
    This is negative semi-definite because $h^T D^2 F_x h = -2 h_1^2
    \leq 0$. Also, graphing the function would make it clear that
    $x_1^*=0$, $x_2^* \in \R$ are all (non-strict) local maxima. 
  \end{example}
\end{frame}

\begin{frame}
  \begin{example}
    $F: \R^2 \to \R$, $F(x_1,x_2) = -x_1^2 + x_2^4$. The first order
    condition is $DF_x = (-2x_1,4 x_2^3) = 0$, so the $x^* = (0,0)$ is a
    critical point. The Hessian is  
    \[ D^2 F_x = \begin{pmatrix} -2 & 0 \\
      0 & 12 x_2^2 \end{pmatrix} \]
    This is negative semi-definite at $0$ because $h^T D^2 F_0 h = -2 h_1^2
    \leq 0$. However, $0$ is not a local maximum because $F(0,x_2)>
    F(0,0)$ for any $x_2 \neq 0$. $0$ is also not a local minimum
    because $F(x_1,0) < F(0,0)$ for all $x_1 \neq 0$.
  \end{example}
\end{frame}

\begin{frame}
  \begin{theorem}
    Let $F: U \to \R$ be twice continuously differentiable on $U$ and
    let $x^*$ in the interior of $U$ be a local maximizer (or minimizer) of
    $F$. Then $DF_{x^*} = 0$ and $D^2 F_{x^*}$ is negative (or positive)
    semi-definite. 
  \end{theorem}
  \begin{proof}
    Using the same notation and setup as in the proof of theorem
    \ref{thm:soc}, 
    \begin{align*}
      0 > F(x^* + tq ) - F(x^*) = & t^2 q^T D^2 F_{x^*} q + r(x,tq)  \\
      0 >  & t^2\left(q^T D^2 F_{x^*} q + \frac{r(x,tq)}{t^2} \right) 
    \end{align*}
    Because $\lim_{t \to 0} \frac{r(x,tq)}{t^2} = 0$, for any $\epsilon
    >0$ $\exists \delta > 0$ such that if $|t|< \delta$, then
    \begin{align*}
      0 > & t^2\left(q^T D^2 F_{x^*} q + \frac{r(x,tq)}{t^2} \right) \geq  t^2
      \left(q^T D^2 F_{x^*} q - \epsilon \right) \\
      t^2 \epsilon >& t^2 q^T D^2 F_{x^*} q 
    \end{align*}
    This is only possible for all $\epsilon >0$ if $q^T D^2 F_{x^*}
    q\leq 0$. 
  \end{proof}
\end{frame}

\section{Definite matrices}

\begin{frame}
  \frametitle{Definite matrices}
  \begin{itemize}
  \item Second order condition depends on checking whether $x^T A x$
    is always positive or negative for symmetric $A$
  \item $1$ by $1$: $x^T A x = ax^2$, negative definite if $a<0$
  \item $2$ by $2$:
    \begin{align*}
      x^T A x = & \begin{pmatrix} x_1 \\ x_2 \end{pmatrix}^T 
      \begin{pmatrix}
        a & b \\ b & c 
      \end{pmatrix}
      \begin{pmatrix} x_1 \\ x_2 \end{pmatrix}
      \\ 
      = & a x_1^2 + 2b x_1 x_2 + c x_2^2
      = & a\left(x_1 + \frac{2b}{a} x_1 x_2 + \frac{b^2}{a^2}
        x_2^2\right) - \frac{b^2}{a} x_2^2 + c x_2^2 \\
      = &a\left(x_1 + \frac{b}{a} x_2\right)^2 + \frac{ac-b^2}{a} x_2^2 
    \end{align*}
    Thus $x^T A x < 0$ for all $x \neq 0$ if $a < 0$ and
    $\det A = ac - b^2 > 0$
  \end{itemize}
\end{frame}

\begin{frame}
  \begin{definition}
    Let $A$ by an $n$ by $n$ matrix. The $k$ by $k$ submatrix 
    \[ A_k = \begin{pmatrix} a_{11} & \cdots a_{1k} \\
      \vdots & & \vdots \\
      a_{k1} & \cdots & a_{kk} \end{pmatrix} \]
    is the \textbf{$k$th leading principal submatrix} of $A$. The
    determinant of $A_k$ is the $k$th order \textbf{leading principal
      minor} of $A$.
  \end{definition}
\end{frame}  

\begin{frame}
  \begin{theorem}
    Let $A$ be an $n$ by $n$ symmetric matrix. Then 
    \begin{enumerate}
    \item $A$ is positive definite if and only if all $n$ of its leading
      principal minors are strictly positive.
    \item $A$ is positive semi-definite if and only if all $n$ of its leading
      principal minors are weakly positive.
    \item $A$ is negative definite if and only if all $n$ of its leading
      principal minors alternate in sign as follows: $\det A_1 < 0$,
      $\det A_2 > 0$, $\det A_3 < 0$, etc.
    \item $A$ is negative semi-definite if and only if all $n$ of its
      leading principal minors weakly alternate in sign as follows:
      $\det A_1 \leq 0$, $\det A_2 \geq 0$, $\det A_3 \leq 0$, etc
    \item $A$ is indefinite if and only if none of the five above cases
      hold, and $\det A_k \neq 0$ for at least one $k$.
    \end{enumerate}
  \end{theorem}
\end{frame}

\begin{frame}\frametitle{Positive definite}
  \includegraphics[width=\linewidth]{pd} 
\end{frame}

\begin{frame}\frametitle{Negative definite}
  \includegraphics[width=\linewidth]{nd} 
\end{frame}

\begin{frame}\frametitle{Positive semi-definite}
  \includegraphics[width=\linewidth]{psd} 
\end{frame}

\begin{frame}\frametitle{Negative semi-definite}
  \includegraphics[width=\linewidth]{nsd} 
\end{frame}

\begin{frame}\frametitle{Indefinite}
  \includegraphics[width=\linewidth]{id} 
\end{frame}

\subsection{Eigenvectors and eigenvalues}

\begin{frame}\frametitle{Eigenvalues and eigenvectors}
  \begin{definition}
    If $A$ is an $n$ by $n$ matrix, $\lambda$ is a scalar, $v \in \R^n$
    with $\norm{v} = 1$, 
    and 
    \[ Av = \lambda v \]
    then $\lambda$ is a \textbf{eigenvalue} of $A$ and $v$ is an
    \textbf{eigenvector}. 
  \end{definition}
\end{frame}

\begin{frame}
  \begin{lemma}
    Let $A$ be an $n$ by $n$ matrix and $\lambda$ a scalar. Each of the
    following are equivalent.
    \begin{enumerate}
    \item\label{eig1} $\lambda$ is a eigenvalue of $A$.
    \item\label{eig2} $A - \lambda I$ is singular.
    \item\label{eig3} $\det(A - \lambda I ) = 0$.
    \end{enumerate}
  \end{lemma}
  \begin{proof}
    We know that (\ref{eig2}) and (\ref{eig3}) from our results on
    systems of linear equations and matrices. Also, if $A - \lambda I$
    is singular, then the null space of $A - \lambda I$ contains
    non-zero vectors. Choose $v \in \mathcal{N}(A-\lambda I)$ such that
    $v \neq 0$. Then $(A-\lambda I) v = 0$, so $A(v/\norm{v}) =
    \lambda(v/\norm{v})$ as in the definition of eigenvalues. 
  \end{proof}
\end{frame}

\begin{frame}
  \begin{itemize}
  \item $\chi_A:\R\to\R$ defined by $\chi_A(x) = \det(A - x I)$
    is called the \textbf{characteristic polynomial} of $A$
    \begin{itemize} 
    \item Polynomial of order $n$.
    \item $n$ roots, possibly not real and not distinct
    \item For symmetric nonsingular matrices, the roots are real and
      distinct
    \end{itemize}
  \end{itemize}
  \begin{lemma}
    Let $A$ be an $n$ by $n$ matrix with $n$ distinct real eigenvalues
    then the eigenvectors of $A$ are linearly independent.
  \end{lemma}
\end{frame}

\begin{frame}
  \begin{itemize}
  \item So if $A$ has $n$ distinct eigenvalues, it has $n$ linearly
    independent eigenvectors.
  \item Can make the eigenvectors orthogonal
  \item Make matrix $V$ with columns composed of the orthogonal
    eigenvectors, then $V$ is orthogonal, so $V^{-1} = V^T$. 
  \item Definition of eigenvalues implies
    \[ A V = V \Lambda \]
    where $\Lambda$ is diagonal matrix of eigenvalues.
  \end{itemize}
  \begin{theorem}[Eigendecomposition]
    Let $A$ be an $n$ by $n$ non-singular symmetric matrix, then $A$ has
    $n$ distinct real eigenvalues
    \[ A = V \Lambda V^T \]
    where $\Lambda$ is the diagonal matrix consisting of the eigenvalues
    of $A$ and the columns of $V$ are the eigenvectors of $A$, and $V$
    is an orthogonal matrix.  
  \end{theorem}
\end{frame}

\begin{frame}
  \begin{theorem}
    If $A$ is an $n$ by $n$ symmetric non-singular matrix with
    eigenvalues $\lambda_1, ..., \lambda_n$, then
    \begin{enumerate}
    \item $\lambda_i > 0$ for all $i$, iff $A$ is positive definite,
    \item $\lambda_i \geq 0$ for all $i$, iff $A$ is positive semi-definite,
    \item $\lambda_i < 0$ for all $i$, iff $A$ is negative definite,    
    \item $\lambda_i \leq 0$ for all $i$, iff $A$ is negative
      semi-definite,
    \item if some $\lambda_i > 0$ and some $\lambda_j < 0$, then $A$ is
      indefinite.
    \end{enumerate}
  \end{theorem}
\end{frame}
\begin{frame}
  \begin{proof}
    Let $A = V \Lambda V^T$ be the eigendecomposition of $A$. Let $x
    \neq 0 \in \R^n$. Then $x^T V = z^T \neq 0$ because $V$ is
    nonsingular. Also, 
    \[ x^T A x = x^T V \Lambda V^T x = z^T \Lambda z. \]
    Since $\Lambda$ is diagonal we have
    \[ z^T \Lambda z = z_1^2 \lambda_1 + ... + z_n^2 \lambda_n \]
    Thus, $x^T A  x= z^T \Lambda z > 0$ for all $x$ iff $\lambda_i>0$
    for each $i$. The other parts of the theorem follow similarly. 
  \end{proof}
\end{frame}

\section{Global maximum and minimum}

\begin{frame}
  \frametitle{Global maximum and minimum}
  \begin{itemize}
  \item First and second order conditions give nice way of finding
    local maxima
  \item Global maximum: find all interior local maxima and compare
    them with each other and the value of $F$ on the boundary
  \item Tedious if many local maxima or boundary of $U$ is complicated 
  \item Generally, no nice necessary and sufficient condition for
    global maximum
  \item One sufficient condition is that concave functions on convex
    sets have a unique global maximum
  \end{itemize}
\end{frame}

\begin{frame}
  \begin{definition}
    Let $f:U \to \R$. $f$ is \textbf{convex} if for all $x,y \in U$ with
    $\ell(x,y) \subseteq U$ we have $f(tx + (1-t)y) \leq tf(x) + (1-t)
    f(y)$ for all $t \in [0,1]$.
  \end{definition}
  \begin{definition}
    Let $f:U \to \R$. $f$ is \textbf{concave} if for all $x,y \in U$ with
    $\ell(x,y) \subseteq U$ we have $f(tx + (1-t)y) \geq tf(x) + (1-t)
    f(y)$ for all $t \in [0,1]$.
  \end{definition}
\end{frame}

\begin{frame}
  \begin{theorem}
    Let $F: U \to \R$ be twice continuously differentiable and $U
    \subseteq \R^n$ be convex. Then,
    \begin{enumerate}
    \item The following three conditions are equivalent:
      \begin{enumerate}
      \item $F$ is a concave function on $U$
      \item $F(y) - F(x) \leq DF_x(y-x)$ for all $x,y \in U$,
      \item $D^2 F_x$ is negative semi-definite for all $x, y \in U$
      \end{enumerate}
    \item If $F$ is a concave function on $U$ and $DF_{x^*} = 0$ for
      some $x^* \in U$, then
      $x^*$ is the global maximizer of $F$ on $U$.
    \end{enumerate}
  \end{theorem}
\end{frame}

\section{Applications}

\subsection{Profit maximization}

\begin{frame}
  \frametitle{Application: profit maximization}
  
\end{frame}

\subsubsection{Competitive multi-product firm}
Suppose a firm has produces $k$ goods using $n$ inputs with production
function $f: \R^n \to \R^k$. The prices of the goods are $p$, and the
prices of the inputs are $w$, so that the firm's profits are 
\[ \Pi(x) =  p^T f(x) -  w^Tx. \]
The firm chooses $x$ to maximize profits.
\[ \max_x p^T f(x) - w^T x \]
The first order condition is
\[ p^T Df_{x^*} - w = 0. \]
or without using matrices,
\[ \sum_{j=1}^k p_j \frac{\partial f_j}{\partial x_i}(x^*) = w_i \]
for $i=1,..., n$. The second order condition is that 
\[ D[p^T Df]_{x^*} = \begin{pmatrix} \sum_{j=1}^k p_j \frac{\partial^2
    f_j}{\partial x_1^2}(x^*) & \cdots & \sum_{j=1}^k p_j \frac{\partial^2
    f_j}{\partial x_1\partial x_n}(x^*)  \\ \vdots & & \vdots \\
  \sum_{j=1}^k p_j \frac{\partial^2
    f_j}{\partial x_1\partial x_n}(x^*) & \cdots & \sum_{j=1}^k p_j
  \frac{\partial^2 f_j}{\partial x_n^2}(x^*) \end{pmatrix} 
\]
must be negative semidefinite. 

\subsubsection{Multi-product monopolist}
Consider the same setup as before, but now with a monopolist who
recognizes that prices depend on output. Let the inverse demand
function be $P(q)$ where $P:\R^k \to \R^k$. Now the firm's problem
is 
\[ \max_x P(f(x))^T f(x) - w^T x \]
The first order condition is
\[ Df_{x^*}^T DP_{f(x^*)}^T f(x^*) + P(f(x^*)) Df_{x^*} - w = 0 \]
or without matrices,
\[ \sum_{j=1}^k \left( \sum_{l=1}^k \frac{\partial P_j}{\partial q_l}(f(x^*))
  \frac{\partial f_l}{\partial x_i}(x^*) f_l(x^*) + 
  P_j(f(x^*)) \frac{\partial f_j}{\partial x_i}(x^*) f_j(x^*) \right)
= w_i \]
We can get something a bit more interpretable by writing this in terms
of elasticities. Recall that the elasticity of demand for good $l$
with respect to the price of good $j$ is $(\epsilon^l_j)^{-1}=
\frac{\partial P_j}{\partial q_l} \frac{q_l}{P_l}$. Then,
\begin{align*}
  \sum_{j=1}^k P_j(f(x^*)) \left(\sum_{l=1}^k (\epsilon^l_j)^{-1} \frac{\partial
      f_l}{\partial x_i}(x^*) +  \frac{\partial f_j}{\partial x_i}(x^*) f_j(x^*) \right)
  = & w_i \\
  \sum_{j=1}^k P_j(f(x^*)) \left[\frac{\partial f_j}{\partial x_i}(x^*)
    \left(1 + (\epsilon_j^j)^{-1} \right) + \sum_{l \neq j}
  (\epsilon_j^l)^{-1} \frac{\partial f_l}{\partial x_i}(x^*) \right] = & w_i 
\end{align*}
There is a way to compare the price and quantity produced by the
monopolist to the competitive firm. There are also things that can be
said about the comparison between a single product monopolist ($k=1$)
vs a multi-product monopolist. It might be interesting to derive some
of these results. To begin with let $k = 1$ and compare the monopolist
to the competitive firm. Under some reasonable assumptions on $f$ and
$P$, you can show that $x^m < x^c$ where $x^m$ is $x^*$ for the
monopolist and $x^c$ is $x^*$ for the competitive firm. You can also
show that $p^m>p^c$ and that $|\epsilon^1_1| < 1$ for the
monopolist. (Although I left it out of the notation above, $\epsilon$
depends on $x$). 

\end{document}
