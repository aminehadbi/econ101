\documentclass[12pt,reqno]{amsart}

\usepackage{amsmath}
\usepackage{amsfonts}
\usepackage{graphicx}
%\usepackage{epstopdf}
\usepackage{hyperref}
\usepackage[left=1in,right=1in,top=0.9in,bottom=0.9in]{geometry}
\usepackage{multirow}
\usepackage{verbatim}
\usepackage{fancyhdr}
%\usepackage[small,compact]{titlesec} 

%\usepackage{pxfonts}
%\usepackage{isomath}
\usepackage{mathpazo}
%\usepackage{arev} %     (Arev/Vera Sans)
%\usepackage{eulervm} %_   (Euler Math)
%\usepackage{fixmath} %  (Computer Modern)
%\usepackage{hvmath} %_   (HV-Math/Helvetica)
%\usepackage{tmmath} %_   (TM-Math/Times)
%\usepackage{cmbright}
%\usepackage{ccfonts} \usepackage[T1]{fontenc}
%\usepackage[garamond]{mathdesign}
\usepackage{color}
\usepackage[normalem]{ulem}

\newtheorem{theorem}{Theorem}[section]
\newtheorem{conjecture}{Conjecture}[section]
\newtheorem{corollary}{Corollary}[section]
\newtheorem{lemma}{Lemma}[section]
\newtheorem{proposition}{Proposition}[section]
\theoremstyle{definition}
\newtheorem{assumption}{}[section]
%\renewcommand{\theassumption}{C\arabic{assumption}}
\newtheorem{definition}{Definition}[section]

\newtheorem{step}{Step}[section]
\newtheorem{remark}{Comment}[section]
\newtheorem{example}{Example}[section]
\newtheorem*{example*}{Example}

\linespread{1.1}

\pagestyle{fancy}
%\renewcommand{\sectionmark}[1]{\markright{#1}{}}
\fancyhead{}
\fancyfoot{} 
%\fancyhead[LE,LO]{\tiny{\thepage}}
\fancyhead[CE,CO]{\tiny{\rightmark}}
\fancyfoot[C]{\small{\thepage}}
\renewcommand{\headrulewidth}{0pt}
\renewcommand{\footrulewidth}{0pt}

\fancypagestyle{plain}{%
\fancyhf{} % clear all header and footer fields
\fancyfoot[C]{\small{\thepage}} % except the center
\renewcommand{\headrulewidth}{0pt}
\renewcommand{\footrulewidth}{0pt}}

\makeatletter
\renewcommand{\@maketitle}{
  \null 
  \begin{center}%
    \rule{\linewidth}{1pt} 
    {\Large \textbf{\textsc{\@title}}} \par
    {\normalsize \textsc{Paul Schrimpf}} \par
    {\normalsize \textsc{\@date}} \par
    {\small \textsc{University of British Columbia}} \par
    {\small \textsc{Economics 526}} \par
    \rule{\linewidth}{1pt} 
  \end{center}%
  \par \vskip 0.9em
}
\makeatother

\newcommand{\argmax}{\operatornamewithlimits{arg\,max}}
\newcommand{\argmin}{\operatornamewithlimits{arg\,min}}
\def\inprobLOW{\rightarrow_p}
\def\inprobHIGH{\,{\buildrel p \over \rightarrow}\,} 
\def\inprob{\,{\inprobHIGH}\,} 
\def\indist{\,{\buildrel d \over \rightarrow}\,} 
\def\F{\mathbb{F}}
\def\R{\mathbb{R}}
\newcommand{\gmatrix}[1]{\begin{pmatrix} {#1}_{11} & \cdots &
    {#1}_{1n} \\ \vdots & \ddots & \vdots \\ {#1}_{m1} & \cdots &
    {#1}_{mn} \end{pmatrix}}
\newcommand{\iprod}[2]{\left\langle {#1} , {#2} \right\rangle}
\newcommand{\norm}[1]{\left\Vert {#1} \right\Vert}
\newcommand{\abs}[1]{\left\vert {#1} \right\vert}
\renewcommand{\det}{\mathrm{det}}
\newcommand{\rank}{\mathrm{rank}}
\newcommand{\spn}{\mathrm{span}}
\newcommand{\row}{\mathrm{Row}}
\newcommand{\col}{\mathrm{Col}}
\renewcommand{\dim}{\mathrm{dim}}
\newcommand{\prefeq}{\succeq}
\newcommand{\pref}{\succ}
\newcommand{\seq}[1]{\{{#1}_n \}_{n=1}^\infty }
\renewcommand{\to}{{\rightarrow}}


\title{Unconstrained optimization}
\date{\today}

\begin{document}

\maketitle

Today's lecture is about unconstrained optimization. If you're
following along in the syllabus, you'll notice that we've skipped the
fourth topic, eigenvalues and definite matrices. We will cover these
things as part of our study of optimization. 

\section{Notation and definitions}
An optimization problem refers to finding the maximum or minimum of a
function, perhaps subject to some constraints. In economics, the most
common optimization problems are utility maximization and profit
maximization. Because of this, we will state most of our definitions
and results for maximization problems. Of course, we could just as
well state each definition and result for a minimization problem by
reversing the sign of most inequalities.

In this lecture we will be interested in unconstrained optimization
problems such as
\[ \max_{x\in U} F(x) \]
where $x \in U \subseteq \R^n$ and $F: U \to \R$. If $F^* =
\max_{x\in U} F(x)$, we
mean that $F(x) \leq F^*$ for all $x \in U$ and $F(x^*) = F^*$ for
some $x^* \in U$. 
\begin{definition}\label{d:max}
  $F^* = \max_{x \in U} F(x)$ is the \textbf{maximum} of $F$ on $U$ if
  $F(x) \leq F^*$ for all $x \in U$ and $F(x^*) = F^*$ for some $x^*
  \in U$
\end{definition}
There may be more than one such $x^*$. We denote
the set of all $x^*$ such that $F(x^*) = F^*$ by $\argmax_{x \in U} F(x)$ and
might write $x^* \in \argmax_{x \in U} F(x)$, or, if we know there is only one
such, $x^*$, we sometimes write $x^* = \argmax_{x \in U} F(x)$.
\begin{definition} \label{d:maxer}
  $x^* \in U$ is a \textbf{maximizer} of $F$ on $U$ if $F(x*)
  = \max_{x \in U} 
  F(x)$. The set of all maximizers is denoted $\argmax_{x \in U}
  F(x)$.
\end{definition}
\begin{definition}\label{d:smaxer}
  $x^* \in U$ is a \textbf{strict maximizer} of $F$ on $U$ if $F(x*) > F(x)
  $ for all $x \in U$ with $x \neq x*$. 
\end{definition}
Recall the definition of a local maximum from lecture 8.
\begin{definition}\label{d:lmax}
  $F$ has a \textbf{local maximum} at $x$ if $\exists \delta > 0$ such
  that $F(y) \leq F(x)$ for all $y \in  N_\delta(x) \cap U$. Each such
  $x$ is called a \textbf{local maximizer} of $F$. If $F(y) < F(x)$
  for all $y \neq x$, $y \in N_\delta(x) \cap U$, then we say $F$ has
  a \textbf{strict local maximum} at $x$.
\end{definition}
When we want to be explicit about the distinction between local
maximum and the maximum in definition \ref{d:max}, we refer to the
later as the global maximum.  

\begin{example}
  Here are some examples of functions from $\R \to \R$ and their
  maxima and minima.
  \begin{enumerate}
  \item $F(x) = x^2$ is minimized at $x = 0$ with minimum $0$. 
  \item $F(x) = c$ has minimum and maximum $c$. Any $x$ is a
    maximizer.
  \item $F(x) = \cos(x)$ has maximum $1$ and minimum $-1$. $2\pi n$
    for $n \in \mathbb{Z}$ is a maximizer. 
  \item $F(x) = \cos(x) + x/2$ has no global maximizer or minimizer,
    but has many local ones.
  \end{enumerate}
\end{example}



\section{First order conditions}
In lecture 8, we proved that if $F$ has a local maximum at $x$, then
$DF_x = 0$. We restate that theorem and proof here.
\begin{theorem}
  Let $U \subseteq \R^n$, $F: U \to \R$,  and suppose $F$ has a local
  maximum or minimum at $x$, $F$ is differentiable at $x$, and $x \in
  \mathrm{interior}(U)$ . Then $DF_x = 0$.  
\end{theorem}
\begin{proof}
  We will write the proof for when $F$ has a local maximum at $x$. The
  exact same reasoning works when $F$ has a local minimum. 

  Since $x$ is in the interior of $U$, we can choose $\delta>0$ such
  that $N_\delta(x) \subset U$. Since $x$ is a local maximum we can
  also choose $x$ such that $F(x) \geq F(y)$ for all $y \in
  N_\delta(x)$.  Since $F$ is differentiable, we can write
  \[ \frac{F(x+h) - F(x)}{\norm{h}} =\frac{ DF_x h +
    r(x,h)}{\norm{h}} \] where $\lim_{h \to 0}
  \frac{|r(x,h)|}{\norm{h}} = 0$. Let $h = t v$ for some $v \in \R^n$
  with $\norm{v} =1$ and $t \in \R$. If $D F_x v > 0$, then for $t>0$
  small enough, we would have $\frac{F(x+tv) - F(x)}{|t|} = D
  F_x v + \frac{r(x,tv)}{|t|} > D
  F_x v / 2 > 0$ and $F(x+tv)> F(x)$ in contradiction to $x$ being a
  local maximum. Similarly, if $D F_v v < 0$ then for $t<0$ and small,
  we would have $\frac{F(x+tv) - F(x)}{|t|} = -DF_x v +
  \frac{r(x,tv)}{|t|} > -D f_x v / 2 > 0$ and $F(x+tv)> F(x)$. Thus,
  it must be that $D F_x v = 0$ for all $v$, i.e. $D F_x = 0$.  
\end{proof}

The first order condition is the fact that $DF_x = 0$ is a necessary
condition for $x$ to be a local maximizer or minimizer of $F$. 
\begin{definition}
  Any point such that $DF_x = 0$ is call a \textbf{critical point} of
  $F$. 
\end{definition}
If $F$ is differentiable, $F$ cannot have local minima or maxima
(=local extrema) at non-critical points. $F$ might have a local
extrema its critical points, but it does not have to. Consider $F(x) =
x^3$ $F'(0) = 0$, but $0$ is not a local maximizer or minimizer of
$F$. Similarly, if $F:\R^2 \to \R$, $F(x) = x_1^2 - x_2^2$, then $DF_0
= 0$, but $0$ is not a local minimum or maximum of $F$.

\section{Second order conditions}

To determine whether a given critical point is a local minimum or
maximum or neither we can look at the second derivative of the
function. Let $F:\R^n \to \R$ and suppose $x^*$ is a critical
point. Then $DF_{x^*} = 0$. To see if $x^*$ is a local maximum, we
need to look at $F(x^* + h)$ for small $h$. If $F$ is twice
continuously differentiable, we can take a second order Taylor
expansion of $F$ around $x^*$.
\[ F(x^*+h) = F(x^*) + DF_{x^*} h + h^T D^2 F_{x*} h + r(x^*,h) \]
where $D^2 F_{x^*}$ is the matrix of $F$'s second derivatives. It is
called the \textbf{Hessian} of $F$. 
\[ D^2 F_{x^*} = \begin{pmatrix} \frac{\partial ^2 F}{\partial x_1^2}
  & \cdots & \frac{\partial ^2 F}{\partial x_1\partial x_n} \\
  \vdots & \ddots & \vdots \\
  \frac{\partial ^2 F}{\partial x_1\partial x_n} & \cdots &
  \frac{\partial ^2 F}{\partial x_n}^2 \end{pmatrix}.
\]
Since $x^*$ is a critical, point $DF_{x^*} = 0$, so
\[ F(x^*+h) - F(x^*) = h^T D^2 F_{x*} h + r(x^*,h). \]
We can see that $x^*$ is a local maximum if 
\[ h^T D^2 F_{x*} h + r(x^*,h) \leq 0 \]
for all $h \neq 0$, $\norm{h} < \delta$ for some $\delta > 0$. We know
that $r(x^*,h)$ is small so, we expect that the above inequality will
be true if $h^T D^2 F_{x*} h \leq 0$ for all $h \neq 0$. The Hessian,
$D^2 F_{x*}$ is just some symmetric $n$ by $n$ matrix, and $h^T D^2
F_{x*} h$ is a quadratic form in $h$. This motivates the following
definition. 
\begin{definition}
  Let $A$ be a symmetric matrix, then $A$ is
  \begin{itemize}
  \item \textbf{Negative definite} if $x^T A x < 0$ for all $x \neq 0$
  \item \textbf{Negative semi-definite} if $x^T A x \leq 0$ for all $x \neq 0$
  \item \textbf{Positive definite} if $x^T A x > 0$ for all $x \neq 0$
  \item \textbf{Positive semi-definite} if $x^T A x \geq 0$ for all $x
    \neq 0$
  \item \textbf{Indefinite} if $\exists$ $x_1$ s.t. $x_1^T A x_1 > 0$ and
    some other $x_2$ such that $x_2^T A x_2 < 0$.
  \end{itemize}  
\end{definition}
In the next section we will derive some conditions on $A$ that ensure
it is negative (semi-)definite. For now, just observe that if $D^2
F_{x*}$ is negative semi-definite, then $x^*$ must be a local
maximum. If $D^2F_{x^*}$ is negative definite, then $x^*$ is a strict local
maximum. The following theorem restates the results of this
discussion.
\begin{theorem}\label{thm:soc}
  Let $F: U \to \R$ be twice continuously differentiable on $U$ and
  let $x^*$ be a critical point in the interior of $U$. If
  \begin{enumerate}
  \item\label{c:smax} The Hessian, $D^2 F_{x^*}$ is negative definite,
    then $x^*$ is a strict local maximizer.  
  \item The Hessian, $D^2 F_{x^*}$ is positive definite, then $x^*$ is
    a strict local minimizer. 
  \item The Hessian, $D^2 F_{x^*}$ is indefinite, $x^*$ is neither a
    local min nor a local max.
  \item\label{c:ambig} The Hessian is positive or negative
    semi-definite, then $x^*$ could be a local maximum, minimum, or
    neither. 
  \end{enumerate}
\end{theorem}
\begin{proof}
  We will only prove the first case. The second and third cases can be
  proven similarly. We will go over some examples of the fourth
  case. As in the discussion preceding the theorem, $x^*$ is a local
  minimizer if
  \[ F(x^*+h) - F(x^*) = h^T D^2 F_{x^*} h + r(x^*,h) \leq 0. \]
  We can rewrite this as
  \begin{align*}
    h^T D^2 F_{x^*} h + r(x^*,h) = & h^T D^2 F_{x^*} h + h^T h
    \frac{r(x^*,h)}{\norm{h}^2} 
  \end{align*}
  Factoring, we have
  \begin{align*}
    h^T D^2 F_{x^*} h + h^T h \frac{r(x^*,h)}{\norm{h}^2} = 
    & 
    h^T \left( D^2 F_{x^*}+ \frac{r(x^*,h)}{\norm{h}^2} \right) h.
  \end{align*}
  From our theorem on Taylor series, we know $\lim_{h \to 0}
  \frac{r(x^*,h)}{\norm{h}^2} = 0$. Then for any $\epsilon > 0$
  $\exists$ $\delta > 0$ such that if $\norm{h} < \delta$, then
  $\frac{|r(x^*,h)|}{\norm{h}^2} < \epsilon$. So,
  \begin{align*}
    h^T \left( D^2 F_{x^*}+ \frac{r(x^*,h)}{\norm{h}^2} \right) h
    \leq & h^T \left( D^2 F_{x^*}+ \epsilon \right) h \\
    \leq & h^T D^2 F_{x*} h + \epsilon \norm{h}^2
  \end{align*}
  Let $t = \norm{h}$ and $q = \frac{h}{\norm{h}}$, so $h = t q$. 
  Then we have
   \begin{align*}
    h^T \left( D^2 F_{x^*}+ \frac{r(x^*,h)}{\norm{h}^2} \right) h
    \leq & t^2 (q^T D^2 F_{x*} q) + \epsilon t^2
  \end{align*}
  We can pick $\epsilon < \inf_{\norm{q} =1} |q^T D^2
  F_{x*} q|$. The set $\{q: \norm{q} = 1\}$ is compact so there is
  some $q$ that achieves this minimum, and $q^T D^2 F_{x^*} q < 0$, so
  $\epsilon >0$.\footnote{If we didn't have compactness, we might have ended
    up with $\epsilon =0$, which is not allowed.} 
  Then for all $|t|<\delta$ and $\norm{q} = 1$, 
  \[ t^2 (q^T D^2 F_{x*} q) + \epsilon t^2 < t^2
  (q^T D^2 F_{x*} q) + |(q^T D^2 F_{x*} q)|t^2 < 0 \]
  where the last inequality follows from $q^T D^2 F_{x^*} q< 0$.
  Thus, 
  \[  F(x^*+h) - F(x^*) < 0 \]
  for all $\norm{h} < \delta$.
\end{proof}

When the Hessian is not positive definite, negative definite, or
indefinite, the result of this theorem is ambiguous. Let's go over
some examples of this case.
\begin{example}
  $F:\R \to \R$, $F(x) = x^4$. The first order condition is $4x^3 =
  0$, so $x^* = 0$ is the only critical point. The Hessian is $F''(x)
  = 12x^2 = 0$ at $x^*$. However, $x^4$ has a strict local minimum at
  $0$.
\end{example}

\begin{example}
  $F: \R^2 \to \R$, $F(x_1,x_2) = -x_1^2$. The first order condition is
  $DF_x = (-2x_1,0) = 0$, so the $x_1^*=0$, $x_2^* \in \R$ are all
  critical points. The Hessian is
  \[ D^2 F_x = \begin{pmatrix} -2 & 0 \\
    0 & 0 \end{pmatrix} \]
  This is negative semi-definite because $h^T D^2 F_x h = -2 h_1^2
  \leq 0$. Also, graphing the function would make it clear that
  $x_1^*=0$, $x_2^* \in \R$ are all (non-strict) local maxima. 
\end{example}

\begin{example}
  $F: \R^2 \to \R$, $F(x_1,x_2) = -x_1^2 + x_2^4$. The first order
  condition is $DF_x = (-2x_1,4 x_2^3) = 0$, so the $x^* = (0,0)$ is a
  critical point. The Hessian is  
  \[ D^2 F_x = \begin{pmatrix} -2 & 0 \\
    0 & 12 x_2^2 \end{pmatrix} \]
  This is negative semi-definite at $0$ because $h^T D^2 F_0 h = -2 h_1^2
  \leq 0$. However, $0$ is not a local maximum because $F(0,x_2)>
  F(0,0)$ for any $x_2 \neq 0$. $0$ is also not a local minimum
  because $F(x_1,0) < F(0,0)$ for all $x_1 \neq 0$.
\end{example}
In each of these examples, the second order condition is inconclusive
because $h^T D^2 F_{x^*} h = 0$ for some $h$. In these cases we could
determine whether $x^*$ is a local maximum, local minimum, or neither
by either looking at higher derivatives of $F$ at $x^*$, or look at
$D^2 F_{x}$ for all $x$ in a neighborhood of $x^*$. We will not often
encounter cases where the second order condition is inconclusive, so
we will not study these possibilities in detail.

The converse of theorem \ref{thm:soc} is nearly true. If $x^*$ is a
local maximizer, then $D^2 F_{x^*}$ must be negative semi-definite.
\begin{theorem}
  Let $F: U \to \R$ be twice continuously differentiable on $U$ and
  let $x^*$ in the interior of $U$ be a local maximizer (or minimizer) of
  $F$. Then $DF_{x^*} = 0$ and $D^2 F_{x^*}$ is negative (or positive)
  semi-definite. 
\end{theorem}
\begin{proof}
  Using the same notation and setup as in the proof of theorem
  \ref{thm:soc}, 
  \begin{align*}
    0 > F(x^* + tq ) - F(x^*) = & t^2 q^T D^2 F_{x^*} q + r(x,tq)  \\
    0 >  & t^2\left(q^T D^2 F_{x^*} q + \frac{r(x,tq)}{t^2} \right) 
  \end{align*}
  Because $\lim_{t \to 0} \frac{r(x,tq)}{t^2} = 0$, for any $\epsilon
  >0$ $\exists \delta > 0$ such that if $|t|< \delta$, then
  \begin{align*}
    0 > & t^2\left(q^T D^2 F_{x^*} q + \frac{r(x,tq)}{t^2} \right) \geq  t^2
    \left(q^T D^2 F_{x^*} q - \epsilon \right) \\
    t^2 \epsilon >& t^2 q^T D^2 F_{x^*} q 
  \end{align*}
  This is only possible for all $\epsilon >0$ if $q^T D^2 F_{x^*}
  q\leq 0$. 
\end{proof}
I do not expect you to remember the proofs of the last two
theorems. However, it is important to remember the second order
condition, and know how to check it. To check the second order
condition, we need some practical way to tell whether a matrix is
positive or negative (semi-)definite. 

\section{Definite matrices}

The second order condition says that a critical point is a local
maximum if the Hessian is negative definite. In this section we will
develop some conditions for whether a matrix is negative
definite. Let's start with the simplest case where $A$ is a one by one
matrix. Then $x^T A x = x^2 a$ is negative definite if and only if $a
< 0$. $A$ is negative semi-definite if $a\leq 0$. What if $A$ is a two
by two symmetric matrix? Then,
\begin{align*}
  x^T A x = & \begin{pmatrix} x_1 \\ x_2 \end{pmatrix}^T 
  \begin{pmatrix}
    a & b \\ b & c 
  \end{pmatrix}
  \begin{pmatrix} x_1 \\ x_2 \end{pmatrix}
  \\ 
  = & a x_1^2 + 2b x_1 x_2 + c x_2^2
\end{align*}
Completing the square we get
\begin{align*}
  x^T A x = & a\left(x_1 + \frac{2b}{a} x_1 x_2 + \frac{b^2}{a^2}
    x_2^2\right) - \frac{b^2}{a} x_2^2 + c x_2^2 \\
  = &a\left(x_1 + \frac{b}{a} x_2\right)^2 + \frac{ac-b^2}{a} x_2^2 
\end{align*}
Thus $x^T A x < 0$ for all $x \neq 0$ if $a < 0$ and
$\frac{ac-b^2}{a}<0$, i.e. $ac - b^2 > 0$. Notice that $ac - b^2 =
\det A$. So $A$ is negative definite if $a<0$ and $\det A > 0$. If we
wanted $A$ to be positive definite, we would need $a>0$ and $\det A >
0$. For semi-definite we get the same thing with weak instead of
strict inequalities. It turns out that in general this sort of pattern
continues. We can determine whether $A$ is negative definite by
looking at the determinants of certain submatrices of $A$.
\begin{definition}
  Let $A$ by an $n$ by $n$ matrix. The $k$ by $k$ submatrix 
  \[ A_k = \begin{pmatrix} a_{11} & \cdots a_{1k} \\
    \vdots & & \vdots \\
    a_{k1} & \cdots & a_{kk} \end{pmatrix} \]
  is the \textbf{$k$th leading principal submatrix} of $A$. The
  determinant of $A_k$ is the $k$th order \textbf{leading principal
    minor} of $A$.
\end{definition}

\begin{theorem}
  Let $A$ be an $n$ by $n$ symmetric matrix. Then 
  \begin{enumerate}
  \item $A$ is positive definite if and only if all $n$ of its leading
    principal minors are strictly positive.
  \item $A$ is positive semi-definite if and only if all $n$ of its leading
    principal minors are weakly positive.
  \item $A$ is negative definite if and only if all $n$ of its leading
    principal minors alternate in sign as follows: $\det A_1 < 0$,
    $\det A_2 > 0$, $\det A_3 < 0$, etc.
  \item $A$ is negative semi-definite if and only if all $n$ of its
    leading principal minors weakly alternate in sign as follows:
    $\det A_1 \leq 0$, $\det A_2 \geq 0$, $\det A_3 \leq 0$, etc
  \item $A$ is indefinite if and only if none of the five above cases
    hold, and $\det A_k \neq 0$ for at least one $k$.
  \end{enumerate}
\end{theorem}
The proof of this theorem is a bit tedious, so we will skip it. There
is a proof in chapter 16.4 of Simon and Blume. Figure
\ref{f:def} shows each of the five types of definiteness.

\begin{figure}\caption{Definite quadratic forms \label{f:def}}
  \begin{tabular}{cc} 
    {Positive definite} $x^2+y^2$ &  {Negative definite}
    $-x^2-y^2$ \\
    \includegraphics[width=0.48\linewidth]{pd} &
    \includegraphics[width=0.48\linewidth]{nd}
    \\
    {Positive semi-definite} $x^2$ &  {Negative
      semi-definite} $-y^2$
    \\ 
    \includegraphics[width=0.48\linewidth]{psd} &
    \includegraphics[width=0.48\linewidth]{nsd}   
    \\
    {Indefinite} $x^2 - y^2$ &  \\
    \includegraphics[width=0.48\linewidth]{id} & 
  \end{tabular}
\end{figure}

\subsection{Eigenvectors and eigenvalues}

Another way to check whether a matrix is negative definite is by
looking at the matrix's eigenvalues. Eigenvalues are of interest in
their own right as well because eigenvalues have many other uses, such
as determining the stability of systems of difference and differential
equations. 
\begin{definition}
  If $A$ is an $n$ by $n$ matrix, $\lambda$ is a scalar, $v \in \R^n$
  with $\norm{v} = 1$, 
  and 
  \[ Av = \lambda v \]
  then $\lambda$ is a \textbf{eigenvalue} of $A$ and $v$ is an
  \textbf{eigenvector}. 
\end{definition}
If $v$ is an eigenvector of $A$ with eigenvalue $\lambda$, then 
\[ A(tv) = \lambda (tv) \]
for any $t \neq 0$. Thus, the requirement that $\norm{v} = 1$ is just
a normalization to pin down $v$. There are a few equivalent ways of
defining eigenvalues, some of which 
are given by the following lemma.
\begin{lemma}
  Let $A$ be an $n$ by $n$ matrix and $\lambda$ a scalar. Each of the
  following are equivalent.
  \begin{enumerate}
  \item\label{eig1} $\lambda$ is a eigenvalue of $A$.
  \item\label{eig2} $A - \lambda I$ is singular.
  \item\label{eig3} $\det(A - \lambda I ) = 0$.
  \end{enumerate}
\end{lemma}
\begin{proof}
  We know that (\ref{eig2}) and (\ref{eig3}) from our results on
  systems of linear equations and matrices. Also, if $A - \lambda I$
  is singular, then the null space of $A - \lambda I$ contains
  non-zero vectors. Choose $v \in \mathcal{N}(A-\lambda I)$ such that
  $v \neq 0$. Then $(A-\lambda I) v = 0$, so $A(v/\norm{v}) =
  \lambda(v/\norm{v})$ as in the definition of eigenvalues. 
\end{proof}
The function $\chi_A:\R\to\R$ defined by $\chi_A(x) = \det(A - x I)$
is called the \textbf{characteristic polynomial} of $A$. It will be a
polynomial of order $n$. You may know from some other math course that
polynomials of degree $n$ have $n$ roots, some of which might be
not be real and some of which might not be distinct. We use the fact
that symmetric nonsingular matrices have $n$ distinct real
eigenvalues without proving it.

Suppose that $A$ has $n$ distinct real eigenvalues, $\lambda_1, ...,
\lambda_n$ with corresponding eigenvectors $v_1, ..., v_n$. Then the
eigenvectors a linearly independent.
\begin{lemma}
  Let $A$ be an $n$ by $n$ matrix with $n$ distinct real eigenvalues
  then the eigenvectors of $A$ are linearly independent.
\end{lemma}
\begin{proof}
  Suppose the eigenvectors are not linearly independent. Then we could
  write\footnote{Possibly we cannot do this for $v_1$, but we can for
    some $v_i$. Without loss of generality, assume that we can for
    $v_1$.}
  \[ v_1 = c_2 v_2 + ... + c_n v_n. \]
  with at least one $c_k \neq 0$.
  By the definition of eigenvalues and eigenvectors, we have
  \begin{align*} A v_1 = & \lambda_1 v_1 \\
    A(c_2 v_2 + ... + c_n v_n) = & \lambda_1 \left(c_2 v_2 + ... + c_n
      v_n\right) \\
    c_2 \lambda_2 v_2 + ... + c_n \lambda_n v_n = & \lambda_1
    \left(c_2 v_2 + ... + c_n v_n\right) \\
    c_2 (\lambda_2 - \lambda_1) v_2 + ... + c_n (\lambda_n -
    \lambda_1) v_n = & 0
  \end{align*}
  Since the $\lambda_i$ are distinct, $\lambda_k - \lambda_1 \neq 0$
  for any $k$, so this says that $v_2, ..., v_n$ are linearly
  dependent as well. We can repeat this argument to show that $v_3,
  ..., v_n$ are linearly dependent, and then repeat it again and again
  and eventually show that $v_{n-1}$ and $v_n$ are linearly
  dependent. Then $v_{n-1} = bv_n$ for some $b \neq 0$, and 
  \[ A v_{n-1} = \lambda_{n-1} v_{n-1} = \lambda_{n-1} (b v_n) \]
  and 
  \[ A v_{n-1} = A b v_n = b \lambda_n v_n. \]
  This implies that $\lambda_n = \lambda_{n-1}$, contrary to the
  eigenvalues being distinct. Therefore, if the eigenvalues are
  distinct, the eigenvectors must be linearly independent.   
\end{proof}
So if $A$ has $n$ distinct eigenvalues, it has $n$ linearly
independent eigenvectors. If we make $\Lambda$ a diagonal matrix
consisting of the eigenvalues of $A$ and $V$ a matrix whose columns are
the eigenvectors of $A$, then 
\[ A V = V \Lambda. \]
Moreover, the columns of $V$ are linearly independent, so $V$ is
invertible and $A = V \Lambda V^{-1}$. Furthermore, the eigenvectors
can be made to be orthogonal to one another. If $v_1$ and $v_2$ are
not orthogonal set, $\tilde{v}_2 = \frac{v_2 - v_1^T v_2
  v_1}{\norm{v_2 - v_1^T v_2 v_1}}$, etc. Then $V$ will be an
orthogonal matrix, so $V^{-1} = V^T$. This relationship is called the
eigendecomposition of $A$.
\begin{theorem}[Eigendecomposition]\label{thm:eigen}
  Let $A$ be an $n$ by $n$ non-singular symmetric matrix, then $A$ has
  $n$ distinct real eigenvalues
  \[ A = V \Lambda V^T \]
  where $\Lambda$ is the diagonal matrix consisting of the eigenvalues
  of $A$ and the columns of $V$ are the eigenvectors of $A$, and $V$
  is an orthogonal matrix.  
\end{theorem}

Using the eigendecomposition, we can relate eigenvalues to the
definiteness of a matrix.
\begin{theorem}
  If $A$ is an $n$ by $n$ symmetric non-singular matrix with
  eigenvalues $\lambda_1, ..., \lambda_n$, then
  \begin{enumerate}
  \item $\lambda_i > 0$ for all $i$, iff $A$ is positive definite,
  \item $\lambda_i \geq 0$ for all $i$, iff $A$ is positive semi-definite,
  \item $\lambda_i < 0$ for all $i$, iff $A$ is negative definite,    
  \item $\lambda_i \leq 0$ for all $i$, iff $A$ is negative
    semi-definite,
  \item if some $\lambda_i > 0$ and some $\lambda_j < 0$, then $A$ is
    indefinite.
  \end{enumerate}
\end{theorem}
\begin{proof}
  Let $A = V \Lambda V^T$ be the eigendecomposition of $A$. Let $x
  \neq 0 \in \R^n$. Then $x^T V = z^T \neq 0$ because $V$ is
  nonsingular. Also, 
  \[ x^T A x = x^T V \Lambda V^T x = z^T \Lambda z. \]
  Since $\Lambda$ is diagonal we have
  \[ z^T \Lambda z = z_1^2 \lambda_1 + ... + z_n^2 \lambda_n \]
  Thus, $x^T A  x= z^T \Lambda z > 0$ for all $x$ iff $\lambda_i>0$
  for each $i$. The other parts of the theorem follow similarly. 
\end{proof}

\section{Global maximum and minimum}

The second order condition (\ref{thm:soc}) along with the first order
condition gives a nice way of finding local maxima of a function, but
what about the global maximum?  In general, you could use the first and
second order conditions to find each of the local maxima of $F$ in the
interior of $U$. You then must compare each of these local maxima with
each other and the value of $F$ on the boundary of $U$ to find the
global maximum of $F$ on $U$. If there are many local maxima or the
boundary of $U$ is complicated, this procedure can be quite
complicated. 
It would be nice if there was some simpler necessary and
sufficient for a global maximum. Unfortunately, there is no such
general necessary and sufficient condition. There is, however, a
sufficient condition that is sometimes useful.
\begin{definition}
  Let $f:U \to \R$. $f$ is \textbf{convex} if for all $x,y \in U$ with
  $\ell(x,y) \subseteq U$ we have $f(tx + (1-t)y) \leq tf(x) + (1-t)
  f(y)$ for all $t \in [0,1]$.
\end{definition}
Equivalently, $F$ is convex if the set $\{(y,x): x \in U, y \geq
f(x)\} \subseteq U \times \R$ is convex. This set is called the
epigraph of $f$. If you draw the graph of the function, this is the
set of points above the function.
\begin{definition}
  Let $f:U \to \R$. $f$ is \textbf{concave} if for all $x,y \in U$ with
  $\ell(x,y) \subseteq U$ we have $f(tx + (1-t)y) \geq tf(x) + (1-t)
  f(y)$ for all $t \in [0,1]$.
\end{definition}
Equivalently, $F$ is concave if the set $\{(y,x): x \in U, y \leq
f(x)\}$ is convex.
\begin{theorem}
  Let $F: U \to \R$ be twice continuously differentiable and $U
  \subseteq \R^n$ be convex. Then,
  \begin{enumerate}
  \item For maximization:
    \begin{enumerate}
    \item The following three conditions are equivalent:
      \begin{enumerate}
      \item $F$ is a concave function on $U$
      \item $F(y) - F(x) \leq DF_x(y-x)$ for all $x,y \in U$,
      \item $D^2 F_x$ is negative semi-definite for all $x, y \in U$
      \end{enumerate}
    \item If $F$ is a concave function on $U$ and $DF_{x^*} = 0$ for
      some $x^* \in U$, then
      $x^*$ is the global maximizer of $F$ on $U$.
    \end{enumerate}
  \item For minimization:
    \begin{enumerate}
    \item The following three conditions are equivalent:
      \begin{enumerate}
      \item $F$ is a convex function on $U$
      \item $F(y) - F(x) \geq DF_x(y-x)$ for all $x,y \in U$,
      \item $D^2 F_x$ is positive semi-definite for all $x, y \in U$
      \end{enumerate}
    \item If $F$ is a convex function on $U$ and $DF_{x^*} = 0$ for
      some $x^* \in U$, then
      $x^*$ is the global minimizer of $F$ on $U$.
    \end{enumerate}
  \end{enumerate}
\end{theorem}
This theorem is often summarized by saying that convex minimization
problems (i.e. the set $U$ and function $F$ are both convex) have
unique solutions.  We will not prove this theorem until after covering
convexity in more detail, which we may or may not have time for. An
important fact about convex optimization problems is that their
solutions can be efficiently computed. A general optimization problem
is what is known as NP-hard, which in practice means that you can
write down what looks like a reasonable not too large optimization
problem whose solution takes prohibitively long to compute. In
contrast convex optimization problems can be solved in polynomial
time. In particular, if $f$ is convex, $x$ is $n$ dimensional, the
steps needed to compute $f(x)$ is $M$, and you want to find $\hat{x}$
such that $\abs{f(\hat{x}) - f(x^\ast)} < \epsilon$, then there are
algorithms that can find $\hat{x}$ in
$O\left(n(n^3+M)\log(1/\epsilon)\right)$ steps. In fact, there are
algorithms for special types of convex problems that can solve them
nearly as fast as a least squares problem or a linear program. In
practice, this means that most convex problems can be solved quickly
with up to about a one thousand variables, and some convex problems
can be solved very quickly even with tens or hundreds of thousands of
variables. The discovery that convex optimization problems can be
solved efficiently was fairly recent. The main theoretical
breakthroughs began in the late eighties and early nineties, and it
remains an active area of research.

\section{Applications}

\subsection{Profit maximization}

\subsubsection{Competitive multi-product firm}
Suppose a firm has produces $k$ goods using $n$ inputs with production
function $f: \R^n \to \R^k$. The prices of the goods are $p$, and the
prices of the inputs are $w$, so that the firm's profits are 
\[ \Pi(x) =  p^T f(x) -  w^Tx. \]
The firm chooses $x$ to maximize profits.
\[ \max_x p^T f(x) - w^T x \]
The first order condition is
\[ p^T Df_{x^*} - w = 0. \]
or without using matrices,
\[ \sum_{j=1}^k p_j \frac{\partial f_j}{\partial x_i}(x^*) = w_i \]
for $i=1,..., n$. The second order condition is that 
\[ D[p^T Df]_{x^*} = \begin{pmatrix} \sum_{j=1}^k p_j \frac{\partial^2
    f_j}{\partial x_1^2}(x^*) & \cdots & \sum_{j=1}^k p_j \frac{\partial^2
    f_j}{\partial x_1\partial x_n}(x^*)  \\ \vdots & & \vdots \\
  \sum_{j=1}^k p_j \frac{\partial^2
    f_j}{\partial x_1\partial x_n}(x^*) & \cdots & \sum_{j=1}^k p_j
  \frac{\partial^2 f_j}{\partial x_n^2}(x^*) \end{pmatrix} 
\]
must be negative semidefinite. 

\subsubsection{Multi-product monopolist}
Consider the same setup as before, but now with a monopolist who
recognizes that prices depend on output. Let the inverse demand
function be $P(q)$ where $P:\R^k \to \R^k$. Now the firm's problem
is 
\[ \max_x P(f(x))^T f(x) - w^T x \]
The first order condition is
\[ Df_{x^*}^T DP_{f(x^*)}^T f(x^*) + P(f(x^*)) Df_{x^*} - w = 0 \]
or without matrices,
\[ \sum_{j=1}^k \left( \sum_{l=1}^k \frac{\partial P_j}{\partial q_l}(f(x^*))
  \frac{\partial f_l}{\partial x_i}(x^*) f_l(x^*) + 
  P_j(f(x^*)) \frac{\partial f_j}{\partial x_i}(x^*) f_j(x^*) \right)
= w_i \]
We can get something a bit more interpretable by writing this in terms
of elasticities. Recall that the elasticity of demand for good $l$
with respect to the price of good $j$ is $(\epsilon^l_j)^{-1}=
\frac{\partial P_j}{\partial q_l} \frac{q_l}{P_l}$. Then,
\begin{align*}
  \sum_{j=1}^k P_j(f(x^*)) \left(\sum_{l=1}^k (\epsilon^l_j)^{-1} \frac{\partial
      f_l}{\partial x_i}(x^*) +  \frac{\partial f_j}{\partial x_i}(x^*) f_j(x^*) \right)
  = & w_i \\
  \sum_{j=1}^k P_j(f(x^*)) \left[\frac{\partial f_j}{\partial x_i}(x^*)
    \left(1 + (\epsilon_j^j)^{-1} \right) + \sum_{l \neq j}
  (\epsilon_j^l)^{-1} \frac{\partial f_l}{\partial x_i}(x^*) \right] = & w_i 
\end{align*}
There is a way to compare the price and quantity produced by the
monopolist to the competitive firm. There are also things that can be
said about the comparison between a single product monopolist ($k=1$)
vs a multi-product monopolist. It might be interesting to derive some
of these results. To begin with let $k = 1$ and compare the monopolist
to the competitive firm. Under some reasonable assumptions on $f$ and
$P$, you can show that $x^m < x^c$ where $x^m$ is $x^*$ for the
monopolist and $x^c$ is $x^*$ for the competitive firm. You can also
show that $p^m>p^c$ and that $|\epsilon^1_1| < 1$ for the
monopolist. (Although I left it out of the notation above, $\epsilon$
depends on $x$). 


\end{document}
