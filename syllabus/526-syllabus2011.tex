\documentclass[10pt]{article}
\usepackage{amsmath}
\usepackage{amsfonts}
\usepackage{graphicx}
%\usepackage{epstopdf}
\usepackage{hyperref}
\usepackage[left=.9in,right=.9in,top=.7in,bottom=.7in]{geometry}
%\usepackage[left=1.1in,right=1.1in,top=.7in,bottom=.7in]{geometry}
\usepackage{multirow}
\usepackage{verbatim}
\usepackage{fancyhdr}
\usepackage[small,compact]{titlesec} 
\usepackage{color}
\usepackage{ulem}
%\usepackage[screen,nopanel,gray]{pdfscreen}
%\screensize{4in}{5.33in}
%\margins{0.75in}{0.75in}{0.75in}{0.75in}


%\usepackage{pxfonts}
%\usepackage{isomath}
\usepackage{mathpazo}
%\usepackage{arev} %     (Arev/Vera Sans)
%\usepackage{eulervm} %_   (Euler Math)
%\usepackage{fixmath} %  (Computer Modern)
%\usepackage{hvmath} %_   (HV-Math/Helvetica)
%\usepackage{tmmath} %_   (TM-Math/Times)
%\usepackage{cmbright}
%\usepackage{ccfonts} \usepackage[T1]{fontenc}
%\usepackage[garamond]{mathdesign}

\newcommand{\argmax}{\operatornamewithlimits{arg\,max}}
\newcommand{\argmin}{\operatornamewithlimits{arg\,min}}
\newcommand{\myTable}[1]{\begin{table}[b]\caption{#1}
    \begin{minipage}{\linewidth} \begin{center}}
\newcommand{\myTableEnd}{\end{center}\end{minipage}\end{table}}

\def\inprobLOW{\rightarrow_p}
\def\inprobHIGH{\,{\buildrel p \over \rightarrow}\,} 
\def\inprob{\,{\inprobHIGH}\,} 
\def\indist{\,{\buildrel d \over \rightarrow}\,} 

\newtheorem{theorem}{Theorem}

%\pagestyle{fancy}
\renewcommand{\sectionmark}[1]{\markright{#1}{}}
 
%\fancyhead[LE,LO]{\thepage}
%\fancyhead[CE,CO]{\rightmark}
%\fancyfoot[C]{}

\title{Economics 526 - Mathematics for Economists}
\date{Term 1, 2011-12}
\author{Paul Schrimpf}

\makeatletter
\renewcommand{\@maketitle}{
  \null
  \begin{center}%
    {\large \textbf{\@title} \par \normalsize \@date \par \@author}%
  \end{center}%
  \par} 
\makeatother

\begin{document}

\maketitle

This course is a course of mathematics for students in our Economics
MA program. It covers the powerful mathematical tools, which often
appear in the modern economic literature. It is essential for the
students' success in other courses to master the materials taught in
this course.  

Colin Caines (ccaines@interchange.ubc.ca) and Yawen Liang
(Neway.liang@gmail.com) are the course TA's.  Information about the
course will be sent via e-mail and posted on the course web site on
UBC Vista. The URL of the course web page is:
\url{https://www.vista.ubc.ca/webct/logon/7603551231151}. My email is
\href{mailto:schrimp@mail.ubc.ca}{schrimpf@mail.ubc.ca}, and my office
phone number is 604-822-5360.

\section{Schedule}
\begin{tabular}{l c c l}
  \hline 
  & \textbf{Day(s)} & \textbf{Time} & \textbf{Location} \\
  Lecture & Monday \& Wednesday & 2:00pm-4:00pm &
  Lasserre 102
  \\
  Discussion Session & Friday & 2:00pm-4:00pm &
  Brock Hall Annex 2365
  \\
  Office hours \\
  \; Paul & Tuesday & 3:30pm-4:30pm & Buchanan Tower 926 \\
  \; Colin & TBA & & \\ 
  \; Yawen & TBA & & \\ \hline
  Exams \\
  \; Midterm & Wednesday, October 5th & 2:00pm-4:00pm & Lasserre 102 \\
  \; Final & Wednesday, November 9th & 2:00pm-4:00pm & Lasserre 102
  \\ \hline 
\end{tabular} \\
Office hours are subject to change. Any changes in office hours will
be posted on the course web page. If you cannot come to my office
hours, feel free to drop by anytime or email me to schedule an
appointment.  

\section{Course Work}

The course work will consist of a series of problem sets and two
exams. 

\subsection{Problem Sets}
Problem sets will be due approximately weekly.  Students should solve
each problem set and turn in their work by the specified due
date. Also, students should study the solution to each problem set
carefully, regardless of their performances in the problem
set. Students may work together on problem sets, but each student
should write their own solutions.

\subsection{Exams}
Two in-class exams will be given on October 5 and November 9. If a
student misses the first exam due to an objectively verifiable,
unavoidable emergency situation, the student's grade will be
determined by setting his first exam's grade equal to his second
exam's grade. If a student misses the first exam for any other
reasons, the student receives zero points for the first exam. 

\subsection{Grading}
The course grade weights the problem sets by 0.15, the First exam by
0.375 and the second exam by 0.425. If a student finds an error in the
grading of a problem set or exam, the student should contact the
teaching assistants (for problem sets) or me (for an exam) within one
week of the day the graded problem sets or exams were made
available. We will then regrade the entire problem set or exam, and
your total score may increase or decrease as a result.  The
distribution of grades in this course for the past five years is shown
below.
\href{http://www.arts.ubc.ca/faculty-amp-staff/resources/courses-and-grading/grading-guidelines.html}
{The Faculty of Arts grading guidelines} recommend that 5-25\% of the
class receive A's (80-100) and not more than 75\% receive A's or B's
(68-79). Grades may be scaled up to meet these guidelines and/or match
the historical distribution of grades, but they will not be scaled
down. 

\begin{centering}
  \includegraphics[width=0.8\linewidth]{526gradeDist}
\end{centering}

\section{Textbook}
The textbook of this course is Carl P. Simon and Laurence Blume,
\textit{Mathematics for Economists}, Norton, 1994.  There will be
additional material posted on the course web page.

\section{Course Outline}

The topics of the course is outlined below. The corresponding part of
the textbook and the projected number of lectures of each topic are
indicated between parentheses. 

\begin{enumerate}
\item Sets, Numbers, and Proofs (Appendix A1; 0.5 lectures)
\item Linear Algebra
  \begin{enumerate}
  \item Systems of Linear Equations (Chapters 6-7; 1 lecture)
  \item Matrix Algebra (Chapters 8-9, 26; 1.5 lecture)
  \item Euclidean Spaces as Linear Spaces (Chapters 10-11, 27; 2 lectures)
  \end{enumerate}
\item Calculus of Several Variables
  \begin{enumerate}
  \item Topology of the Euclidean Spaces (Chapter 12, 29 (up to
    Section 29.3); 1 lecture) 1 \item Functions (Chapter 13; 1.5
    lecture)
  \item  Differential Calculus (Chapters 14, 30, 1 lecture)
  \item Contraction Principle, Inverse Function Theorem, and Implicit
    Function Theorem (Chapter 15; 0.5 lecture)
  \end{enumerate}
\item Eigenvalues, Eigenvectors, and Definite Matrices (Chapter 23,
  16; 1 lecture)
\item Optimization
  \begin{enumerate}
  \item Unconstrained Optimization (Chapter 17; 0.5 lectures)
  \item Constrained Optimization (Chapters 18-19; 1.5 lectures)
  \item Homogeneous and Homothetic Functions (Chapter 20; 1 lectures)
  \item Concave and Quasiconcave Functions (Chapter 21; 1 lecture)
  \end{enumerate}
\item Dynamic programming and Optimal Control (1 lecture) 
  % guido and ivan's notes on ocw, daron's book chs 6, 7 , 16
\item Comparative statics without differentiability: Supermodularity
  and Topkis's theorem (1 lecture)
  \begin{itemize}
  \item Reference: Amir, Rabah (2005). ``Supermodularity and
    Complementarity in Economics: An Elementary Survey,''
    \textit{Southern Economic Journal}. 71(3):
    636-660. \url{http://www.jstor.org/stable/20062066}
  \end{itemize}
\end{enumerate}

\end{document}

