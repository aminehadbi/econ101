\documentclass[compress]{beamer} 
\usepackage{amsmath}
\usepackage{amsfonts}
\usepackage{graphicx}
%\usepackage{epstopdf}
\usepackage{hyperref}
%\usepackage{multirow}
\usepackage{verbatim}
%\usepackage[small,compact]{titlesec} 
%\usecolortheme{beaver}
\usetheme{Hannover}
%\usecolortheme{whale}

%\usepackage{pxfonts}
%\usepackage{isomath}
%\usepackage{mathpazo}
%\usepackage{arev} %     (Arev/Vera Sans)
%\usepackage{eulervm} %_   (Euler Math)
%\usepackage{fixmath} %  (Computer Modern)
%\usepackage{hvmath} %_   (HV-Math/Helvetica)
%\usepackage{tmmath} %_   (TM-Math/Times)
\usepackage{tgheros}
%\usepackage{cmbright}
%\usepackage{ccfonts} \usepackage[T1]{fontenc}
%\usepackage[garamond]{mathdesign}
\usepackage{color}
\usepackage{ulem}

\newcommand{\argmax}{\operatornamewithlimits{arg\,max}}
\newcommand{\argmin}{\operatornamewithlimits{arg\,min}}
\def\inprobLOW{\rightarrow_p}
\def\inprobHIGH{\,{\buildrel p \over \rightarrow}\,} 
\def\inprob{\,{\inprobHIGH}\,} 
\def\indist{\,{\buildrel d \over \rightarrow}\,} 
\def\pref{\succeq}
\def\Re{\mathbb{R}}
\def\rel{\,{\buildrel R \over \sim}\,}


% \newtheorem{theorem}{Theorem}[section]
% \newtheorem{conjecture}{Conjecture}[section]
% \newtheorem{corollary}{Corollary}[section]
% \newtheorem{lemma}{Lemma}[section]
% \newtheorem{proposition}{Proposition}[section]
% \theoremstyle{definition}
% \newtheorem{assumption}{}[section]
% %\renewcommand{\theassumption}{C\arabic{assumption}}
% \newtheorem{definition}{Definition}[section]
% \newtheorem{step}{Step}[section]
% \newtheorem{remark}{Comment}[section]
% \newtheorem{example}{Example}[section]
% \newtheorem*{example*}{Example}

%\linespread{1.1}

%\renewcommand{\sectionmark}[1]{\markright{#1}{}}


\title{Sets, Numbers, and Proofs}
\author{Paul Schrimpf}
\institute{UBC \\ Economics 526}
\date{\today}

\begin{document}

\frame{\titlepage}

\begin{frame}
  \tableofcontents  
\end{frame}


%%%%%%%%%%%%%%%%%%%%%%%%%%%%%%%%%%%%%%%%%%%%%%%%%%%%%%%%%%%%%%%%%%%%%% 
\section{Sets \label{s:sets}} 

\begin{frame}
  \frametitle{Sets}
  \begin{itemize}
  \item A \textbf{set} is any well-specified collection of
    elements
  \item Notation and examples:
    \begin{align*} 
      a \in & A   & 
      A = & \{4 , 5, 6 \} \\
      A = & \{n \in \mathbb{N}: 3 < n < 7 \}  & 
      A_k = & \{ n \in \mathbb{N}: n > k \} \\ 
      B = & \{ A_1, A_6, A_7 \} & \emptyset 
    \end{align*}
  \end{itemize}
\end{frame}
\subsection{Economic examples}

\begin{frame}
  \begin{example}\label{ex:sampleSpace}[Sample space]
    In a random experiment, the set of all possible outcomes is called
    the \textbf{sample space}. E.g. for the roll of a dice, the sample
    space if $\{1,2 ,3 , 4, 5, 6\}$. An \textbf{event} is any subset of
    the sample space.
  \end{example}
\end{frame}

\begin{frame}
  \begin{example}\label{ex:games}[Games]
    A game is a model of strategic decision making. A game consists of a
    finite set of $n$ players, say $N = \{1,2,..., n\}$. Each player $i
    \in N$ chooses an action $a_i$ from a set of actions $A_i$. The
    outcome of the game depends on the actions chosen by all players. 
  \end{example}
\end{frame}

\begin{frame}
  \begin{example}\label{ex:conset}[Consumption set]
    The \textbf{consumption set} is the set of all feasible consumption
    bundles. Suppose there are $n$ commodities. A consumer chooses a
    consumption bundle $\mathbf{x} = (x_1, x_2, ..., x_n)$. Consumption
    cannot be negative, so the consumption set is a subset of $\Re^n_+ =
    \{(x_1, ..., x_n) : x_1 \geq 0, x_2 \geq 0, ... x_n \geq 0 \}$.
  \end{example}
\end{frame}

\subsection{Set operations}

\begin{frame}
  \frametitle{Set operations and comparisons}
  \begin{itemize}
  \item \textbf{Union:} $A \cup B = \{x: x\in A \text{ or } x \in B\}$.
  \item \textbf{Intersect:} $A \cap B = \{x: x \in A \text{ and } x \in
    B\}$. 
  \item \textbf{Minus:} $A \setminus B = \{ x: x\in A \text{ and }
    \not\in B \}$
  \item \textbf{Complement:} $A^c = \{x \in U \setminus A\}$
  \item \textbf{Subset:} $A \subseteq B$, $A \subset B$.
  \end{itemize}
\end{frame}

\subsection{Equivalence between logic and set
  operations \label{s:setLogic}} 

\begin{frame}
  \frametitle{Sets and logic}
  \begin{itemize}
  \item \textbf{Predicate}: is a function, $p(x)$, from some set, $X$,
    to $\{T, F \}$
  \item Truth set of $p(x)$ is $\{x \in X: p(x) = T \}$
  \end{itemize}
  \begin{theorem}[Equivalence of truth sets and
    predicates \label{thm:setLogic}] Let $p(x)$
    and $q(x)$ be predicates and $P$ and $Q$ be
    the associated truth sets. Then if $\tilde{p}(x) = \begin{cases} T
      \text{ if } x \in P \\ 
      F \text{ if } x \not\in P \end{cases}$, we have $p(x) = \tilde{p}(x)
    \forall x \in X$. Also, 
    \begin{enumerate}
    \item\label{i1} $p(x) \wedge q(x)$ iff $x \in P \cap Q$
    \item\label{i2} $p(x) \vee q(x)$ iff $x \in P \cup Q$
    \item\label{i3} $\sim p(x)$ iff $x \in P^c$
    \item\label{i4} $p(x) \Rightarrow q(x)$ iff $P \subseteq Q$
    \end{enumerate}
  \end{theorem}
\end{frame}

\begin{frame}
  \begin{corollary}
    Let $X$, $Y$, and $Z$ be sets contained in some universe $U$. 
    The following sets from columns A and B are equivalent.
    \begin{centering}
      \begin{tabular}{l r}
        A       & B \\ \hline
        $(X \cup Y)^c$ & $X^c \cap Y^c$ \\
        $(X \cap Y)^c$ & $X^c \cup Y^c$ \\
        $X \cap (Y \cup Z)$ & $(X \cap Y) \cup (X \cap Z)$ \\
        $X \cup (Y \cap Z)$ & $(X \cup Y) \cap (X \cup Z)$
      \end{tabular}
    \end{centering}
  \end{corollary}
\end{frame}

\section{Cardinality \label{s:card}}

\begin{frame}
  \frametitle{Cardinality}
  \begin{itemize}
  \item How can we compare the size of infinite sets?
  \item Cardinality of set $A$ denoted $|A|$. 
  \item $|A| = |B|$ if $\exists$ one-to-one and onto mapping between
    $A$ and $B$
  \item Cardinality is either
    \begin{itemize}
    \item Finite
    \item Countable $=|\mathbb{N}|$
    \item Uncountable $>|\mathbb{N}|$ 
    \end{itemize}
  \end{itemize}
\end{frame}

\begin{frame}
  \begin{lemma}
    $\mathbb{Z}$ is countable. 
  \end{lemma}
\end{frame}  

\begin{frame}
  \begin{theorem} \label{thm:countsubset}
    Every infinite subset of a countable set $A$ is countable.
  \end{theorem}
\end{frame}

\begin{frame}
  \begin{theorem}
    The rational numbers are ??????
  \end{theorem}
\end{frame}

\begin{frame}
  \begin{theorem}
    The real numbers are ??????
  \end{theorem}
\end{frame}

\begin{frame}
  \frametitle{Remarks}
  \begin{itemize}
  \item Sets bigger than $\mathbb{R}$?
    \begin{itemize}
    \item Power set of $A$ always has cardinality larger than $A$
    \end{itemize}
  \item Sets bigger than $\mathbb{N}$ and smaller than $\mathbb{R}$?
    \begin{itemize}
    \item Continuum hypothesis 
    \end{itemize}
  \end{itemize}
\end{frame}

%%%%%%%%%%%%%%%%%%%%%%%%%%%%%%%%%%%%%%%%%%%%%%%%%%%%%%%%%%%%%%%%%%%%%%
\section{Numbers \label{s:numbers}}

\begin{frame}
  \frametitle{Numbers}
  \begin{itemize}
  \item Natural, integer, rational, real: where do they come from?
    what makes them special?
  \item Natural: take as given (but could construct from logic or set
    theory) 
  \end{itemize}
\end{frame}

\begin{frame}
  \begin{itemize}
  \item Properties of addition:
    \begin{itemize}
    \item[1] \emph{Closure} if $a, b \in \mathbb{N}$, so is $a + b$ 
    \item[2] \emph{Associative} $a + (b + c) = (a + b) + c$. 
    \item[3] \emph{Identity} $\exists 0$ s.t. $a + 0 = a$,
    \item[4] \emph{Inverse} $\forall a$, $\exists b$ s.t. $a + b = 0$
    \end{itemize}
  \item Given natural numbers, if we demand addition have (3) and (4),
    we need integers    
  \item If we also demand multiplication has property (4), we need
    rationals
  \item Addition and multiplication are also
    \begin{itemize}
    \item[5] \emph{Commutative} $ a + b = b + a$
    \item[6] \emph{Distributive} $a (b + c) = ab + ac$
    \end{itemize} 
  \item A set with $+$ and $\times$ s.t. (1)-(6) is a field
  \end{itemize}
\end{frame}

\begin{frame}
  \frametitle{More properties of rationals}
  \begin{itemize}
  \item An \textbf{ordered set} is a set, $A$, and a relation, $<$,
    such that (i) $\forall a,b \in A$ either $a < b$ or $a = b$ or $a
    > b$; and (ii) if $a < b$ and $b < c$ then $a < c$.
  \item An \textbf{ordered field} is a field that is an ordered set
    and addition and multiplication preserve the ordering in that (i)
    if $b<c$ then $a + b < a + c$ (ii) if $a>0$ and $b>0$ then $ab>0$
  \end{itemize}
\end{frame}

\begin{frame}
  \frametitle{The problem with rationals}
  \begin{theorem}
    $\sqrt{2} \not\in \mathbb{Q}$.
  \end{theorem}
  \begin{itemize}
  \item $s \in S$ is an \textbf{upper bound} of $A$ if $s > a \forall
    a \in A$.
  \item $s$ is a \textbf{least upper bound} of $A$ if $s$ is an upper
    bound of $A$ and if $r < s$, then $r$ is not an upper bound of
    $A$. 
  \item $S$ has the \textbf{least-upper-bound property} if whenever $A
    \subset S$ has an upper bound, $A$ has a least upper bound.
  \end{itemize}
  \begin{theorem}
    $\mathbb{Q}$ does not have the least-upper-bound property.
  \end{theorem}
\end{frame}

\subsection{Real numbers}
\begin{frame}
  \frametitle{Real numbers}
  \begin{theorem}
    There exists an ordered field, $\mathbb{R}$, that has the least
    upper bound property. $\mathbb{R}$ contains $\mathbb{Q}$. Moreoever,
    $\mathbb{R}$ is ``unique''.
  \end{theorem}
\end{frame}

\section{Relations}

\begin{frame}\frametitle{Relations}
  \begin{definition}[Relation]
    A \textbf{relation} on two sets $A$ and $B$ is any subset of $A
    \times B$, $R \subseteq A \times B$. We usually denote relations by
    $a \rel b$ if $(a,b) \in R$ (where $\rel$ could be some other symbol).
  \end{definition}
\end{frame}

\begin{frame}\frametitle{Properties of relations}
  A relation $\rel$ on $A$ is 
  \begin{itemize}
  \item \textbf{reflexive} if $a \rel a$ $\forall a \in A$,
  \item \textbf{symmetric} if $a \rel b$ implies $b \rel a$,
  \item \textbf{transitive} if $a \rel b$ and $b \rel c$ implies
    $a \rel c$,
  \item \textbf{antisymmetric} if $a \rel b$ and $b \rel a$
    implies $a=b$,
  \item \textbf{asymmetric} if $a \rel b$ implies $b$ is not
    $\rel a$, and
  \item \textbf{complete} if either $a \rel b$ or $b \rel a$ or
    both $\forall a, b \in A$. 
  \end{itemize}
\end{frame}

\begin{frame}
\begin{example}[Preference relation] 
  A consumer's preference relation, $\pref$, is a relation on her
  consumption set, $X$. $x \pref y$ means that the consumer likes the
  bundle of goods $x$ at least as much of the bundle of goods $y$. We
  usually assume that preference relations are complete and
  transitive.
\end{example}
\end{frame}

\begin{frame}[shrink]
  \frametitle{Equivalence}
  \begin{itemize}
  \item \textbf{equivalence relation} $=$ complete,transitive, \&
    symmetric
  \item \textbf{equivalence class} of $x$ is $\sim(x) = \{ a
    \in X: a \sim x \}$
  \end{itemize}
  \begin{example}[Indifference]
    \begin{itemize}
    \item Take $\pref$, a preference relation on $X$
    \item  Indifference relation $x \sim y$ if $x \pref y$ and $y
      \pref x$
    \item Indifference classes $=$ indifference curves
    \end{itemize}
  \end{example}
\end{frame}



\end{document}
