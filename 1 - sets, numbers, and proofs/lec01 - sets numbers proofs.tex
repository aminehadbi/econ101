\documentclass[12pt,reqno]{amsart}
\usepackage{amsmath}
\usepackage{amsfonts}
\usepackage{graphicx}
%\usepackage{epstopdf}
\usepackage{hyperref}
\usepackage[left=1in,right=1in,top=0.9in,bottom=0.9in]{geometry}
\usepackage{multirow}
\usepackage{verbatim}
\usepackage{fancyhdr}
%\usepackage[small,compact]{titlesec} 

%\usepackage{pxfonts}
%\usepackage{isomath}
\usepackage{mathpazo}
%\usepackage{arev} %     (Arev/Vera Sans)
%\usepackage{eulervm} %_   (Euler Math)
%\usepackage{fixmath} %  (Computer Modern)
%\usepackage{hvmath} %_   (HV-Math/Helvetica)
%\usepackage{tmmath} %_   (TM-Math/Times)
%\usepackage{cmbright}
%\usepackage{ccfonts} \usepackage[T1]{fontenc}
%\usepackage[garamond]{mathdesign}
\usepackage{color}
\usepackage{ulem}

\newcommand{\argmax}{\operatornamewithlimits{arg\,max}}
\newcommand{\argmin}{\operatornamewithlimits{arg\,min}}
\def\inprobLOW{\rightarrow_p}
\def\inprobHIGH{\,{\buildrel p \over \rightarrow}\,} 
\def\inprob{\,{\inprobHIGH}\,} 
\def\indist{\,{\buildrel d \over \rightarrow}\,} 

\newtheorem{theorem}{Theorem}[section]
\newtheorem{conjecture}{Conjecture}[section]
\newtheorem{corollary}{Corollary}[section]
\newtheorem{lemma}{Lemma}[section]
\newtheorem{proposition}{Proposition}[section]
\theoremstyle{definition}
\newtheorem{assumption}{}[section]
%\renewcommand{\theassumption}{C\arabic{assumption}}
\newtheorem{definition}{Definition}[section]
\newtheorem{step}{Step}[section]
\newtheorem{remark}{Comment}[section]
\newtheorem{example}{Example}[section]
\newtheorem*{example*}{Example}

\linespread{1.1}

\pagestyle{fancy}
%\renewcommand{\sectionmark}[1]{\markright{#1}{}}
\fancyhead{}
\fancyfoot{} 
%\fancyhead[LE,LO]{\tiny{\thepage}}
\fancyhead[CE,CO]{\tiny{\rightmark}}
\fancyfoot[C]{\small{\thepage}}
\renewcommand{\headrulewidth}{0pt}
\renewcommand{\footrulewidth}{0pt}

\fancypagestyle{plain}{%
\fancyhf{} % clear all header and footer fields
\fancyfoot[C]{\small{\thepage}} % except the center
\renewcommand{\headrulewidth}{0pt}
\renewcommand{\footrulewidth}{0pt}}

\makeatletter
\renewcommand{\@maketitle}{
  \null 
  \begin{center}%
    \rule{\linewidth}{1pt} 
    {\Large \textbf{\textsc{\@title}}} \par
    {\normalsize \textsc{Paul Schrimpf}} \par
    {\normalsize \textsc{\@date}} \par
    {\small \textsc{University of British Columbia}} \par
    {\small \textsc{Economics 526}} \par
    \rule{\linewidth}{1pt} 
  \end{center}%
  \par \vskip 0.9em
}
\makeatother

\title{Sets, Numbers, and Proofs}
\date{\today}

\begin{document}

\maketitle
 
This lecture covers some of the most basic elements of mathematics ---
sets, proofs, and numbers. Section \ref{s:sets} introduces sets and
some related concepts. The summer math review covered logical
propistions and the operations. Section \ref{s:setLogic} explores the
relationship between sets and logic. Carter sections 1.1 and 1.2 are
good reading for more information on sets and orderings.  Section
\ref{s:card} briefly discusses cardinality and introduces countable
and uncountable sets. Section \ref{s:numbers} is about familiar sets
of numbers, including the integers, rationals, and real numbers. The
properties of these sets of numbers that make them distinct are
discussed in detail.  This lecture will not explicitly discuss proofs,
but it does include some examples of proofs. For an explicit
discussion of proof techniques, see Simon and Blume appendix A1 or
Kwong's ``Introduction to Logic and Proofs'' from the summer review
material.  Some of the material in this lecture, particularly the
sections on cardinality and constructing the real numbers, is more
abstract and less practical than most of what will be covered in this
course. Do not worry if you find this material difficult. It will not
be essential to the later parts of the course. This material is
included because I think it is especially interesting, and it
illustrates rigorous mathematical thinking and proofs well.

%%%%%%%%%%%%%%%%%%%%%%%%%%%%%%%%%%%%%%%%%%%%%%%%%%%%%%%%%%%%%%%%%%%%%% 
\section{Sets \label{s:sets}} 

A \textbf{set} is any well-specified collection of
elements. Sets are conventionally denoted by capital letters, and
elements of a set are usually denoted by lower case letters. The
notation, $a \in A$, means that $a$ is a member of the set $A$. A set
can be defined by listing its elements inside braces. For example,
\[ A = \{ 4, 5, 6 \} \]
means that $A$ is a set of three elements with members $4$, $5$, and
$6$. The members of a set need not be explicitly listed. Instead, they
can be defined by some logical relation. For example, the same set $A$
could be written
\begin{equation}
  A = \{ n \in \mathbb{N} : 3 < n < 7 \}  \label{e:s1}
\end{equation}
where $\mathbb{N} = \{1, 2, 3, ... \}$ is the natural numbers. The
expression in (\ref{e:s1}) could be read as, ``the set of natural
numbers, $n$, such that three is less than $n$ is less than 7.''
Sometimes $|$ will be used to mean ``such that'' instead of $:$. The
elements of sets need not be simple things like numbers. For example,
if $A_k = \{ n \in \mathbb{N}: n > k \}$ is the set of natural numbers
greater than $k$, then you could have a set of sets, $B = \{ A_1,
A_{10}, A_{6} \}$.  Sets are unordered, so the previous definition of
$B$ is the same as $B = \{ A_1, A_6, A_{10} \}$. Also, sets do not
contain duplicates, so for example, $\{ 1, 1, 2 \} \equiv \{1, 2
\}$. Sets can be empty. The empty set, also called the null set, is
denoted by $\emptyset$ or, less commonly, $\{ \}$.

\subsection{Economic examples}

Naturally sets appear all over economics.\footnote{These examples come
from chapter 1 of Carter.} 
\begin{example}[Sample space]
  In a random experiment, the set of all possible outcomes is called
  the \textbf{sample space}. E.g. for the roll of a dice, the sample
  space if $\{1,2 ,3 , 4, 5, 6\}$. An \textbf{event} is any subset of
  the sample space.
\end{example}

\begin{example}[Games]
  ALSO USE AS EXAMPLE OF PRODUCT AND POWER SET
\end{example}

\begin{example}[Consumption set]
  ALSO USE AS EXAMPLE OF PREFERENCE RELATION
\end{example}


\subsection{Set operations}

Given two sets $A$ and $B$, a new set can be formed with the following
three operations:
\begin{enumerate}
\item \textbf{Union:} $A \cup B = \{x: x\in A \text{ or } x \in B\}$.
\item \textbf{Intersect:} $A \cap B = \{x: x \in A \text{ and } x \in
  B\}$. 
\item \textbf{Minus:} $A \setminus B = \{ x: x\in A \text{ and }
  \not\in B \}$
\item \textbf{Product:} $A \times B = \{(x,y): x \in A, y \in B \}$
\item \textbf{Power set:} $\mathcal{P}(A) =$ set of all subsets of $A$ 
\end{enumerate}
Often, we will discuss sets that are all subsets of some universal
set, $U$. In this case, the \textbf{complement} of $A$ in $U$ is $A^c
= U \setminus A$. If every element of $A$ is also in $B$, then we say that
$B$ contains $A$ and write $B \supseteq A$, or $A$ is a subset of $B$
and write $A \subseteq B$. If, additionally, there exists $b \in B$
such that $b \not\in A$, then we say that $A$ is a proper subset of $B$,
which is denoted by $A \subset B$ or $B \supset A$. If we have an
indexed collection of sets, $\{A_k \}_{k \in \mathcal{K}}$, we may
may take the union or intersection of all these sets and denote it as
$\cup_{k \in \mathcal{K}} A_k $ or $\cap_{k \in \mathcal{K}} A_k$. 

\subsection{Equivalence between logic and set
  operations \label{s:setLogic}} 
The set operations, union, $\cup$, and intersect, $\cap$, appear
somewhat similar to the logical operations, or, $\vee$, and and,
$\wedge$. This is not merely a coincidence. Predicates and logical
operations are essentially equivalent to sets and set
operators. Recall from the summer review that a \textbf{predicate} is
a statement that may be either true or false depending on the value of
its variable. For example, $y^2 = 9$, is a predicate that depends on
the variable $y$. More formally, a predicate is a
function\footnote{Somewhat informally, a function from $X$ to $Y$
  takes each $x \in X$ and associates it with a single $y \in Y$. To
  be precise, a function is an ordered triple of sets $(X,Y,F)$ where
  $X$ is the domain, $Y$ is the codomain, and $F \subseteq X \times Y$
  consists of ordered pairs $(x,y): x\in X, y \in Y$. We will not need
  this formal definition in this course. }
from a set, $X$, to the set $\{T, F \}$.  

Given a predicate, $p(x)$, we can construct its truth set, $P = \{x
\in X: p(x) = T\}$. Similarly, given a set, $P$, we can construct the
predicate,
$p(x) = \begin{cases} T \text{ if } x \in P \\
  F \text{ if } x \not\in P \end{cases}$. The following theorem shows
that this mapping is one-to-one and establishes the relationship
between set and logic operations.
\begin{theorem}[Equivalence of truth sets and
  predicates \label{thm:setLogic}] Let $p(x)$
  and $q(x)$ be predicates and $P$ and $Q$ be
  the associated truth sets. Then if $\tilde{p}(x) = \begin{cases} T
    \text{ if } x \in P \\ 
    F \text{ if } x \not\in P \end{cases}$, we have $p(x) = \tilde{p}(x)
  \forall x \in X$. Also, 
  \begin{enumerate}
  \item\label{i1} $p(x) \wedge q(x)$ iff $x \in P \cap Q$
  \item\label{i2} $p(x) \vee q(x)$ iff $x \in P \cup Q$
  \item\label{i3} $\sim p(x)$ iff $x \in P^c$
  \item\label{i4} $p(x) \Rightarrow q(x)$ iff $P \subseteq Q$
  \end{enumerate}
\end{theorem}
\begin{proof}
  Suppose $p(x) = T$. Then $x \in P$ and $\tilde{p}(x) = T$ by
  definition. Similarly, if $p(x) = F$, then $\tilde{p}(x) = F$. Thus,
  $p(x) = \tilde{p}(x)$.  (\ref{i1})-(\ref{i3}) also follow directly
  from our definition of $p(x)$, $q(x)$, $P$, and $Q$.

  To show (\ref{i4}), suppose $p(x) \Rightarrow q(x)$. By definition
  of $P$, $\forall x \in P$, $p(x) = T$, which implies $q(x) = T$, so
  $x \in Q$. Conversely, suppose $P \subseteq Q$. Then $x \in P$
  implies both $p(x) = T$ and $x \in Q$, which also implies $q(x) =
  T$. 
\end{proof}

\begin{corollary}
  Let $X$, $Y$, and $Z$ be sets contained in some universe $U$. 
  The following sets from columns A and B are equivalent.
  \begin{centering}
    \begin{tabular}{l r}
      A       & B \\ \hline
      $(X \cup Y)^c$ & $X^c \cap Y^c$ \\
      $(X \cap Y)^c$ & $X^c \cup Y^c$ \\
      $X \cap (Y \cup Z)$ & $(X \cap Y) \cup (X \cap Z)$ \\
      $X \cup (Y \cap Z)$ & $(X \cup Y) \cap (X \cup Z)$
    \end{tabular}
  \end{centering}
\end{corollary}
\begin{proof}
  This follows from theorem \ref{thm:setLogic} above and Theorem 1
  from the review material on Logic. Let $x(a)$, $y(a)$, $z(a)$ be the
  predicates associated with $X$, $Y$, and $Z$. From theorem
  \ref{thm:setLogic}, $a \in (X \cup Y)^c$ iff $\sim(x(a) \vee
  y(a))$. From theorem 1 from the logic review, $\sim(x(a) \vee y(a))$
  iff $(\sim x(a) ) \wedge (\sim y(a))$. The remainder of the proof
  proceeds similarly and is left as an exercise.
\end{proof}

\subsection{Cardinality \label{s:card}}

\footnote{This section based on Chapter 2 of Rudin, Walter
  (1976). \textit{Principles of Mathematical
    Analysis}. McGraw-Hill.}Sometimes, we want to compare the size of
two sets. This is easy when sets are finite; we simply count how many
elements each has. It is not so easy to compare the size of infinite
sets. Consider, for example, the natural numbers, $\mathbb{N}$, the
integers $\mathbb{Z}$, rationals, $\mathbb{Q}$, and real numbers,
$\mathbb{R}$. Let $|A|$ denote the ``size'' of $A$ (we will define it
precisely later). We know that
\[ \mathbb{N} \subset \mathbb{Z} \subset \mathbb{Q} \subset
\mathbb{R}, \]
so it seems sensible to say that 
\[ |\mathbb{N}| < |\mathbb{Z}| < |\mathbb{Q}| < |\mathbb{R}|. \] 
On the other hand, the even integers are a subset of $\mathbb{Z}$, but
since we can write the set of even integers as $\{2x: x \in \mathbb{Z}
\}$, it doesn't seem like there are any more integers than even
integers. It was questions like these that led Georg Cantor to invent
set theory in the 1870's. 

A function (aka mapping), $f:A \rightarrow B$ is called
\textbf{one-to-one} (aka injective) if for every $b \in B$ the set
$\{a: f(a) = b \}$ is either a singleton or empty. $f$ is called
\textbf{onto} (aka surjective) if $\forall b \in B$ $\exists a \in A:
f(a) = b$.  If there exists a one-to-one mapping of $A$ onto $B$ (aka
bijection or one-to-one correspondence), then we say that $A$ and $B$
have the same \textbf{cardinal number} (or cardinality) and write $|A|
= |B|$.  Let $J_n = \{1, ..., n \}$. $A$ is \textbf{finite} if $|A| =
|J_n|$.  $A$ is \textbf{countable} if $|A| = |\mathbb{N}|$. $A$ is
\textbf{uncountable} if $A$ is neither finite nor countable.  You
should verify that the relation $|A| = |B|$ is reflexive ($|A| =
|A|$), symmetric ($|A| = |B|$ implies $|B| = |A|$), and transitive (if
$|A| = |B|$ and $|B| = |C|$ then $|A| = |C|$). 

\begin{lemma}
  $\mathbb{Z}$ is countable. 
\end{lemma}
\begin{proof}
  We can construct a bijection between $\mathbb{Z}$ and $\mathbb{N}$
  as follows:
  \begin{align*}
    \begin{matrix}
      \mathbb{Z}: & 0, & -1, & 1, & 2, & -2, & 3, & -3, ... \\
      \mathbb{N}: & 1, & 2, & 3, & 4, & 5, & 6, & 7,  ...
    \end{matrix}
  \end{align*}
  Or as a formula, $f:\mathbb{N} \rightarrow \mathbb{Z}$ with 
  \[ f(n) = \begin{cases} (n-1)/2 \text{ if } n \text{ odd} \\
    -n/2 \text{ if } n \text{ even}. \end{cases} \]  
\end{proof}

\begin{theorem} \label{thm:countsubset}
  Every infinite subset of a countable set $A$ is countable.
\end{theorem}
\begin{proof}
  $A$ is countable, so there exists a bijection from $A$ to
  $\mathbb{N}$. We can use this mapping to arrange the elements of $A$
  in a sequence, $\{a_{n}\}^\infty_{n=1}$. Let $B$ be an infinite
  subset of $A$. Let $n_1$ be the smallest number such that $a_{n_1}
  \in B$. Given $\{a_{n_1}, ... , a_{n_{k-1}}\}$, let $n_k$ be the
  smallest number such that $a_{n_k} \in B \\ \{a_{n_1}, ... ,
  a_{n_{k-1}}\}$. Such an $n_k$ always exists since $B$ is
  infinite. Also, $B = \{a_{n_k} \}_{k=1}^\infty$ since otherwise
  there would be a $b \in B$, but $b\not\in A$. Thus, $f(k) = a_{n_k}$ is
  a one-to-one correspondence between $B$ and $\mathbb{N}$.
\end{proof}

\begin{theorem}
  The rational numbers are countable.
\end{theorem}
\begin{proof}
  Consider the following arrangement of positive rational numbers:
  \begin{align*}
    \begin{array}{ccccc} 
      1/1 & 2/1 & 3/1 & 4/1 & \cdots \\
      1/2 & 2/2 & 3/2 & 4/2 & \cdots \\
      1/3 & 2/3 & 3/3 & 4/3 & \cdots \\
      \vdots &  &  & \ddots 
    \end{array}
  \end{align*}
  Starting in the top left and going back and forth diagonally, we get
  the following sequence:
  \begin{align*}
    1/1, 1/2, 2/1, 1/3, 2/2, 3/1, ...
  \end{align*}
  Adding zero and the negative rationals, we can write e.g.\
  \begin{align*}
    & 0,1/1, -1/1, 1/2, -1/2, 2/1, -2/1, 1/3, -1/3, 2/2, -2/2, 3/1,
    ... \\
    = & q_1, q_2, q_3 , q_4, ... \\
  \end{align*}
  Continuing on in this way, we could list all rational numbers. Some
  of these fractions represent the same number and can be
  removed. Thus, we obtain a correspondence between the rationals and
  an infinite subset of $\mathbb{N}$. However, by theorem
  \ref{thm:countsubset}, this subset is countable, so the rationals
  are also countable.
\end{proof}

\begin{theorem}
  The real numbers are uncountable.
\end{theorem}
\begin{proof}
  We have not rigorously defined the real numbers, so we will take for
  granted the following: every infinite decimal expansion,
  (e.g. $0.135436080...$) represents a unique real number in $[0,1)$,
  except for expansions that end in all zeros or nines, which are
  equivalent. Suppose we have constructed a surjective mapping from
  $\mathbb{N}$ to $(0,1)$. Then we can list all real numbers in
  $(0,1)$ as
  \begin{align*}
    r_1 = 0.d_{11} d_{12} d_{13} ... \\
    r_2 = 0.d_{21} d_{22} d_{23} ... \\
    \ddots
  \end{align*}
  where each $d_{ij} \in \{0,1,...,9\}$, and no expansion ends in all
  nines. Consider the following number $x^* = 0.d^*_1 d^*_2 d^*_3 ...$. where 
  \[ 
  d^*_n
  = \begin{cases} d_{nn} + 1 \text{ if } d_{nn} \neq 8 \\
    0 \text{ if } d_{nn} = 8 
  \end{cases} .
  \]  
  $x^*$ is in $(0,1)$, but $x^* \neq r_n$ for any $n$ because $d^*_n
  \neq d_{nn}$. Thus, we have a contradiction, and there cannot be a
  onto mapping from $\mathbb{N}$ to $(0,1)$.  If there is no
  surjective mapping from $\mathbb{N}$ to $(0,1)$, there can be no
  surjective mapping from $\mathbb{N}$ to $\mathbb{R}$ since $(0,1)
  \subset \mathbb{R}$.
\end{proof}
Countable sets are said to have cardinality $\aleph_0$ (``aleph
null''). Note that an implication of theorem \ref{thm:countsubset} is
that $\aleph_0$ is the smallest infinite cardinal number. The real
numbers have cardinality of the continuum, sometimes written
$2^{\aleph_0}$ or $\mathbf{c}$. You might be wondering whether there
are larger cardinal numbers. The answer is yes. The set of all subsets
of a set, $A$, called the \textbf{power set} of $A$, always has larger
cardinality $2^{|A|}$ (the proof of this is similar to the proof that
the real numbers are uncountable). 

A final question to ask yourself is whether there are sets with
cardinality between $\aleph_0$ and $2^{\aleph_0}$. The answer to that
question is whatever you want it to be. The conjecture that there are
no cardinal numbers between $\aleph_0$ and $2^{\aleph_0}$ is known as
the continuum hypothesis. It was proposed by Cantor in the 1870s. In
1900, Hilbert made a famous list of 23 important unsolved problems in
mathematics. The continuum hypothesis was the first. In 1940,
G\"{o}del showed that the continuum hypothesis cannot be disproved
from the standard axioms that lie at the foundation of mathematics.
In 1963, Cohen showed that the continuum hypothesis cannot be proved
from the standard axioms. This is an example of G\"{o}del's
incompleteness theorem, a very interesting result that we won't be
able to cover in this course. Loosely speaking, G\"{o}del's
incompleteness theorem says that for any non-trivial set of
assumptions and system of logic, you can make statements consistent
with the system of logic that cannot be proven or disproven from the
assumptions.

%%%%%%%%%%%%%%%%%%%%%%%%%%%%%%%%%%%%%%%%%%%%%%%%%%%%%%%%%%%%%%%%%%%%%%
\section{Numbers \label{s:numbers}}

We have been assuming familiarity with the natural numbers, integers,
rationals, and real numbers. This section explores some properties of
these sets of numbers and heuristically describes how these sets of
numbers are constructed. It may appear silly and slightly confusing to
try to be ``rigorous'' about something like real numbers that we
already feel like we understand.  Much of mathematics is about finding
and describing patterns that apply to abstract objects. Many of the
abstract objects that we will study are similar to the real numbers in
some ways, but different in others. Examples of things that are
similar to the real numbers include complex numbers, vector spaces,
matrices, and sets of functions. Some of these things we will be able
to add and multiple just like real numbers, but not all of them. A
natural sort of question is: this class of objects shares properties
X, Y, and Z with the real numbers; what theorems that we know about
the real numbers will also be true of this class of objects? Before
answering this sort of question we have to be precise about what
properties the real numbers have.

We will take for granted that we understand what the natural numbers
are. Note, however, that it is possible to rigorously construct the
natural numbers from a simple list of assumptions using logic or set
theory. We will also take for given that we know how to add and
multiply natural numbers. Addition has the following nice properties.
\begin{itemize}
\item[1] \emph{Closure} if $a, b \in \mathbb{N}$, so is $a + b$ 
\item[2] \emph{Associative} $a + (b + c) = (a + b) + c$. 
\end{itemize}
If we demand that addition also has 
\begin{itemize}
\item[3] \emph{Identity} $\exists 0$ s.t. $a + 0 = a$,
\item[4] \emph{Inverse} $\forall a$, $\exists b$ s.t. $a + b = 0$
\end{itemize}
then we must expand the natural numbers to include the integers,
$\mathbb{Z}$. Multiplication also satisfies these four analogous
properties:
\begin{itemize}
\item[1'] \emph{Closure} if $a, b \in A$, so is $ab$ 
\item[2'] \emph{Associative} $a (b c) = (a b)  c$. 
\item[3'] \emph{Identity} $\exists 1$ s.t. $a 1= a$,
\item[4'] \emph{Inverse} $\forall a \neq 0$, $\exists b$ s.t. $ab = 1$
\end{itemize}
However, if we want multiplicative inverses to exist for all $z\in
\mathbb{Z}$, then we must further expand our set of numbers to the
rationals, $\mathbb{Q}$. Addition and multiplication are also
\begin{itemize}
\item[5] \emph{Commutative} $ a + b = b + a$
\item[6] \emph{Distributive} $a (b + c) = ab + ac$
\end{itemize}
To summarize: if we start with the natural numbers, and then demand
that multiplication and addition have these six properties, we end up
with the rational numbers. 

More generally, we could study a set $A$ combined with one or two
operations that satisfy certain properties. The branch of mathematics
that studies these sort of objects is abstract algebra. We will not be
studying algebra in detail, but it may be useful to be familiar with
some basic terms. A \textbf{group} is a set and operation,
$(A,\oplus)$ such that $A$ is closed under $\oplus$, $\oplus$ is
associative, there exists an identity, and inverses exist under
$\oplus$ (i.e.\ properties 1-4). If $\oplus$ is also commutative, we
call $(A,\oplus)$ an abelian (or commutative) group. Examples of
groups include $(\mathbb{Z},+)$ and $(\mathbb{Q},\cdot)$. A
\textbf{ring} is a set with two operations, $(A,\oplus,\odot)$ such
that $(A,\oplus)$ is a group, and $\odot$ has properties 1-3 and
6. $(\mathbb{Z},+,\cdot)$ is a ring.  One ring that will come up
repeatedly in this course is the set of all $n$ by $n$ matrices with
the usual matrix addition and multiplication. A \textbf{field} is set
with two operations such that 1-6 hold for both
operations. $(\mathbb{Q},+,\cdot)$ is a field. Another field that you
may have encountered is the complex numbers with the usual addition
and multiplication. If you're interested you may want to verify that
the integers modulo any number is a ring, and the integers modulo any
prime number if a field.

\subsection{Real numbers}

The rational numbers are pretty nice; they're a field with the six
properties listed above. However, $\mathbb{Q}$ does not contain all
the numbers that we think it should. For example,
\begin{theorem}
  $\sqrt{2} \not\in \mathbb{Q}$
\end{theorem}
\begin{proof}
  Suppose $\sqrt{2} \in \mathbb{Q}$. Then $\sqrt{2} = p / q$ where $p$
  and $q$ are not both even. If we square both sides, we get
  \begin{align*}
    2 = & p^2 / q^2 \\
    2 q^2 = & p^2.
  \end{align*}
  Hence, $p^2$ must be even. From the review, then $p$ must also be
  even, say $p = 2m$. Then we have
  \begin{align*}
    2 q^2 = & 2(2 m^2) \\
    q^2 = & 2 m^2,
  \end{align*}
  which means $q$ must also be even, contrary to our starting
  assumption. 
\end{proof}
Apparently, the rationals have some holes in them that we should fill
in. To do so in a unique way, we need to define another property of
the rational numbers. A \textbf{totally ordered set} is a set, $A$,
and a relation, $<$, such that (i) (total) $\forall a,b \in A$ either
$a < b$ or $a = b$ or $a > b$; and (ii) (transitive) if $a < b$ and $b
< c$ then $a < c$. An \textbf{ordered field} is a field that is a
totally ordered set and addition and multiplication preserve the
ordering in that (i) if $b<c$ then $a + b < a + c$ (ii) if $a>0$ and
$b>0$ then $ab>0$.

We need one more definition. Simon and Blume state that one property
of real numbers that will be used throughout the book is the least
upper bound property. It turns out that this property is not only
useful; it lies at the foundation of the real numbers. Let $S$ be an
ordered set and $A \subset S$. $s \in S$ is an \textbf{upper bound} of
$A$ if $s > a \forall a \in A$. $s$ is a \textbf{least upper bound}
(aka supremum) of $A$ if $s$ is an upper bound of $A$ and if $r < s$,
then $r$ is not an upper bound of $A$. $S$ has the
\textbf{least-upper-bound property} (aka complete or Dedekind
complete) if whenever $A \subset S$ has an upper bound, $A$ has a
least upper bound. Given that $\sqrt{2} \not\in \mathbb{Q}$, it should
not be surprising that the rational numbers are not complete. You will
prove this fact on the first problem set. 

\begin{theorem}[Real numbers]
  There exists an ordered field, $\mathbb{R}$, that has the least
  upper bound property. $\mathbb{R}$ contains $\mathbb{Q}$. Moreoever,
  $\mathbb{R}$ is ``unique''.
\end{theorem}

The proof of this is surprisingly long, so we will not go over it in
detail. Existence can be proven by construction. One method involves
constructing real numbers as Dedekind cuts. A Dedekind cut is a
nonempty subset of the rationals, $A \subset \mathbb{Q}$, such that
(i) if $p \in A$, $q \in \mathbb{Q}$, and $q < p$, then $q \in A$ and
(ii) if $p \in A$ then $p<r$ for some $r \in A$ (i.e. $A$ has no
greatest element. For example, the Dedekind cut associated with
$\sqrt{2}$ would be $\{p \in \mathbb{Q}: p^2 < 2\}$). We would then
define addition, multiplication, and ordering of these cuts in the
natural way and verify that all the properties above are
satisfied. See Rudin for details if you are interested.

The ``uniqueness'' is harder to prove. $\mathbb{R}$ is unique in the
sense that any two ordered fields with the least-upper-bound property
are isomorphic (there exists a bijection between them that preserves
multiplication, addition, and ordering). The proof proceeds by
supposing that $\mathbb{R}$ and $\mathbb{F}$ are two ordered fields
with the least-upper-bound property and then shows that there is an
isomorphism between them.



\end{document}
