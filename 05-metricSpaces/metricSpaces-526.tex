
\documentclass[12pt,reqno]{amsart}

\usepackage{amsmath}
\usepackage{amsfonts}
\usepackage{amssymb}

\usepackage{graphicx}
%\usepackage{epstopdf}
\usepackage[colorlinks=true]{hyperref}
\usepackage[left=1in,right=1in,top=0.9in,bottom=0.9in]{geometry}
\usepackage{multirow}
\usepackage{verbatim}
\usepackage{fancyhdr}
\usepackage{mdframed}
\usepackage{natbib}
\usepackage{multicol}
\renewcommand{\cite}{\citet}

%\usepackage[small,compact]{titlesec} 

%\usepackage{pxfonts}
%\usepackage{isomath}
\usepackage{mathpazo}
%\usepackage{arev} %     (Arev/Vera Sans)
%\usepackage{eulervm} %_   (Euler Math)
%\usepackage{fixmath} %  (Computer Modern)
%\usepackage{hvmath} %_   (HV-Math/Helvetica)
%\usepackage{tmmath} %_   (TM-Math/Times)
%\usepackage{cmbright}
%\usepackage{ccfonts} \usepackage[T1]{fontenc}
%\usepackage[garamond]{mathdesign}
\usepackage{xcolor}
\usepackage[normalem]{ulem}

\newtheorem{theorem}{Theorem}[section]
\newtheorem{conjecture}{Conjecture}[section]
\newtheorem{corollary}{Corollary}[section]
\newtheorem{lemma}{Lemma}[section]
\newtheorem{proposition}{Proposition}[section]
\theoremstyle{definition}
\newtheorem{assumption}{}[section]
%\renewcommand{\theassumption}{C\arabic{assumption}}
\newtheorem{definition}{Definition}[section]

\newtheorem{step}{Step}[section]
\newtheorem{rem}{Comment}[section]
\newtheorem{ex}{Example}[section]
\newtheorem{hist}{History}[section]
\newtheorem*{ex*}{Example}
\newtheorem{exer}{Exercise}[section]


\newenvironment{example}
  {\begin{mdframed}\begin{ex}}
  {\end{ex}\end{mdframed}}

\newenvironment{history}
  {\begin{mdframed}\begin{hist}}
  {\end{hist}\end{mdframed}}

\newenvironment{exercise}
  {\begin{mdframed}\begin{exer}}
  {\end{exer}\end{mdframed}}

\newenvironment{remark}
  {\begin{mdframed}\begin{rem}}
  {\end{rem}\end{mdframed}}



\linespread{1.1}

\pagestyle{fancy}
%\renewcommand{\sectionmark}[1]{\markright{#1}{}}
\fancyhead{}
\fancyfoot{} 
%\fancyhead[LE,LO]{\tiny{\thepage}}
\fancyhead[CE,CO]{\tiny{\rightmark}}
\fancyfoot[C]{\small{\thepage}}
\renewcommand{\headrulewidth}{0pt}
\renewcommand{\footrulewidth}{0pt}

\fancypagestyle{plain}{%
\fancyhf{} % clear all header and footer fields
\fancyfoot[C]{\small{\thepage}} % except the center
\renewcommand{\headrulewidth}{0pt}
\renewcommand{\footrulewidth}{0pt}}

\makeatletter
\renewcommand{\@maketitle}{
  \null 
  \begin{center}%
    \rule{\linewidth}{1pt} 
    {\Large \textbf{\textsc{\@title}}} \par
    {\normalsize \textsc{Paul Schrimpf}} \par
    {\normalsize \textsc{\@date}} \par
    {\small \textsc{University of British Columbia}} \par
    {\small \textsc{Economics 526}} \par
    \rule{\linewidth}{1pt} 
  \end{center}%
  \par \vskip 0.9em
}
\makeatother

\definecolor{ubcblue}{cmyk}{1.00,.62,.0,.52}
\definecolor{ubcgrey}{cmyk}{.42,.08,.0,.40}

\hypersetup{
  linkcolor  = ubcblue,
  citecolor  = ubcgrey,
  urlcolor   = ubcblue,
  colorlinks = true
}

\usepackage{mathtools}
\newcommand{\verteq}{\rotatebox{90}{$\,=$}}
\newcommand{\equalto}[2]{\underset{\scriptstyle\overset{\mkern4mu\verteq}{#2}}{#1}}
\newcommand{\nulls}{\mathrm{null}}
\newcommand{\range}{\mathrm{range}}

\newcommand{\argmax}{\operatornamewithlimits{arg\,max}}
\newcommand{\argmin}{\operatornamewithlimits{arg\,min}}
\def\inprobLOW{\rightarrow_p}
\def\inprobHIGH{\,{\buildrel p \over \rightarrow}\,} 
\def\inprob{\,{\inprobHIGH}\,} 
\def\indist{\,{\buildrel d \over \rightarrow}\,} 
\def\F{\mathbb{F}}
\def\R{\mathbb{R}}
\def\Er{\mathrm{E}}
\def\Pr{\mathrm{P}}
\def\En{\mathbb{E}_n}
\def\Pn{\mathbb{P}_n}
\newcommand{\gmatrix}[1]{\begin{pmatrix} {#1}_{11} & \cdots &
    {#1}_{1n} \\ \vdots & \ddots & \vdots \\ {#1}_{m1} & \cdots &
    {#1}_{mn} \end{pmatrix}}
\newcommand{\iprod}[2]{\left\langle {#1} , {#2} \right\rangle}
\newcommand{\norm}[1]{\left\Vert {#1} \right\Vert}
\newcommand{\abs}[1]{\left\vert {#1} \right\vert}
\renewcommand{\det}{\mathrm{det}}
\newcommand{\rank}{\mathrm{rank}}
\newcommand{\spn}{\mathrm{span}}
\newcommand{\row}{\mathrm{Row}}
\newcommand{\col}{\mathrm{Col}}
\renewcommand{\dim}{\mathrm{dim}}
\newcommand{\prefeq}{\succeq}
\newcommand{\pref}{\succ}
\newcommand{\seq}[1]{\{{#1}_n \}_{n=1}^\infty }
\renewcommand{\to}{{\rightarrow}}
\newcommand{\corres}{\overrightarrow{\rightarrow}}
\def\rel{\,{\buildrel R \over \sim}\,}

 


\title{Metric spaces, topology, and continuity}
\date{\today}

\begin{document}

\maketitle

\begin{quotation}
  Always consider a problem under the minimum structure in which it
  makes sense. ... one is naturally led to the study of problems with
  a kind of minimal and intrinsic structure. Besides the fact that it
  is much easier to find the crux of the matter in a simple structure
  than in a complicated one, there are not so many really basic
  structures, so one can hope that they will remain of interest for a
  very long time. --- \cite{talagrand2005}
\end{quotation}

This lecture focuses on metric spaces, topology, and continuity.
Similar material is covered in chapters 12 and 29, of \cite{sb1994},
1.3 of \cite{carter2001}, chapter 2 of \cite{fuente2000}, and any
textbook on real analysis, such as \cite{rudin1976} or \cite{tao2006}.

Many proofs in mathematics rely on showing some approximation can be
made arbitrarily close. For example, showing that the first order
conditions are necessary for an optimum relied on approximating the
objective function with a first order expansion. To facilitate such
arguments, we need some good ways to measure closeness. The most
obvious way is to define some sort of distance. This is what metric
spaces do. Given a measure of distance, we can think about convergence
of sequences, continuity of functions, et cetera. We will see that a
distance also gives rise to a classification of sets (as open and
closed), and these open and close sets can also be used to describe
convergence of sequences, continuity, and so on. This is what topology
is about. Topology gives us another way of thinking about closeness
without referring directly to distance. This can sometimes lead to
easier proofs and new insights. There also exist sets that can be
assigned a topology, but for which there does not 
exist an equivalent distance.\footnote{In econometrics, you might
  eventually encounter weak convergence and hear references to the
  topology of weak convergence. This is an example of a non-metrizable
  topology.} 

\section{Sequences and limits}

A \textbf{sequence} is a list of elements, $\{x_1, x_2, ... \}$ or
$\seq{x}$ or sometimes just $\{x_n\}$. Although the notation for a
sequence is similar to the notation for a set, they should not be
confused. Sequences are different from sets in that the order of
elements in a sequence matters, and the same element can appear many
times in a sequence.  Some examples of sequences with $x_i \in \R$
include
\begin{enumerate}
\item\label{sfib} $\{ 1, 1, 2, 3, 5, 8, ... \}$
\item\label{sc} $\{ 1, \frac{1}{2}, \frac{1}{3}, \frac{1}{4}, ... \}$
\item\label{sd} $\{\frac{1}{2}, \frac{-2}{3}, \frac{3}{4}, \frac{-4}{5},
  \frac{5}{6}, ... \}$
\end{enumerate}
Some sequences, like \ref{sc}, have elements that all get closer and
closer to some fixed point. We say that these types of sequences
converge. A sequence that does not converge diverges. Some divergent
sequences like \ref{sfib}, increase without bound. Other divergent
sequences, like \ref{sd}, are bounded, but they do not converge to any
single point. 

To analyze sequences with elements that are not necessarily real
numbers, we need to be able to say how far apart the entries in the
sequence are. 
\begin{definition}
  A \textbf{metric space} is a set, $X$, and function $d:X\times X
  \rightarrow \R$ called a \textbf{metric} (or distance) such that
  $\forall x, y, z \in X$
  \begin{enumerate}
  \item $d(x,y) > 0$ unless $x=y$ and then $d(x,x) = 0$
  \item (symmetry) $d(x,y) = d(y,x)$
  \item (triangle inequality) $d(x,y) \leq d(x,z) + d(z,y)$.
  \end{enumerate}
\end{definition}
\begin{example}
  $\R$ is a metric space with $d(x,y) = |x - y|$. 
\end{example}
\begin{example}
  Any normed vector space\footnote{In the past, we covered vector
    spaces before metric spaces, so this example made more sense
    here. Now it can be safely skipped.} is a metric space with $d(x,y) =
  \norm{x-y}$. 
\end{example}
The most common metric space that we will encounter will be $\R^n$
with the Euclidean metric, $d(x,y) = \norm{x-y} = \sqrt{\sum_{i=1}^n
  (x_i - y_i)^2}$.  
\begin{definition}
  A sequence $\{x_n\}_{n=1}^\infty$ in a metric space \textbf{converges} to $x$
  if $\forall \epsilon > 0$ $\exists N$ such that 
  \[ d(x_n, x) < \epsilon \]
  for all $n \geq N$. We call $x$ the \textbf{limit} of $\seq{x}$ and
  write $\lim_{n \rightarrow \infty} x_n = x$ or $x_n \rightarrow x$. 
\end{definition}
\begin{example}
  The sequence $\{ 1, \frac{1}{2}, \frac{1}{3}, \frac{1}{4}, ... \} =
  \{1/n\}_{n=1}^\infty$ converges. To see this, take any
  $\epsilon>0$. Then $\exists$ $N$ such that $1/N < \epsilon$. For all
  $n \geq N$, $d(1/n,0) = 1/n < \epsilon$. 
\end{example}
If a sequence does not converge, it diverges. 
\begin{example}
  The Fibonacci sequence, $\{ 1, 1, 2, 3, 5, 8, ... \}$ diverges.
\end{example}
\begin{definition}
  $a$ is an \textbf{accumulation point} of $\seq{x}$ if $\forall
  \epsilon > 0$ $\exists$ infinitely many $x_i$ such that 
  \[ d(a, x_i ) < \epsilon. \]
\end{definition}
\begin{example}
  The sequence $\{\frac{1}{2}, \frac{-2}{3}, \frac{3}{4},
  \frac{-4}{5}, \frac{5}{6}, ... \}$ has two accumulation points, $1$ 
  and $-1$.
\end{example}
The limit of any convergent sequence is an accumulation point of the
sequence. In fact, it is the only accumulation point.
\begin{lemma}
  If $x_n \rightarrow x$, then $x$ is the only accumulation point of
  $\seq{x}$. 
\end{lemma}
\begin{proof}
  Let $\epsilon>0$ be given. By the definition of convergence,
  $\exists N$ such that 
  \[ d(x_n,x) < \epsilon \]
  for all $n \geq N$. $\{n\in \mathbb{N}: n\geq N\}$ is infinite, so
  $x$ is an accumulation point. 

  Suppose $x'$ is another accumulation point. Then $\forall
  \epsilon>0$ $\exists N$ and $N'$ such that if $n \geq N$ and $n \geq
  N'$, then $d(x_n, x) < \epsilon/2$ and $d(x_n,x') < \epsilon/2)$. By
  the triangle inequality, $d(x,x') \leq d(x_n,x') + d(x_n,x) <
  \epsilon$. Since this inequality holds for any $\epsilon$, it must
  be that $d(x,x') = 0$. $d$ is a metric, so then $x = x'$, and the
  limit of sequence is the sequence's unique accumulation point. 
\end{proof}
The third example of a sequence at the start of this section,
\ref{sd}, shows that the converse of this lemma is false. Not every
accumulation point is a limit. 
\begin{definition}
  Given $\seq{x}$ and any sequence of positive integers, $\{n_k\}$
  such that $n_1 < n_2 < ... $ we call $\{x_{n_k}\}$ a
  \textbf{subsequence} of $\seq{x}$. 
\end{definition}
In example \ref{sd}, there are two accumulation points, $-1$ and $1$,
and you can find subsequences that converge to these points. 
\begin{lemma}
  Let $a$ be an accumulation point of $\{x_n\}$. Then $\exists$
  a subsequence that converges to $a$. 
\end{lemma}
\begin{proof}
  We can construct a subsequence as follows. Let $\{\epsilon_k\}$ be a
  sequence that converges to zero with $\epsilon_k >0 \forall k$, (for
  example, $\epsilon_k = 1/k$). By the definition of accumulation
  point, for each $\epsilon_k$ $\exists$ infinitely many $x_n$ such
  that 
  \begin{align}
    d(x_n, a) < \epsilon_k \label{ieq:a}
  \end{align}
  Pick any $x_{n_1}$ such that (\ref{ieq:a}) holds for
  $\epsilon_1$. For $k>1$, pick $n_k \neq n_{j}$ for all $j<k$ and
  such that (\ref{ieq:a}) holds for $\epsilon_k$. Such an $n_k$ always
  exists because there are infinite $x_n$ that satisfy
  (\ref{ieq:a}). By construction, $\lim_{k\rightarrow \infty} x_{n_k}
  = a$ (you should verify this using the definition of limit).
\end{proof}

Convergence of sequences is often preserved by arithmetic
operations, as in the following two theorems.
\begin{theorem}
  Let $\{x_n\}$ and $\{y_n\}$ be sequences in $\R^n$ (or any normed
  vector space)
  $V$. If $x_n \to x$ and $y_n \to y$, then
  \[ x_n + y_n \to x + y. \]
\end{theorem}
\begin{proof}
  Let $\epsilon > 0$ be given. Then $\exists$ $N_x$ such that for all
  $n \geq N_x$, 
  \[ d(x_n,x) < \epsilon/2,\] and $\exists N_y$ such that
  for all $n \geq N_y$,  
  \[ d(y_n,y) < \epsilon/2. \]
  Let $N =\max\{N_x,N_y\}$. Then for all $n \geq N$, 
  \begin{align*}
    d(x_n + y_n,x+y) = \norm{(x_n + y_n) - (x+y)} \leq & \norm{x_n -
      x} + \norm{y_n - y} \\
    < & \epsilon/2 + \epsilon /2  = \epsilon.
  \end{align*}  
\end{proof}
\begin{theorem}
  Let $\{x_n \}$ be a sequence in a normed vector space with scalar
  field $\R$ and let $\{c_n\}$ be a sequence in $\R$. If
  $x_n \to x$ and $c_n \to c$ then 
  \[ x_n c_n \to x c. \]
\end{theorem}
\begin{proof}
  Left as an exercise.
\end{proof}
In fact, later we will see that if $f(\cdot,\cdot)$ is
continuous, then $\lim f(x_n, y_n) = f(x,y)$. The previous two
theorems are examples of this with $f(x,y) = x+y$ and $f(c,x) = c x$,
respectively.

\subsection{Series}

Infinite sums or series are formally defined as the limit of the
sequence of partial sums. 
\begin{definition}
  Let $\seq{x}$ be a sequence in a normed vector space. Let $s_n =
  \sum_{i=1}^n x_i$ denote the sum of the first $n$ elements of the
  sequence. We call $s_n$ the $n$th partial sum. We define the sum of
  all the $x_i$s as
  \[ \sum_{i=1}^\infty x_i \equiv \lim_{n \to \infty} s_n \]
  This is called a(n infinite) \textbf{series}. 
\end{definition}

\begin{example}
  Let $\beta \in \R$. $\sum_{i=0}^\infty \beta^i$ is called a
  geometric series. Geometric series appear often in
  economics, where $\beta$ will be the subjective discount factor or
  perhaps $1/(1+r)$.  Notice  that
  \begin{align*}
    s_n = & 1 + \beta + \beta^2 + \cdots + \beta^n \\
    = & 1 + \beta ( 1 + \beta + \cdots + \beta^{n-1} ) \\
    = & 1 + \beta ( 1 + \beta + \cdots + \beta^{n-1} + \beta^n) -
    \beta^{n+1} \\
    s_n(1 - \beta) = & 1 - \beta^{n+1} \\
    s_n = & \frac{1 - \beta^{n+1}}{1-\beta} ,
  \end{align*}
  so, 
  \begin{align*}
    \sum_{i=0}^\infty \beta^i = & \lim s_n \\
    = & \lim \frac{1 - \beta^{n+1}}{1-\beta} \\
    = & \frac{1}{1-\beta} \text{ if } |\beta|<1.
  \end{align*}
\end{example}


%%%%%%%%%%%%%%%%%%%%%%%%%%%%%%%%%%%%%%%%%%%%%%%%%%%%%%%%%%%%%%%%%%%%%%
\subsection{Cauchy sequences}

We have defined convergent sequences as ones whose entries all get
close to a fixed limit point. This means that all the entries of the
sequence are also getting closer together. You might imagine a
sequence where the entries get close together without necessarily
reaching a fixed limit.
\begin{definition}
  A sequence $\seq{x}$ is a \textbf{Cauchy} sequence if for any
  $\epsilon > 0$ $\exists N$ such that for all $i,j\geq N$,
  $d(x_i,x_j) < \epsilon$.
\end{definition}
It turns that in $\R^n$ Cauchy sequences and convergent sequences are
the same. This is a consequence of the way the real numbers are
defined.  
\begin{assumption}
  Every Cauchy sequence in $\R$ converges.
\end{assumption}
There is a more detailed discussion of this in Chapter 29.1 of Simon
and Blume. If you want more practice with the sort of proofs in this
lecture, it would be good to read that section. The convergence of
Cauchy sequences in the real numbers is equivalent to the least upper
bound property that is discussed in the appendix to the lecture notes
on sets.

Cauchy sequences do not converge in all metric spaces. For example,
the rational numbers are a metric space, and any sequence of rationals
that converges to an irrational number in $\R$ is a Cauchy sequence in
$\mathbb{Q}$ but has no limit in $\mathbb{Q}$. Having Cauchy sequences
converge is necessary for proving many theorems, so we have a special
name for metric spaces where Cauchy sequences converge.
\begin{definition}
  A metric space, $X$, is \textbf{complete} if every Cauchy sequence of
  points in $X$ converges in $X$.
\end{definition}
%Completeness is important for so many results that complete version of
%vector spaces are named after the mathematicians who first studied
%them extensively. A \textbf{Banach space} is a complete normed vector
%space. A \textbf{Hilbert space} is a complete inner product
%space.\footnote{According to legend, the term Hilbert space became
%  widely used without Hilbert himself realizing it. In 1929 von Neumann gave a lecture
%  about some theorem involving Hilbert spaces with David Hilbert in
%  attendance. At the end of the lecture, Hilbert asked, ``Dr. von
%  Neumann, I would very much like to know, what after all is a Hilbert
%  space?'' \cite{krantz2002}.} Since any inner product space is a normed vector space with
%norm $\norm{x} = \sqrt{\iprod{x}{x}}$, any Hilbert space is also a
%Banach space. 
\begin{example}
  $\R^n$ is a complete metric space.  A brief argument follows. You
  may want to state the details as an exercise. Let $\{\mathbf{x}_i\}$
  be a Cauchy sequence in $\R^n$. Each coordinate of
  $\mathbf{x}_i = (x_{1i}, ... , x_{ni})$ is a Cauchy sequence in
  $\R$. $\R$ is complete, so each coordinate has a limit,
  $x_{ji} \to x_j$ for $j = 1, ..., n$. Finally, show that
  $\mathbf{x} = (x_1, ..., x_n)$ is the limit of the original sequence
  in $\R^n$.
\end{example}
\begin{example}
  $\ell^p = \{ (x_1, x_2, ...)\, s.t.\, x_i \in \R, \sum_{i=1}^\infty |x_i|^p <
  \infty \}$ with metric
  \[ d_p(x,y) = \left( \sum_{i=1}^\infty |x_i - y_i|^p \right)^{1/p} \]
  is a complete metric space. 
    
  Showing that $\ell^p$ is complete is slightly tricky because you
  have deal with a sequence of $\mathbf{x}_i \in \ell^p$, each element
  of which is itself an infinite sequence. 
  
  To show that $\ell^p$ is complete, let $\seq{\mathbf{x}}$ be a
  Cauchy sequence. Let $\norm{x-y}_p$ denote $d_p(x,y)$.

  Denote the elements of $\mathbf{x}_i$ by $x_{i1},
  x_{i2}, ... $. First, let's show that for any $n$, $x_{1n}, x_{2n},
  ... $ is a Cauchy sequence in $\R$. Let $\epsilon > 0$. Since
  $\seq{\mathbf{x}}$ is Cauchy, $\exists N_\epsilon$ such that for all $i,j
  \geq N_\epsilon$, 
  \[  \norm{\mathbf{x}_i - \mathbf{x}_j }_p < \epsilon. \]
  Since 
  \[ \norm{\mathbf{x}_i - \mathbf{x}_j }_p^p = \left( \sum_{m=1}^\infty
    \abs{x_{i m} - x_{j m}}^p \right) \]
  All terms in the sum on the right are non-negative and the sum
  includes $\abs{x_{i n} - x_{jn}}$, so 
  \begin{align*} \abs{x_{i n} - x_{j n} }^p \leq & \norm{\mathbf{x}_i -
      \mathbf{x}_j }_p^p \\
    \abs{x_{i n} - x_{j n} } \leq & 
    \norm{\mathbf{x}_i - \mathbf{x}_j }_p  
  \end{align*}
  Therefore, $\abs{x_{i n} - x_{j n}} < \epsilon$ for all $i,j \geq
  N_\epsilon$, i.e. $x_{1n}, x_{2n}, ...$ is a Cauchy sequence in
  $\R$. $\R$ is complete, so it has some limit. Denote the limit by
  $x^\ast_n$.  

  Now we will show that $\mathbf{x}^\ast = (x_{1}^\ast, x_{2}^\ast,
  ... )$ is the limit of $\seq{\mathbf{x}}$.  First, we should show
  that $\mathbf{x}^\ast \in \ell^p$. Let 
  \begin{align*}
    s_m^\ast = \sum_{n=1}^m |x^\ast_n|^p.
  \end{align*}
  We need to show that $\lim s_m^\ast $ exists.  Since
  $\seq{\mathbf{x}}$ is Cauchy, $\exists j$ such that if $i \geq j$,
  $\norm{\mathbf{x}_i - \mathbf{x}_j} < 1$. Using the triangle
  inequality, 
  \[ \norm{\mathbf{x}_i} \leq \norm{\mathbf{x}_i - \mathbf{x}_j} +
  \norm{\mathbf{x}_j} = 1 + \norm{\mathbf{x}_j} \equiv M \]
  for all $i \geq j$ and some fixed $j$. Thus, $\norm{\mathbf{x}_i}
  \leq M$ for some constant $M$ and all $i \geq j$. Then, 
  \[
    s_m^\ast = \sum_{n=1}^m |x_{n}^\ast|^p = \lim_{i \to \infty}
    \sum_{n=1}^m |x_{in}|^p\leq M^p.
  \]
  The final inequality comes from the fact that $ \sum_{n=1}^m |x_{in}|^p \leq \sum_{n=1}^\infty
  |x_{in}|^p = \norm{\mathbf{x}_i}^p \leq M^p$. Importantly, the upper
  bound $M^p$ does not depend on $i$ or $m$. Thus, we have shown that
  \[ s_m^\ast \leq M^p \]
  for all $m$. $s_m^\ast$ is a bounded weakly increasing sequence in
  $\R$, so it must converge.\footnote{Let $\seq{x} \in \R$ and suppose $x_1
    \leq x_2 \leq x_3 \leq ...$ and $\seq{x}$ is bounded, then we will
    show $\seq{x}$ converges. Suppose not. Then the sequence has no
    accumulation points. In particular, $x_i$ is not an accumulation
    point of the sequence for any $i$ i.e. there is an $\epsilon>0$
    such that for all $i$ there are finitely many $j$ with $d(x_i,
    x_j) < \epsilon$. Then we can construct a subsequence
    by choosing $j_k$ such that $j_k> j_{k-1}$ and $|x_{j_k} -
    x_{j_{k-1}} | > \epsilon$. But then 
    \begin{align*}
      x_{j_k} = & x_{j_1} + (x_{j_2} - x_{j_1}) + (x_{j_3} - x_{j_2}) +
      ... + (x_{j_k} - x_{j_{k-1}} ) \\
      \geq & x_{j_1} + (k-1) \epsilon 
    \end{align*}
    which is not bounded.}
  
  Finally, we should show that $\seq{\mathbf{x}}$ converges to
  $\mathbf{x}^\ast$. Let $\epsilon>0$. 
  Since the original sequence is Cauchy, there is an $N$ such that if
  $i,j > N$, then
  \begin{align*}
    \sum_{m=1}^M \abs{x_{im} - x_{jm}}^p \leq \norm{\mathbf{x}_i -
      \mathbf{x}_j}^p < \epsilon
  \end{align*}
  for all $M$. Therefore, 
  \begin{align*}
    \lim_{j \to \infty} \sum_{m=1}^M \abs{x_{im} - x_{jm}}^p =
    \sum_{m=1}^M\abs{x_{im} - x_m^\ast} < \epsilon 
  \end{align*}
  for all $i \geq N$ and all $M$. Thus, 
  \[
  \norm{\mathbf{x}_i - \mathbf{x}^\ast} = \lim_{M \to \infty}
  \sum_{m=1}^M\abs{x_{im} - x_m^\ast} < \epsilon 
  \]
  for all $i \geq N$, so the sequence converges.
\end{example}

%%%%%%%%%%%%%%%%%%%%%%%%%%%%%%%%%%%%%%%%%%%%%%%%%%%%%%%%%%%%%%%%%%%%%%
\section{Open sets}

\begin{definition}
  Let $X$ be a metric space and $x \in X$. A \textbf{neighborhood} of
  $x$ is the set 
  \[ N_\epsilon (x) = \{y \in X: d(x,y) < \epsilon. \]
\end{definition}
A neighborhood is sometimes also called an open
$\epsilon$\textbf{-ball} of $x$ and written $B_{\epsilon}(x)$. I will
try to stick with the $N_\epsilon(x)$ notation, but I might sometimes
use ball and neighborhood interchangeably.
\begin{definition}
  A set, $S \subseteq X$ is \textbf{open} if $\forall x \in S$,
  $\exists$ $\epsilon>0$ such that 
  \[ N_\epsilon(x) \subset S. \]
\end{definition}
For every point in an open set, you can find a small neighborhood
around that point such that the neighborhood lies entirely within the
set. 
\begin{example}
  Any open interval, $(a,b) = \{x \in \R: a<x<b\}$, is an open set. 
\end{example}

\begin{theorem} $\,$
  \begin{enumerate} \label{thm:uio}
  \item Any union of open sets is open. (finite or infinite)
  \item The \emph{finite} intersection of open sets is open.
  \end{enumerate}
\end{theorem}
\begin{proof}
  Let $S_j$, $j \in J$ be a collection of open sets. Pick any $j_0 \in
  J$. If $x \in \cup_{j \in J} S_j$, then there must be $\epsilon_{j_0} >
  0$ such that $N_{\epsilon_{j_0}} (x) \subset S_{j_0}$. It is
  immediate that $N_{\epsilon_{j_0}} (x) \subset \cup_{j \in J} S_j$
  as well. 

  Let $S_1, .., S_k$ be a finite collection of open sets. For each $i$
  $\exists \epsilon_i > 0$ such that $N_{\epsilon_i}(x) \subset
  S_i$. Let $\underline{\epsilon} = \min_{i \in \{1,..., k\}}
  \epsilon_i$. Then $\underline{\epsilon}>0$ since it is the minimum
  of a finite set of positive numbers. Also,
  $N_{\underline{\epsilon}}(x) \subset S_i$ for each $i$, so
  $N_{\underline{\epsilon}}(x) \subset \cap_{i=1}^k S_i$. 
\end{proof}
\begin{definition}
  The \textbf{interior} of a set $A$ is the union of all open sets
  contained in $A$. It is denoted as $\mathrm{int}(A)$.
\end{definition}
From the previous, theorem, we know that the interior of any set is
open. 
\begin{example}
  Here some examples of the interior of sets in $\R$.
  \begin{enumerate}
  \item $A = (a,b)$, $\mathrm{int}(A) = (a,b)$.
  \item $A = [a,b]$, $\mathrm{int}(A) = (a,b)$.
  \item $A = \{1, 2, 3, 4, ... \}$, $\mathrm{A} = \emptyset$
  \end{enumerate}  
\end{example}

\begin{exercise} \label{ex:openConvergence}
  Let $X$ be a metric space and $\seq{x}$ a sequence in $X$. Show that
  $x_n \to x$ if and only if for every open set $U$ containing $x$
  $\exists N$ such that $x_n \in U$ for all $n \geq N$.
\end{exercise}

\section{Closed sets}

A closed set is almost like the opposite of an open set.  However, a
set can be both open and closed, and a set can be neither, so they are
not exactly opposites.
\begin{definition}
  A set $S \subseteq X$ is closed if its complement, $S^c$, is open. 
\end{definition}
For any metric space, $X$, the whole space, $X$ is open. Therefore,
the $X^c = \emptyset$ is closed.  The empty set is considered open in
any metric space. For now, you could just consider this to be by
convention, but below we will that closed sets can also be defined by
convergence of sequences. By that definition, the whole space is
closed, so its complement, the empty set must be open. Anyway, the
point is that the empty set and entire metric space are always both
open and closed.

Closed sets behavior under union and intersections is the mirror image
of that of open sets. 
\begin{theorem} $\;$\label{thm:uic}
  \begin{enumerate}
  \item The intersection of any collection of closed sets is closed.
  \item\label{uic2} The union of any finite collection of closed sets is closed. 
  \end{enumerate}
\end{theorem}
\begin{proof}
  Let $C_j$, $j\in J$ be a collection of closed sets. Then 
  $ \left(\cap_{j \in J} C_j\right)^c = \cup_{j \in J}
  C^c_j $. 
  $C^c_j$ are open, so by theorem \ref{thm:uio},  $\cup_{j \in J}
  C^c_j = \left(\cap_{j \in J} C_j\right)^c =$ is open.  
  
  The proof of part \ref{uic2} is similar. 
\end{proof}

\begin{example}[Closed sets]
  Some examples of closed sets include
  \begin{enumerate}
  \item $[a,b] \subseteq \R$
  \item $\{(x,y) \in \R^2: x^2 + y^2 \leq 1\}$
  \item $\{x \in \R^n: A x = b \}$ where $A$ is an $m \times n$ matrix
    and $b \in \R^m$
  \end{enumerate}
\end{example}

\begin{example}[Discrete metric and topology]
  Let $X$ be any set. Define a metric on $X$ by
  \[ \Delta(x,y) = \begin{cases} 1 & \text{ if } x\neq y \\
      0 & \text{ if } x = y
    \end{cases}
  \]
  This is a metric because it is positive definite, symmetric, and
  satisfies the triangle inequality. This metric is called the
  discrete metric. In this space, the set of any single point $\{x \}$
  is open since $N_{\epsilon}(x) = \{x\} \subseteq \{x\}$ for any
  $\epsilon < 1$. Any union of open sets is open, so all subsets of
  $X$ are open. Since every set is open, the complement of every set is
  open, so every set is closed as well.  
\end{example}

Closed sets can also be defined as sets that contain the limit of any
convergent sequence in the set. Simon and Blume use this
definition. The next theorem shows that their definition is equivalent
to ours.
\begin{theorem}\label{thm:clim}
  Let $\{x_n\}$ be any convergent sequence with each element contained
  in a set $C$. Then $\lim x_n = x \in C$ for all such $\{x_n\}$ if
  and only if $C$ is closed.
\end{theorem}
\begin{proof}
  First, we will show that any set that contains the limit points of
  all its sequences is closed. Let $x \in C^c$. Consider
  $N_{1/n}(x)$. If for any $n$, $N_{1/n}(x) \subset C^c$, then $C^c$
  is open, and $C$ is closed as desired. If for all $n$, $N_{1/n}(x)
  \not \subset C^c$, then $\exists y_n \in N_{1/n}(x) \cap C$. The
  sequence $\{y_n\}$ is in $C$ and $y_n \to x$. However, by assumption
  $C$ contains the limit of any sequence within it. Therefore, there
  can be no such $x$, and $C^c$ must be open and $C$ is closed.

  Suppose $C$ is closed. Then $C^c$ is open. Let $\{x_n\}$ be in $C$
  and $x_n \to x$. Then $d(x_n, x) \to 0$, and for any $\epsilon > 0$,
  $\exists x_n \in N_\epsilon(x)$. Hence, there can be no $\epsilon$
  neighborhood of $x$ contained in $C^c$. $C^c$ is open by assumption,
  so $x \not\in C^c$ and it must be that $x \in C$. 
\end{proof}

\begin{definition}
  The \textbf{closure} of a set $S$, denoted by $\overline{S}$ (or
  $\mathrm{cl}(S)$), is the intersection of all closed sets containing $S$.
\end{definition}

%\begin{definition}
%  A \textbf{limit point} of a set $S$ is any $x$ such that for any
%  $\epsilon>0$, $N_\epsilon(x) \cap S \neq \emptyset$.
%\end{definition}
\begin{example}
  If $S$ is closed, $\overline{S} = S$. 
\end{example}
\begin{example}
  $\overline{(0,1]} = [0,1]$
\end{example}
\begin{lemma}\label{lem:closure}
  $\overline{S}$ is the set of limits of convergent sequences in $S$.
\end{lemma}
\begin{proof}
  Let $\{x_n\}$ be a convergent sequence in $S$ with limit $x$. If $C$
  is any closed set containing $S$, then $\{x_n\}$ is in $C$ and by
  theorem \ref{thm:clim}, $x \in C$. Therefore, $x \in \overline{S}$. 
  
  Let $x \in \overline{S}$. For any $\epsilon>0$, $N_\epsilon(x) \cap
  S \neq \emptyset$ because otherwise $N_\epsilon(x)^c$ is a closed
  set containing $S$, but not $x$. Therefore, we can construct a
  sequence $x_n \in S \cap N_{1/n}(x)$ that converges to $x$ and is in
  $S$. 
\end{proof}
\begin{example}
  $\overline{\{1/n\}_{n \in \mathbb{N}}} = \{0, 1, 1/2, 1/3, ... \}$
\end{example}
\begin{definition}
  The \textbf{boundary} of a set $S$ is $\overline{S} \cap
  \overline{S^c}$. 
\end{definition}
\begin{example}
  The boundary of $[0,1]$ is $\{0,1\}$.
\end{example}
\begin{example}
  The boundary of the unit ball, $\{x \in \R^2: \norm{x} < 1\}$ is the
  unit circle, $\{x \in \R^2: \norm{x} = 1\}$.
\end{example}
\begin{lemma}
  If $x$ is in the boundary of $S$ then $\forall \epsilon>0$,
  $N_\epsilon(x) \cap S \neq \emptyset$ and $N_\epsilon(x) \cap S^c
  \neq \emptyset$. 
\end{lemma}
\begin{proof}
  As in the proof of lemma \ref{lem:closure}, all
  $\epsilon$-neighborhoods of $x \in \overline{S}$ must intersect with
  $S$. The same applies to $S^c$. 
\end{proof}

Exercise \ref{ex:openConvergence} and theorem \ref{thm:clim} show that
there is an important relationship between convergence of sequences
and open and closed sets. Given a definition of what it means for a
sequence to converge, we could use theorem \ref{thm:clim} to define
closed sets. Open sets could then be defined as the complement of
closed sets. Conversely, if we specify which sets are open and closed,
we can then define convergence of sequences as in exercise
\ref{ex:openConvergence}. Metrics, convergence of sequences, and open
and closed sets are three different ways of describing notions of
proximity and continuity.

%%%%%%%%%%%%%%%%%%%%%%%%%%%%%%%%%%%%%%%%%%%%%%%%%%%%%%%%%%%%%%%%%%%%%%%%%%%%%%%%
\section{Compact sets}

Compact sets are a generalization of finite sets. Compact sets are
essential for proving many important theorems. Compact sets have a
somewhat difficult to understand definition, but they are incredibly
useful.

\begin{definition}
  An \textbf{open cover} of a set $S$ is a collection of open sets,
  $\{G_\alpha\}$  $\alpha \in \mathcal{A}$ such that $S \subseteq
  \cup_{\alpha \in \mathcal{A}} G_\alpha$. 
\end{definition}
\begin{example}
  Some open covers of $\R$ are:
  \begin{itemize}
  \item $\{\R\}$ 
  \item $\{(-\infty, 1), (-1, \infty)$
  \item $\{..., (-3, -1), (-2, 0),(-1,1), (0,2), (1,3), ... \}$
  \item $\{(x,y) : x<y \}$
  \end{itemize}
  The first two are finite open covers since they consist of finitely
  many open sets. The third is a countably infinite open cover. The
  fourth is an uncountably infinite open cover. 
\end{example}
\begin{example}
  Let $X$ be a metric space and $A \subseteq X$. The set of open balls
  of radius $\epsilon$ centered at all points in $A$ is an open cover
  of $A$. If $A$ is finite / countable / uncountable, then this open
  cover will also be finite / countable / uncountable. 
\end{example}
Open covers of the form in the previous example are often used to
prove some property applies to all of $A$ by verifying the property in
each small $N_\epsilon(x)$. Unfortunately, this often involves taking
a maximum or sum of something for each set in the open cover. When the
open cover is infinite, it can be hard to ensure that the infinite sum
or maximum stays finite. When we have a finite open cover, we know
that things will remain finite. 
\begin{definition}
  A set $K$ is \textbf{compact} if every open cover of $K$ has a
  finite subcover. 
\end{definition}
By a finite subcover, we mean that there is finite subset of the sets
in the open cover, $G_{\alpha_1},
... G_{\alpha_k}$ such that $S \subset \cup_{j=1}^k
G_{\alpha_j}$. Compact sets are a generalization of finite sets. Many
facts that are obviously true of finite sets are also true for compact
sets, but not true for infinite sets that are not compact. Suppose we
want to show a set has some property. If the set is compact, we can
cover it with a finite number of small $\epsilon$ balls and then we
just need to show that each small ball has the property we want. We
will see many concrete examples of this technique in the next few
weeks.
\begin{example}
  $\R$ is not compact. $\{..., (-3, -1), (-2, 0),(-1,1), (0,2), (1,3),
  ... \}$ is an infinite cover, but if we leave out any single
  interval (the one beginning with $n$) we will fail to cover
  some number ($n+1$).
\end{example}
\begin{example}
  Let $K = \{x\}$, a set of a single point. Then $K$ is compact. Let
  $\{G_\alpha\}_{\alpha \in \mathcal{A}}$ be an open cover of
  $K$. Then $\exists$ $\alpha$ such that $x \in G_\alpha$. This single
  set is a finite subcover.
\end{example}
\begin{example}
  Let $K = \{x_1, ..., x_n\}$ be a finite set. Then $K$ is
  compact.  Let
  $\{G_\alpha\}_{\alpha \in \mathcal{A}}$ be an open cover of
  $K$. Then for each $i$, $\exists$ $\alpha_i$ such that $x_i \in
  G_{\alpha_i}$. 
  The collection $\{G_{\alpha_1}, ... G_{\alpha_n} \}$ is a finite
  subcover.
\end{example}
\begin{example}
  $(0,1) \subseteq \R$ is not compact. $\{(1/n,1)\}_{n=2}^\infty$ is
  an open cover, but there can be no finite subcover. Any finite
  subcover would have a largest $n$ and could not contain,
  e.g. $1/(n+1)$. 
\end{example}
\begin{example}
  Pick any $x \in \R^n$. Let $K = \{x\frac{1}{2},
  x\frac{2}{3}, x\frac{3}{4}, ... \}$. Then $K$ is not
  compact. Consider the open cover $N_{\norm{x}\frac{1}{3(n+2)^2}}(x
  \frac{n}{n+1})$ for $n=1,2, ...$. Assuming $x \neq 0$, each of these
  neighborhoods contains exactly one point of $K$, so there is no
  finite subcover. 
\end{example}


Before using compactness, let's investigate how being compact relates
to other properties of sets, such as closed/open. 
\begin{lemma}\label{lem:compactClosed}
  Let $X$ be a metric space and $K \subseteq X$. If $K$ is compact,
  then $K$ is closed.
\end{lemma}
\begin{proof}
  Let $x \in K^c$. The collection $\{N_{d(x.y)/3}(y)\}$, $y\in K$ is an
  open cover of $K$. $K$ is compact, so there is a finite subcover,
  $N_{d(x,y_1)/3}(y_1), ... , N_{d(x,y_n)/3}(y_n)$.  For each $i$,
  $N_{d(x,y_i)/3}(y_i) \cap N_{d(x,y_i)/3}(x) = \emptyset$, so 
  \[ \cap_{i=1}^n N_{d(x,y_i)/3}(x) \]
  is an open neighborhood of $x$ that is contained in $K^c$. $K^c$ is
  open, so $K$ is closed.
\end{proof}
\begin{lemma}
  Let $X$ be a metric space, $C \subseteq K \subseteq X$. If $K$ is
  compact and $C$ is closed. Then $C$ is also compact.
\end{lemma}
\begin{proof}
  Let $\{G_\alpha\}_{\alpha \in \mathcal{A}}$ be an open cover for
  $C$. Then $\{G_\alpha\}_{\alpha \in \mathcal{A}}$ plus $C^c$ is an
  open cover for $K$. Since $K$ is compact there is a finite
  subcover. Since $C \subseteq K$, the finite subcover also covers
  $C$. Therefore, $C$ is compact. 
\end{proof}
An equivalent definition of compactness is in terms of collection of
sets with ``the finite intersection property.'' A collection of sets
$\{C_\alpha \}_{\alpha \in \mathcal{B}}$ has the finite intersection
property if for all finite subsets, $F \subseteq \mathcal{A}$, the
intersection, $\cap_{\alpha \in F} C_\alpha$ is not empty.
\begin{lemma}
  Let $X$ be a metric space and $K \subseteq X$. $K$ is compact if and
  only if for every collection of closed subsets, $\{C_\alpha\}_{\alpha
  \in \mathcal{B}}$ with $C_\alpha \subseteq K$, with the finite
  intersection property, the intersection of all the subsets is not
  empty i.e. 
  $\cap_{\alpha \in \mathcal{B}} C_\alpha$ is not empty.
\end{lemma}
\begin{proof}
  Left as an exercise. The statement in the lemma is basically the
  contrapositive of the definition of compact, combined with the
  observation that the complement of closed sets are open and vice
  versa, and the fact that $(A \cup B)^c = A^c \cap B^c$.
\end{proof}

The definition of compactness is somewhat abstract. We just saw that
compact sets are always closed. Another property of compact sets is
that they are bounded. 
\begin{definition}
  Let $X$ be a metric space and $S \subseteq X$. $S$ is
  \textbf{bounded} if $\exists x_0 \in S$ and $r \in \R$ such that 
  \[ d(x,x_0) < r \]
  for all $x \in S$.
\end{definition}
A bounded set is one that fits inside an open ball of finite
radius. For subsets of $\R$ this definition is equivalent to there
being a lower and upper bound for the set. For subsets of a normed
vector space, if $S$ is bounded then there exists some $M$ such that
$\norm{x} < M$ for all $x \in S$. 
\begin{lemma}
  Let $K \subseteq X$ be compact. Then $K$ is bounded. 
\end{lemma}
\begin{proof}
  Pick $x_0 \in K$. $\{N_{r}(x_0) \}_{r \in \R}$ is an open cover of
  $K$, so there must be a finite subcover. The finite subcover has
  some maximum $r^*$. Then $K \subseteq N_{r^*}(x_0)$, so $K$ is bounded.
\end{proof}
This lemma along with lemma \ref{lem:compactClosed} show that if a set
is compact then it is also closed and bounded. In $\R^n$, the converse
is also true.
\begin{theorem}[Heine-Borel]\label{thm:hb}
  A set $S \subseteq \R^n$ is compact if and only if it is closed and
  bounded. 
\end{theorem}
\begin{proof}
  We already showed that if $S$ is compact, then it is closed and
  bounded. 

  Now suppose $S$ is closed and bounded. Since $S$ is bounded, it is a
  subset of some $n$-dimensional cube, say $[-a,a]^n$ (i.e.\ the set
  of all vectors $x = (x_1,..,x_n)$ with $-a \leq x_i \leq a$).  We
  will show $[-a,a]^n$ is compact, and then use the fact that a
  closed subset of a compact set is compact.

  Let's just show $[-a,a]^n$ is compact for $n=1$. The argument for
  larger $n$ is similar, but the notation is more cumbersome. If
  $[-a,a]$ is not compact, then there is an infinite open cover with
  no finite subcover, say $\{G_\alpha\}_{\alpha \in \mathcal{A}}$. If
  we cut the interval into two halves, $[-a,0]$ and $[0,a]$, at least
  one of them must have no finite subcover. We can repeat this
  argument many times to get nested closed intervals of length
  $a/(2^k)$ for any $k$. Call the $k$th interval $I_k$. We claim that
  $\cap_{k=1}^\infty I_k \neq \emptyset$. To show this take the
  sequence of lower endpoints of the intervals, call it
  $\seq{x}$. This is a Cauchy sequence, so it converges to some limit,
  $x_0$. Also, for any $k$, $\{x_n\}_{n=k}^\infty$ is a sequence in
  $I_k$. $I_k$ is closed so $x_0 \in I_k$. Thus $x_0 \in
  \cap_{k=1}^\infty I_k$. On the other hand, $I_k \subset \cup_{\alpha
    \in \mathcal{A}} G_\alpha$ for all $k$. Therefore, $x_0$ must be
  in some open $G_\alpha$ as part of this cover. Then $\exists
  \epsilon>0$ such that $N_\epsilon(x_0) \subset G_\alpha$. However,
  for $k>1/\epsilon$, $I_k \subset N_\epsilon(x_0) \subset G_\alpha$,
  and then $I_k$ has a finite subcover. Therefore, $[-a,a]$ must be
  compact. 

  The argument for $n>1$ is very similar. For $n=2$, we would divide
  the square $[-a,a]^2$ into four smaller squares. For $n=3$, we would
  divide the cube into eight smaller cubes. In general we would divide
  the hypercube $[-a,a]^n$ into $2^n$ hypercubes with half the side
  length.
\end{proof}
You may wonder whether closed and bounded sets are always compact. We
will see that all finite dimensional real vector spaces are isomorphic
to $\R^n$. In any such space, sets are compact iff they are closed and
bounded. However, in infinite dimensional spaces, there are closed and
bounded sets that are not compact. The argument in the previous proof
does not apply to infinite-dimensional spaces because an infinite
dimensional hypercube can only be divided into infinitely many
hypercubes with half the side length.
\begin{example}\label{ex:ellinf}
  $\ell^\infty = \{ (x_1, x_2, ...) : \sup_{i} |x_i| < \infty$ with
  norm $\norm{x} = \sup_{i} |x_i|$ is a normed vector space. Let $e_i$
  be the element of all $0$s except for the $i$th position, which is
  $1$. Then $E = \{e_i\}_{i=1}^\infty$ is closed and bounded. However,
  $E$ is not compact because $\{N_{1/2}(e_i)\}_{i=1}^\infty$ is an
  open cover with no finite subcover. 
\end{example}
It is always true that a closed subset of a compact set is compact.
\begin{lemma}
  Let $C \subseteq K \subseteq X$, where $X$ is a metric space.  If
  $K$ is compact and $C$ is closed, then $C$ is compact. 
\end{lemma}
\begin{proof}
  Let $\{G_{\alpha}\}_{\alpha \in \mathcal{A}}$ be an open cover for
  $C$. $C$ is closed, so $C^c$ is open. Also, $C^c$ along with
  $\{G_{\alpha}\}_{\alpha \in \mathcal{A}}$ is an open cover of
  $K$. $K$ is compact so there exists a finite subcover. This finite
  subcover (with $C^c$ removed) is also a finite subcover of
  $C$. Therefore, $C$ is compact. 
\end{proof}

We saw that closed sets contain the limit points of all their
convergent sequences. There is also a relationship between compactness
and sequences. 
\begin{definition}
  Let $X$ be a metric space and $K \subseteq X$. $K$ is
  \textbf{sequentially compact} if every sequence in $K$ has an
  accumulation point in $K$.
\end{definition}
Sometimes this definition is written as: $K$ is sequentially compact
if every sequence in $K$ has a subsequence that converges in
$K$. Compactness implies sequential compactness. 
\begin{lemma}\label{lem:compactSeqCompact}
    Let $X$ be a metric space and $K \subseteq X$ be compact. Then $K$
    is sequentially compact. 
\end{lemma}
\begin{proof}
  Let $\seq{x}$ be a sequence in
  $K$. Pick any $\epsilon>0$, $N_\epsilon(x)$, $x \in K$ is an open
  cover of $K$, so there is a finite subcover. Therefore, one of the
  $\epsilon$ neighborhoods must contain an infinite number of the
  elements from the sequence. Call this neighborhood
  $N_\epsilon(x_1^\ast)$. Pick the smallest $n$ such that $x_n \in
  N_\epsilon(x_1^\ast)$ and call it $n_1$. $\overline{N_{\epsilon}(x_1^\ast)}
  \cap K$ is a closed subset of the compact set $K$, so is itself
  compact. Repeat the above argument with $\epsilon/2$ in place of
  $\epsilon$ and $\overline{N_{\epsilon}(x_1^\ast)} \cap K$ in place of $K$
  to find an $n_2$, $n_3$, etc. Then the subsequence $x_{n_1},
  x_{n_2}, ...$ is a Cauchy sequence, so it converges. Its limit is an
  accumulation point. Its limit must be in $K$ because $K$ is compact,
  and so, closed. Therefore, $K$ is sequentially compact.
\end{proof}

In $\R^n$, a set is sequentially compact iff it is compact iff it is
closed and bounded.
\begin{theorem}[Bolzano-Weierstrass] \label{thm:bw}
  A set $S \subseteq \R^n$ is closed and bounded if and only if it is
  sequentially compact. 
\end{theorem}
\begin{proof}
  Let $S$ be closed and bounded. By the Heine-Borel theorem
  (\ref{thm:hb}), $S$ is compact. By lemma
  \ref{lem:compactSeqCompact}, $S$ is sequentially compact. 

  Let $S$ be sequentially compact. Let $\{x_n\}$ be a convergent
  sequence in $S$. Its limit is an accumulation point, so it must be
  in $S$. Therefore, $S$ is closed. To show $S$ is bounded, pick $x_0
  \in S$. Suppose $\exists x_1 \in S$ such that $d(x_1, x_0) \geq 1$,
  and $x_2 \in S$ such that $d(x_2, x_0) \geq 2$ etc. This sequence is
  not Cauchy because of the reverse triangle inequality,  
  \begin{align*}
    d(x_i,x_j) \geq & \abs{d(x_i,x_0) - d(x_j,x_0) } = |i-j|
  \end{align*}  
  This would be a sequence in $S$ with no accumulation
  points. Therefore, it must not always be possible to find such
  $x_n$. In other words, $S$ must be bounded.
\end{proof}

\begin{remark}
  This theorem is sometimes stated as ``each bounded sequence in $\R^n$
  has a convergent subsequence.'' As an exercise, you may want to
  verify that this statement is equivalent to the one above.
\end{remark}
Simon and Blume also prove this theorem in chapter 29.2. They do not
prove the Heine-Borel theorem first though, so their proof is of the
Bolzano-Weierstrass theorem is longer. Perhaps unsurprisingly, the
details of their proof are somewhat similar to our proof of the
Heine-Borel theorem.

In $\R^n$, compactness, sequential compactness, and closed and bounded
are all the same. In general metric spaces, this need not be true. We
saw above that in $\ell^\infty$, there are
closed and bounded sets which are not compact.  However it is always
true that sequential compactness and compactness are the same for
metric spaces. We already showed that compactness implies sequential
compactness.  The proof that sequential compactness implies
compactness is a somewhat long and difficult, and you may want to skip
it unless you are especially interested.
\begin{theorem}
  Let $X$ be a metric space and $K \subseteq X$. $K$ is compact if and
  only if $K$ is sequentially compact. 
\end{theorem}
\begin{proof}
  Lemma \ref{lem:compactSeqCompact} shows that if $K$ is compact, then
  $K$ is sequentially compact.
  
  Suppose every sequence in $K$ has a convergent subsequence with a
  limit point in $K$. Let $G_{\alpha}$, $\alpha \in \mathcal{A}$ be an
  open cover of $K$. $\mathcal{A}$ could be uncountable, so we will
  begin by showing that there must be a countable subcover.  Let
  $n=1$. Pick $x_1 \in K$. If possible choose $x_2 \in K$ such that
  $d(x_1,x_2) \geq 1/n$. Repeat this process, choosing $x_j$ in $K$
  such that $d(x_j,x_i) \geq 1/n$ for each $i<j$. Eventually this will
  no longer be possible because otherwise we could construct a
  sequence with no convergent subsequence. When it is no longer
  possible, set $n = n+1$. This gives a countable collection of open
  neighborhoods $N_{1/n}(x_i)$ that cover $K$ for each $n$ and get
  arbitrarily small as $n$ increases. Call these neighborhoods
  $\eta_j$ for $j = 1,2,..$. Let $J$ be set of all $\eta_j$ such that
  $\eta_j \subseteq G_\alpha$ for some $\alpha$. $J$ is a subset of a
  countable set, so $J$ is countable. Note that $\cup_{j \in J} \eta_j
  \supset K$ because if $x \in K$, then $x \in G_\alpha$ for some
  $\alpha$, and then $\exists \epsilon$ such that $N_{\epsilon}(x) \in
  G_{\alpha}$ and $\exists j$ s.t. $\eta_j \subset N_{\epsilon}(x)$.
  Finally, for each $j \in J$ choose $G_{\alpha_j}$ such that $\eta_j
  \subseteq G_{\alpha_j}$. Such $\alpha_j$ exist by construction. Also
  $\cup_{j \in J} G_{\alpha_j} \supset \cup_{j \in J} \eta_j \supset
  K$. So $G_{\alpha_j}$ is a countable subcover.

  If $G_{\alpha_j}$ has no finite subcover, then for each $n$, 
  \[ F_n = \left(\cup_{i=1}^n G_{\alpha_i}\right)^c \cap K \] is not
  empty (if it were empty, then $\cup_{i=1}^n G_{\alpha_i}$ would be a
  finite subcover). Choose $x_n \in F_n$. Then $\{x_n\}$ is a sequence in $K$,
  and it must have a convergent subsequence with a limit, $x_0$, in
  $K$. However, each $F_{i+1} \subset F_i$ and $F_i$ are all
  closed. Therefore, the sequence $\{x_j\}_{j=i}^\infty$ is also in
  $F_i$ and so is its limit. Then $x_0 \in \cap_{i=1}^\infty
  F_i$. However, 
  \[ \cap_{i=1}^\infty F_i = \left(\cup_{i=1}^\infty
    G_{\alpha_i}\right)^c \cap K, \]  
  but $G_{\alpha_i}$ is a countable cover of $K$, which implies 
  \[ \cap_{i=1}^\infty F_i = \left(\cup_{i=1}^\infty
    G_{\alpha_i}\right)^c \cap K = \emptyset\] 
  and we have a contradiction. Therefore, $G_{\alpha_j}$ must have a
  finite subcover, and $K$ is compact.
\end{proof}

\begin{remark}
  There are non-metric spaces where sequential compactness and
  compactness are not equivalent. One can define open sets on a space
  without a metric by simply specifying which sets are open and making
  it such that theorem \ref{thm:uio} holds. Such a space is called a
  topological space. You can then define closed sets, compact sets,
  and sequential compactness in terms of open sets.  Exercise
  \ref{ex:openConvergence} showed that it is possible to define the
  convergence of sequences using only open sets, without referring to
  a metric at all. Similarly, you can define continuity of functions
  in terms of open and closed sets. Topology is the branch of
  mathematics that studies topological spaces. One interesting
  observation is that on $\R^n$, if a set is open with respect to some
  $p$-norm, then it is also open with respect to any other
  $p$-norm. Thus, we say that $\R^n$ with the $p$-norms are
  topologically equivalent or homeomorphic. Properties like continuity
  and compactness are the same regardless of what $p$-norm we use.
  
  Topological spaces that are not metric spaces sound exotic, but they
  do sometimes appear. An example of a non-metrizable topological
  space is the following. Let $X = \{f: \R \to \R\}$. Define
  convergence of sequences in this space as $f_n \to f$ if
  $f_n(x) \to f(x)$ for all $x \in \R$. Let $S \subset X$ be closed if
  $S$ contains the accumulation points of all sequences in $S$. This
  definition of closed and open sets is called the topology of
  pointwise convergence. This space and definition of convergence
  sounds reasonable (and is reasonable), but there is no metric on
  this space that leads to the same definition of convergence and
  closed sets.

  In an area of econometrics and statistics called empirical process
  theory, you often have to work with something called the topology of
  weak convergence, which has a somewhat similar definition and is
  also non-metrizable. We saw in example \ref{ex:ellinf} that in an
  infinite dimensional space, the unit sphere (or a closed
  neighborhood of any radius) is not compact. It is also not
  sequentially compact. Econometricians care about sequential
  compactness to show that limits and asymptotic distributions
  exist. Working with weak convergence instead of a metrizable
  topology makes many nice sets sequentially compact.
\end{remark}

To review, in $\R^n$ a set is compact if any of the following four
things hold:
\begin{enumerate}
\item For every open cover there exists a finite subcover,
\item Every collection of closed sets with the finite intersection
  property has a non-empty intersection
\item Every sequence in the set has a convergent subsequence, or
\item The set is closed and bounded.
\end{enumerate}
The first two are always equivalent. 
In infinite dimensional spaces, closed and bounded sets need not be
compact, but compact sets are always closed and bounded. In any metric
space, a set is compact if and only if it is sequentially compact. 

%%%%%%%%%%%%%%%%%%%%%%%%%%%%%%%%%%%%%%%%%%%%%%%%%%%%%%%%%%%%%%%%%%%%%%%%%%%%%%%%
\section{Functions and continuity}

We have already used functions in this course, so perhaps we should
have defined them earlier. Anyway, a \textbf{function} from a set $A$
to a set $B$ is a rule that assigns to each $a \in A$ one and only one
$b \in B$. If we want to call this function $f$, we denote this by
$f:A \to B$, which is read as ``$f$ is a function from $A$ to $B$'' or
simply ``$f$ from $A$ to $B$.'' The set $A$ is called the
\textbf{domain} of $f$. $B$ is called the \textbf{target} of $f$. The
set 
\[ \{y  \in B:  f(x) = y \text{ for some } x \in A \} \]
is called the \textbf{image} or \textbf{range} of $f$. 


\begin{example}$\;$
\begin{enumerate}
\item Production functions: $f:\R^2 \to \R$ 
  \begin{itemize}
  \item Linear $ f(x_1,x_2) = a_1 x_1 + a_2 x_2 $
  \item Cobb-Douglas: $f(x_1,x_2) = K x_1^{\alpha_1} x_2^{\alpha_2}$
  \item Constant elasticity of substitution: $f(x_1,x_2) = K (c_1
    x_1^{-a} + c_2 x_2^{-a})^{-b/a}$
  \end{itemize}
\item Utility functions: $u: \R^T \to \R$
  \begin{itemize}
  \item Constant relative risk aversion: $u(c_1,...,c_T) =
    \sum_{t=1}^T \beta^t \frac{c_t^{1-\gamma}}{1-\gamma}$
  \item Constant absolute risk aversion: $u(c_1,...,c_T) =
    \sum_{t=1}^T \beta^t (-e^{-\alpha c_t})$
  \end{itemize}
\item Demand function with constant elasticity,  $D:\R^3 \to \R^2$
  \[ D(p_1,p_2,y) = \begin{pmatrix} M p_1^{\alpha_{11}}
    p_2^{\alpha_{12}} y^{\beta_1} \\
    M p_1^{\alpha_{21}} p_2^{\alpha_{22}} y^{\beta_2}
  \end{pmatrix}
  \]  
  where $p_1$ and $p_2$ are the prices of two goods and $y$ is income.
\end{enumerate}
I do not expect you to remember the names of these functions, but it
is very likely that you will repeatedly encounter them this year. 
\end{example}

A continuous function is a function without any jumps or
holes. Formally,
\begin{definition}
  A function $f:X \to Y$ where $X$ and $Y$ are metric spaces is
  \textbf{continuous} at $x$ if whenever $\seq{x}$ converges to $x$ in
  $X$, then $f(x_n) \to f(x)$ in $Y$.  
\end{definition}
We simply say that $f$ is continuous if it is continuous at every
$x \in X$.  There are some equivalent definitions of continuity that
are also useful. You may have seen continuity defined as in the
following lemma.
\begin{lemma}\label{lem:ced}
  $f: X \to Y$ is continuous at $x$ if and only if for every
  $\epsilon>0$ $\exists$ $\delta >0$ such that $d(x,x') < \delta $
  implies $d(f(x),f(x')) < \epsilon$.
\end{lemma}
\begin{proof}
  Suppose $f$ is continuous and there is an $\epsilon>0$ such that
  such that for any $\delta>0$, $d(x,x') < \delta$ does not imply
  $d(f(x),f(x'))< \epsilon$. Then by letting $\delta = 1/n$ we can
  construct a convergent sequence by choosing $x_n$ such that
  $d(x,x_n) < 1/n$ and $d(f(x),f(x_n)) \geq \epsilon$. $x_n \to x$,
  but $f(x_n) \not\to f(x)$. Therefore, if $f$ is continuous it must
  be impossible to construct such a sequence. This means that there
  must be some $\delta>0$ such that $d(x,x') < \delta$ implies
  $d(f(x),f(x')) < \epsilon$.

  Now suppose $\forall \epsilon>0$ $\exists \delta>0$ such that
  $d(x,x')<\delta$ implies $d(f(x),f(x')) <\epsilon$. Let $x_n \to
  x$. Then $\exists N$ s.t. $n\geq N$ implies $d(x,x_n) < \delta$. But
  this implies $d(f(x),f(x_n)) < \epsilon$, so $f$ is continuous.
\end{proof}
Both the definition of continuity and (\ref{lem:ced}) are about $f$
being continuous at a point $x$. We say that $f$ is continuous on $U
\subseteq X$ if $f$ is continuous at all $x \in U$.

A third way of defining continuity is in terms of open sets. First,
another definition.
\begin{definition}
  Let $f: X \to Y$. $f$ is continuous The \textbf{preimage} of $V \subseteq Y$ is the
  set in $X$, $f^{-1}(V)$ defined by
  \[ f^{-1} (V) = \{ x \in X: f(x) \in V \} \]
\end{definition}
A function is continuous if and only if the preimage of any open set
is open. 
\begin{lemma}\label{lem:copen}
  $f:X \to Y$ is continuous at $x$ if and only if for all open $V$
  with $f(x) \in V$,
  $\exists$ open $U \subseteq X$ such that $x \in U \subseteq f^{-1}(V)$.
\end{lemma}
\begin{proof}
  %On problem set.
  Suppose for all open $V \subseteq Y$ that $\exists$ open $U$ such
  that $x \in U \subseteq f^{-1}(V)$. We want to show that then $f$ is
  continuous on $U$. To do that, let $x_n \to x$ and let
  $\epsilon>0$. $N_\epsilon(f(x))$ is open, so by assumption,
  $\exists$ open $U$ such that
  $x \in U \subseteq f^{-1}(N_{\epsilon}(f(x)))$. By the definition of
  open sets, $\exists$ $\delta > 0$ such that
  $N_{\delta}(x) \subseteq U \subseteq f^{-1}(N_{\epsilon}(f(x)))$. By
  the definition of $x_n \to x$, $\exists N$ such that if $n \geq N$,
  $x_n \in N_{\delta}(x)$. Then $x_n \in f^{-1}(N_{\epsilon}(f(x)))$,
  so $f(x_n) \in N_{\epsilon}(f(x))$, i.e.
  \[ d\left(f(x_n),f(x) \right) < \epsilon. \]
  Therefore, $f(x_n) \to f(x)$. 

  Conversely, suppose $f$ is continuous. Let $V \subseteq Y$ be
  open. We want to show that $\exists$ open $U$ such that
  $x \in U \subseteq f^{-1}(V)$. Suppose there is no such $U$. Then
  for any $\epsilon > 0$,
  $\exists \tilde{x}_{\epsilon} \not\in f^{-1}(V)$ with
  \[ d(x,\tilde{x}_{\epsilon}) < \epsilon. \]
  Pick a sequence of $\epsilon_n$ that converges to zero, such as
  $\epsilon_n = 1/n$. Then the associated $\tilde{x}_n \to
  x$. However, since each $\tilde{x}_n \not\in f^{-1}(V)$,
  $f(\tilde{x}_n) \in V^c$. But then having $f(\tilde{x}_n) \to f(x)$
  would mean that $V^c$ is not closed, which contradict $V$ being
  open. Thus, there must exist an open $U$ such that $x \in U
  \subseteq f^{-1}(V)$  when $f$ is continuous.
\end{proof}
Since the a set is open if and only its complement is closed, we can
also define continuity using closed sets.
\begin{corollary}
  $f:X \to Y$ is continuous if and only if $f^{-1}(V)$ is closed for all
  closed $V \subseteq Y$. 
\end{corollary}
\begin{proof}
  Let $V \subseteq Y$ be closed. Then $V^c$ is open. Also, note that
  the complement of the preimage of $V$ is the preimage of $V^c$. In
  symbols,
  \begin{align*}
    f^{-1}(V)^c = \{x \in X: f(x) \not\in V \} = \{x \in X: f(x) \in V^c
    \} = f^{-1}(V^c).
  \end{align*}
  From lemma \ref{lem:copen}, $f$ is continuous iff $f^{-1}(V^c) =
  f^{-1}(V)^c$ is open for all open sets $V^c$, which is true iff
  $f^{-1}(V)$ is closed for all closed sets $V$.
\end{proof}

Earlier we saw that convergence of sequences is preserved by
arithmetic. Since continuity can be defined using sequences, it should
be no surprise that continuity is also preserved by arithmetic.
\begin{theorem}
  Let $f:X \to Y$ and $g:X \to Y$ be continuous and $X$ and $Y$ be
  vector spaces. Then $(f+g)(x) = f(x) + g(x)$ is continuous.
\end{theorem}
\begin{proof}
  If $f$ and $g$ are continuous, then by definition $f(x_n) \to f(x)$
  and $g(x_n) \to g(x)$ whenever $x_n \to x$. From the previous
  lecture the limit of a (finite) sum is the sum of limits, so 
  $f(x_n) + g(x_n) \to f(x) + g(x)$, and $f+g$ is continuous.
\end{proof}
Similar results can be shown for subtraction, multiplication, etc,
whenever they are well defined. 

Continuity is also preserved by composition.
\begin{theorem}
  Let $f:X \to Y$ and $g:Y \to Z$ be continuous where $X$, $Y$, and
  $Z$ are metric spaces. Then $f \circ g$ is continuous, where
  \[ (f \circ g)(x) = f(g(x)). \]
\end{theorem}  
\begin{proof}
  Let $x_n \to x$. $g$ is continuous, so $g(x_n) \to g(x)$. $f$ is
  also continuous, so $f(g(x_n)) \to f(g(x))$.
\end{proof}
$f\circ g$ is called the composition of $f$ and $g$.

\subsection{Onto, one-to-one, and inverses}
We have already used the concepts of onto, one-to-one, and inverses.
We restate the definitions here.
\begin{definition}
  $f:X \to Y$ is \textbf{one-to-one} or \textbf{injective} if for all
  $x_1, x_2 \in X$, 
  \[ f(x_1) = f(x_2) \]
  if and only if $x_1 = x_2$.
\end{definition}
Equivalently, $f$ is injective if for each $y \in Y$, the set $\{x:
f(x) = y\}$ is either a singleton or empty.  In terms of a nonlinear
equation, if $f$ is one-to-one, then $f(x) = b$ has at most one
solution. 
\begin{definition}
  $f:X \to Y$ is \textbf{onto} or \textbf{surjective} if $\forall y
  \in Y$, $\exists x \in X$ such that $f(x) = y$.
\end{definition}
In terms of a nonlinear equation, if $f$ is onto, then $f(x) = b$ has
at least one solution. When $f$ is one-to-one and onto, we say that
$f$ is \textbf{bijective}. A bijective function has an inverse.
\begin{definition}
  If $f:X \to Y$ is bijective, then the \textbf{inverse} of $f$,
  written $f^{-1}$ satisfies
  \[ f(f^{-1} (y)) = y  \]
  and 
  \[ f^{-1} ( f(x) ) = x. \]
\end{definition}

\begin{remark}
  While writing these notes, I briefly tried to prove that if $f:X \to
  Y$ is bijective and continuous, then $f^{-1}$ is continuous. I could
  not do this, which is good, because that statement is false. You have
  to be a little creative in defining $X$ and $Y$ to come up with a
  counterexample. Let $X = [0,2\pi)$ and $Y = \{(x,y) \in \R^2: x^2 +
  y^2 = 1 \}$. Then $f(x) = (cos(x), sin(x))$ is bijective and
  continuous, but $f^{-1}$ is not continuous at $(1,0)$. 
  
  This counterexample is actually related to a fundamental fact in
  topology. You may remember from last lecture that topology is about
  studying spaces with open and closed sets that do not necessarily
  have a metric. One thing that people are interested when studying
  such spaces is finding a continuous (in both directions) bijections
  between them. Loosely speaking, two topological spaces will have a
  continuous bijection between them if one can be bent and stretched
  from one into the form of another. You cannot bend a circle into an
  interval without breaking the circle, so there is no continuous
  bijection between the circle and an interval. When there is a
  continuous bijection between two spaces, they have the same
  collection of open sets, so to a topologist, they are the same. We
  then call the spaces homeomorphic (or topologically
  isomorphic). Loosely speaking, spaces will be homeomorphic if they
  are the same dimension and their shapes have the same number of
  holes.  The circle has one hole, an interval has none, so they are
  not topologically isomorphic. I'd be remiss not to make a joke now,
  so here goes: Why did the topologist eat her/his coffee mug and
  drink from his/her donut?  Because they're topologically
  isomorphic. Hahaha.
\end{remark}

We begin the course by studying how to solve optimization
problems. However, we did not worry too much about conditions that
ensure an optimum exists. One fundamental result that ensures the
existence of optima is Weierstrass's theorem.
\begin{theorem}[Weierstrass]
  Let $X$ be a metric space and $f:X \to \R$ be continuous and
  $K \subset X$ be compact and nonempty. Then $\exists x^* \in K$ such
  that $f(x^*) \geq f(x) \forall x \in K$.
\end{theorem}
\begin{proof}
  First, we show that $\sup_{x \in K} f(x)$ exists. 
  $\R$ has the least upper bound property, so it is enough to
  show that $f$ is bounded. Let $\epsilon>0$. For each $x$, let $x \in
  U_x$ be open and $U_x \subseteq f^{-1}(N_\epsilon(x))$. Such $U_x$
  exist since $f$ is continuous. Also, $\{U_x\}_{x \in K}$ is open
  cover of $K$. $K$ is compact, so there exists a finite subcover. Let
  this subcover be $U_{x_1}, ..., U_{x_n}$. For any $x \in K$, $x \in
  U_{x_j}$ for some $j$ and 
  \[ f(x) \leq f(x_j) + \epsilon. \]
  Hence,
  \[ f(x) \leq \max_{j\in\{1, ..., n\}} f(x_j) + \epsilon \]
  and $f$ is bounded on $K$. $\R$ has the least upper bound property
  --- any set bounded above has a least upper bound, so
  $\bar{f}=\sup_{x \in K} f(x)$ exists.

  Let $\bar{f} = \sup_{x \in K} f(x)$. Let $y_n = \bar{f} -
  1/n$. By definition of supremum, for each $n$, $\exists$ $x_n$
  such that $y_n \leq f(x_n) \leq \bar{f}$. Since $K$ is compact,
  the sequence $\{x_n\}$ must have a convergent subsequence
  $x_{n_k} \to x \in K$. $f$ is continuous, so $f(x_{n_k}) \to
  f(x)$. Also, by construction, $d(f(x_{n_k}) , \bar{f} ) \to 0$, so
  it must be that $f(x) = \bar{f}$. 
\end{proof}

%%%%%%%%%%%%%%%%%%%%%%%%%%%%%%%%%%%%%%%%%%%%%%%%%%%%%%%%%%%%%%%%%%%%%%%%
\section{Correspondences \label{sec:corr}} 

\footnote{This section is largely based on section 2.1.5 of
  \cite{carter2001}.}A function, $f:X \to Y$ associates exactly one
element of $Y$, $f(x)$, with each $x \in X$. Often we encounter things
that are like functions, but for each $x \in X$, there are multiple
elements of $Y$. We call this generalization of a function as
correspondence.
\begin{definition}
  A \textbf{correspondence} from a set $X$ to a set $Y$, is a rule
  that assigns to each a $x \in X$ a subset of $Y$. We denote a
  correspondence by $\phi: X \corres Y$.
\end{definition}
An equivalent definition is that $\phi: X \corres Y$ is a function
from $X$ to the power set of $Y$. Correspondences appear often in
economics, especially as constraint sets in optimization problems.
\begin{example}[Budget correspondence]
  Suppose there are $n$ goods with prices $p \in \R^n$. Then given
  income of $m$, a consumer can afford $\chi(p,m)=\{ x \in X \subseteq
  \R^n: p'x \leq m\}$, which defines a correspondence $\chi: \R^{n+1}
  \corres X$. We can write the consumer's problem of maximizing
  utility subject to the budget constraint as
  \begin{align*}
    \max_{x \in \chi(p,m)} u(x) 
  \end{align*}
  If this problem has a solution, then the indirect utility function
  is the maximized utility,
  \[ v(p,m) = \max_{x \in \chi(p,m)} u(x). \]
  The demand correspondence (usually function) is
  \[ x^*(p,m) = \argmax_{x \in \chi(p,m)} u(x). \] 
  Such maximization problems are central to economics. To derive
  properties of the indirect utility and demand functions it is often
  useful to treat the budge set as a correspondence.
\end{example}
Correspondences also appear in economics in models with multiple
equilibria, such as many games.

Defining continuity is a bit more complicated for correspondences than
for functions. A function can either be continuous or it can jump. A
correspondence can also expand or contract. For example, consider
$\xi:\R \corres \R$ defined by 
\[ \xi(x) = \begin{cases} 
  [0,1] & \text{ if } x > 0 \\
  [1/4,3/4] & \text{ if } x \leq 0 
\end{cases} \]
and $\psi: \R \corres \R$ defined by
\[ \psi(x) = \begin{cases} 
  [0,1] & \text{ if } x \geq 0 \\
  [1/4,3/4] & \text{ if } x < 0 
\end{cases} \]
Both these correspondences are somewhat continuous because they
contain a continuous function, e.g. $f(x) = 1/2$,  for all
$x$. However, they are also somewhat discontinuous because the
corresponding set changes suddenly at $0$. Motivated by this
observation we define the following:
\begin{definition}
  A correspondence, $\phi: X \corres Y$ is \textbf{upper
    hemicontinuous} at $x$ if for all sequences $x_n \to x$ and $y_n
  \in \phi(x_n)$ with $y_n \to y$, then $y \in \phi(x)$.
\end{definition}
In the previous example, $\psi$ is upper hemicontinuous at $0$, but
$\xi$ is not. To see this consider $x_n = 1/n$ and $y_n = 1$.
\begin{definition}
  A correspondence, $\phi: X \corres Y$ is \textbf{lower
    hemicontinuous} at $x$ if for all sequences $x_n \to x$ and $y \in
  \phi(x)$, there exists a subsequence, $x_{nk}$ and $y_k \in
  \phi(x_{nk})$ with $y_k \to y$.
\end{definition}
In the previous example, $\xi$ is lower hemicontinuous at $0$, but
$\psi$ is not. To see this consider $x_n = -1/n$ and $y = 1$. 
\begin{definition}\label{def:corrcont}
  We say that a correspondence is \textbf{continuous} if it is both
  upper and lower hemicontinuous.
\end{definition}
At all $x \neq 0$, $\xi$ and $\psi$ are continuous.

\begin{exercise}
  A correspondence $\phi:X \corres Y$ can also be viewed as a function
  $\phi:X \to \mathcal{P}(Y)$. There is a metric on the set of compact
  subsets of $Y$ (which is a subset of $\mathcal{P}(Y)$) that leads to
  the same definition of continuity as in definition
  \ref{def:corrcont}. What is that metric?
\end{exercise}

\bibliographystyle{jpe}
\bibliography{../526}

\end{document}
