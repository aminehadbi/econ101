
\documentclass[12pt,reqno]{amsart}

\usepackage{amsmath}
\usepackage{amsfonts}
\usepackage{amssymb}

\usepackage{graphicx}
%\usepackage{epstopdf}
\usepackage[colorlinks=true]{hyperref}
\usepackage[left=1in,right=1in,top=0.9in,bottom=0.9in]{geometry}
\usepackage{multirow}
\usepackage{verbatim}
\usepackage{fancyhdr}
\usepackage{mdframed}
\usepackage{natbib}
\usepackage{multicol}
\renewcommand{\cite}{\citet}

%\usepackage[small,compact]{titlesec} 

%\usepackage{pxfonts}
%\usepackage{isomath}
\usepackage{mathpazo}
%\usepackage{arev} %     (Arev/Vera Sans)
%\usepackage{eulervm} %_   (Euler Math)
%\usepackage{fixmath} %  (Computer Modern)
%\usepackage{hvmath} %_   (HV-Math/Helvetica)
%\usepackage{tmmath} %_   (TM-Math/Times)
%\usepackage{cmbright}
%\usepackage{ccfonts} \usepackage[T1]{fontenc}
%\usepackage[garamond]{mathdesign}
\usepackage{xcolor}
\usepackage[normalem]{ulem}

\newtheorem{theorem}{Theorem}[section]
\newtheorem{conjecture}{Conjecture}[section]
\newtheorem{corollary}{Corollary}[section]
\newtheorem{lemma}{Lemma}[section]
\newtheorem{proposition}{Proposition}[section]
\theoremstyle{definition}
\newtheorem{assumption}{}[section]
%\renewcommand{\theassumption}{C\arabic{assumption}}
\newtheorem{definition}{Definition}[section]

\newtheorem{step}{Step}[section]
\newtheorem{rem}{Comment}[section]
\newtheorem{ex}{Example}[section]
\newtheorem{hist}{History}[section]
\newtheorem*{ex*}{Example}
\newtheorem{exer}{Exercise}[section]


\newenvironment{example}
  {\begin{mdframed}\begin{ex}}
  {\end{ex}\end{mdframed}}

\newenvironment{history}
  {\begin{mdframed}\begin{hist}}
  {\end{hist}\end{mdframed}}

\newenvironment{exercise}
  {\begin{mdframed}\begin{exer}}
  {\end{exer}\end{mdframed}}

\newenvironment{remark}
  {\begin{mdframed}\begin{rem}}
  {\end{rem}\end{mdframed}}



\linespread{1.1}

\pagestyle{fancy}
%\renewcommand{\sectionmark}[1]{\markright{#1}{}}
\fancyhead{}
\fancyfoot{} 
%\fancyhead[LE,LO]{\tiny{\thepage}}
\fancyhead[CE,CO]{\tiny{\rightmark}}
\fancyfoot[C]{\small{\thepage}}
\renewcommand{\headrulewidth}{0pt}
\renewcommand{\footrulewidth}{0pt}

\fancypagestyle{plain}{%
\fancyhf{} % clear all header and footer fields
\fancyfoot[C]{\small{\thepage}} % except the center
\renewcommand{\headrulewidth}{0pt}
\renewcommand{\footrulewidth}{0pt}}

\makeatletter
\renewcommand{\@maketitle}{
  \null 
  \begin{center}%
    \rule{\linewidth}{1pt} 
    {\Large \textbf{\textsc{\@title}}} \par
    {\normalsize \textsc{Paul Schrimpf}} \par
    {\normalsize \textsc{\@date}} \par
    {\small \textsc{University of British Columbia}} \par
    {\small \textsc{Economics 526}} \par
    \rule{\linewidth}{1pt} 
  \end{center}%
  \par \vskip 0.9em
}
\makeatother

\definecolor{ubcblue}{cmyk}{1.00,.62,.0,.52}
\definecolor{ubcgrey}{cmyk}{.42,.08,.0,.40}

\hypersetup{
  linkcolor  = ubcblue,
  citecolor  = ubcgrey,
  urlcolor   = ubcblue,
  colorlinks = true
}

\usepackage{mathtools}
\newcommand{\verteq}{\rotatebox{90}{$\,=$}}
\newcommand{\equalto}[2]{\underset{\scriptstyle\overset{\mkern4mu\verteq}{#2}}{#1}}
\newcommand{\nulls}{\mathrm{null}}
\newcommand{\range}{\mathrm{range}}

\newcommand{\argmax}{\operatornamewithlimits{arg\,max}}
\newcommand{\argmin}{\operatornamewithlimits{arg\,min}}
\def\inprobLOW{\rightarrow_p}
\def\inprobHIGH{\,{\buildrel p \over \rightarrow}\,} 
\def\inprob{\,{\inprobHIGH}\,} 
\def\indist{\,{\buildrel d \over \rightarrow}\,} 
\def\F{\mathbb{F}}
\def\R{\mathbb{R}}
\def\Er{\mathrm{E}}
\def\Pr{\mathrm{P}}
\def\En{\mathbb{E}_n}
\def\Pn{\mathbb{P}_n}
\newcommand{\gmatrix}[1]{\begin{pmatrix} {#1}_{11} & \cdots &
    {#1}_{1n} \\ \vdots & \ddots & \vdots \\ {#1}_{m1} & \cdots &
    {#1}_{mn} \end{pmatrix}}
\newcommand{\iprod}[2]{\left\langle {#1} , {#2} \right\rangle}
\newcommand{\norm}[1]{\left\Vert {#1} \right\Vert}
\newcommand{\abs}[1]{\left\vert {#1} \right\vert}
\renewcommand{\det}{\mathrm{det}}
\newcommand{\rank}{\mathrm{rank}}
\newcommand{\spn}{\mathrm{span}}
\newcommand{\row}{\mathrm{Row}}
\newcommand{\col}{\mathrm{Col}}
\renewcommand{\dim}{\mathrm{dim}}
\newcommand{\prefeq}{\succeq}
\newcommand{\pref}{\succ}
\newcommand{\seq}[1]{\{{#1}_n \}_{n=1}^\infty }
\renewcommand{\to}{{\rightarrow}}
\newcommand{\corres}{\overrightarrow{\rightarrow}}
\def\rel{\,{\buildrel R \over \sim}\,}

 

\title{Sets and their properties}
\date{\today}

\begin{document}

\maketitle

Much (perhaps all) of mathematics is about studying sets of objects
with particular properties. 

Section \ref{s:sets} introduces sets and some related concepts.
Section \ref{s:card} briefly discusses cardinality and introduces
countable and uncountable sets. Section \ref{s:relations} is about
relations, especially orders, which are used to state Arrow's
impossibility theorem. The appendix section \ref{s:numbers} is about
familiar sets of numbers, including the integers, rationals, and real
numbers. The properties of these sets of numbers that make them
distinct are discussed.

\subsection*{References}

Section \ref{s:sets} on sets is partly based on chapter 1 of
\cite{carter2001}. Any similar high-level mathematical economics
textbook covers similar material. Examples include \cite{fuente2000},
\cite{ok2007}, and \cite{corbae2009}. Textbooks on real analysis, such
as \cite{rudin1976} and \cite{tao2006}, also typically start with a
section about sets.

Section \ref{s:card} about cardinality is largely based on chapter 2
\citet{rudin1976}. Chapter B of \cite{ok2007} covers similar
material. Weeks 2 and 3 of the notes of \cite{tao2003} (on which
\cite{tao2006} is based) also cover cardinality.

Section \ref{s:relations} about relations is based on chapter 1.2 of
\cite{carter2001}. Arrow's impossibility theorem first appeared in
\cite{arrow1950}. \cite{feldman1974} is a more approachable, simplified
proof of the theorem.

The appendix section \ref{s:numbers} is based on \cite{rudin1976}, but
any textbook on real analysis will cover similar
material. \cite{tao2006} (or the note version \cite{tao2003}) is
especially detailed and careful in its construction of the real
numbers. 


%%%%%%%%%%%%%%%%%%%%%%%%%%%%%%%%%%%%%%%%%%%%%%%%%%%%%%%%%%%%%%%%%%%%%% 
\section{Sets \label{s:sets}} 

A \textbf{set} is any well-specified collection of
elements.\footnote{``Well-specified'' is somewhat ambiguous, and this
  ambiguity can lead to trouble such as Russell's paradox or Cantor's
  paradox. We'll ignore these paradoxes, but rest assured that they
  can be avoided by more carefully defining ``well-specified.''
} Sets are conventionally denoted by capital letters, and
elements of a set are usually denoted by lower case letters. The
notation, $a \in A$, means that $a$ is a member of the set $A$. A set
can be defined by listing its elements inside braces. For example,
\[ A = \{ 4, 5, 6 \} \]
means that $A$ is a set of three elements with members $4$, $5$, and
$6$. The members of a set need not be explicitly listed. Instead, they
can be defined by some logical relation. For example, the same set $A$
could be written
\begin{equation}
  A = \{ n \in \mathbb{N} : 3 < n < 7 \}  \label{e:s1}
\end{equation}
where $\mathbb{N} = \{1, 2, 3, ... \}$ is the natural numbers. The
expression in (\ref{e:s1}) could be read as, ``the set of natural
numbers, $n$, such that 3 is less than $n$ is less than 7.''
Sometimes $|$ will be used to mean ``such that'' instead of $:$. The
elements of sets need not be simple things like numbers. For example,
if $A_k = \{ n \in \mathbb{N}: n > k \}$ is the set of natural numbers
greater than $k$, then you could have a set of sets, $B = \{ A_1,
A_{10}, A_{6} \}$.  Sets are unordered, so the previous definition of
$B$ is the same as $B = \{ A_1, A_6, A_{10} \}$. Also, sets do not
contain duplicates, so for example, $\{ 1, 1, 2 \} \equiv \{1, 2
\}$. Sets can be empty. The empty set, also called the null set, is
denoted by $\emptyset$ or, less commonly, $\{ \}$.

\subsection{Economic examples}


Sets appear all over economics.\footnote{These examples come
from chapter 1 of Carter.} 
\begin{example}\label{ex:sampleSpace}[Sample space]
  In a random experiment, the set of all possible outcomes is called
  the \textbf{sample space}. E.g. for the roll of a dice, the sample
  space if $\{1,2 ,3 , 4, 5, 6\}$. An \textbf{event} is any subset of
  the sample space.
\end{example}

\begin{example}\label{ex:games}[Games]
  A game is a model of strategic decision making. A game consists of a
  finite set of $n$ players, say $N = \{1,2,..., n\}$. Each player $i
  \in N$ chooses an action $a_i$ from a set of actions $A_i$. The
  outcome of the game depends on the actions chosen by all players. 
\end{example}

\begin{example}\label{ex:conset}[Consumption set]
  The \textbf{consumption set} is the set of all feasible consumption
  bundles. Suppose there are $n$ commodities. A consumer chooses a
  consumption bundle $\mathbf{x} = (x_1, x_2, ..., x_n)$. Consumption
  cannot be negative, so the consumption set is a subset of $\R^n_+ =
  \{(x_1, ..., x_n) : x_1 \geq 0, x_2 \geq 0, ... x_n \geq 0 \}$.
\end{example}

\subsection{Set operations}

Given two sets $A$ and $B$, a new set can be formed with the following
operations:
\begin{enumerate}
\item \textbf{Union:} $A \cup B = \{x: x\in A \text{ or } x \in B\}$.
\item \textbf{Intersect:} $A \cap B = \{x: x \in A \text{ and } x \in
  B\}$. 
\item \textbf{Minus:} $A \setminus B = \{ x: x\in A \text{ and }
  \not\in B \}$
\item \textbf{Product:} $A \times B = \{(x,y): x \in A, y \in B \}$
\item \textbf{Power set:} $\mathcal{P}(A) =$ set of all subsets of $A$ 
\end{enumerate}
Often, we will discuss sets that are all subsets of some universal
set, $U$. In this case, the \textbf{complement} of $A$ in $U$ is $A^c
= U \setminus A$. If we have an indexed collection of sets, $\{A_k
\}_{k \in \mathcal{K}}$, we may take the union or intersection of all
these sets and denote it as $\cup_{k \in \mathcal{K}} A_k $ or
$\cap_{k \in \mathcal{K}} A_k$.

\subsection{Set relations}
If every element of $A$ is also in $B$, then we say that
$B$ contains $A$ and write $B \supseteq A$, or $A$ is a subset of $B$
and write $A \subseteq B$. If, additionally, there exists $b \in B$
such that $b \not\in A$, then we say that $A$ is a proper subset of $B$,
which is denoted by $A \subset B$ or $B \supset A$. 

\begin{example}[\ref{ex:games} Games continued] 
  In a game subsets of players are called \textbf{coalitions}. The set
  of all coalitions is the power set of the set of players,
  $\mathcal{P}(N)$. 

  The \textbf{action space} of a game is the set of all possible
  outcomes or combinations of actions, $A = A_1 \times A_2 \times
  ... \times A_n$. An element of $A$, $a = (a_1, a_2, ..., a_n)$ is
  called an \textbf{action profile}. 
\end{example}

%%%%%%%%%%%%%%%%%%%%%%%%%%%%%%%%%%%%%%%%%%%%%%%%%%%%%%%%%%%%%%%%%%%%%%%%%%%%%%%%
\subsection{Cardinality \label{s:card}}

\footnote{This section based on Chapter 2 \citet{rudin1976}.} 
Sometimes, we want to compare the size of two
sets. This is easy when sets are finite; we simply count how many
elements each has. It is not so easy to compare the size of infinite
sets. Consider, for example, the natural numbers, $\mathbb{N}$, the
integers $\mathbb{Z}$, rationals, $\mathbb{Q}$, and real numbers,
$\mathbb{R}$. Let $|A|$ denote the ``size'' of $A$ (we will define it
precisely later). We know that
\[ \mathbb{N} \subset \mathbb{Z} \subset \mathbb{Q} \subset
\mathbb{R}, \]
so it seems sensible to say that 
\[ |\mathbb{N}| < |\mathbb{Z}| < |\mathbb{Q}| < |\mathbb{R}|. \] 
On the other hand, the even integers are a subset of $\mathbb{Z}$, but
since we can write the set of even integers as $\{2x: x \in \mathbb{Z}
\}$, it doesn't seem like there are any more integers than even
integers. It was questions like these that led Georg Cantor to pioneer
set theory in the 1870's. 

A function (aka mapping), $f:A \rightarrow B$ is called
\textbf{one-to-one} (aka injective) if for every $b \in B$ the set
$\{a: f(a) = b \}$ is either a singleton or emptyo. $f$ is called
\textbf{onto} (aka surjective) if $\forall b \in B$ $\exists a \in A:
f(a) = b$.  If there exists a one-to-one mapping of $A$ onto $B$ (aka
bijection or one-to-one correspondence), then we say that $A$ and $B$
have the same \textbf{cardinal number} (or cardinality) and write $|A|
= |B|$.  Let $J_n = \{1, ..., n \}$. $A$ is \textbf{finite} if $|A| =
|J_n|$.  $A$ is \textbf{countable} if $|A| = |\mathbb{N}|$. $A$ is
\textbf{uncountable} if $A$ is neither finite nor countable.  You
should verify that the relation $|A| = |B|$ is reflexive ($|A| =
|A|$), symmetric ($|A| = |B|$ implies $|B| = |A|$), and transitive (if
$|A| = |B|$ and $|B| = |C|$ then $|A| = |C|$). 

\begin{lemma}
  $\mathbb{Z}$ is countable. 
\end{lemma}
\begin{proof}
  We can construct a bijection between $\mathbb{Z}$ and $\mathbb{N}$
  as follows:
  \begin{align*}
    \begin{matrix}
      \mathbb{Z}: & 0, & -1, & 1, & 2, & -2, & 3, & -3, ... \\
      \mathbb{N}: & 1, & 2, & 3, & 4, & 5, & 6, & 7,  ...
    \end{matrix}
  \end{align*}
  Or as a formula, $f:\mathbb{N} \rightarrow \mathbb{Z}$ with 
  \[ f(n) = \begin{cases} (n-1)/2 \text{ if } n \text{ odd} \\
    -n/2 \text{ if } n \text{ even}. \end{cases} \]  
\end{proof}

\begin{theorem} \label{thm:countsubset}
  Every infinite subset of a countable set $A$ is countable.
\end{theorem}
\begin{proof}
  $A$ is countable, so there exists a bijection from $A$ to
  $\mathbb{N}$. We can use this mapping to arrange the elements of $A$
  in a sequence, $\{a_{n}\}^\infty_{n=1}$\footnote{By this notation,
    we mean an infinite ordered list of elements of $A$, i.e. $a_1,
    a_2, a_3, ...$.}. Let $B$ be an infinite subset of $A$. Let $n_1$
  be the smallest number such that $a_{n_1} \in B$. Given $n_{k-1}$,
  let $n_k$ be the smallest number greater than $n_{k-1}$ such that
  $a_{n_k} \in B$. Such an $n_k$ always exists since $B$ is
  infinite. Also, $B = \{a_{n_k} \}_{k=1}^\infty$ since otherwise
  there would be a $b \in B$, but $b\not\in A$. Thus, $f(k) = a_{n_k}$
  is a one-to-one correspondence between $B$ and $\mathbb{N}$.
\end{proof}

\begin{theorem}
  The rational numbers are countable.
\end{theorem}
\begin{proof}
  Consider the following arrangement of positive rational numbers:
  \begin{align*}
    \begin{array}{ccccc} 
      1/1 & 2/1 & 3/1 & 4/1 & \cdots \\
      1/2 & 2/2 & 3/2 & 4/2 & \cdots \\
      1/3 & 2/3 & 3/3 & 4/3 & \cdots \\
      \vdots &  &  & \ddots 
    \end{array}
  \end{align*}
  Starting in the top left and going back and forth diagonally, we get
  the following sequence:
  \begin{align*}
    1/1, 1/2, 2/1, 1/3, 2/2, 3/1, ...
  \end{align*}
  Adding zero and the negative rationals, we can write e.g.\
  \begin{align*}
    & 0,1/1, -1/1, 1/2, -1/2, 2/1, -2/1, 1/3, -1/3, 2/2, -2/2, 3/1,
    ... \\
    = & q_1, q_2, q_3 , q_4, ... \\
  \end{align*}
  Continuing on in this way, we could list all rational numbers. Some
  of these fractions represent the same number and can be
  removed. Thus, we obtain a correspondence between the rationals and
  an infinite subset of $\mathbb{N}$. However, by theorem
  \ref{thm:countsubset}, this subset is countable, so the rationals
  are also countable.
\end{proof}

\begin{theorem}
  The real numbers are uncountable.
\end{theorem}
\begin{proof} (Cantor's diagonal argument)
  We have not rigorously defined the real numbers, so we will take for
  granted the following: every infinite decimal expansion,
  (e.g. $0.135436080...$) represents a unique real number in $[0,1)$,
  except for expansions that end in all zeros or nines, which are
  equivalent\footnote{E.g. $0.199... = 0.200...$}. 
  
  We will use proof by contradiction to prove the theorem. Proof by
  contradiction is a common technique that works by showing that if
  the theorem were false, then we could prove something that
  contradicts what we know is true. 

  Suppose the theorem is false. Then we can construct a surjective
  mapping from $\mathbb{N}$ to $(0,1)$. That is we can list all real
  numbers in $(0,1)$ as
  \begin{align*}
    \begin{array}{c c ccc c}
      r_1 & =  0. & d_{11}& d_{12}& d_{13}& ... \\
      r_2 & =  0. &d_{21}& d_{22}& d_{23}& ... \\
      r_3 & =  0. &d_{31}& d_{32}& d_{33}& ... \\
      \vdots & & \vdots 
    \end{array}
  \end{align*}
  where each $d_{ij} \in \{0,1,...,9\}$, and no expansion ends in all
  nines. We will now show that there is a real number in $(0,1)$ that
  is not in the list. Let $x^* = 0.d^*_1 d^*_2 d^*_3 ...$. where
  $d^*_n$ is chosen such that $d^*_n \neq d_{nn}$ and $x^*$ is sure
  not to end in all nines. There are many possibilities, but to be
  concrete, let's set
  \[ 
  d^*_n
  = \begin{cases} d_{nn} + 1 \text{ if } d_{nn} < 8 \\
    0 \text{ if } d_{nn} \geq 8 
  \end{cases} .
  \]      
  $x^*$ is in $(0,1)$, but $x^* \neq r_n$ for any $n$ because $d^*_n
  \neq d_{nn}$. Thus, we have a contradiction, and there cannot be a
  onto mapping from $\mathbb{N}$ to $(0,1)$.  If there is no
  surjective mapping from $\mathbb{N}$ to $(0,1)$, there can be no
  surjective mapping from $\mathbb{N}$ to $\mathbb{R}$ since $(0,1)
  \subset \mathbb{R}$. 
\end{proof}
Countable sets are said to have cardinality $\aleph_0$ (``aleph
null''). Note that an implication of theorem \ref{thm:countsubset} is
that $\aleph_0$ is the smallest infinite cardinal number. The real
numbers have cardinality of the continuum, sometimes written
$2^{\aleph_0}$ or $\mathbf{c}$. You might be wondering whether there
are larger cardinal numbers. The answer is yes. The set of all subsets
of a set, $A$, called the \textbf{power set} of $A$, always has larger
cardinality $2^{|A|}$ (the proof of this is similar to the proof that
the real numbers are uncountable). 

A final question to ask yourself is whether there are sets with
cardinality between $\aleph_0$ and $2^{\aleph_0}$. The answer to that
question is whatever you want it to be. The conjecture that there are
no cardinal numbers between $\aleph_0$ and $2^{\aleph_0}$ is known as
the continuum hypothesis. It was proposed by Cantor in the 1870s. In
1900, Hilbert made a famous list of 23 important unsolved problems in
mathematics. The continuum hypothesis was the first. In 1940,
G\"{o}del showed that the continuum hypothesis cannot be disproved
from the standard axioms that lie at the foundation of mathematics.
In 1963, Cohen showed that the continuum hypothesis cannot be proved
from the standard axioms. This is an example of G\"{o}del's
incompleteness theorem, a very interesting result that we won't be
able to cover in this course. Loosely speaking, G\"{o}del's
incompleteness theorem says that for any non-trivial set of
assumptions and system of logic, you can make statements consistent
with the system of logic that cannot be proven or disproven from the
assumptions.

%%%%%%%%%%%%%%%%%%%%%%%%%%%%%%%%%%%%%%%%%%%%%%%%%%%%%%%%%%%%%%%%%%%%%%

\section{Relations \label{s:relations}}

There is not much more we can say about generic sets. Fortunately,
sets used in economics and mathematics typically have some additional
properties that we can utilize. Basic mathematics studies real
numbers. Numbers have many properties: they are ordered, they can be
added and multiplied, etc. Most of the abstract sets that we will
study will have some, but not all, of the properties of real
numbers. We will begin by studying ordered sets. 

Orderings, or, more generally, relations, are important in economics
because they can be used to represent preferences. Relations are
things like $=$, $<$, $\leq$, and $\subset$. Formally,
\begin{definition}[Relation]
  A \textbf{relation} on two sets $A$ and $B$ is any subset of $A
  \times B$, $R \subseteq A \times B$. We usually denote relations by
  $a \rel b$ if $(a,b) \in R$ (where $\rel$ could be some other symbol).
\end{definition}
Although we define relations in terms of a subset of the product set,
it's usually easier to just think about relations as a rule expressing
the relationship between elements of $A$ and $B$. 
For most relations, $A=B$, and then we say $\rel$ is a relation on $A$. 
\begin{example}
  Let $A=B=\R$. Then $<$ is associated with $R_{<} = \{(a,b) \in \R^2
  : a<b \}$. 
\end{example}
Relations usually have some of following properties.
\begin{definition}[Properties of relations]  
  A relation $\rel$ on $A$ is 
  \begin{itemize}
  \item \textbf{reflexive} if $a \rel a$ $\forall a \in A$,
  \item \textbf{transitive} if $a \rel b$ and $b \rel c$ implies
    $a \rel c$,
  \item \textbf{symmetric} if $a \rel b$ implies $b \rel a$,
  \item \textbf{antisymmetric} if $a \rel b$ and $b \rel a$
    implies $a=b$,
  % \item \textbf{asymmetric} if $a \rel b$ implies $b$ is not
  %   $\rel a$, and
  \item \textbf{complete} if either $a \rel b$ or $b \rel a$ or
    both $\forall a, b \in A$. 
  \end{itemize}
\end{definition}
\begin{exercise}
  As an exercise, you may want to work out which of the above
  properties $=$, $<$, and $\leq$ on $\R$ have.
\end{exercise}
Slightly confusingly, symmetric and antisymmetric are not
opposites. The usual equality in $\R$, $=$ is both symmetric and
antisymmetric.
\begin{example}[Preference relation] 
  A consumer's preference relation, $\pref$, is a relation on her
  consumption set, $X$. $x \pref y$ means that the consumer likes the
  bundle of goods $x$ at least as much of the bundle of goods $y$. We
  will assume that preference relations are complete,
  transitive.  
\end{example}
\begin{exercise}
  Show that any complete and transitive order is also reflexive.
\end{exercise}

\subsection{Equivalence relations}
An \textbf{equivalence relation} is a relation that is reflexive,
transitive, and symmetric. If $\sim$ is an equivalence relation on
$X$, then the \textbf{equivalence class} of $x$ is $\sim(x) = \{ a
\in X: a \sim x \}$. Since $x \sim x$, all $x
\in X$ must be in some equivalence class. Also, since equivalence
relations are symmetric, each $x \in X$ is in only one equivalence
class. 
\begin{example}[Indifference]
  Let $\pref$ be a preference relation on $X$. Then we can define an
  equivalence relation by $x \sim y$ if $x \pref y$ and $y \pref
  x$. This relation is called the indifference relation. The
  equivalence classes of $\sim$ are called indifference classes. You
  are probably familiar with graphs of indifference
  curves. Indifference curves are indifference classes. 
\end{example}
\begin{example}[Isoquants]
  Consider a production function $f: \R^n \to \R$. We can define an
  equivalence relation on $\R^n$ by $x \sim z$ iff $f(x) = f(z)$. The
  equivalence classes of this preference relation are the isoquants of
  the production function. 
\end{example}

\subsection{Order relations}
A relation that is transitive and reflexive but not symmetric is
called an \textbf{order}. Preference relations are orders. Any order
$\preceq$ induces another relation $\prec$ defined by $x \prec y$ if
$x \preceq y$ and $y \npreceq x$. The reflexive order $\preceq$ is
often called a non-strict order, and the non-reflexive $\prec$
is called a strict order.

The usual order on $\R$, $\leq$ has two additional properties that a
generic order need not have. First, any set in $\R$ with an upper
bound has a least upper bound. $y$ is an upper bound of $ A \subseteq
\R$, if $y \geq x$ for all $x \in A$. $y$ is the least upper bound of
$A$ if there is no $z < y$ that is also an upper bound of $A$. Least
upper bounds in $\R$ are unique. Among other things, uniqueness of
least upper bounds is important for ensuring that optimization
problems are well-defined. A \textbf{partial order} is a relation that
is transitive, reflexive, and antisymmetric. Not all elements of a
partially ordered set need to be comparable.  
\begin{example}
  Let $S$ be set. The power set of $S$ (set of all subsets) is
  partially ordered by $\subseteq$.
\end{example}
\begin{example}
  A partial order on $\R^n$ can be defined by $x \leq y$ if $x_i \leq
  y_i$ for all $i$.
\end{example}
A preference relation is not a partial order because it allows
indifference between goods that are not the same. 

Another aspect of the usual $\leq$ order on $\R$ is that all elements
are comparable, i.e. it is complete. A relation that is complete,
transitive, and reflexive is called a (non-strict) \textbf{weak order}
(or sometimes just an ordering). Preference relations are weak
orders.\footnote{The ``weak'' refers to the fact that $x \preceq y$
  and $y \preceq x$ does not imply $x=y$. This is somewhat confusing
  because in economics we might read $x \preceq y$ as ``$y$ is weakly
  preferred to $x$'', where weakly refers to allowing indifference,
  not the fact that preference relations are weak orders.}

\subsection{Pareto order}

\footnote{Based on example 1.4.3 of \cite{carter2001}.}In this
subsection we will derive an interesting economic result --- Arrow's
impossibility theorem --- purely by thinking about orders.  Suppose we
have $n$ individuals each with some preference, $\pref_i$ on set
$X$. Think of $X$ as some set of outcomes for society. For example,
each element of $X$ could specify how much consumption to give each
individual. The (non-strict) Pareto order on $X$ is defined as
\[ x \pref^P y \text{ if } x \pref_i y \text{ for all } i=1,...,n \]
We say that $x$ Pareto dominates $y$ if $x \succ^P y$. $x$ is Pareto
optimal (or Pareto efficient) if there is no $y$ such that $y \succ^P
x$. 
\begin{exercise}
  Which of the properties of a relation does the Pareto order have?
\end{exercise}
Remember that any relation on $X$ can be represented by $R \subseteq X
\times X$. If $R_i$ is the set associated with $\pref_i$, i.e. $R_i =
\{(x,y): x \pref_i y \}$, then the set associated with $\pref^P$ is
$\cap_{i=1}^n R_i$. 

The Pareto order depends on the $n$ individual preferences. In other
words the Pareto order is a function of individual preferences. Since
each individual preference can be represented by a subset of $X \times
X$, we can think of the Pareto order as a function $\mathcal{P}(X
\times X)^n \to \mathcal{P}(X \times X)$. Thinking of orders as
subsets of $X \times X$ (equivalently elements of $\mathcal{P}(X\times
X)$) may not be especially helpful, but it is essential to remember
that the Pareto order is a function of individual preferences.

The Pareto order will usually not be complete. For example take
$n=2$.  Let elements of $X$ be consumption bundles for person $1$ and
$2$, $(x_1,x_2)$, such that $x_1 + x_2 = c$, for some constant
$c$. Then we would expect that $(c,0) \pref_1 (0,c)$ and $(0,c)
\pref_2 (c,0)$, so $(c,0)$ and $(0,c)$ are not Pareto comparable. 

Social choice theory is about how best to combine individual
preferences to make social decisions.  It would be great if we could
somehow complete the Pareto order, because then we could just choose
the maximal outcome under the completed Pareto order.
\cite{arrow1950} showed that this is impossible to do in an appealing
manner. To precisely state Arrow's result, we first need a couple more
definitions. 

A \textbf{social choice rule} is a rule for combining individual
preferences $\pref_i$ into a single weak order $\succeq$. In other
words, a social choice rule is a function from
the set of $n$ individual preferences to the
set of weak orders. Just like the Pareto order, a social choice rule
is a function of individual preferences. Unlike the Paretor order, a
social choice rule is required to be complete (weak orders are
complete).

We will denote the social ordering by
$F(\pref_1, ..., \pref_n) = \succeq$. A social choice rule $F$
completes the Pareto order if $\succeq$ is complete and for all $x,y$
such that $x \pref^P y$, the social order agrees $x \succeq
y$. Although not made explicit by this notation, it is important to
keep in mind that both the Pareto order, $\pref^P$, and the social
order, $\succeq$, are functions of individual preferences.

A social choice rule satisfies the \textbf{independence of irrelevant
  alternatives} (IIA) if for every $A \subseteq X$, if $\pref_i$ and
$\pref_i'$ are two sets of individuals preferences that agree on $A$,
\[ x \pref_i y \iff x \pref_i' y \; \forall x,y \in A \text{ for each
} i\] 
then $\succeq=F(\pref_1, ..., \pref_n)$ and $\succeq'=F(\pref_1', ...,
\pref_n')$ also agree on $A$. Independence 
of irrelevant alternatives says that the question of whether society
likes apples better than bananas should not depend on what anyone
thinks about the comparison between apples and oranges (or oranges and
kiwis, etc). It is an intuitively desirable feature of a social
choice rule.

One social choice rule that completes the Pareto order is the
dictatorial ordering. That is, we let any one person's preference be
society's preference. Of course, this is not a very desirable social
ordering.

\begin{theorem}[Arrow's impossibility theorem]
  It is impossible to complete the non-strict Pareto ordering in a way that
  is not dictatorial and is independent of irrelevant alternatives
  when $|X| \geq 4$. 
\end{theorem}
\cite{feldman1974} contains a very concrete proof for when $n=2$ and
$|X|=3$. We will prove the general theorem, but if you find the
general argument confusing (and it is a bit difficult), then reading
\cite{feldman1974} might be helpful.
\begin{proof}
  Suppose we have a social choice rule, $F$ that is consistent with
  the Pareto order and obeys the irrelevance of independent
  alternatives. We will prove the theorem in three steps, but first a
  definition will be useful.

  A group of individuals, $S \subseteq \{1,\cdots,n\}$ is
  \textbf{decisive} over $x, y \in X$ if for any individual preference
  relations if $x \succ_i y$ for all $i \in S$ implies $x \succ y$,
  where $\succ=F(\succ_1,..., \succ_n)$. Being decisive is a property
  of the subgroup $S$ and the decision rule $F$. It is something that
  holds for all possible individual preferences.
  
  \begin{enumerate}
  \item \emph{Field expansion lemma}: Suppose $|X|\geq 4$ and that there is one pair of
    alternatives, $x$ and $y$ such that some group $S$ is decisive
    over $x$, $y$.  We will show that $S$ must also be decisive over
    all other pairs of alternatives.  Let $z, w\in X$ and suppose that
    for all $i \in S$, $w \succ_i z$. By the irrelevance of
    independent alternatives for any individuals preferences
    $\succ_i'$ such that $w \succ_i' z$ if and only if $w \succ_i z$ for all $i$,
    it must be that $w \succ z$ if and only if $w \succ' z$, where
    $\succ'=F(\succ_1',...,\succ_n')$ and
    $\succ=F(\succ_1,...,\succ_n)$.  This means that we can freely
    change preferences among alternatives as long as we hold the
    preference between $w$ and $z$ fixed. Consider preferences such
    that $w \succ_j' x$ and $y \succ_j' z$ for all $j$, and $x
    \succ_i' y$ for all $i \in S$. Then $y \succ^{'P} z$ and $w
    \succ^{'P} x$. Since the social order must agree with the Pareto
    order, we also have $y \succ' z$ and $w \succ' x$. By assumption
    $S$ is decisive over $x$ and $y$, so $x \succ' y$. By transitivity
    $w \succ' z$. Like we argued earlier, by the independence of
    irrelevant alternatives, it must also be that $w \succ z$, i.e.\
    $S$ is also decisive over $w$ and $z$
  \item \emph{Group contraction lemma}: We will show that if $S$ is
    decisive and $|S| \geq 2$, then there exists $S' \subset S$ such
    that $S'$ is decisive. 
    
    Let $z \in X$, $z \neq x,y$. Partition $S$ into two groups. That is, let $S_1
    \cup S_2 = S$ where $S_1 \cap S_2 = \emptyset$, and $S_1$ and $S_2$
    are not empty. Consider preferences $\succ_i$ such that $x \succ_i
    y$ for all $i \in S$. $S$ is decisive, so then, $x \succ y$ as
    well.

    If $S_1$ is not decisive, then from the previous part, $S_1$ is
    not decisive over any pair. In particular, $S_1$ is not decisive
    over $z$ and $x$. This means that we can find preferences $x
    \succ_i z$ for all $i \in S_1$ such that the social
    order $\succ=F(\succ_1,
    ..., \succ_2)$ prefers $z$ to $x$, $z \succeq x$. Moreover, by the
    independence of irrelevant alternatives, we can make these
    preferences be such that $x \succ_i y \succ_i z$ for all $i \in
    S_1$. 

    If $S_2$ is also not decisive, then $S_2$ is not decisive over any
    pair, including $z$ and $y$. Thus, far we have placed assumptions
    about preferences between $z$ and $x$ and $x$ and $y$. We have not
    said anything about preferences between $z$ and $y$. By the
    independence of irrelevant alternatives, we are free to specify
    preferences between $z$ and $y$ which will still satisfy all the
    conditions about $z$ and $x$ above.  This means that we can find
    preferences such that $z \succ_i y$ for all $i \in S_2$ but the
    social order (weakly) prefers $y$ to $z$, $y \succeq z$.  To
    summarize, assuming neither $S_1$ nor $S_2$ is decisive, we have
    constructed preferences such that
    \begin{itemize}
    \item for all $i \in S_1$, $x \succ_i y \succ_i z$, but $z \succeq
      x$, and
    \item for all $i \in S_2$, $z  \succ_i y$ and $x \succ_i y$, but $y \succeq z$
    \end{itemize}
    However, $S$ being decisive implies $x \succ y$, which contradicts $y
    \succeq z \succeq x$. Therefore, either $S_1$ or $S_2$ must be decisive.    
  \item To conclude the proof, notice that the set $S$ of everyone is
    decisive for any social ordering that agrees with the Pareto
    ordering. By the group contraction lemma, there is a decisive
    proper subset $S'$. We can apply the group contraction lemma again
    and again until we get to a decisive set with one element, i.e. a
    dictator.
  \end{enumerate}    
\end{proof}
So what does this theorem mean? One implication is that any
non-dictatorial social choice rule that agrees with the Pareto order
must either (i) not actually be a weak order (usually not trnsitive)
or (ii) violate the independence of irrelevant alternatives.
\begin{example}[Binary voting]
  Suppose $n$ is odd (to avoid ties) and the social choice rule is
  determined by majority vote. That is, 
  \[ x \succeq y \iff x \succeq_i y \text{ for at least $(n+1)/2$
    individuals }. \]
  Voting is complete, agrees with the Pareto order, and is
  non-dictatorial. Voting also obeys the independence or irrelevant
  alternatives. Whether $x \succeq y$ only depends on individual
  preferences between $x$ and $y$. So, how can this be possible? Well,
  voting does not always lead to a transitive order, so it does not
  qualify as a social choice rule.  One of the
  requirements of a social choice rule is that it is a weak order
  (given any individual preferences), so it
  must be transitive. Transitivity was essential at the end of step 2
  of our proof of the impossibility theorem.  
\end{example}
We could also try to construct social choice rules through other
voting arrangements. 
\begin{example}[Majority voting]
  Suppose the social choice rule is to let each person vote on their
  most preferred outcome. Then outcomes are ranked by the number of
  votes received. This social choice rule is transitive and complete
  (so it is a weak order), but it does not obey the independence of
  irrelevant alternatives. We see this often in elections with 3
  candidates, where the presence of a third fringe candidate can
  influence which of two mainstream candidates wins.
\end{example}



\appendix
\section{Numbers \label{s:numbers}}

We have been assuming familiarity with the natural numbers, integers,
rationals, and real numbers. This section explores some properties of
these sets of numbers and heuristically describes how these sets of
numbers are constructed. It may appear silly and slightly confusing to
try to be ``rigorous'' about something like real numbers that we
already feel like we understand.  Much of mathematics is about finding
and describing patterns that apply to abstract objects. Many of the
abstract objects that we will study are similar to the real numbers in
some ways, but different in others. Examples of things that are
similar to the real numbers include complex numbers, vector spaces,
matrices, and sets of functions. Some of these things we will be able
to add and multiple just like real numbers, but not all of them. A
natural sort of question is: this class of objects shares properties
X, Y, and Z with the real numbers; what theorems that we know about
the real numbers will also be true of this class of objects? Before
answering this sort of question we have to be precise about what
properties the real numbers have.

We will take for granted that we understand what the natural numbers
are. Note, however, that it is possible to rigorously construct the
natural numbers from a simple list of assumptions using logic or set
theory. We will also take for given that we know how to add and
multiply natural numbers. Addition has the following nice properties.
\begin{itemize}
\item[1] \emph{Closure} if $a, b \in \mathbb{N}$, so is $a + b$ 
\item[2] \emph{Associative} $a + (b + c) = (a + b) + c$. 
\end{itemize}
If we demand that addition also has 
\begin{itemize}
\item[3] \emph{Identity} $\exists 0$ s.t. $a + 0 = a$,
\item[4] \emph{Inverse} $\forall a$, $\exists b$ s.t. $a + b = 0$
\end{itemize}
then we must expand the natural numbers to include the integers,
$\mathbb{Z}$. Multiplication also satisfies these four analogous
properties:
\begin{itemize}
\item[1'] \emph{Closure} if $a, b \in A$, so is $ab$ 
\item[2'] \emph{Associative} $a (b c) = (a b)  c$. 
\item[3'] \emph{Identity} $\exists 1$ s.t. $a 1= a$,
\item[4'] \emph{Inverse} $\forall a \neq 0$, $\exists b$ s.t. $ab = 1$
\end{itemize}
However, if we want multiplicative inverses to exist for all $z\in
\mathbb{Z}$, then we must further expand our set of numbers to the
rationals, $\mathbb{Q}$. Addition and multiplication are also
\begin{itemize}
\item[5] \emph{Commutative} $ a + b = b + a$
\item[6] \emph{Distributive} $a (b + c) = ab + ac$
\end{itemize}
To summarize: if we start with the natural numbers, and then demand
that multiplication and addition have these six properties, we end up
with the rational numbers. 

More generally, we could study a set $A$ combined with one or two
operations that satisfy certain properties. The branch of mathematics
that studies these sort of objects is abstract algebra. We will not be
studying algebra in detail, but it may be useful to be familiar with
some basic terms. A \textbf{group} is a set and operation,
$(A,\oplus)$ such that $A$ is closed under $\oplus$, $\oplus$ is
associative, there exists an identity, and inverses exist under
$\oplus$ (i.e.\ properties 1-4). If $\oplus$ is also commutative, we
call $(A,\oplus)$ an abelian (or commutative) group. Examples of
groups include $(\mathbb{Z},+)$ and $(\mathbb{Q},\cdot)$. A
\textbf{ring} is a set with two operations, $(A,\oplus,\odot)$ such
that $(A,\oplus)$ is a group, and $\odot$ has properties 1-3 and
6. $(\mathbb{Z},+,\cdot)$ is a ring.  One ring that will come up
repeatedly in this course is the set of all $n$ by $n$ matrices with
the usual matrix addition and multiplication. A \textbf{field} is a set
with two operations such that 1-6 hold for both
operations. $(\mathbb{Q},+,\cdot)$ is a field. Another field that you
may have encountered is the complex numbers with the usual addition
and multiplication. If you're interested you may want to verify that
the integers modulo any number is a ring, and the integers modulo any
prime number if a field.

\subsection{Real numbers}

The rational numbers are pretty nice; they're a field with the six
properties listed above. However, $\mathbb{Q}$ does not contain all
the numbers that we think it should. For example,
\begin{theorem}
  $\sqrt{2} \not\in \mathbb{Q}$
\end{theorem}
\begin{proof}
  Suppose $\sqrt{2} \in \mathbb{Q}$. Then $\sqrt{2} = p / q$ where $p$
  and $q$ are not both even. If we square both sides, we get
  \begin{align*}
    2 = & p^2 / q^2 \\
    2 q^2 = & p^2.
  \end{align*}
  Hence, $p^2$ must be even. From the review, then $p$ must also be
  even, say $p = 2m$. Then we have
  \begin{align*}
    2 q^2 = & 2(2 m^2) \\
    q^2 = & 2 m^2,
  \end{align*}
  which means $q$ must also be even, contrary to our starting
  assumption. 
\end{proof}
Apparently, the rationals have some holes in them that we should fill
in. To do so in a unique way, we need to define another property of
the rational numbers. A \textbf{totally ordered set} is a set, $A$,
and a relation, $<$, such that (i) (total) $\forall a,b \in A$ either
$a < b$ or $a = b$ or $a > b$; and (ii) (transitive) if $a < b$ and $b
< c$ then $a < c$. An \textbf{ordered field} is a field that is a
totally ordered set and addition and multiplication preserve the
ordering in that (i) if $b<c$ then $a + b < a + c$ (ii) if $a>0$ and
$b>0$ then $ab>0$.

We need one more definition. Simon and Blume state that one property
of real numbers that will be used throughout the book is the least
upper bound property. It turns out that this property is not only
useful; it lies at the foundation of the real numbers. Let $S$ be an
ordered set and $A \subset S$. $s \in S$ is an \textbf{upper bound} of
$A$ if $s \geq a \forall a \in A$. $s$ is a \textbf{least upper bound}
(aka supremum) of $A$ if $s$ is an upper bound of $A$ and if $r < s$,
then $r$ is not an upper bound of $A$. $S$ has the
\textbf{least-upper-bound property} (aka complete or Dedekind
complete) if whenever $A \subset S$ has an upper bound, $A$ has a
least upper bound. Given that $\sqrt{2} \not\in \mathbb{Q}$, it should
not be surprising that the rational numbers are not complete. %You will
%prove this fact on the first problem set. 

\begin{theorem}[Real numbers]
  There exists an ordered field, $\mathbb{R}$, that has the least
  upper bound property. $\mathbb{R}$ contains $\mathbb{Q}$. Moreoever,
  $\mathbb{R}$ is ``unique''.
\end{theorem}

The proof of this is surprisingly long, so we will not go over it in
detail. Existence can be proven by construction. One method involves
constructing real numbers as Dedekind cuts. A Dedekind cut is a
nonempty subset of the rationals, $A \subset \mathbb{Q}$, such that
(i) if $p \in A$, $q \in \mathbb{Q}$, and $q < p$, then $q \in A$ and
(ii) if $p \in A$ then $p<r$ for some $r \in A$ (i.e. $A$ has no
greatest element. For example, the Dedekind cut associated with
$\sqrt{2}$ would be $\{p \in \mathbb{Q}: p^2 < 2\}$). We would then
define addition, multiplication, and ordering of these cuts in the
natural way and verify that all the properties above are
satisfied. See \citet{rudin1976} for details if you are interested.

The ``uniqueness'' is harder to prove. $\mathbb{R}$ is unique in the
sense that any two ordered fields with the least-upper-bound property
are isomorphic (there exists a bijection between them that preserves
multiplication, addition, and ordering). The proof proceeds by
supposing that $\mathbb{R}$ and $\mathbb{F}$ are two ordered fields
with the least-upper-bound property and then shows that there is an
isomorphism between them.

\bibliographystyle{jpe}
\bibliography{../526}



\end{document}
