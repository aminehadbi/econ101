\documentclass[compress]{beamer} 
\usepackage{amsmath}
\usepackage{amsfonts}
\usepackage{graphicx}
%\usepackage{epstopdf}
\usepackage{hyperref}
\usepackage{multirow}
\usepackage{verbatim}
%\usepackage[small,compact]{titlesec} 
%\usecolortheme{beaver}
\usetheme{Hannover}
%\usecolortheme{whale}

%\usepackage{pxfonts}
%\usepackage{isomath}
%\usepackage{mathpazo}
%\usepackage{arev} %     (Arev/Vera Sans)
%\usepackage{eulervm} %_   (Euler Math)
%\usepackage{fixmath} %  (Computer Modern)
%\usepackage{hvmath} %_   (HV-Math/Helvetica)
%\usepackage{tmmath} %_   (TM-Math/Times)
\usepackage{tgheros}
%\usepackage{cmbright}
%\usepackage{ccfonts} \usepackage[T1]{fontenc}
%\usepackage[garamond]{mathdesign}
\usepackage{color}
\usepackage{ulem}

\setbeamertemplate{navigation symbols}{}
\AtBeginSection[] % Do nothing for \section*
{ \frame{\sectionpage} }

\newcommand{\argmax}{\operatornamewithlimits{arg\,max}}
\newcommand{\argmin}{\operatornamewithlimits{arg\,min}}
\def\inprobLOW{\rightarrow_p}
\def\inprobHIGH{\,{\buildrel p \over \rightarrow}\,} 
\def\inprob{\,{\inprobHIGH}\,} 
\def\indist{\,{\buildrel d \over \rightarrow}\,} 
\def\F{\mathbb{F}}
\def\R{\mathbb{R}}
\newcommand{\gmatrix}[1]{\begin{pmatrix} {#1}_{11} & \cdots &
    {#1}_{1n} \\ \vdots & \ddots & \vdots \\ {#1}_{m1} & \cdots &
    {#1}_{mn} \end{pmatrix}}
\newcommand{\iprod}[2]{\left\langle {#1} , {#2} \right\rangle}
\newcommand{\norm}[1]{\left\Vert {#1} \right\Vert}
\newcommand{\abs}[1]{\left\vert {#1} \right\vert}
\renewcommand{\det}{\mathrm{det}}
\newcommand{\rank}{\mathrm{rank}}
\newcommand{\spn}{\mathrm{span}}
\newcommand{\row}{\mathrm{Row}}
\newcommand{\col}{\mathrm{Col}}
\renewcommand{\dim}{\mathrm{dim}}
\newcommand{\prefeq}{\succeq}
\newcommand{\pref}{\succ}


\title{Applications}
\author{Paul Schrimpf}
\institute{UBC \\ Economics 526}
\date{\today}

\begin{document}

\frame{\titlepage}
\setcounter{tocdepth}{2}

\begin{frame}
  \tableofcontents  
\end{frame}

\section{Portfolio analysis}

\begin{frame}
  \frametitle{Portfolio analysis}
  \begin{itemize}
  \item Model of portfolio analysis taken from Simon and Blume
  \item Good way of reviewing results about matrices and systems of
    linear equations 
  \end{itemize}
\end{frame}

\begin{frame}
  \frametitle{Setup}
  \begin{itemize}
  \item $A$ assets
  \item $S$ states of nature
  \item Initial value of $i = v_i$
  \item After state revealed, value $y_{si}$
  \item Realized return $R_{si} = \frac{y_{si}}{v_i}$
  \item Matrix of returns $\underbrace{\mathcal{R}}_{S \times A}$
  \end{itemize}
\end{frame}

\begin{frame}
  \frametitle{Setup}
  \begin{itemize}
  \item Choosing budget shares $x_i$, $\sum_{i=1}^A x_i = 1$ (but
    individual shares could be positive or negative), call $x =
    (x_1,...,x_A)$ a portfolio
  \end{itemize}
\end{frame}

\begin{frame} \frametitle{Riskless portfolios}
  \begin{definition}
    A portfolio is \textbf{riskless} if 
    \begin{align*}
      \sum_{i=1}^A R_{1i} x_i  =  \cdots   =\sum_{i=1}^A R_{Si} x_i 
    \end{align*}
  \end{definition}
  \begin{itemize}
  \item Solution to $\mathbf{c} = (c, ..., c)$
    \begin{align*}
      \underbrace{\mathcal{R}}_{S \times A} \underbrace{x}_{A \times 1} =
      \underbrace{\mathbf{c}}_{S \times 1}
    \end{align*}
  \item Exists if $\mathbf{c} \in \col(\mathcal{R})$
  \end{itemize}
\end{frame}

\begin{frame}\frametitle{Insurable states}
  \begin{definition}
    A state $s^*$ is \textbf{insurable} if $\exists$ portfolio $x$ such
    that $\sum_{a=1}^A R_{s^*a} x_a > 0$ and $\sum_{a=1}^A R_{sa} x_a =
    0$ for all $s \neq s^*$. 
  \end{definition}
  \begin{itemize}
  \item $s^*$ insurable iff $\exists$ solution to
    \[ \mathcal{R}x = e_{s^*} \] 
  \item So if $e_{s^*} \in \col(\mathcal{R})$
  \item All states insurable iff $\col(\mathcal{R}) = \R^S$
  \end{itemize}
\end{frame}

\begin{frame}
  \frametitle{Duplicable portfolios}
  \begin{definition}
    A portfolio $x$ is \textbf{duplicable} if there is
    another portfolio $w$ such that $\sum_{i=1}^A x_i = \sum_{i=1}^A w_i
    $ and 
    \[ \mathcal{R} x = \mathcal{R} w. \]
  \end{definition}
  \begin{itemize}
  \item i.e.\  multiple solutions to
    \begin{align*}
      \begin{pmatrix} \mathcal{R} \\ 1 \cdots 1 \end{pmatrix} x 
      = & \begin{pmatrix} \mathcal{R} \\ 1 \cdots 1 \end{pmatrix} w
      \\
      \widetilde{\mathcal{R}} x = \widetilde{\mathcal{R}} w
    \end{align*}
    if and only if $\dim \mathcal{N}(\widetilde{\mathcal{R}})> 0$ if and
    only if $\rank \widetilde{\mathcal{R}} < A$
  \end{itemize}
\end{frame}

\section{First and second welfare theorems}

\begin{frame}
  \frametitle{Welfare theorems}
  \begin{itemize}
  \item[1st:] Every competitive equilibrium is Pareto efficient (under
    some conditions)
  \item[2nd:] Every Pareto efficient allocation is a competitive
    equilibrium (under some stronger conditions)
  \item We will
    \begin{itemize}
    \item Carefully define competitive equilibrium and Pareto
      efficiency
    \item State and prove theorems
    \end{itemize}
  \end{itemize}
\end{frame}

\begin{frame}
  \frametitle{Welfare theorems}
  \begin{itemize}
  \item Proofs will use:
    \begin{itemize}
    \item Vector spaces, linear transformations
      \begin{itemize}
      \item Should mostly understand
      \end{itemize}
    \item Convexity, separating hyperplane theorem
    \item Continuity --- $\forall \epsilon > 0 \exists \delta > 0$
      s.t. ...
      \begin{itemize}
      \item Maybe difficult to follow, but we will see a lot more of
        this in calculus
      \end{itemize}
    \end{itemize}
  \item Economic issues to think about:
    \begin{itemize}
    \item The result of the two theorems, their assumptions, and how
      they relate to the real world
    \item General role of assumptions in proofs, so can conjecture
      what happens when assumptions are false 
    \end{itemize}
  \end{itemize}
\end{frame}

\subsection{Lines, planes, and hyperplanes }

\begin{frame}
  \frametitle{Lines, planes, and hyperplanes}
  \begin{definition}
    A \textbf{hyperplane} in $\R^n$ is an $n-1$ dimensional affine
    subspace. Equivalently, a hyperplane is the set of solutions to a
    single equation with $n$ variables.
  \end{definition}
  Any hyperplane can be written in the form:
  \[ H_{\xi,c} = \{x: \iprod{\xi}{x} = c \} \] where $c \in \R$, $\xi \in
  \R^n$, and $\norm{\xi} = 1$.
\end{frame}

\begin{frame}
  \frametitle{Separating hyperplanes}
  \begin{definition}
    A set $S \subseteq \R^n$ is convex if $\forall x_1, x_2 \in S$ and
    $\lambda \in (0,1)$, we have $x_1 \lambda + x_2(1-\lambda) \in S$.  
  \end{definition}
  \begin{theorem}[Separating hyperplane theorem]
    If $S_1$ and $S_2 \subseteq \R^n$ are convex. If $S_1 \cap S_2 =
    \emptyset$ then there exists a hyperplane, $H_{\xi c} = \{ x:
    \xi'x = c \}$ such that
    \[ \iprod{s_1}{\xi} \leq c \leq \iprod{s_2}{\xi} \]
    for all $s_1 \in S_1$ and $s_2 \in S_2$. We say that $H_{\xi,c}$
    separates $S_1$ and $S_2$. 
  \end{theorem}
  \begin{itemize}
  \item Draw picture
  \end{itemize}
\end{frame}

\begin{frame}
  \begin{centering}
    \includegraphics[width=\linewidth]{separatingHyperplane}
  \end{centering}
\end{frame}
%%%%%%%%%%%%%%%%%%%%%%%%%%%%%%%%%%%%%%%%%%%%%%%%%%%%%%%%%%%%%%%%%%%%%%%%% 


\subsection{Setup}

\begin{frame}
  \frametitle{Setup}
  \begin{itemize}
  \item Commodities $s \in S$, a normed vector space (e.g. $S$ linear
    subspace of $\R^n$)
  \item Consumers $i=1,..., I$ with consumption possibility set $X_i
    \subseteq S$
    (think of as physical constraint, not budget constraint)
  \item Preference relation $\prefeq_i$ s.t.
    \begin{enumerate}
    \item (total) $\forall x, z \in X_i$, either $x \prefeq_i z$ or $z
      \prefeq_i x$ or both,
    \item (transitive) $\forall x, w, z \in X_i$, if $x \prefeq_i w$ and $w
      \prefeq_i z$ then $x \prefeq_i z$,
    \item (reflexive) $\forall x \in X_i$, $x \prefeq_i x$.
    \end{enumerate}
    Denote strict preference $\pref_i$, and indifference $\simeq_i$
    \begin{itemize}
    \item Could come from  utility function i.e.\ $x \prefeq_i z$ iff
      $u_i(x) \geq u_i(z)$, but does not have to
    \end{itemize}    
  \end{itemize}
\end{frame}

\begin{frame}
  \frametitle{Setup}
  \begin{itemize}
  \item Firms $j=1,...,J$ produce $y_j$ from within production
    possibility sets $Y_j \subseteq S$
    \begin{itemize}
    \item Example: 3 commodities, food $f$, clothing $c$, and labor
      $l$. Production function $f = F^f(l)$, and $c = F^c(l)$,
      \[ Y_j = \{ (f,c,l) \in S: l \leq 0 \wedge f \leq F^f(\alpha |l|)
      \wedge c \leq F^c((1-\alpha) |l|) \text{ for some } \alpha \in
      [0,1]\}. \] 
    \end{itemize}
  \item $I+J$ tuple $((x_i),(y_j))$ is an \textbf{allocation} 
  \item Allocation is feasible if $x_i \in X_i \forall i$, $y_j \in
    Y_j \forall j$, and the market clears, 
    $\sum_{i=1}^I x_i = \sum_{j=1}^J y_j$.
  \end{itemize}
\end{frame}  

\begin{frame}
  \begin{definition}
    An allocation, $((x_i^0),(y_j^0))$, is \textbf{Pareto efficient}
    if it is a feasible and there is no other feasible allocation,
    $((x_i),(y_j))$, such that $x_i \prefeq_i x_i^0$ for all $i$ and
    $x_i \pref_i x_i^0$ for some $i$.
  \end{definition}

  \begin{definition}
    A price system is a continuous linear transformation $p:S
    \rightarrow \R$. (If $S$ is $n$-dimensional then $p$ is any $1
    \times n$ matrix)
  \end{definition}
\end{frame}

\begin{frame}
  \begin{definition}
    An allocation, $((x_i^0),(y_j^0))$, along with a price system, $p$,
    is a \textbf{competitive equilibrium} if 
    \renewcommand{\theenumi}{C\arabic{enumi}}
    \begin{enumerate}
    \item\label{c1} The allocation is feasible
    \item\label{c2} (Utility maximization) For each $i$ and $x \in X_i$
      if $px \leq px_i^0$ 
      then $x_i^0 \prefeq_i x$, 
    \item\label{c3} (Profit maximization) For each $j$ if $y \in Y_j$
      then $p y \leq p y_j^0$ 
    \end{enumerate}
    \renewcommand{\theenumi}{\roman{enumi}}
  \end{definition}
\end{frame}

\subsection{first welfare theorem}

\begin{frame}
  \frametitle{first welfare theorem}
  \begin{definition}
    preference relation $\pref_i$ has the \textbf{local non-satiation
      condition} if for each $x \in x_i$ and $\epsilon > 0$ $\exists x'
    \in x_i$ such that $\norm{x - x'} \leq \epsilon $ and $x' \pref_i x$.
  \end{definition}
  
  \begin{theorem}[first welfare theorem]
    if $((x_i^0),(y^0_j))$ and $p$ is a competitive equilibrium and all
    consumers' preferences have the local non-satiation condition, then
    $((x_i^0),(y_i^0))$ is pareto efficient.
  \end{theorem}
\end{frame}  

\begin{frame}
  \frametitle{proof of first welfare theorem}
  \begin{itemize}
  \item will show contradiction
  \item suppose competitive equilibrium not pareto efficient, then
    $\exists$ a pareto improvement $((x_i),(y_j))$ another feasible
    allocation s.t.
    \begin{itemize}
    \item $\exists $ at least one $i^*$ with $x_{i^*} \pref_i x_i^0$
      \begin{itemize}
      \item the contrapositive of \ref{c2} implies $p x_{i^*} > p
        x_i^0$
      \end{itemize}
    \item all other $x_i \prefeq_i x_i^0$ 
      \begin{itemize}
      \item local non-satiation and continuity of $p$ imply $p x_i
        \geq p x_i^0$  (next slide)
      \end{itemize}
    \end{itemize}
  \end{itemize}
\end{frame}

\begin{frame}
  \frametitle{proof of first welfare theorem}
  \begin{itemize}
  \item local non-satiation and continuity of $p$ imply $p x_i
    \geq p x_i^0$  
    \begin{itemize}
    \item if $p x_i < p x_i^0$, then local non-satiation implies
      $\exists$ $x_i'$ very close to $x_i$ such that $x_i' \pref_i x_i
      \prefeq_i x_i^0$. 
    \item $p$ continuous means that if $x_i'$ is close enough to
      $x_i$, then $p x_i'$ will be close enough to $p x_i$ that 
      $p x_i' < p x_i^0$ as well. 
    \item but $x_i' \pref_i x_i$ and $p x_i' < p x_i^0$ contradicts
      the definition of competitive equilibrium.
    \end{itemize}
  \item so in this pareto improvement, we have
    \begin{align}
     \sum_{i=1}^i p x_i > \sum_{i=1}^i p x_i^0
    \end{align}
    next, we will show this contradicts profit maximization
  \end{itemize}
\end{frame}

\begin{frame}
  \frametitle{proof of first welfare theorem}
  \begin{itemize}
  \item start with market clearing, apply $p$, rearrange:
    \begin{align}
      \sum_{i=1}^i x_i & = \sum_{j=1}^j y_j  \\
      p(\sum_{i=1}^i x_i) =  \sum_{i=1}^i p x_i & 
      = \sum_{j=1}^j p y_j  = p(\sum_{j=1}^j y_j)  \\ 
    \end{align}
  \item same thing for $x_i^0$ and $y_j^0$ with inequality from last
    slide gives
    \begin{align}
      \sum_{j=1}^j p y_j > \sum_{j=1}^j p y_j^0 
    \end{align}
    but then there is at least on $j$ with $p y_j > p y_j^0$,
    contradicting \ref{c3}
  \end{itemize}  
\end{frame}

\subsection{Second welfare theorem}
\begin{frame}
  \frametitle{Second welfare theorem}
  \begin{definition}
    A preference relation, $\prefeq_i$, is \textbf{convex} if whenever $x
    \prefeq_i z$ and $y \prefeq_i z$, then $\lambda x + (1-\lambda) y
    \prefeq_i z$ for all $\lambda \in [0,1]$. 
  \end{definition}
  
  \begin{definition}
    A preference relation, $\prefeq_i$, is \textbf{continuous} if
    for any  $x \pref_i z$ there exists a $\delta >
    0$ such that for all $x'$ with $\norm{x - x'}<\delta$ we have $x'
    \pref_i z$.
  \end{definition}
\end{frame}

\begin{frame}
  \frametitle{Second welfare theorem}
  \begin{theorem}[Second welfare theorem]
    Assume the preferences of each consumer are convex, locally
    non-satiated, and continuous, and that $X_i$ is convex and
    non-empty.  
    Also assume that $Y_j$ is convex and non-empty for each
    firm $j$. 
    
    Suppose $((x_i^e), (y_j^e))$ is a Pareto efficient allocation such
    that for any price system, $p$, there is always a cheaper bundle of
    goods, i.e.\ $\exists x_i \in X_i$ s.t. $p x_i < p x_i^e$ for each
    $i$. Then there exists a price system, $p^e$ such that $((x_i^e),
    (y_j^e))$ and $p^e$ is a competitive equilibrium.
  \end{theorem}  
\end{frame}

\begin{frame}
  Compared to first welfare theorem three more assumptions:
  \begin{enumerate}
  \item Convexity of preferences, consumption possibility sets, and
    production possibility sets
  \item Continuity of preferences
  \item Cheaper good condition
  \end{enumerate}
\end{frame}

\begin{frame}
  \frametitle{Proof of second welfare theorem}
  \begin{itemize}
  \item Price system will come from separating hyperplane theorem,
    first need two disjoint, convex sets
  \item Let $V_i = \{ x \in X_i : x \pref_i
    x_i^e \}$, set sum of $V_i$ be
    \begin{align*}
      V = \{ \chi \in S: \chi = \sum_{i=1}^I x_i \text{ for some } x_i
      \in V_i \}  
    \end{align*}
    $V_i$ convex, so $V$ convex
  \item Let 
    \begin{align*}
      Y = \{ \psi \in S: \psi = \sum_{i=j}^J y_j \text{
        for some } y_j \in Y_j \}  
    \end{align*}
    also convex.
  \end{itemize}
\end{frame}

\begin{frame}
  \frametitle{Proof of second welfare theorem}
  \begin{itemize}
  \item Show $Y$ and $V$ disjoint
    \begin{itemize}
    \item Suppose $\chi \in Y \cap V$. Then $\exists x_i \in V_i$ and
      $y_j \in Y_j$ such that $\chi = \sum_{i=1}^I x_i =
      \sum_{j=1}^J$, is feasible and $x_i \pref_i x_i^e$
    \item Contradicts $((x_i^e),(y_i^e))$ being Pareto efficient
    \end{itemize}
  \item Separating hyperplane theorem $\Rightarrow$ $\exists p$ s.t.
    \begin{align}
      p \chi \geq p \psi \label{ieq:p}
    \end{align}
  \item Next, show $((x_i^e), (y_j^e))$ with $p$ is a competitive
    equilibrium 
  \end{itemize}
\end{frame}

\begin{frame}
  \frametitle{Proof of second welfare theorem}
  \begin{itemize}
  \item Show $p \chi\geq p \chi^e = p \psi^e \geq p
    \psi$ for all $\chi \in V$, $\psi \in Y$
    \begin{itemize}
    \item $\psi^e \in Y$, so $p \chi \geq p \psi^e$
      $\forall \chi \in V$.
    \item $\chi^e = \sum x_i^e = \sum y_j^e = \psi^e$, so 
      \begin{align} p \chi \geq p \chi^e = p \psi^e \end{align} 
    \item Local non-satiation implies we can find $x_i' \pref_i x_i^e$
      very close to $x_i^e$. Continuity of $p$ means that for $x_i'$
      close enough we have $|p \chi^e - p\chi'| < \epsilon$ for any
      $\epsilon >0$. Therefore $\forall \psi \in Y$
      \begin{align}
        p\chi^e + \epsilon > p\chi' \geq p \psi
      \end{align}
      so $p\chi^e \geq p \psi$
    \end{itemize}  
  \end{itemize}
\end{frame}

\begin{frame}
  \frametitle{Proof of second welfare theorem}
  \begin{itemize}
  \item Show $p x_i \geq p x_i^e $ for each $i$ and
    all $x_i \in V_i$ 
    \begin{itemize}
    \item If not there $p x_i + \epsilon < p x_i^e$  with $x_i \in
      V_i$. 
    \item Local non-satiation and continuity of $p$ means we can
      choose other $x_k \in V_k$ such that $\left\vert \sum_{k\neq i}
        (p x_k - p x_k^e) \right\vert < \epsilon / 2$
    \item But then $p \chi < p \chi^e$ contradicting the previous
      slide 
    \end{itemize}
  \item Same reasoning but using convexity of $Y_j$ instead of local
    non-satiation means $p y_j \leq y_j^e$ for each $j$ and all $y_j
    \in Y_j$ i.e. firms are profit maximizing, \ref{c3} holds
  \end{itemize}
\end{frame}

\begin{frame}
  \frametitle{Proof of second welfare theorem}
  \begin{itemize}
  \item Have shown: if $x_i \pref_i x_i^e$ then  $p x_i \geq p x_i^e$
  \item Need: if $x_i \pref_i x_i^e$ then  $p x_i > p x_i^e$
  \item Use cheaper good condition
    \begin{itemize}
    \item If $p x_i = p x_i^e$ and $\exists x_i' \in X_i$ s.t. $p x_i'
      < p x_i^e$, then for all $\lambda \in (0,1)$,
      \begin{align}
        p(\lambda x_i +(1-\lambda) x_i') < p x_i^e
      \end{align}
    \item Continuous preferences implies for $\lambda$ close to $0$, 
      \begin{align}
        \lambda x_i + (1-\lambda)x_i' \pref_i x_i^e      
      \end{align}
      so $\lambda x_i + (1-\lambda) x_i' \in V_i$, but that
      contradicts what we have shown.
    \end{itemize}
  \end{itemize}
\end{frame}

\end{document}
