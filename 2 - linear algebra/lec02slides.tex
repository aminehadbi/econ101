\documentclass[compress]{beamer} 
\usepackage{amsmath}
\usepackage{amsfonts}
\usepackage{graphicx}
%\usepackage{epstopdf}
\usepackage{hyperref}
\usepackage{multirow}
\usepackage{verbatim}
%\usepackage[small,compact]{titlesec} 
%\usecolortheme{beaver}
\usetheme{Hannover}
%\usecolortheme{whale}

%\usepackage{pxfonts}
%\usepackage{isomath}
%\usepackage{mathpazo}
%\usepackage{arev} %     (Arev/Vera Sans)
%\usepackage{eulervm} %_   (Euler Math)
%\usepackage{fixmath} %  (Computer Modern)
%\usepackage{hvmath} %_   (HV-Math/Helvetica)
%\usepackage{tmmath} %_   (TM-Math/Times)
\usepackage{tgheros}
%\usepackage{cmbright}
%\usepackage{ccfonts} \usepackage[T1]{fontenc}
%\usepackage[garamond]{mathdesign}
\usepackage{color}
\usepackage{ulem}

\newcommand{\argmax}{\operatornamewithlimits{arg\,max}}
\newcommand{\argmin}{\operatornamewithlimits{arg\,min}}
\def\inprobLOW{\rightarrow_p}
\def\inprobHIGH{\,{\buildrel p \over \rightarrow}\,} 
\def\inprob{\,{\inprobHIGH}\,} 
\def\indist{\,{\buildrel d \over \rightarrow}\,} 
\def\Pr{\mathrm{P}}
\newcommand{\rank}{\mathrm{rank}}

% \newtheorem{theorem}{Theorem}[section]
% \newtheorem{conjecture}{Conjecture}[section]
% \newtheorem{corollary}{Corollary}[section]
% \newtheorem{lemma}{Lemma}[section]
% \newtheorem{proposition}{Proposition}[section]
\theoremstyle{definition}
% \newtheorem{assumption}{}[section]
% %\renewcommand{\theassumption}{C\arabic{assumption}}
% \newtheorem{definition}{Definition}[section]
% \newtheorem{step}{Step}[section]
% \newtheorem{remark}{Comment}[section]
%\newtheorem{example}{Example}[section]
\newtheorem{question}{Question}
% \newtheorem*{example*}{Example}

%\linespread{1.1}

%\renewcommand{\sectionmark}[1]{\markright{#1}{}}


\title{Systems of linear equations}
\author{Paul Schrimpf}
\institute{UBC \\ Economics 526}
\date{\today}

\begin{document}

\frame{\titlepage}

%\section{Overview}

%\begin{frame}
%  \frametitle{Overview}
%  \begin{itemize}
  % \item Review last lecture
  %   \begin{itemize}
  %   \item Two more theorems on countability (helpful for homework) 
  %   \item Gaussian and Gauss-Jordan elimination
  %   \end{itemize}
%  \item Solutions to systems of linear equations
%    \begin{itemize}
%    \item Existence
%    \item Uniqueness
%    \item Characterizing the set of solutions
%    \end{itemize}
%  \end{itemize}
%\end{frame}

\section*{Cardinality review}
\begin{frame}
  \frametitle{More cardinality}    
  \begin{itemize}
  \item Let $A$ and $B$ be sets, the \textbf{Cartesion product} of $A$
    and $B$ is
    \[ A \times B := \{(a, b) : a \in A, b \in B\} \]
  \end{itemize}
  
  \begin{question}
    If $A$ and $B$ are countable is $A \times B$ countable?
  \end{question}
\end{frame}

\begin{frame}
  \frametitle{More cardinality}    
  \begin{theorem}
    If $A$ and $B$ are countable, then so is $A \times B$.
  \end{theorem}
  \begin{proof}
    \begin{itemize}
    \item $A$ and $B$ are countable, so we can write $A = \{a_1, a_2,
      ...\}$ and $B = \{b_1, b_2, ... \}$.
    \item Consider 
      \begin{align*}
        \begin{pmatrix} 
          (a_1, b_1) & (a_1, b_2) & (a_1, b_3) & \cdots \\
          (a_2, b_1) & (a_2, b_2) & (a2, b_3) & \cdots \\
          (a_3, b_1) & (a_3, b_2) & (a_3, b_3) & \cdots \\
          \vdots &               &           & \ddots
        \end{pmatrix}
      \end{align*}
    \item Count along the diagonals
    \end{itemize}
  \end{proof}  
\end{frame}

\begin{frame}
  \frametitle{More cardinality}    
  \begin{question}
    If $A_1, A_2, A_3, ...$ are each countable is $\cup_{n=1}^\infty
    A_n$ countable?
  \end{question}
\end{frame}

\begin{frame}
  \frametitle{More cardinality}    
  \begin{theorem}
    If $\{A_n\}$ are countable, then so is $\cup_{n=1}^\infty A_n$.
  \end{theorem}
  \begin{proof}
    \begin{itemize}
    \item $A_n$ is countable, so we can write $A_n = \{a_{1n}, a_{2n},
      ...\}$ 
    \item Consider 
      \begin{align*}
        \begin{pmatrix} 
          a_{11} & a_{12} & a_{13} & \cdots \\
          a_{21} & a_{22} & a_{23} & \cdots \\
          a_{31} & a_{32} & a_{33} & \cdots \\          
          \vdots &      &       & \ddots
        \end{pmatrix}
      \end{align*}
    \item Count along the diagonals
    \end{itemize}
  \end{proof}  
\end{frame}

%\section{Solving systems of linear equations}

\begin{frame}
  \tableofcontents  
\end{frame}

\section{Introduction and definitions}
\begin{frame} 
  \frametitle{Systems of linear equations}
  \begin{itemize} 
  \item Example:
    \begin{align*}
      5 x_1 - 7 x_2 = & 9 \\
      -8 x_1 + x_2 = & 0.
    \end{align*}
  \item In general:
    \begin{align*}
      a_{11} x_1 + a_{12} x_2 + ... + a_{1n} x_n = & b_1 \\
      a_{21} x_1 + a_{22} x_2 + ... + a_{2n} x_n = & b_2 \\
      \vdots & \vdots \\
      a_{m1} x_1 + a_{m2} x_2 + ... + a_{mn} x_n = & b_m ,
    \end{align*}
  \end{itemize}
\end{frame}

\begin{frame}
  \frametitle{Coefficient matrix}
  \begin{itemize}
  \item Coefficient matrix $A$
    \begin{align*}
      \begin{pmatrix} a_{11} &  \cdots & a_{1n} \\
        \vdots & & \vdots \\
        a_{m1} & \cdots & a_{mn} 
      \end{pmatrix} 
      \begin{pmatrix} x_1 \\ \vdots \\ x_n
      \end{pmatrix} = & \begin{pmatrix} b_1 \\ \vdots \\ b_m
      \end{pmatrix} \\
      A \mathbf{x} = & \mathbf{b}
    \end{align*}
  \item Augmented coefficient matrix $\hat{A}$
    \begin{align*}
      \hat{A} =   \begin{pmatrix} a_{11} &  \cdots & a_{1n} & b_1 \\
        \vdots & & \vdots & \vdots \\
        a_{m1} & \cdots & a_{mn} & b_m 
      \end{pmatrix}  = \begin{pmatrix} A \mathbf{b}
      \end{pmatrix}
    \end{align*}
  \end{itemize}
\end{frame}

\begin{frame}
  \frametitle{Example: Markov model of employment}
  \begin{itemize}
  \item Let $s_t = \begin{cases} 1 \text{ if employed at time } t \\
      0 \text{ if unemployed at time } t
    \end{cases}$
  \item Random process described by $P(s_t | s_{t-1}, s_{t-2}, ... )$
  \item \textbf{Markovian}: $P(s_t | s_{t-1}, s_{t-2}, ... ) = P(s_t |
    s_{t-1})$
    \begin{itemize}
    \item Probability of being employed tomorrow only depends on
      whether you're employed today and not the more distant past
    \end{itemize}
  \item \textbf{Stationary distribution}: $q$ stationary if when $P(s_t = 1) = q(1)$ and $P(s_t
    = 0) = q(0)$ today, then also have $P(s_{t+1} = 1) = q(1)$ and
    $P(s_{t+1}=0) = q(0)$ tomorrow 
    \[ q(s) = \sum_{s_0 \in \{0, 1\} } P(s | s_0) q(s_0) \]
  \end{itemize}
\end{frame}

\begin{frame}
  \frametitle{Example: Markov model of employment}
  \begin{itemize}
  \item Stationary distribution satisfies:
    \begin{align*}
      q(1) = & P(1|1) q(1) + P(1|0) q(0) \\
      q(0) = & P(0|1) q(1) + P(0|0) q(0) \\
      1 = & q(1) + q(0) 
    \end{align*}
    system of linear equations for unknowns $q(1)$ and $q(2)$
  \item Coefficient matrix and augmented coefficient matrix $=$ ?
  \end{itemize}
\end{frame}

\begin{frame}
  \frametitle{Questions to be answered:}
  Given a system of linear equations:
  \begin{enumerate}
  \item Does any solution exist?
  \item How many solutions?
  \item How can a solution be computed?
  \end{enumerate}  
\end{frame}

\section{Gaussian elimination}

\begin{frame}
  \frametitle{Equation/row operations}
  \begin{itemize}
  \item Familiar with solving equations by:
    \begin{itemize}
    \item Substitution
    \item Elimination
    \end{itemize}
  \item Use \textbf{equation (row) operations} that preserve set of
    solutions
    \begin{enumerate}
    \item Multiply an equation by a non-zero constant,
    \item Add a multiple of one equation to another, and
    \item Interchange two equations.
    \end{enumerate}
  \end{itemize}
\end{frame}

\begin{frame}
  \frametitle{Row echelon form}
  \begin{itemize}
  \item \textbf{Row echelon form}: each row begins with more zeros
    than the row above it or the row is all zeros
  \item Examples: 
    \begin{align*}
      \begin{pmatrix} 
        a_{11} & a_{12} \\
        0     & a_{22}
      \end{pmatrix}
    \end{align*}
    \begin{align*}
      \begin{pmatrix}
        a_{11} & a_{12} & a_{13} \\
        0     & 0     & a_{23} 
      \end{pmatrix}
    \end{align*}
  \item Once a system is in row echelon form it is easy to solve by
    back substitution
  \end{itemize}
\end{frame}

\begin{frame}
  \frametitle{Gaussian elimination}
  \begin{itemize}
  \item Systematic way to transform system of equations to row echelon
    form
    \begin{enumerate}
    \item\label{ge1} Identify the first column to contain any non-zero
      elements, call this column $c^*$.
    \item\label{ge2} Interchange rows so that a nonzero entry appears at
      the top of column $c^*$. 
      % Divide the new first row by its first
      % non-zero entry, so its first non-zero entry becomes one. 
    \item\label{ge3} Add a multiple of the first row to each of the rows
      below so that the entries in column $c^*$ below the first row are
      zero.
    \item\label{ge4} Repeat \ref{ge1}-\ref{ge2} on the submatrix
      consisting of the lower right part of the original matrix below the
      first row and to the right of column $c^*$. Stop if this submatrix
      has no columns or has no rows.
    \end{enumerate}
  \end{itemize}
\end{frame}

\begin{frame}
  \frametitle{Gaussian elimination: example} 
  \begin{align*}
    \begin{array}{ccccccc}
      x_1  &+ & x_2  &+& 3 x_3 &=& 0 \\
      2x_1 &+ & 3x_2 &+&  7 x_3 &=& 9 \\
      3x_1 &+ & 5x_2 &+& 13x_3  &=& 1 
    \end{array}
  \end{align*}
\end{frame}

\section{Existence of solutions}
\begin{frame}
  Transform the following system into row echelon form:
  \begin{align*}
    x + 2 y - z = & 2 \\
    4 y + z = & 5 \\
    -2x - 4y + 2z = & 1.
  \end{align*}
\end{frame}

\begin{frame}
  \frametitle{Existence of solutions}
  \begin{itemize}
  \item \textbf{rank} of $A$ is the number of nonzero rows in its row
    echelon form
  \item Is rank well-defined?
  \end{itemize}
  
  \begin{lemma}\label{lem:rankcr}
    The rank of a matrix $A$ is always less than or equal to the number
    of columns of $A$ and less than or equal to the number of rows of $A$.
  \end{lemma}
  
  \begin{lemma}\label{lem:rankaug}
    Let $A$ be a coefficient matrix and $\hat{A}$ be an augmented
    coefficient matrix. Then $\rank A \leq \rank \hat{A}$.
  \end{lemma}
\end{frame}


\begin{frame}
  \frametitle{Existence of solutions}
  \begin{theorem}[Existence of solutions]\label{thm:exist}
    A system of linear equations with coefficient matrix $A$ and
    augmented coefficient matrix $\hat{A}$ has a solution (perhaps more
    than one) if and only if $\rank A = \rank \hat{A}$.
  \end{theorem}
  \begin{proof}
    See notes.
  \end{proof}
\end{frame}

\begin{frame}
  Consider the system:
  \begin{align*}
    4 y + z = & 5 \\    
    x + 2 y - z = & 2 \\
    -8y - 2z = & -10.
  \end{align*}
  \begin{itemize}
  \item What if $b_3 \neq -10$?
  \end{itemize}  
\end{frame}



\begin{frame}\frametitle{Multiple solutions means infinite solutions}
  \begin{lemma}\label{lem:sinf}
    Suppose $\mathbf{x}_1$ and $\mathbf{x}_2$ are two distinct
    solutions to the system of equations $A \mathbf{x} =
    \mathbf{b}$. Then the system of equations has (uncountably)
    infinitely many solutions.
  \end{lemma}
\end{frame}

\begin{frame}\frametitle{Solution existence for any $b$}
  \begin{theorem}[Solution existence]\label{thm:sexist}
    A system of linear equations with coefficient matrix $A$ will have a
    solution for any choice of $b_1, ..., b_m$ if and only if $\rank A$
    is equal to the number of rows of $A$.
  \end{theorem}

  \begin{corollary}
    For any system of equations with more equations than variables,
    (i.e.\ an overdetermined system)
    there exists a choice of $\textbf{b}$ such that no solutions exist. 
  \end{corollary}
\end{frame}

\section{Uniqueness of solutions}

\begin{frame}\frametitle{Uniqueness}
  \begin{theorem}[Solution uniqueness] \label{thm:sunique}
    Any system of equations with coefficient matrix $A$ has at most one
    solution for any $b_1, ... , b_m$ if and only if $\rank A$ equals
    the number of columns of $A$. 
  \end{theorem}

  \begin{corollary}\label{cor:smult}
    If $\rank A$ is less than the number of columns of $A$ then either
    no solutions exists or multiple solutions exists.
  \end{corollary}
  
  \begin{corollary}
    $A$ is nonsingular (always has a unique solution) if and only if
    $A$ has an equal number of columns and rows ($A$ is square) and
    has rank equal to its number of columns (or rows).
  \end{corollary}
\end{frame}

\begin{frame}
  \frametitle{Summary}
  \begin{centering}
  \begin{tabular}{c|c|c|c} 
    & \multicolumn{2}{c}{$\rank A = \rank \hat{A}$} &  
    $\rank A < \rank \hat{A} $ \\ 
    & $\rank A = \mathrm{columns}(A)$ & 
    $\rank A < n$ & \\ \hline
    $0$ &  &  & X \\
    $1$ & X & & \\
    $\infty$ & & X & \\ \hline
  \end{tabular}
  \end{centering}
\end{frame}

\section{Set of solutions}

\begin{frame}
  \frametitle{Sets of solutions}
\begin{figure}\label{fig:solnSets}
  \begin{tabular}{ccc}
    $-2x + y = 1$
    &
    $\begin{array}{cc} 
      x - y - z = & 0 \\
      -2x + y + 3z = & 1 
    \end{array}$    
    & 
    $
    -2x - y +  z =  2
    $
    \\
    \includegraphics[width=0.25\linewidth]{fig1} & 
    \includegraphics[width=0.25\linewidth]{fig2} & 
    \includegraphics[width=0.25\linewidth]{fig3} 
  \end{tabular}
\end{figure}
\end{frame}

\begin{frame}
  \begin{definition}
    The set $S \subseteq \mathbb{R}^n$ is called a \textbf{linear
      subspace} if it is closed under (i) scalar multiplication and (ii)
    addition in other words, if 
    \begin{itemize}
    \item[(i)] for every $(x_1, ..., x_n)\in S$ and $a \in \mathbb{R}$,
      we have $(a x_1, a x_2, ..., a x_n) \in S$, and
    \item[(ii)] for every $(x_1, ..., x_n)\in S$ and $(y_1, ..., y_n)\in
      S$, we have
      $(x_1 + y_1, ..., x_n + y_n)  \in S$
    \end{itemize}
  \end{definition}
\end{frame}

\begin{frame}
  \begin{definition}
    A set of vectors in $\mathbb{R}^n$, $\{\textbf{x}_j = (x^j_1,...,
    x^j_n)\}_{j=1}^J$, is \textbf{linearly independent} if the only
    solution to 
    \begin{align*}
      \sum_{j=1}^J c_j \textbf{x}_j = 0 
    \end{align*}
    is $c_1 = c_2 = ... = c_J = 0$. 
  \end{definition}

  \begin{definition}
    The dimension of a linear subspace $S \subseteq \mathbb{R}^n$ is the
    cardinality of the largest set of linearly independent elements in
    $S$. 
  \end{definition} 
\end{frame}

\begin{frame}
  \begin{theorem}[Rouch\'{e}-Capelli] \label{thm:rc} A system of linear
    equations with $n$ variables has a solution if and only if the rank
    of its coefficient matrix, $A$, is equal to the rank of its
    augmented matrix, $\hat{A}$. If a solution exists and $\rank A$ is
    equal to its number of columns, the solution is unique. If a
    solution exists and $\rank A$ is less than its number of columns,
    there are infinite solutions. In this case the set of solutions
    forms a linear subspace of dimension $n - \rank A$.
  \end{theorem}
\end{frame}

\begin{frame}
  \frametitle{Example: Markov model of employment}
  \begin{itemize} 
  \item Employment $s_t \in \{u, e\}$.
  \item Markovian: $\Pr(s_t | s_{t-1}, s_{t-2} , ... ) = \Pr(s_t|
    s_{t-1})$
  \item Stationary distribution: $(\pi_e, \pi_u)$ such that if $s_t = e$
    with probability $\pi_e$ and $u$ with probability $\pi_u$, then
    $s_{t+1}$ has same distribution, i.e.\ 
    \begin{align*}
       \Pr(e|e) \pi_e + \Pr(e|u) \pi_u = & \pi_e \\
       \Pr(u|e) \pi_e + \Pr(u|u) \pi_u = & \pi_u
    \end{align*}
  \end{itemize}
\end{frame}
\begin{frame}
  \frametitle{Example: Markov model of employment}
  \begin{itemize}
  \item Stationary distribution is a solution to:
  \begin{align*}
      (p_{ee}-1) \pi_e + p_{eu} \pi_u = & 0\\
      p_{ue} \pi_e + (p_{uu}-1) \pi_e= & 0\\
      \pi_e + \pi_u = & 1 .
    \end{align*}
  \item Questions:
    \begin{enumerate}
    \item Does any solution exist? 
    \item How many solutions exist?
    \item How can a solution be computed?
    \end{enumerate}    
  \end{itemize}
\end{frame}

\end{document}
