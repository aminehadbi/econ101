\documentclass[12pt,reqno]{amsart}

\usepackage{amsmath}
\usepackage{amsfonts}
\usepackage{graphicx}
%\usepackage{epstopdf}
\usepackage{hyperref}
\usepackage[left=1in,right=1in,top=0.9in,bottom=0.9in]{geometry}
\usepackage{multirow}
\usepackage{verbatim}
\usepackage{fancyhdr}
%\usepackage[small,compact]{titlesec} 

%\usepackage{pxfonts}
%\usepackage{isomath}
\usepackage{mathpazo}
%\usepackage{arev} %     (Arev/Vera Sans)
%\usepackage{eulervm} %_   (Euler Math)
%\usepackage{fixmath} %  (Computer Modern)
%\usepackage{hvmath} %_   (HV-Math/Helvetica)
%\usepackage{tmmath} %_   (TM-Math/Times)
%\usepackage{cmbright}
%\usepackage{ccfonts} \usepackage[T1]{fontenc}
%\usepackage[garamond]{mathdesign}
\usepackage{color}
\usepackage{ulem}

\newcommand{\argmax}{\operatornamewithlimits{arg\,max}}
\newcommand{\argmin}{\operatornamewithlimits{arg\,min}}
\def\inprobLOW{\rightarrow_p}
\def\inprobHIGH{\,{\buildrel p \over \rightarrow}\,} 
\def\inprob{\,{\inprobHIGH}\,} 
\def\indist{\,{\buildrel d \over \rightarrow}\,} 
\newcommand{\rank}{\mathrm{rank}}

\newtheorem{theorem}{Theorem}[section]
\newtheorem{conjecture}{Conjecture}[section]
\newtheorem{corollary}{Corollary}[section]
\newtheorem{lemma}{Lemma}[section]
\newtheorem{proposition}{Proposition}[section]
\theoremstyle{definition}
\newtheorem{assumption}{}[section]
%\renewcommand{\theassumption}{C\arabic{assumption}}
\newtheorem{definition}{Definition}[section]
\newtheorem{step}{Step}[section]
\newtheorem{remark}{Comment}[section]
\newtheorem{example}{Example}[section]
\newtheorem*{example*}{Example}

\linespread{1.1}

\pagestyle{fancy}
%\renewcommand{\sectionmark}[1]{\markright{#1}{}}
\fancyhead{}
\fancyfoot{} 
%\fancyhead[LE,LO]{\tiny{\thepage}}
\fancyhead[CE,CO]{\tiny{\rightmark}}
\fancyfoot[C]{\small{\thepage}}
\renewcommand{\headrulewidth}{0pt}
\renewcommand{\footrulewidth}{0pt}

\fancypagestyle{plain}{%
\fancyhf{} % clear all header and footer fields
\fancyfoot[C]{\small{\thepage}} % except the center
\renewcommand{\headrulewidth}{0pt}
\renewcommand{\footrulewidth}{0pt}}

\makeatletter
\renewcommand{\@maketitle}{
  \null 
  \begin{center}%
    \rule{\linewidth}{1pt} 
    {\Large \textbf{\textsc{\@title}}} \par
    {\normalsize \textsc{Paul Schrimpf}} \par
    {\normalsize \textsc{\@date}} \par
    {\small \textsc{University of British Columbia}} \par
    {\small \textsc{Economics 526}} \par
    \rule{\linewidth}{1pt} 
  \end{center}%
  \par \vskip 0.9em
}
\makeatother

\title{Systems of Linear Equations}
\date{\today}

\begin{document}

\maketitle

This lecture analyzes systems of linear equations. It is largely based
on Chapters 6-7 of Simon and Blume. 

Systems of linear appear throughout economics. There are some
interesting economic models that naturally have a linear
structure. Chapter 6 of Simon and Blume gives five examples: taxation
and deductions, Leontief production, Markovian employment, IS-LM, and
investment and arbitrage. Linear systems also arise as local
approximations to nonlinear systems. Therefore, understanding linear
systems is essential for understanding nonlinear systems as well.

A system of linear equations is any set of equations in which the
unknown only appear linearly. An example of a linear system with 2
unknowns and 2 equations is
\begin{align*}
  5 x_1 - 7 x_2 = & 9 \\
  -8 x_1 + x_2 = & 0.
\end{align*}
In general, a linear system with $m$ equation and $n$ unknowns can be
written
\begin{align*}
  a_{11} x_1 + a_{12} x_2 + ... + a_{1n} x_n = & b_1 \\
  a_{21} x_1 + a_{22} x_2 + ... + a_{2n} x_n = & b_2 \\
  \vdots & \vdots \\
  a_{m1} x_1 + a_{m2} x_2 + ... + a_{mn} x_n = & b_m ,
\end{align*}
where $a_{ij}$ and $b_i$ are given, and $x_j$ are unknown.  This
system of equations be also be written in matrix form.
\begin{align*}
  \begin{pmatrix} a_{11} &  \cdots & a_{1n} \\
    \vdots & & \vdots \\
    a_{m1} & \cdots & a_{mn} 
  \end{pmatrix} 
  \begin{pmatrix} x_1 \\ \vdots \\ x_n
  \end{pmatrix} = & \begin{pmatrix} b_1 \\ \vdots \\ b_m
  \end{pmatrix} \\
  A \mathbf{x} = & \mathbf{b},
\end{align*}
where the $m$ by $n$ matrix $A$ is called the \textbf{coefficient
  matrix}, $\mathbf{x}$ is an $n\times 1$ vector of unknowns and
$\mathbf{b}$ is an $m \times 1$ vector.  We can represent a system of
equations slightly more compactly by working with the
\textbf{augmented coefficient matrix},
\begin{align*}
  \hat{A} =   \begin{pmatrix} a_{11} &  \cdots & a_{1n} & b_1 \\
    \vdots & & \vdots & \vdots \\
    a_{m1} & \cdots & a_{mn} & b_m 
  \end{pmatrix}  = \begin{pmatrix} A \mathbf{b}
  \end{pmatrix}.
\end{align*}

\subsection*{Example: Markov model of employment}

Let $s_t$ be some variable, like employment, that is randomly changing
over time. We call this a random process (or stochastic process). In
general, the probability of being employed at time $t$ could depend on
the entire history of $s_{t-1}, s_{t-2}, ...$. We could write it as
$P(s_t|s_{t-1}, s_{t-2}, ...)$. We call $s_t$ \textbf{Markovian} if
instead the probability of being employed at time $t$ only depends on
$s_{t-1}$.
\[ P(s_t|s_{t-1}, s_{t-2}, ...) = P(s_t | s_{t-1}) \] In this case,
$\{s_t\}$ is said to be a Markov process. Given a Markov process, we
are often interested in its stationary distribution. A stationary
distribution is a distribution that $s_t$ that will stay the same as
time changes. If $s_t \in S$ is discrete, a stationary distribution
must satisfy 
\[ q(s) = \sum_{s_0 \in S} P(s|s_0)q(s_0) \]
for all $s \in S$. 

Markov processes appear in many areas of economics. It is usually
easier to work with a Markov process than a general stochastic
process.  Often, we assume variables are Markovian to make a model
tractable. This is really why we are about to assume employment is a
Markov process. Other times, economic theory implies that some
variable must follow a Markov process. This sometimes happens with
asset prices. Markov processes are also very important for Bayesian
estimation.

If employment follows a Markov process, then its evolution over time
is completely described by four probabilities: the probability that
someone who is employed today is employed tomorrow, the probability
that someone who is employed today is unemployed tomorrow, the
probability that someone who is unemployed today is employed tomorrow,
and the probability that someone who is unemployed today is also
unemployed tomorrow. We will denote these four probabilities by
$p_{ee}$, $p_{ue}$, $p_{eu}$, and $p_{uu}$. Given these for
probabilities, we might be interested in the equilibrium employment
rate, i.e.\ the employment rate in the stationary distribution of the
process.  Let $\pi_e$ and $\pi_u$ be stationary employment and
unemployment rates. They must satisfy 
\begin{align*}
  \pi_e =& p_{ee} \pi_e + p_{eu} \pi_u \\
  \pi_u =& p_{ue} \pi_e + p_{uu} \pi_e \\
  1 = & \pi_e + \pi_u,
\end{align*}
or, equivalently, in the general form written above
\begin{align*}
  (p_{ee}-1) \pi_e + p_{eu} \pi_u = & 0\\
  p_{ue} \pi_e + (p_{uu}-1) \pi_e= & 0\\
  \pi_e + \pi_u = & 1 .
\end{align*}
The augmented matrix for this system is
\[ \hat{A} = \begin{pmatrix} p_{ee} - 1 & p_{eu} & 0 \\
  p_{ue} & p_{uu}-1 & 0 \\
  1 & 1 & 1 \end{pmatrix}. \]
Three questions to ask about such a system of equations are:
\begin{enumerate}
\item Does any solution exist? 
\item How many solutions exist?
\item How can a solution be computed?
\end{enumerate}
We will begin by examining the first question.  Then, we will see that
the answers to the first two questions depend on the coefficients of
the system of equations. 

\section{Solving systems of equations}
You likely already have experience solving small systems of
equations. The two basic techniques are
\begin{enumerate}
\item substitution: solve for one variable in terms of the others and
  substitute 
\item elimination: add a multiple of one equation to another to
  eliminate one variable.
\end{enumerate}
One way of viewing elimination is that transforms one system of
equations to another that is easier to solve, while ensuring the
solution remains the same. Three basic \textbf{equation operations}
that we can perform while preserving the solution of the system are:
\begin{enumerate}
\item Multiply an equation by a non-zero constant,
\item Add a multiple of one equation to another, and
\item Interchange two equations.
\end{enumerate}
We could also perform these operations on the augmented coefficient
matrix. We then call them \textbf{row operations} instead.  Given a
reasonably small system of equations, you might be able to solve
the system without thinking too carefully about the steps
involved. However, if we want to solve large systems of equations (or
write a computer programmer to solve large systems of equations), we
will need to think carefully about the steps involved. 

\subsection{Gaussian elimination}

Gaussian elimination is the process of using these operations to
transform the augmented matrix of a system of equation into row
echelon form.  A matrix is in \textbf{row echelon form} if each row
begins with more zeros than the row above it or the row is all
zeros. The first non-zero entry in each row of a matrix in row echelon
form is called a \textbf{pivot}.  \textbf{Gaussian elimination} can be
performed as follows:
\renewcommand{\theenumi}{GE\arabic{enumi}}
\begin{enumerate}
\item\label{ge1} Identify the first column to contain any non-zero
  elements, call this column $c^*$.
\item\label{ge2} Interchange rows so that a nonzero entry appears at
  the top of column $c^*$. 
  % Divide the new first row by its first
  %non-zero entry, so its first non-zero entry becomes one. 
\item\label{ge3} Add a multiple of the first row to each of the rows
  below so that the entries in column $c^*$ below the first row are
  zero.
\item\label{ge4} Repeat \ref{ge1}-\ref{ge2} on the submatrix
  consisting of the lower right part of the original matrix below the
  first row and to the right of column $c^*$. Stop if this submatrix
  has no columns or has no rows.
\end{enumerate}

\begin{example}[Gaussian elimination]
  Consider the system
  \begin{align*}
    \begin{array}{ccccccccc} 
      &   & 3x_2 & + & 2x_3 & - & 4x_4 & = & 4 \\
      6x_1 & - &  x_2 &   &      & +  & x_4 & = & -2 \\
      x_1 & + & x_2  & +   & x_3 &    &     & = & 1 \\
      &   &     &     & 4x_3 & - & x_4 & = &  3
    \end{array}
  \end{align*}
  The augmented matrix for this system is 
  \begin{align*}
    \hat{A} = & \begin{pmatrix} 
      0 &  3 &  2 & -4 & 4 \\
      6 & -1 &  0 & 1  & -2 \\
      1 & 1  &  1 & 0  & 1 \\
      0 & 0  & 4 & -1  & 3 
    \end{pmatrix}
  \end{align*}
  Following the steps above:
  \begin{itemize}
  \item[\ref{ge1}] $c^* = 1$
  \item[\ref{ge2}] Swap 2nd and 1st row.
    \begin{align*}
      \begin{pmatrix} 
        6 & -1 &  0 & 1  & -2 \\
        0 &  3 &  2 & -4 & 4 \\
        1 & 1  &  1 & 0  & 1 \\
        0 & 0  & 4 & -1  & 3 
      \end{pmatrix}
    \end{align*}
  \item[\ref{ge3}] Add $-1/6(row$ 1$)$ to row 3.
    \begin{align*}
      \begin{pmatrix} 
        6 & -1 &  0 & 1  & -2 \\
        0 &  3 &  2 & -4 & 4 \\
        0 & 7/6  &  1 & -1/6  & 4/3 \\
        0 & 0  & 4 & -1  & 3 
      \end{pmatrix}
    \end{align*}
  \item[\ref{ge1}] Now ignoring first row and column, $c^* = 2$.
  \item[\ref{ge2}] Leave row 2 where it is.
  \item[\ref{ge3}] Add $-7/18$ row 2 to row 3.
    \begin{align*}
      \begin{pmatrix} 
        6 & -1 &  0 & 1  & -2 \\
        0 &  3 &  2 & -4 & 4 \\
        0 & 0  &  2/9 & 25/18 & -4/9 \\
        0 & 0  & 4 & -1  & 3 
      \end{pmatrix}
    \end{align*}  
  \item[\ref{ge1}] Now ignoring first to columns and rows, $c^* = 3$.
  \item[\ref{ge2}] Leave row 3.
  \item[\ref{ge3}] Add $-18$ row 3 to row 4.
    \begin{align*}
      \begin{pmatrix} 
        6 & -1 &  0 & 1  & -2 \\
        0 &  3 &  2 & -4 & 4 \\
        0 & 0  &  2/9 & 14/9 & -4/9 \\
        0 & 0  & 0 & -26  & 11
      \end{pmatrix}
    \end{align*}  
  \end{itemize}
  Given the above row echelon form, it is relatively easy to solve the
  system of equations. From the last row, we know $x_4 =
  -11/26$. Substituting into the second last row, we get $x_3 = 9/2
  (-4/9 + 14/9\times 11/26)$. Back substituting back to the first row
  gives a complete solution.
\end{example}

It is always possible to transform a matrix into row echelon
form. Moreover, we can prove that the above procedure always work.
\begin{theorem}[Existence of row echelon form] \label{thm:ge}
  Any matrix can be put into row echelon form using Gaussian
  elimination. 
\end{theorem}
\begin{proof}
  Let $m$ be the number of rows of a matrix and $n$ be the number of
  columns. We will prove the theorem by induction on $m$ and $n$. Any
  $1$ by $n$ matrix is already in row echelon form. Also, given any
  $m$ by $1$ matrix, if it is all zeros, it is already in row-echelon
  form. Otherwise, it contains a nonzero entry. We can move this
  nonzero entry to the first row. Then we can add a multiple of the
  first entry to all other entries to make all entries after the first
  into zeros. Thus, any $m$ by $1$ matrix can be put into row echelon
  form.
  
  Now suppose we can put matrix with less than $m$ rows or less than
  $n$ columns into row echelon form. Given an $m$ by $n$ matrix either
  the first column contains a nonzero entry or it does not. If the
  first column is all zeros, then we may ignore it, and we just have
  to transform the remainging $m$ by $n-1$ matrix into row echelon
  form, which can be done due to our inductive assumption. 
  
  If the first column is not all zeros, then we can follow steps
  (\ref{ge1}-\ref{ge3}) to make the first element of the matrix
  nonzero and all other entries in the first column zero. We can then
  just work on transforming the remaining $m-1$ by $n$ matrix after
  the first row into row echelon form, which again is possible by our
  inductive assumption.
\end{proof}
Note that the row echelon form of a matrix is not unique because, for
example, we could multiply any row by a constant and the matrix would
still be in row echelon form. 

\subsection{Gauss-Jordan elimination}

A matrix is in \textbf{reduced row echelon form} if it is in row
echelon form with each pivot equal to one and each column that
contains a pivot has no other non-zero entries. For example,
\begin{align*}
  \begin{pmatrix}
    1 & 0 & 0 & b_1 \\
    0 & 1 & 0 & b_2 \\
    0 & 0 & 1 & b_3 
  \end{pmatrix}
\end{align*}
is in reduced row echelon form. The solution to a system of linear
equations is given immediately by its augmented matrix in reduced row
echelon form. In the previous example, the solution is $x_1 = b_1$,
$x_2 = b_2$, and $x_3 = b_3$. 
\textbf{Gauss-Jordan elimination} transforms a matrix into reduced row
echelon form and can be performed as follows:
\renewcommand{\theenumi}{GJ\arabic{enumi}}
\begin{enumerate}
\item\label{gj1} Put the matrix into row echelon form by performing
  Gaussian elimination
\item\label{gj2} Divide the bottom row by its pivot. 
\item\label{gj3} Add a multiple of the bottom row to each row above it
  such that the column above the bottom row's pivot is made equal to
  all zeros.
\item\label{gj4} Repeat \ref{gj2} and \ref{gj3} with the next row up.
\end{enumerate}
\begin{example}[Gauss-Jordan elimination]
  Suppose we have performed Gaussian elimination to get
  \[ \begin{pmatrix}
    2 & 3 & 0 & -1 \\
    0 & -1 & -2 & 3 \\
    0 & 0 & 7 & 14 
  \end{pmatrix} 
  \]
  Following the steps above, we get
  \begin{align*}
    \begin{pmatrix}
      2 & 3 & 0 & -1 \\
      0 & -1 & -2 & 3 \\
      0 & 0 & 1 & 2 
    \end{pmatrix} & (\text{\ref{gj2}}) \\
    \begin{pmatrix}
      2 & 3 & 0 & -1 \\
      0 & -1 & 0 & 7 \\
      0 & 0 & 1 & 2 
    \end{pmatrix} & (\text{\ref{gj3}}) \\
    \begin{pmatrix}
      2 & 3 & 0 & -1 \\
      0 & 1 & 0 & -7 \\
      0 & 0 & 1 & 2 
    \end{pmatrix} & (\text{\ref{gj2}}) \\
    \begin{pmatrix}
      2 & 0 & 0 & 20 \\
      0 & 1 & 0 & -7 \\
      0 & 0 & 1 & 2 
    \end{pmatrix} & (\text{\ref{gj3}}) \\
    \begin{pmatrix}
      1 & 0 & 0 & 10 \\
      0 & 1 & 0 & -7 \\
      0 & 0 & 1 & 2 
    \end{pmatrix} & (\text{\ref{gj2}}) 
  \end{align*}
\end{example}

As with row echelon form and Gaussian elimination, we can prove that
a reduced row echelon form always exists and Gauss-Jordan elimination
can produce it.
\begin{theorem}[Existence of reduced row echelon form]\label{thm:gj}
  Any matrix can be put into reduced row echelon form using
  Gauss-Jordan elimination.
\end{theorem}
\begin{proof}
  Let $A$ by an $m$ by $n$ matrix. By Theorem \ref{thm:ge}, $A$ can
  be transformed into row echelon form. Steps \ref{gj2}-\ref{gj4} will
  transform $A$ into reduced row echelon form. 
\end{proof}

\section{Existence of solutions}

We now have a method for solving systems of equations. Will this
method always work? The answer is no. It is easy to write down systems
of equations that cannot be solved. For example, $\begin{array}{ccc} x
  & = & 2 \\ - x & = & 3 \end{array}$, has no solutions. However, it
is not always so obvious when a system of equations has no solutions. 
\begin{example}
  Consider the system:
  \begin{align*}
    x + 2 y - z = & 2 \\
    4 y + z = & 5 \\
    -2x - 4y + 2z = & 1.
  \end{align*}
  Let's transform this system into row echelon form. Let $\hat{A}
  \simeq \hat{B}$ mean that the systems of equations represent by
  $\hat{A}$ and $\hat{B}$ have the same solution.
  \begin{align*}
    \begin{pmatrix} 
      1 & 2 & -1 & 2 \\
      0 & 4 & 1 & 5 \\ 
      -2 & -4 & 2 & 1 
    \end{pmatrix}  \simeq &     
    \begin{pmatrix} 
      1 & 2 & -1 & 2 \\
      0 & 4 & 1 & 5 \\ 
      0 & 0 & 0 & 3 
    \end{pmatrix} 
  \end{align*}
  The third equation in the transformed system is 
  \[ 0 x + 0y + 0z = 3, \] 
  which has no solution. Then, the entire transfrom system must have
  no solution. By construction, the transformed system has the same
  solutions as the original system, so the original system of
  equations must also have no solution. 
\end{example}

In the preceding example, we saw a system of equations with no
solution because its row echelon form had a row with all zeros except
for the final column. This observation applies more generally. 
\begin{definition}
  The \textbf{rank} of a matrix is the number of nonzero rows in its
  row echelon form. 
\end{definition}
This definition is slighty problematic because, as stated earlier, the
row echelon form of a matrix is not unique. To show that rank is well
defined, we should prove that any row echelon form of a matrix has the
same number of nonzero rows. We will prove that rank is a well-defined
a little later in the course. 
\begin{lemma}\label{lem:rankcr}
  The rank of a matrix $A$ is always less than or equal to the number
  of columns of $A$ and less than or equal to the number of rows of $A$.
\end{lemma}
\begin{proof}
  The first claim follows form the definition of row echelon form. If
  each row of the row echelon form of $A$ must start with more zeros
  than the preceding row, then there can be at most as many nonzero
  rows as there are columns. $A$ also cannot have more nonzero rows
  than total rows, so the second claim is trivial.
\end{proof}
\begin{lemma}\label{lem:rankaug}
  Let $A$ be a coefficient matrix and $\hat{A}$ be an augmented
  coefficient matrix. Then $\rank A \leq \rank \hat{A}$.
\end{lemma}
\begin{proof}

\end{proof}

\begin{theorem}[Existence of solutions]\label{thm:exist}
  A system of linear equations with coefficient matrix $A$ and
  augmented coefficient matrix $\hat{A}$ has a solution (perhaps more
  than one) if and only if $\rank A = \rank \hat{A}$.
\end{theorem}
\begin{proof}
  Suppose $\rank A \neq \rank \hat{A}$. From \ref{lem:rankaug}, we
  know that $\rank A < \rank \hat{A}$. There is a zero row in the row
  echelon form of $A$ and a corresponding nonzero row in the row
  echelon form of $\hat{A}$. The equation associated with this row is
  of the form
  \[ 0 x_1 + ... + 0 x_n = b'_m \]
  for some $b'_m \neq 0$. As in the example, this equation has no
  solution, so the system has no solution. 

  Now suppose $\rank A = \rank \hat{A}$. Let $A'$ be the row echelon
  form of $A$. We can prove the existence of solutions by induction on
  the number of rows. If $A'$ is $1$ by $n$ with a nonzero entry, say
  $a_{1j}$, we can produce a solution by choosing any values for
  $\{x_{1}, ... , x_{n}\} \setminus \{x_j\}$ and set
  \begin{align*}
    x_{j} = \frac{1}{a_{1j_1}} \left(b_1 - \sum_{k \neq j} a_{1k}
      x_{k} \right).
  \end{align*}
  
  Now let $A$ be $m$ by $n$ and suppose we have proven the claim for
  all $(m-1)$ by $n$ matrices. If the $m$th row of $A'$ has all zeros,
  then we may ignore it and only work with the first $m-1$ rows. If
  the $m$th row of $A'$ has a nonzero entry, let $a_{mj}'$ be the
  first nonzero entry in the $m$th row. Choose any values for
  $x_{j+1},...,x_n$ and set
  \begin{align*}
    x_{j} = \frac{1}{a_{1j_1}} \left(b_m' - \sum_{k = j+1}^n a'_{mk}
      x_{k} \right).
  \end{align*}
  Substitute these values of $x_j, ..., x_n$ into the $m-1$ rows above
  to produce the system
  \begin{align*}
    \hat{B}' = \begin{pmatrix}
      a'_{1,1} & \cdots  & a'_{1,j-1} & b'_1 - \sum_{k=j}^n a_{1,k} x_k  \\
      0 & \ddots &                & \vdots \\ 
          & \cdots  & a_{m-1,j-1} & b'_{m-1} - \sum_{k=j}^n a_{m-1,k}x_k 
    \end{pmatrix}.
  \end{align*}
  We must have $a_{m-1,j-1} \neq 0$ because $a_{m,j} \neq 0$ and
  $\hat{A}$ was in row echelon form. Similarly, all of $\hat{B}$ must
  be row echelon form because $\hat{A}$ was in row echelon form. Thus,
  $\rank B = m-1$ and the coefficient matrix associated with $\hat{B}$
  also has rank $m-1$. By the inductive assumption, there exists $x_1,
  ... , x_{j-1}$ that solve the system represented by $\hat{B}$. This
  solution combined with the values for $x_j , ..., x_n$ described
  above is a solution to the entire original system.
\end{proof}

We now have a nice condition for when there exists at least one
solution to a system of linear equations. If there is a solution can
be more than one? From looking at simple systems like $x + y = 0$, the
answer is clearly yes. If you look carefully at the proof of Theorem
\ref{thm:exist} you might be able to see that multiple solutions will
exist whenever at least one solution exists and the row echelon form
of the coefficient matrix has a row with more than one more zero than
the row preceding it. That is, there will be multiple solutions
whenever the rank of the augmented coefficient matrix is equal to rank
of the coefficient matrix, and rank is less than the number of
variables in the system (which is equal to the number of columns). We
will state these ideas formally and prove them below. Before that,
let's look at another example.
\begin{example}
  Consider the system:
  \begin{align*}
    4 y + z = & 5 \\    
    x + 2 y - z = & 2 \\
    -8y - 2z = & -10.
  \end{align*}
  Let's transform this system into row echelon form. Let $\hat{A}
  \simeq \hat{B}$ mean that the systems of equations represent by
  $\hat{A}$ and $\hat{B}$ have the same solution.
  \begin{align*}
    \begin{pmatrix} 
      0 & 4 & 1 & 5 \\ 
      1 & 2 & -1 & 2 \\
      0 & -8 & 2 & -10 
    \end{pmatrix}  \simeq &     
    \begin{pmatrix} 
      1 & 2 & -1 & 2 \\
      0 & 4 & 1 & 5 \\     
      0 & -8 & 2 & -10 
    \end{pmatrix}  \\
    \simeq & 
    \begin{pmatrix} 
      1 & 2 & -1 & 2 \\
      0 & 4 & 1 & 5 \\     
      0 & 0 & 0 & 0 
    \end{pmatrix}  
  \end{align*}
  For any value of $z\in \mathbb{R}$, $y = \frac{1}{4}(5 - z)$ and $x
  = 2 - \frac{1}{2}(5 - z) + z = -\frac{1+z}{2}$ is a solution to the
  system.  
\end{example}
When solving a system of equations, we call variables whose value is
indeterminate \textbf{free} variables. We call variables whose value
is either completely determined or determined by the value of the free
variables \textbf{basic}. In the above examples, $z$ is a free
variable and $x$ and $y$ are basic. Note that free and basic are
just names that are sometimes useful, but they are not concrete
definitions. In the above example, we could have just as easily
described the set of solutions by saying: $x \in \mathbb{R}$,
$z = 2x - 1$, and $y = \frac{3 - x}{2}$. 

In the above example we saw a system of equations with infinitely many
solutions. It turns out that whenever there is more than one solution,
there must be infinitely many.
\begin{lemma}\label{lem:sinf}
  \footnote{We state and prove this lemma using matrix notation. We
    will study matrix algebra in greater detail in the next
    lecture. If you are uncomfortable with matrix notation and
    operations here, it may help to restate and prove the theorem
    without using matrices.}  
  Suppose $\mathbf{x}_1$ and $\mathbf{x}_2$ are two distinct solutions
  to the system of equations $A \mathbf{x} = \mathbf{b}$. Then the
  system of equations has (uncountably) infinitely many solutions.
\end{lemma}
\begin{proof}
  Let $w \in \mathbb{R}$. Consider $\mathbf{x}(w) = w
  \mathbf{x}_1 + (1 - w)\mathbf{x}_2$. Since $\mathbf{x}_1 \neq
  \mathbf{x}_2$, for each $w$, $\mathbf{x}(w)$ is unique. Also,
  \begin{align*}
    A \mathbf{x}(w) = & A \left( \mathbf{x}_1 w + \mathbf{x}_2 (1-w)
    \right) \\
    = & A \mathbf{x}_1 w + A \mathbf{x}_2 (1-w) \\
    = & b w + b (1- w) \\
    = & b,
  \end{align*}
  so $\mathbf{x}(w)$ is another solution. 
\end{proof}

Another interesting observation about the previous example is that if
were to just change $b_3$ from $-10$ to any other number, it would have
no solutions instead of a unique solution. In particular, the third
row of the row echelon form would be $\begin{pmatrix} 0 & 0 & 0 & b_3
  + 10 \end{pmatrix}$.  It would be good to know when the existence
of solutions depends on the choice of $b$ and when it does not.
\begin{theorem}[Solution existence]\label{thm:sexist}
  A system of linear equations with coefficient matrix $A$ will have a
  solution for any choice of $b_1, ..., b_m$ if and only if $\rank A$
  is equal to the number of rows of $A$.
\end{theorem}
\begin{proof}
  If $\rank$ of $A$ is equal to the number of rows of $A$, then the
  row echelon form of $A$ has no all zero rows. For any choice of $b$,
  the augmented matrix, $\hat{A}$, must also have no all zero
  rows. Hence, $\rank A = \rank \hat{A}$ and by theorem
  \ref{thm:exist}, at least one solution exists. 

  If $\rank A$ is less than the number of rows of $A$, then the last
  row of the row echelon form of $A$, $A'$ has all zeros. If we
  augment produce an augmented matrix in row echelon form $\hat{A}'$
  with $b_m \neq 0$, then the last equation has no solutions. We can
  then reverse the steps of Gaussian elimination used to produce $A'$
  to arrive at an augmented matrix $\hat{A}$ corresponding to a system
  with coefficient matrix $A$ that has no solution. 
\end{proof}
The following corollary is an immediate consequence of the second part
of the above proof.
\begin{corollary}
  For any system of equations with more equations than variables,
  there exists a choice of $\textbf{b}$ such that no solutions exist. 
\end{corollary}
We call a system of equations with more equations than variables
\textbf{overdetermined}. A system with more variables than equations
is called \textbf{underdetermined}. 

\section{Uniqueness of solutions}

\begin{theorem}[Solution uniqueness] \label{thm:sunique}
  Any system of equations with coefficient matrix $A$ has at most one
  solution for any $b_1, ... , b_m$ if and only if $\rank A$ equals
  the number of columns of $A$. 
\end{theorem}
\begin{proof}
  Suppose $\rank A$ is equal to the number of columns of $A$. Then the
  row echelon form of $A$ must be of the form
  \begin{align*}
    \begin{pmatrix}
      a'_{11} & a'_{12} & ... & a'_{1n} \\
      0 & a'_{22} & ... & a'_{2n}  \\
      0 & 0 & \ddots \\
      0 & \cdots & 0 & a_{kn}' \\
      0 & \cdots & \cdots & 0 \\
      \vdots &  &  & \vdots \\
      0 & \cdots & \cdots & 0 
    \end{pmatrix}
  \end{align*}
  The only possible solution is $x_n = b_k/a'_{kn}$, $x_{n-1} =
  \frac{b_{k-1} - a_{k-1,n}' x_n}{a_{k-1,n-1}'}$. Thus, if any
  solution exists, it is unique. 

  Conversely, suppose $\rank A$ is less than the number of columns of
  $A$. We can prove that $A$ can have multiple solutions by performing
  induction on $m$. If $A$ is $1$ by $n$ and has at least one non-zero
  entry, then the system will have infinite solutions for any
  $b_1$. On the other hand, if $A$ is all zeros, then the system has
  infinite solutions for $b_1 = 0$.\footnote{The only other possible
    case is that $A$ is all zeros and $b_1 \neq 0$, in that case $A$
    has no solutions. Thus, if $A$ has any solution, then $A$ has
    multiple solutions. This observation is not needed for this proof,
    but will be used to prove corollary \ref{cor:smult}.}
  
  Now, suppose we have shown that any $m-1$ by $n$ matrix with rank
  less than the number of columns can have multiple solutions. If $A$
  is $m$ by $n$ with rank less than $n$, consider three cases: (i) the
  last row of the row echelon form of $A$ is identically zero, (ii)
  the last row of the row echelon from of $A$ has one non-zero entry
  and (iii) the last row of the row echelon form of $A$ has multiple
  non-zero entries. Every $A$ will fit into one of these three cases.

  In case (iii), we can produce a solution for any value of $x_n$, so
  multiple solutions exist.

  In case (i), we can delete the last row of the row echelon form of
  $A$ without changing its rank. Let $A_1$ be the row echelon form of
  $A$ with the last row deleted. When $b_m = 0$, any solution to $A_1$
  is also a solution to $A$. Furthermore, the rank of $A_1$ must equal
  the rank of $A$, and the number of columns of $A_1$ equals the
  number of columns of $A$. Hence, $A_1$ is $m-1$ by $n$ with rank
  less than the number of columns, so by induction, multiple solutions
  will exist.

  In case (ii), let $A_1$ be the row echelon form of $A$ with last row
  and last column deleted.  Finally, $\rank A_1 = \rank A - 1$ and
  $A_1$ has one column less than $A$, so $A_1$ is $m-1$ by $n-1$ and
  by induction multiple solutions exist. Let $b_1, ..., b_{m-1}$
  combined with $A_1$ produces multiple solutions. Then any solution
  to the system $A_1$ along with $x_{n} = b_m/\tilde{a}_{mn}$, where
  $\tilde{a}_{mn}$ is the last entry in the last row of the row
  echelon form of $A$, will be a solution to the system with
  coefficients $A$ and right hand side $b_1 + \tilde{a}_{1n}
  b_m/\tilde{a}_{mn}, \cdots , b_{m-1} + \tilde{a}_{m-1,n}
  b_m/\tilde{a}_{mn} , b_m$. Thus, multiple solutions to $A$ also
  exist.
\end{proof}
The following corollary is a nice consequence of the above proof.
\begin{corollary}\label{cor:smult}
  If $\rank A$ is less than the number of columns of $A$ then either
  no solutions exists or multiple solutions exists. 
\end{corollary}
\begin{proof}
  As noted in the footnote in the previous proof, the first step
  showed that any $1$ by $n$ matrix with rank less than $n$ has either
  no solutions or multiple solutions. The same inductive argument as
  in the previous proof then shows that whenever any solution exists,
  there must be multiple solutions. 
\end{proof}

We call a coefficient matrix $A$ \textbf{nonsingular} if for any $b_1,
..., b_m$ the system of equations has exactly one solution. Combining
the last two theorems, (\ref{thm:sexist} and \ref{thm:sunique}) we get
the following corollary.
\begin{corollary}
  $A$ is nonsingular if and only if $A$ has an equal number of columns
  and rows ($A$ is square) and has rank equal to its number of columns
  (or rows).
\end{corollary}

We now know conditions under which a system has a solution and when
the solution is unique. To review, we know from theorem
\ref{thm:exist} that a particular system has a solution if and only if
$\rank A = \rank \hat{A}$.  Additionally, if $\rank A$ is equal to its
number of columns, then the solution is unique (theorem
\ref{thm:sunique}).  On the other hand if $\rank A$ is less than the
number of columns, then there are infinite solutions (corollary
\ref{cor:smult} and lemma \ref{lem:sinf}). 

\section{Set of solutions}
A final issue to investigate is: if there are infinite solution, how
can we describe the set of solutions.  Figure \ref{fig:solnSets} plots
the solution sets to three systems of equations with multiple
solutions. 
\begin{figure}\caption{Solution sets \label{fig:solnSets}}
  \begin{tabular}{ccc}
    $-2x + y = 1$
    &
    $\begin{array}{cc} 
      x - y - z = & 0 \\
      -2x + y + 3z = & 1 
    \end{array}$    
    & 
    $
    -2x - y +  z =  2
    $
    \\
    \includegraphics[width=0.3\linewidth]{fig1} & 
    \includegraphics[width=0.3\linewidth]{fig2} & 
    \includegraphics[width=0.3\linewidth]{fig3} 
  \end{tabular}
\end{figure}
Generalizing the these three examples, we can guess that:
\begin{itemize}
\item The set of solutions to an equation with two variables,
  \begin{align*}
    a x + b y = c,
  \end{align*}
  is a line in $\mathbb{R}$.
\item The set of solutions to two equation with
  three variables,
  \begin{align*}
    a_1 x + b_1 y + c_1 z =&  d_1 \\
    a_2 x + b_2 y + c_2 z =&  d_2, \\
  \end{align*}
  is also a line in $\mathbb{R}^3$. 
\item The set of solutions to a single equation with three variables, 
  \begin{align*}
    a_1 x + b_1 y + c_1 z =&  d_1 
  \end{align*}
  is a plane in $\mathbb{R}^2$.
\end{itemize}
We will prove that these three guesses are true next week. In fact, we
will prove that when generalized to systems with more variables, these
guesses remain true. In this lecture we will just state the general
result. To do so, we need some more definitions. We will use the
definition much more when we talk about Euclidean spaces in a week or
so.
\begin{definition}
  The set $S \subseteq \mathbb{R}^n$ is called a \textbf{linear
    subspace} if it is closed under (i) scalar multiplication and (ii)
  addition in other words, if 
  \begin{itemize}
  \item[(i)] for every $(x_1, ..., x_n)\in S$ and $a \in \mathbb{R}$,
    we have $(a x_1, a x_2, ..., a x_n) \in S$, and
  \item[(ii)] for every $(x_1, ..., x_n)\in S$ and $(y_1, ..., y_n)\in
    S$, we have
    $(x_1 + y_1, ..., x_n + y_n)  \in S$
  \end{itemize}
\end{definition}
\begin{definition}
  A set of vectors in $\mathbb{R}^n$, $\{\textbf{x}_j = (x^j_1,...,
  x^j_n)\}_{j=1}^J$, is \textbf{linearly independent} if the only
  solution to 
  \begin{align*}
    \sum_{j=1}^J c_j \textbf{x}_j = 0 
  \end{align*}
  is $c_1 = c_2 = ... = c_J = 0$. 
\end{definition}
Observe that two points on a line are not linearly independent. Three
points on a plane are also not linearly independent. This suggests the
following definition.
\begin{definition}
  The dimension of a linear subspace $S \subseteq \mathbb{R}^n$ is the
  cardinality of the largest set of linearly independent elements in
  $S$. 
\end{definition} 
With this defition, a line has dimension one, a plane has dimension
two, and $\mathbb{R}^n$ has dimension $n$. We can now state a result
that summarizes everything we know about the solutions to linear
systems. 
\begin{theorem}[Rouch\'{e}-Capelli] \label{thm:rc} A system of linear
  equations with $n$ variables has a solution if and only if the rank
  of its coefficient matrix, $A$, is equal to the rank of its
  augmented matrix, $\hat{A}$. If a solution exists and $\rank A$ is
  equal to its number of columns, the solution is unique. If a
  solution exists and $\rank A$ is less than its number of columns,
  there are infinite solutions. In this case the set of solutions
  forms is of the form\footnote{A set of this form is called an affine
    subspace. It is a linear subspace that has been shifted so that it
    no longer necessarily contains the origin.}
  \[ 
  \{s+x^* \in \mathbb{R}^n : s \in S \text{ and } Ax^* = b \} 
  \]
  where $S$ is the linear subspace of dimension $n - \rank A$ defined
  by $S = \{s \in \mathbb{R}^n: As = 0 \}$ and $x^*$
  is a solution to $A x = b$.
\end{theorem}
The first part of this theorem is just a restatement of things we have
already proven. We will prove the last claim, that the set of solutions
forms a linear subspace of dimension $n - \rank A$, next week.

\subsection*{Example: Markov model of employment (continued)}
We can now answer the three questions posed at the beginning of this
lecture. 
\renewcommand{\theenumi}{\arabic{enumi}}
\begin{enumerate}
\item \emph{Does any solution exist? } \\
  A solution exists if $\rank A = \rank \hat{A}$. For this example, 
  \begin{align*}
    A = & \begin{pmatrix} 
      p_{ee} - 1 & p_{eu}  \\ 
      p_{ue} & p_{uu}-1  \\ 
      1 & 1  \end{pmatrix}.  \\
    \hat{A} = & \begin{pmatrix} 
      p_{ee} - 1 & p_{eu} & 0 \\ 
      p_{ue} & p_{uu}-1 & 0 \\ 
      1 & 1 & 1 \end{pmatrix}. 
  \end{align*}
If we perform Gaussian elimination on $A$, we get\footnote{You should
  check that the steps being performed do not involve division by
  zero.}
\begin{align*}
  \begin{pmatrix} 
    p_{ee} - 1 & p_{eu}  \\ 
    p_{ue} & p_{uu}-1  \\ 
    1 & 1  \end{pmatrix} \simeq & 
  \begin{pmatrix} 
    p_{ee} - 1 & p_{eu}  \\ 
    0 & \frac{p_{ee}- 1 - p_{eu}}{p_{ee}-1}  \\
    0 & \frac{(p_{ee}-1) (p_{uu}- 1) - p_{eu} p_{ue}}{p_{ee}-1} 
  \end{pmatrix} \\
  \simeq &  
  \begin{pmatrix} 
    p_{ee} - 1 & p_{eu}  \\ 
    0 & \frac{p_{ee}- 1 - p_{eu}}{p_{ee}-1}  \\
    0 & 0 \end{pmatrix}. \\
\end{align*}
For $\hat{A}$, we get
\begin{align*}
  \begin{pmatrix} 
    p_{ee} - 1 & p_{eu} & 0 \\ 
    1 & 1 & 1 \\
    p_{ue} & p_{uu}-1 & 0 
  \end{pmatrix} 
  \simeq & 
  \begin{pmatrix} 
    p_{ee} - 1 & p_{eu} & 0 \\ 
    0 & \frac{p_{ee}- 1 - p_{eu}}{p_{ee}-1}  & 1 \\
    0 & \frac{(p_{ee}-1) (p_{uu}- 1) - p_{eu} p_{ue}}{p_{ee}-1} & 0 
  \end{pmatrix}  \\
  \simeq & 
  \begin{pmatrix} 
    p_{ee} - 1 & p_{eu} & 0 \\ 
    0 & \frac{p_{ee}- 1 - p_{eu}}{p_{ee}-1} &  1 \\
    0 & 0 & -\frac{(p_{ee}-1) (p_{uu}- 1) - p_{eu} p_{ue}}{p_{ee}-1} 
    \frac{p_{ee} - 1}{p_{ee} - 1 - p_{eu}}.
  \end{pmatrix}
\end{align*}
and we see that the rank $\hat{A}$ will be greater than the rank $A$
of unless 
\[ 
   -\frac{(p_{ee}-1) (p_{uu}- 1) - p_{eu} p_{ue}}{p_{ee}-1} 
    \frac{p_{ee} - 1}{p_{ee} - 1 - p_{eu}} = 0 .
\]
Fortunately, if these are valid probabilities, then $p_{ee} + p_{ue} =
1$ and $p_{uu} + p_{eu} = 1$, so 
\begin{align*}
  (p_{ee}-1) (p_{uu}- 1) - p_{eu} p_{ue} = & (-p_{ue})(-p_{eu}) -
  p_{eu} p_{ue} \\
  = & 0.
\end{align*}
Thus, the system does have a solution. 
\item \emph{How many solutions exist?} \\
  From the above, we see that $\rank A = 2$ provided that $-p_{ue} -
  p_{eu} \neq 0$ and $p_{ee} - 1 \neq 0$. Therefore, there is a unique
  solution. 
\item \emph{How can a solution be computed?} \\
  By Gaussian or Gauss-Jordan elimination, like we did above.
\end{enumerate}


\end{document}
