\documentclass[12pt,reqno]{amsart}

\usepackage{amsmath}
\usepackage{amsfonts}
\usepackage{graphicx}
%\usepackage{epstopdf}
\usepackage{hyperref}
\usepackage[left=1in,right=1in,top=0.9in,bottom=0.9in]{geometry}
\usepackage{multirow}
\usepackage{verbatim}
\usepackage{fancyhdr}
%\usepackage[small,compact]{titlesec} 

%\usepackage{pxfonts}
%\usepackage{isomath}
\usepackage{mathpazo}
%\usepackage{arev} %     (Arev/Vera Sans)
%\usepackage{eulervm} %_   (Euler Math)
%\usepackage{fixmath} %  (Computer Modern)
%\usepackage{hvmath} %_   (HV-Math/Helvetica)
%\usepackage{tmmath} %_   (TM-Math/Times)
%\usepackage{cmbright}
%\usepackage{ccfonts} \usepackage[T1]{fontenc}
%\usepackage[garamond]{mathdesign}
\usepackage{color}
\usepackage{ulem}

\newtheorem{theorem}{Theorem}[section]
\newtheorem{conjecture}{Conjecture}[section]
\newtheorem{corollary}{Corollary}[section]
\newtheorem{lemma}{Lemma}[section]
\newtheorem{proposition}{Proposition}[section]
\theoremstyle{definition}
\newtheorem{assumption}{}[section]
%\renewcommand{\theassumption}{C\arabic{assumption}}
\newtheorem{definition}{Definition}[section]
\newtheorem{step}{Step}[section]
\newtheorem{remark}{Comment}[section]
\newtheorem{example}{Example}[section]
\newtheorem*{example*}{Example}

\linespread{1.1}

\pagestyle{fancy}
%\renewcommand{\sectionmark}[1]{\markright{#1}{}}
\fancyhead{}
\fancyfoot{} 
%\fancyhead[LE,LO]{\tiny{\thepage}}
\fancyhead[CE,CO]{\tiny{\rightmark}}
\fancyfoot[C]{\small{\thepage}}
\renewcommand{\headrulewidth}{0pt}
\renewcommand{\footrulewidth}{0pt}

\fancypagestyle{plain}{%
\fancyhf{} % clear all header and footer fields
\fancyfoot[C]{\small{\thepage}} % except the center
\renewcommand{\headrulewidth}{0pt}
\renewcommand{\footrulewidth}{0pt}}

\makeatletter
\renewcommand{\@maketitle}{
  \null 
  \begin{center}%
    \rule{\linewidth}{1pt} 
    {\Large \textbf{\textsc{\@title}}} \par
    {\normalsize \textsc{Paul Schrimpf}} \par
    {\normalsize \textsc{\@date}} \par
    {\small \textsc{University of British Columbia}} \par
    {\small \textsc{Economics 526}} \par
    \rule{\linewidth}{1pt} 
  \end{center}%
  \par \vskip 0.9em
}
\makeatother

\newcommand{\argmax}{\operatornamewithlimits{arg\,max}}
\newcommand{\argmin}{\operatornamewithlimits{arg\,min}}
\def\inprobLOW{\rightarrow_p}
\def\inprobHIGH{\,{\buildrel p \over \rightarrow}\,} 
\def\inprob{\,{\inprobHIGH}\,} 
\def\indist{\,{\buildrel d \over \rightarrow}\,} 
\def\F{\mathbb{F}}
\def\R{\mathbb{R}}
\newcommand{\gmatrix}[1]{\begin{pmatrix} {#1}_{11} & \cdots &
    {#1}_{1n} \\ \vdots & \ddots & \vdots \\ {#1}_{m1} & \cdots &
    {#1}_{mn} \end{pmatrix}}
\newcommand{\iprod}[2]{\left\langle {#1} , {#2} \right\rangle}
\newcommand{\norm}[1]{\left\Vert {#1} \right\Vert}
\newcommand{\abs}[1]{\left\vert {#1} \right\vert}
\renewcommand{\det}{\mathrm{det}}
\newcommand{\rank}{\mathrm{rank}}
\newcommand{\spn}{\mathrm{span}}
\newcommand{\row}{\mathrm{Row}}
\newcommand{\col}{\mathrm{Col}}
\renewcommand{\dim}{\mathrm{dim}}
\newcommand{\prefeq}{\succeq}
\newcommand{\pref}{\succ}

\title{Applications}
\date{\today}

\begin{document}

\maketitle

\section{Portfolio analysis}

This section presents a basic model of portfolio
analysis.\footnote{This example is taken from sections 6.2, 7.X and
  28.2 of Simon and Blume.} We will setup the model, define some
special types of portfolios, and then use our results on systems of
linear equations and matrices to say things about the existence of
these special types of portfolios. 

Suppose there are $A$ assets and $S$ states of nature. States of
nature are things about which we are uncertain. You could think of the
assets as various stocks and the states of nature could be all
possible combinations of prices of the stocks tomorrow. For this
example, we will assume $A$ and $S$ are finite so that everything can
be represented by matrices, but we could allow them to be infinite and
use abstract linear transformations instead.  Considering an infinite
number of assets is a bit of a stretch, but you could imagine bonds
that mature at time $t$, $t+1$, $t+2$, and so on forever. Of course,
in the real world we don't see bonds that mature arbitrarily far into
the future. On the other hand, allowing for infinite states of nature
$S$ is pretty reasonable. In the model a state of nature is anything
that affects the value of the assets. Each potential combination of
values of the assets is a single state of nature. 

We will think about a single period model. At the start of the period,
the value of asset $i \in \{1,..,A\}$ is $v_i$. Then some state of
nature, $s \in \{1, ..., S\}$, is drawn and the value of the asset
becomes $y_{si}$. The realized return of asset $i$ in state $s$ is
$R_{si} = \frac{y_{si}}{v_i}$. Let $\mathcal{R}$ be the $S$ by $A$
matrix consisting of the $R_{si}$.

The investor has wealth $w_0$ is choosing to buy $n_{i}$ units or
shares of asset $i$. The budget constraint is
\begin{align}
  \sum_{i=1}^A n_i v_i = w_0
\end{align}
Let $x_i = \frac{n_i v_i}{w_0}$ be the share of wealth in asset
$i$. We call $(x_1, ..., x_A)$ a portfolio. 
The return to a portfolio $x = (x_1,...,x_A)$ in state $s$ is
\begin{align}
  R_s = \sum_{i=1}^A \frac{y_{si}}{v_i} x_i = \sum_{i=1}^A R_{si} x_i 
\end{align}
\begin{definition}
  A portfolio is \textbf{riskless} if $x \neq 0$ and
  \begin{align*}
    \sum_{i=1}^A R_{1i} x_i  =  \cdots   =\sum_{i=1}^A R_{Si} x_i 
  \end{align*}
  or in matrix form,
  \begin{align*}
    \underbrace{\mathcal{R}}_{S \times A} \underbrace{x}_{A \times 1} =
    \underbrace{c}_{S \times 1}
  \end{align*}
  \footnote{This definition ignores the budget constraint,
    $\sum_{i=1}^A x_i = 1$. This is okay because if $x$ is riskless,
    so is $\alpha x$ for any scalar $\alpha$, and we can adjust
    $\alpha$ to meet the budget constraint.}
\end{definition}
A riskless portfolio has the same return in all states of nature.

When does a riskless portfolio exist? A riskless portfolio solves
\[ \mathcal{R} x = c \] where $c$ is an $S$ by $1$ vector with all
entries equal. We know that this equation has a solution if $ c \in
\col(\mathcal{R})$. 

If there is a riskless portfolio, then buying that portfolio ensures
the same return in all states of nature. A person may not want an
entirely riskless portfolio, but could be interested in insuring against
risk associated with a particular state. For example, suppose $s^*$ is
a state of the world where the demand for economists has collapsed. In
this state, economists' labor income will be lower. A risk averse
economist would like to insure against this outcome by buying a
portfolio that has a large return in state $s^*$. 
\begin{definition}
  A state $s^*$ is \textbf{insurable} if $\exists$ portfolio $x^*$ such
  that $\sum_{a=1}^A R_{s^*a} x_a^* > 0$ and $\sum_{a=1}^A R_{sa} x_a^* =
  0$ for all $s \neq s^*$. 
\end{definition}
If $s^*$ is insurable, then buying portfolio $x^*$ lets a person
transfer income from other states to $s^*$. 

When are state insurable? If state $s^*$ is insurable, then there must
be a solution to
\[ \mathcal{R}x = e_{s^*} \] As above, state $s^*$ is insurable if and
only if $e_{s^*} \in \col(\mathcal{R})$. If all states are insurable,
then $\col(\mathcal{R}) = \R^S$ and $\rank \mathcal{R} = S$. In
particular, if all states are insurable, then there are at least as
many assets as states. Also, in that case, we must also have a
riskless portfolio. This makes sense, if every state is insurable, we
should be able to eliminate all risk. 

From the above we saw that if $\rank \mathcal{R} = S$ then there are
portfolios that eliminate all risk. For this to be the case there must
be at least as many (linearly independent) assets as states. Is there
anything to be gained by having more assets than states? Potential
realized returns are in $\col(\mathcal{R})$. If $\col(\mathcal{R}) =
\R^S$ already, adding extra assets (which are rows) will not change 
$\col(\mathcal{R})$, so there is no change in the set of possible
outcomes. 
\begin{definition}
  A portfolio $x$ is \textbf{duplicable} if there is
  another portfolio $w$ such that $\sum_{i=1}^A x_i = \sum_{i=1}^A w_i
  $ and 
  \[ \mathcal{R} x = \mathcal{R} w. \]
\end{definition}
Duplicable portfolios satisfy\footnote{Until now, we ignored the
  budget constraint $\sum x_i = 1$, but now we include it. Why was it
  okay to ignore the budget constraint earlier? Why does it matter now?}
\begin{align*}
  \begin{pmatrix} \mathcal{R} \\ 1 \cdots 1 \end{pmatrix} x 
  = \begin{pmatrix} \mathcal{R} \\ 1 \cdots 1 \end{pmatrix} w 
\end{align*}
If $\underbrace{\widetilde{\mathcal{R}}}_{(S+1) \times A}
=   \begin{pmatrix} \mathcal{R} \\ 1 \cdots  1 \end{pmatrix} $, then
the above equation has solutions $x \neq w$ 
if and only if $\dim \mathcal{N}(\widetilde{\mathcal{R}})> 0$. This will
be the case if and only if $\rank \widetilde{\mathcal{R}} < A$. 

%%%%%%%%%%%%%%%%%%%%%%%%%%%%%%%%%%%%%%%%%%%%%%%%%%%%%%%%%%%%%%%%%%%%%%
\section{First and second welfare theorems \label{sec:welfare}}

You may have heard of the first and second welfare theorems
before. The first welfare theorem says that under some conditions,
every competitive equilibrium is Pareto efficient. The second welfare
theorem says that under some stronger conditions, every Pareto
efficient allocation can be achieved by some competitive
equilibrium. We now have (nearly) enough mathematical tools to state
these theorems very precisely and prove them in a very general
setting.  Do not worry if you find the proofs confusing. The proofs
are somewhat long, and they use continuity and convexity, which we
have not yet covered. We will be talking a lot about continuity and
related concepts next week, so this should be a good preview. In terms
of mathematics, the things that you should understand are parts where
we use properties of linear transformations, convexity, and the
separating hyperplane theorem. In terms of economics, the important
things to understand are the two theorems, the assumptions of the
theorems (and how they relate to the real world), and the general role
of the assumptions in the proofs, so you can make an informed
conjecture about what happens when the assumptions are false.


\subsection{Lines, planes, and hyperplanes }

A line in $\R^2$ splits $\R^2$ into two pieces. A plane in $\R^3$
splits it into two pieces. More generally, an $n-1$ dimensional
affine space splits $\R^n$ into two pieces.
\begin{definition}
  A \textbf{hyperplane} in $\R^n$ is an $n-1$ dimensional affine
  subspace. Equivalently, a hyperplane is the set of solutions to a
  single equation with $n$ variables.
\end{definition}
Any hyperplane can be written in the form:
\[ H_{\xi,c} = \{x: \iprod{\xi}{x} = c \} \] where $c \in \R$, $\xi \in
\R^n$, and $\norm{\xi} = 1$.  Hyperplanes play an important role in
optimization. There is one theorem, which we state here without proof,
that is especially useful.  We will use this theorem to prove the
second welfare theorem. First, a definition.
\begin{definition}
  A set $S \subseteq \R^n$ is \textbf{convex} if $\forall x_1, x_2 \in
  S$ and $\lambda \in (0,1)$, we have $x_1 \lambda + x_2(1-\lambda)
  \in S$.
\end{definition}
If a set is convex, when we draw a line between any two points in the
set, the line remains entirely within the set.  In $\R^2$, convex sets
include things shaped like triangles, squares, pentagons, circles,
ellipses, etc. Some non-convex shapes are stars, horseshoes, rings,
etc.

\begin{theorem}[Separating hyperplane theorem] \label{thm:sht}
  If $S_1$ and $S_2 \subseteq \R^n$ are convex and $S_1 \cap S_2 =
  \emptyset$ then there exists a hyperplane, $H_{\xi c} = \{ x: \xi'x
  = c \}$ such that  
  \[ \iprod{s_1}{\xi} \leq c \leq \iprod{s_2}{\xi} \]
  for all $s_1 \in S_1$ and $s_2 \in S_2$. We say that $H_{\xi,c}$
  separates $S_1$ and $S_2$. 
\end{theorem}
Visually, this theorem says that we can draw a hyperplane, $D$,
between $S_1$ and $S_2$. $H$ will then be another hyperplane
orthogonal to $D$ and the projection of $S_1$ on $H$ is disjoint from
the projection of $S_2$ on $H$. See figure \ref{fig:sht} for an
illustration in $\R^2$.

Recall that the projection of $S_1$ on $H$ is the set
\[ P_H S_1 = \{ \iprod{s_1}{\xi}\xi : s_1 \in S_1. \] The projections
are disjoint or almost disjoint)\footnote{Really this argument only
  shows disjointness if $S_1$ and $S_2$ are topologically
  closed, something that we will discuss later.} because if
$\iprod{s_1}{\xi} < \iprod{s_2}{\xi}$ $\forall s_1 \in S_1, s_2 \in
S_2$, we can never have $\iprod{s_1}{\xi}\xi = \iprod{s_2}{\xi}\xi $.
\begin{figure}\caption{Separating hyperplane \label{fig:sht}}
  \begin{centering}
    \includegraphics[width=0.8\textwidth]{separatingHyperplane}
  \end{centering}
\end{figure}
%%%%%%%%%%%%%%%%%%%%%%%%%%%%%%%%%%%%%%%%%%%%%%%%%%%%%%%%%%%%%%%%%%%%%%%%%%% 

\subsection{Setup}

We have some set of commodities, $S$, which we will assume is a normed
vector space. For example, if a world with $n$ goods, $S$ could be
$\R^n$ and for each $s = (s_1,...,s_n) \in S$, $s_j$ represents the
quantity of the $j$th good. These goods include everything that is
bought or sold, including things like food or clothing that we usually
think of as goods, and things like labor and land. There are $I$
consumers, indexed by $i$. Each consumer chooses goods from a feasible
set $X_i \subseteq S$. These $X_i$ are feasible consumption sets, not
budget sets. It is supposed to represent the physical constraints of
the world. For example if there are three goods: food, clothing, and labor
measured in days of labor per day, then $X_i$ might be $[0, \infty)
\times [0,\infty) \times [0, 1]$.  Each consumer has preferences over
$X_i$ represented by a preference relation, $\prefeq_i$ that has the
following properties:
\begin{enumerate}
\item (total) $\forall x, z \in X_i$, either $x \prefeq_i z$ or $z
  \prefeq_i x$ or both,
\item (transitive) $\forall x, w, z \in X_i$, if $x \prefeq_i w$ and $w
  \prefeq_i z$ then $x \prefeq_i z$,
\item (reflexive) $\forall x \in X_i$, $x \prefeq_i x$.
\end{enumerate}
In words, $x \prefeq_i z$ means person $i$ likes the bundle of goods $x$
as much as or more than the bundle of goods $z$. If you wish, you can
think of the preference relation coming from a utility function,
$u_i(x) : X_i \rightarrow \R$ and $x \prefeq_i z$ means $u_i(x) \geq
u_i(z)$.  If $x \prefeq_i z$ but $z \not\prefeq_i x$, then we say that
$x$ is strictly preferred to $z$ and write $x \pref_i z$. If $x
\prefeq_i z$ and $z \prefeq_i x$ we say that person $i$ is indifferent
between $x$ and $z$ and write $x \simeq_i z$.

There are also $J$ firms indexed by $j$. Each firm $j$ chooses
production $y_j$ from production possibility set $Y_j \subseteq
S$. The firm will produce positive quantities of its outputs and
negative quantities of its inputs. Continuing with the example of
three goods, if the firm produces $F^f(l)$ units of food from $l$
units of labor and $F^c(l)$ units of clothing, then production
possibility set could be written: 
\[ Y_j = \{ (f,c,l) \in S: l \leq 0 \wedge f \leq F^f(\alpha |l|)
\wedge c \leq F^c((1-\alpha) |l|) \text{ for some } \alpha \in
[0,1]\}. \] 
Firms produce goods and consumers consume goods. For the market to
\textbf{clear} we must have sum of production equal to the sum of
consumption, 
i.e. 
\begin{align*}
  \sum_{i=1}^I x_i = \sum_{j=1}^J y_j \label{eq:mc}
\end{align*}
We call the $I+J$-tuple of all $x_i$ and $y_j$, $\left( (x_1,..., x_I)
  , (y_1, ..., y_J) \right)$ (which we will sometimes shorted by just
writing $((x_i),(y_j))$) an \textbf{allocation}. An allocation is
\textbf{feasible} if $x_i \in X_i \forall i$, $y_j \in Y_j \forall j$,
and $ \sum_{i=1}^I x_i = \sum_{j=1}^J y_j$.
\begin{definition}
  An allocation, $((x_i^0),(y_j^0))$, is \textbf{Pareto efficient} (or
  Pareto optimal) if it is a feasible and there is no other feasible
  allocation, $((x_i),(y_j))$, such that $x_i \prefeq_i x_i^0$ for all
  $i$ and $x_i \pref_i x_i^0$ for some $i$.
\end{definition}
This definition is just a mathematical way of stating the usual verbal
definition of Pareto efficient. An allocation is Pareto efficient if
there is no other allocation that makes at least one person better off
and no one worse off. 

We are going to be comparing competitive equilibria to Pareto
efficient allocations. To do that we must first define a competitive
equilibrium. A price system is a continuous linear
transformation\footnote{All linear transformations on finite
  dimensional vector spaces are continuous, so matrices are always
  continuous linear transformations. There are discontinuous linear
  operators on infinite dimensional vector spaces (differentiation on
  $C^\infty$ is one example), but they are beyond the scope of this
  course. Also, they are not needed for this proof, unless you want
  allow economies with an infinite number of goods.}, $p:S \rightarrow
\R$, i.e. $p \in S^\ast$. In the case where $S = \R^n$, a price system
is just a $1 \times n$ matrix. The entries in this price matrix are
the prices of each of the $n$ goods. $px$ for $x \in S$ represents the
total expenditure needed to purchase the bundle of goods $x$.
\begin{definition}
  An allocation, $((x_i^0),(y_j^0))$, along with a price system, $p$,
  is a \textbf{competitive equilibrium} if 
  \renewcommand{\theenumi}{C\arabic{enumi}}
  \begin{enumerate}
  \item\label{c1} The allocation is feasible
  \item\label{c2} For each $i$ and $x \in X_i$ if $px \leq px_i^0$
    then $x_i^0 \prefeq_i x$,
  \item\label{c3} For each $j$ if $y \in Y_j$ then $p y \leq p y_j^0$
  \end{enumerate}
  \renewcommand{\theenumi}{\roman{enumi}}
\end{definition}
Condition \ref{c2} says that each consumer must be choosing the most
preferred bundle of goods that he or she can afford. If the preference
relation comes from a utility function, \ref{c2} says that consumers
maximize their utility given prices. Similarly, condition \ref{c3}
says that producers maximize profits.

\subsection{First welfare theorem}

The first welfare theorem requires one additional condition on
preferences.
\begin{definition}
  Preference relation $\pref_i$ has the \textbf{local non-satiation
    condition} if for each $x \in X_i$ and $\epsilon > 0$ $\exists x'
  \in X_i$ such that $\norm{x - x'} \leq \epsilon $ and $x' \pref_i x$.
\end{definition}
This condition says that given any bundle of goods you can find
another bundle very close by that is preferred. If the preference
relation comes from utility function, the utility function having a
non-zero derivative everywhere implies local non-satiation. The
intuition for why the first welfare theorem requires local
non-satiation is that local non-satiation rules out the following
scenario. Suppose person $i$ does not care about clothing at all. Then 
you take clothes away from person $i$, making person $i$ no worse off,
and give them to someone else, making that person better off. However,
there is nothing in the definition of a competitive equilibrium that
prevents person $i$ from having clothes.

\begin{theorem}[First welfare theorem]
  If $((x_i^0),(y^0_j))$ and $p$ is a competitive equilibrium and all
  consumers' preferences have the local non-satiation condition, then
  $((x^0_i),(y^0_i))$ is Pareto efficient.
\end{theorem}
\begin{proof}
  We will prove it by contradiction. Suppose that a competitive
  equilibrium is not Pareto efficient. Then there exists another
  feasible allocation\footnote{This sort of allocation is called a
    Pareto improvement.}, $((x_i),(y_j))$, such that there is at least
  one $x_{i^*} \pref_{i^*} x_{i^*}^0$. The contrapositive of condition
  \ref{c2} in the definition of competitive equilibrium implies that
  then $p x_{i^*} > p x_{i^*}^0$.  For all other $i \neq i^*$ it must
  be that $x_i \prefeq_i x_i^0$. When $x_i \pref_i x_i^0$, by the same
  argument as above, $p x_i > p x_i^0$. When $x_i \simeq_i x_i^0$,
  then we will show that local non-satiation implies $px_i \geq
  px_i^0$. If not and $px_i < p x_i^0$, then by continuity\footnote{We
    have not yet defined continuity, so do not worry if you find this
    part of the proof confusing. A function $f: V\rightarrow W$ where
    $V$ and $W$ are normed vector spaces is continuous if $\forall
    \epsilon > 0$ $\exists \delta > 0$ such that whenever $\norm{x -
      y}<\delta$ we also have $\norm{f(x) - f(y)} < \epsilon$. In this
    proof, $V$ is $S$ , $W$ is $\R$, $f$ is $p$, and $\epsilon$ is
    $|px_i - px_i^0|$. } of $p$ there exists some $\delta > 0$ such
  that for all $x'$ with $\norm{x_i - x'} < \delta$, we have
  \[ | p x_i - p x' | < | px_i - p x_i^0| \]
  and in particular, 
  \[ p x' < px_i^0. \]
  Additionally since preferences are locally non-satiated, there
  exists some $\tilde{x}$ with $\norm{x_i - \tilde{x}}<\delta$ and
  $\tilde{x} \pref_i x_i \simeq_i x_i^0$. However, then we also have
  $\tilde{x} \pref_i x_i^0$ and $p \tilde{x} < p x_i^0$, which contradicts
  $x_i^0$  and $p$ being part of a competitive equilibrium. Thus, we
  can conclude that $p x_i \geq p x_i^0$.

  At this point we have shown that if $((x_i^0),(y_j^0))$ is a
  competitive equilibrium that is not Pareto efficient, then there is
  some other allocation $((x_i),(y_j))$ that is feasible and has $x_i
  \prefeq_i x_i^0$, which implies that $p x_i \geq p x_i^0$. Each
  consumer spends (weakly) more in this alternative, Pareto improving
  allocation. Now we will show that each consumer spending at least as
  much contradicts profit maximization. The total expenditure of
  consumers in the alternate allocation must be greater than in the
  competitive equilibrium because there is one consumer who is
  spending strictly more. That is,
  \begin{align}
    \sum_{i=1}^I p x_i > \sum_{i=1}^I p x_i^0 \label{ieq:ex}
  \end{align}
  The price system is a linear transformation, so
  \[ \sum_{i=1}^I p x_i = p \left(\sum_{i=1}^I x_I \right) \]
  Both allocations are feasible, and, in particular, market clearing
  so
  \begin{align*}
    \sum_{i=1}^I x_i & = \sum_{j=1}^J y_j \\
  \end{align*}
  Applying $p$ to both sides, 
  \begin{align*}
    p\left( \sum_{i=1}^I x_i\right) & = p\left( \sum_{j=1}^J y_j
    \right) \\ 
    = & \sum_{j=1}^J p y_j.
  \end{align*}
  Identical reasoning would show that 
  \[ \sum_{i=1}^I p x_i^0 = \sum_{j=1}^J p y_j^0. \]
  Substituting into (\ref{ieq:ex}) we get
  \begin{align}
    \sum_{j=1}^J p y_j > \sum_{j=1}^J p y_j^0. \label{ieq:prof}
  \end{align}
  But this contradicts profit maximization (\ref{c3}) since $y_j \in
  Y_j$ and we cannot have (\ref{ieq:prof}) if $p y_j \leq p y_j^0$. 
  Therefore, we conclude that there can be no Pareto improvement from
  a competitive equilibrium, i.e.\ any competitive equilibrium is
  Pareto efficient.
\end{proof}

\subsection{Second welfare theorem}
The second welfare theorem is the converse of the first welfare
theorem. The second welfare theorem says that any Pareto efficient
allocation can be achieved by some competitive equilibrium. The second
welfare theorem does not hold quite as generally as the first welfare
theorem. 
\begin{definition}
  A preference relation, $\prefeq_i$, is \textbf{convex} if whenever $x
  \prefeq_i z$ and $y \prefeq_i z$, then $\lambda x + (1-\lambda) y
  \prefeq_i z$ for all $\lambda \in [0,1]$. 
\end{definition}
Alternatively, a preference relation is convex if the set $\{x\in X_i: x
\prefeq_i z\}$ is convex for each $z$. Whenever you have seen convex
indifference curves, the associated preference relation is convex. If
the preference relation is generated by a concave (more generally
quasi-concave) utility function, then the preference relation is
convex.
\begin{definition}
  A preference relation, $\prefeq_i$, is \textbf{continuous} if
  for any  $x \pref_i z$ there exists a $\delta >
  0$ such that for all $x'$ with $\norm{x - x'}<\delta$ we have $x'
  \pref_i z$.
\end{definition}
A continuous preference relation can be generated by a continuous
utility function.

\begin{theorem}[Second welfare theorem]
  Assume the preferences of each consumer are convex, locally
  non-satiated, and continuous, and that $X_i$ is convex and
  non-empty.  
  Also assume that $Y_j$ is convex and non-empty for each
  firm $j$. 

  Suppose $((x_i^e), (y_j^e))$ is a Pareto efficient allocation such
  that for any price system, $p$, there is always a cheaper bundle of
  goods, i.e.\ $\exists x_i \in X_i$ s.t. $p x_i < p x_i^e$ for each
  $i$. Then there exists a price system, $p^e$ such that $((x_i^e),
  (y_j^e))$ and $p^e$ is a competitive equilibrium.
\end{theorem}
\begin{proof}
  We are going to construct the price system by applying the
  separating hyperplane theorem. Let $V_i = \{ x \in X_i : x \pref_i
  x_i^e \}$ be the set of $x$ strictly preferred by person $i$. Let 
  \begin{align*}
    V = \{ \chi \in S: \chi = \sum_{i=1}^I x_i \text{ for some } x_i
    \in V_i \}  
  \end{align*}
  be the set of sums of elements from each $V_i$. The convexity of
  $X_i$ and the preference relation implies that $V_i$ is convex for
  each $i$. That, in turn, implies that $V$ is convex.\footnote{It
    might be a good exercise to prove these claims. } 
  Similarly, if 
  \begin{align*}
    Y = \{ \psi \in S: \psi = \sum_{i=j}^J y_j \text{
      for some } y_j \in Y_j \}  
  \end{align*}
  is the sum of each firms' production possibility set, then $Y$ is
  convex. 

  We have two convex sets. Now, we just need to show that they are
  disjoint, and then we can apply the separating hyperplane
  theorem. Suppose $\chi \in Y \cap V$. Then $\exists x_i \in V_i$ and
  $y_j \in Y_j$ such that $\chi = \sum_{i=1}^I x_i = \sum_{j=1}^J$. This
  is feasible allocation, and $x_i \pref_i x_i^e$ by
  construction. This contradicts $((x_i^e),(y_i^e))$ being Pareto
  efficient. Therefore, $Y \cap V = \emptyset$. o

  Now, by the separating hyperplane theorem, $\exists p$ such
  that\footnote{In the notation of theorem \ref{thm:sht}, $p$ is $\xi$.}
  \begin{align}
    p \chi \geq p \psi \label{ieq:p}
  \end{align}
  for all $\chi \in V$ and $\psi \in Y$. Now we need to verify
  that $((x_i^e),(y_j^e))$ with $p$ is a competitive equilibrium. It
  is feasible because $((x_i^e),(y_j^e))$ is Pareto efficient, and
  feasible by definition.
  
  We now show that (\ref{ieq:p}) holds with equality at $\chi^e =
  \sum_{i=1}^I x_i^e$ and $\psi^e = \sum_{j=1}^J y_j^e$. On the
  one hand, $\chi^e = \psi^e \in Y$, so we must have 
  \begin{align*}
    p \chi \geq p \chi^e 
  \end{align*}
  for all $\chi \in V$. On the other hand, for any$\delta > 0$, by
  local non-satiation, we can find $x_i$ such that $x_i \pref_i x_i^e$
  and $\norm{x_i - x_i^e}<\delta/I$. It follows that
  $\norm{\sum_{i=1}^I x_i - \sum_{i=1}^T x_i^e } < \delta$. $p$ is
  continuous, so
  \[
  \left\vert p\left( \sum_{i=1}^e x_i \right) - p\left(\sum_{i=1}^I
      x_i\right) \right\vert < \epsilon,
  \]
  and we can choose $\epsilon>0$ to be as small as we want. 
  Then, for any $\epsilon >0$, 
  \begin{align*}
    p \chi^e + \epsilon \geq p \psi
  \end{align*}
  for all $y \in Y$. Since this is true for any $\epsilon$, it must be
  that $p \chi^e \geq p \psi$. Therefore, we have now shown
  that
  \begin{align}
    p\chi \geq p \chi^e = p \psi^e \geq p \psi
  \end{align}  
  for all $\chi \in V$ and $\psi \in Y$. 

  It must be then also be that $p x_i \geq p x_i^e $ for each $i$ and
  all $x_i \in V_i$. If not, then there is an $\epsilon>0$ such that
  $p x_i + \epsilon < p x_i^e$, and then using local non-satiation we
  can choose $x_k$ for $k\neq i$ such that $x_k \in V_k$ and 
  \[\left| \sum_{
      k \neq i } px_k - \sum_{k \neq i } p x_k^e \right| < \epsilon/2
  \]
  and then 
  \[ \sum_{k=1}^I p x_k + \epsilon/2 < \sum_{k=1}^I p x_k^e. \]
  Similarly, we must have $p y_j^e \geq p y_j$ for all $y_j \in Y_j$,
  which proves that profit maximization, (\ref{c3}), holds. 

  We have nearly shown that utility maximization, (\ref{c2}), also
  holds. We have shown that for each $i$ if $x_i \pref_i x_i^e$ then
  $p x_i \geq p x_i^e$. To strengthen it to the form in the
  definition, we need to show that $px_i > p x_i^e$. We will use the
  continuity of preferences and the cheaper good condition. Suppose $p
  x_i = p x_i^e$ and $\exists x'_i \in X_i$ such that $p x_i' < p
  x_i^e$. Then for any $\lambda \in (0,1)$, $p(\lambda x_i'
  +(1-\lambda) x_i') < p x_i^e$. Also, by the continuity of
  preferences, for $\lambda$ close enough to $0$, $\lambda x_i +
  (1-\lambda)x_i' \pref_i x_i^e$. However, then $\lambda x_i +
  (1-\lambda) x_i' \in V_i$ contradicting $p(\lambda x_i
  +(1-\lambda) x_i') < p x_i^e$. Therefore, if the cheaper good
  exists, we must have $p x_i < p x_i^e$.   
\end{proof}


\end{document}
