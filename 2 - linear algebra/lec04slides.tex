\documentclass[compress]{beamer} 
\usepackage{amsmath}
\usepackage{amsfonts}
\usepackage{graphicx}
%\usepackage{epstopdf}
\usepackage{hyperref}
\usepackage{multirow}
\usepackage{verbatim}
%\usepackage[small,compact]{titlesec} 
%\usecolortheme{beaver}
\usetheme{Hannover}
%\usecolortheme{whale}

%\usepackage{pxfonts}
%\usepackage{isomath}
%\usepackage{mathpazo}
%\usepackage{arev} %     (Arev/Vera Sans)
%\usepackage{eulervm} %_   (Euler Math)
%\usepackage{fixmath} %  (Computer Modern)
%\usepackage{hvmath} %_   (HV-Math/Helvetica)
%\usepackage{tmmath} %_   (TM-Math/Times)
\usepackage{tgheros}
%\usepackage{cmbright}
%\usepackage{ccfonts} \usepackage[T1]{fontenc}
%\usepackage[garamond]{mathdesign}
\usepackage{color}
\usepackage{ulem}

\setbeamertemplate{navigation symbols}{}
\AtBeginSection[] % Do nothing for \section*
{ \frame{\sectionpage} }

\newcommand{\argmax}{\operatornamewithlimits{arg\,max}}
\newcommand{\argmin}{\operatornamewithlimits{arg\,min}}
\def\inprobLOW{\rightarrow_p}
\def\inprobHIGH{\,{\buildrel p \over \rightarrow}\,} 
\def\inprob{\,{\inprobHIGH}\,} 
\def\indist{\,{\buildrel d \over \rightarrow}\,} 
\def\F{\mathbb{F}}
\def\R{\mathbb{R}}
\newcommand{\gmatrix}[1]{\begin{pmatrix} {#1}_{11} & \cdots &
    {#1}_{1n} \\ \vdots & \ddots & \vdots \\ {#1}_{m1} & \cdots &
    {#1}_{mn} \end{pmatrix}}
\newcommand{\iprod}[2]{\left\langle {#1} , {#2} \right\rangle}
\newcommand{\norm}[1]{\left\Vert {#1} \right\Vert}
\newcommand{\abs}[1]{\left\vert {#1} \right\vert}
\renewcommand{\det}{\mathrm{det}}
\newcommand{\rank}{\mathrm{rank}}
\newcommand{\spn}{\mathrm{span}}
\newcommand{\row}{\mathrm{Row}}
\newcommand{\col}{\mathrm{Col}}
\renewcommand{\dim}{\mathrm{dim}}
\newcommand{\prefeq}{\succeq}
\newcommand{\pref}{\succ}


\title{Vector spaces}
\author{Paul Schrimpf}
\institute{UBC \\ Economics 526}
\date{\today}

\begin{document}

\frame{\titlepage}
\setcounter{tocdepth}{2}

\begin{frame}
  \frametitle{Overview}
  \begin{itemize} 
  \item Main idea: take intuition from $\R^3$ and apply to $\R^n$ and
    other vector spaces
  \end{itemize}
\end{frame}

\begin{frame}
  \tableofcontents  
\end{frame}

\section{Normed vector spaces}
\begin{frame}
  \frametitle{Normed vector spaces}
  \begin{itemize}
  \item Measure length or distance
  \end{itemize}
  \begin{definition}
    A \textbf{normed vector space}, $(V,\F,+,\cdot, \norm{\cdot})$, is a
    vector space with a function, called the \textbf{norm}, from $V$ to
    $\F$ and denoted by $\norm{v}$ with the following properties:
    \begin{enumerate}
    \item $\norm{v} \geq 0$ and $\norm{v} = 0$ iff $v = 0$,
    \item $\norm{\alpha v} = |\alpha|\norm{v}$ for all $\alpha \in \F$,
    \item The triangle inequality holds:
      \[ \norm{v_1+v_2} \leq \norm{v_1} + \norm{v_2} \]
      for all $v_1, v_2 \in V$.
    \end{enumerate}
  \end{definition}
  \begin{itemize}
  \item Shortest distance between two points is a straight line
  \end{itemize}
\end{frame}

\subsection{Examples}
\begin{frame}
  \frametitle{Examples}
  \begin{example}
    $\R^3$ is a normed vector space with norm
    \[ \norm{x} = \sqrt{x_1^2 + x_2^2 + x_3^2}. \] 
    This norm is exactly
    how we usually measure distance. For this reason, it is called the
    Euclidean norm.
    
    More generally, for any $n$, $\R^n$, is a normed vector space with
    norm 
    \[ \norm{x} = \sqrt{\sum_{i=1}^n x_i^2 }. \]
  \end{example}
\end{frame}

\begin{frame}
  \frametitle{Examples}
  \begin{example}
    $\R^n$ with the norm 
    \[ \norm{x}_p = \left( \sum_{i=1}^p |x_i|^p \right)^{1/p} \]
    for $p \in [1,\infty]$\footnote{Where $\norm{x}_\infty = \max_{1\leq
        i \leq n} |x_i| $} is a normed vector space. This norm is called
    the p-norm. 
  \end{example}
\end{frame}  

\begin{frame}
  \frametitle{Examples}
  \begin{example}
    $\mathcal{L}^p(0,1)$ with p-norm 
    \[ \norm{f}_p = \left(\int_0^1 |f(x)|^p dx\right)^{1/p} \] is a
    normed vector space. Moreover, $\mathcal{L}^p(0,1)$ is a different
    space for different $p$. For example, $\frac{1}{x^{1/p}} \not\in
    \mathcal{L}^p(0,1)$, but $\frac{1}{x^{1/p}} \in \mathcal{L}^q(0,1)$
    for $q < p$. 
  \end{example}
\end{frame}

\section{Inner product spaces}
\begin{frame}
  \frametitle{Inner product spaces}
  \begin{itemize}
  \item Measure angles
  \end{itemize}
  \begin{definition}
    A real \textbf{inner product space} is a vector space over the field
    $\R$ with an additional operation called the inner product that is
    function from $V \times V$ to $\mathbb{R}$. We denote the inner
    product of $v_1, v_2 \in V$ by $\iprod{v_1}{v_2}$. It has the
    following properties:
    \begin{enumerate}
    \item Symmetry: $\iprod{v_1}{v_2} = \iprod{v_2}{v_1}$
    \item Linear: $\iprod{a v_1 + b v_2}{v_3} = a \iprod{v_1}{v_3} + b
      \iprod{v_2}{v_3}$ for $a, b \in \R$
    \item Positive definite: $\iprod{v}{v} \geq 0$ and equals $0$ iff
      $v=0$. 
    \end{enumerate} 
  \end{definition}
  \begin{itemize}
  \item Inner product space is also a normed vector space
    \[ \norm{x} = \sqrt{\iprod{x}{x}}. \]
  \end{itemize}
\end{frame}

\begin{frame}
  \frametitle{Example}
  \begin{itemize}
  \item $\R^n$ with the dot product,
    \[ \iprod{x}{y} = x\cdot y = \sum_{i=1}^n x_i y_i = x^T y \] 
  \item Norm induced by the inner product is the Euclidean norm
    \[ \norm{x} = \sqrt{\iprod{x}{x}} = \sqrt{\sum_{i=1}^n x_i^2 } \]
  \end{itemize}
\end{frame}

\begin{frame}
  \frametitle{Measuring angles}
  \begin{itemize}
  \item \begin{align*}
      \norm{x + y}^2 = & \iprod{x+y}{x+y} \\
      = & \iprod{x}{x} + 2\iprod{x}{y} + \iprod{y}{y}
    \end{align*}
  \item In $\R^n$ with the Euclidean norm when $x$ and $y$ are at
    right angles 
    to one other, $\iprod{x}{y} = 0$, and we have the Pythagorean theorem:
    \[ \norm{x + y}^2 = \norm{x}^2 + \norm{y}^2. \]
  \end{itemize}
  
  \begin{definition}
    Let $x,y \in V$, an inner product space. $x$ and $y$ are
    \textbf{orthogonal} iff $\iprod{x}{y} = 0$. 
  \end{definition}
\end{frame}

\begin{frame}
  \frametitle{Measuring angles}
  \begin{theorem}\label{thm:angle}
    Let $u, v \in \R^n$, then the angle between them is
    \[ \theta = \cos^{-1} \frac{\iprod{u}{v}}{\norm{u}\norm{v}}. \]  
  \end{theorem}
  \begin{proof}
    Draw picture.
    \[ \cos \theta = \frac{\norm{tv}}{\norm{u}}. \]
    Use Pythagorean theorem,
    \begin{align*}
      \norm{u}^2 = & \norm{tv}^2 + \norm{u - tv}^2 \\
      \norm{u}^2 = & t^2 \norm{v}^2 + \norm{u}^2 - 2t\iprod{u}{v} + t^2
      \norm{v}^2 
      \\
      2 t \iprod{u}{v} = & 2 t^2 \norm{v}^2 
    \end{align*}
  \end{proof}
\end{frame}

\subsection{Useful inequalities}

\begin{frame}
  \frametitle{Useful inequalities}
  \begin{itemize}
  \item Triangle inequality
  \end{itemize}
  \begin{theorem}[Reverse triangle inequality]
    Let $V$ be a normed vector space and $x,y \in V$. Then
    \[ \left| \norm{x} - \norm{y} \right| \leq \norm{x-y}. \]
  \end{theorem}
  \begin{proof}
    By the usual triangle inequality,
    \begin{align*}
      \norm{x} + \norm{x-y} \geq & \norm{y} \\
      \norm{x-y} \geq \norm{y} - \norm{x}
    \end{align*}
    and
    \begin{align*}
      \norm{y} + \norm{y-x} \geq & \norm{x} \\
      \norm{y-x} \geq \norm{x} - \norm{y}.
    \end{align*}
  \end{proof}
\end{frame}
\begin{frame}
  \frametitle{Useful inequalities}
  
  \begin{theorem}[Cauchy-Schwarz inequality \label{thm:cauchy-schwarz}]
    Let $V$ be an inner product space and let $u,v\in V$. Then,
    \[ \left\vert \iprod{u}{v} \right\vert \leq \norm{u}\norm{v}. \]
  \end{theorem}
  \begin{proof}
    Setup as before, we can show that $t =
    \frac{\iprod{u}{v}}{\norm{v}^2}$. Now, let $z = u-tv$. By the
    Pythagorean theorem,
    \begin{align*}
      \norm{u}^2 = & \norm{tv}^2 + \norm{z}^2 \\
      = & \frac{\iprod{u}{v}^2}{\norm{v}^2} + \norm{z}^2 
    \end{align*}
    $\norm{z}^2\geq 0$, so $
    \norm{u}^2 \geq \frac{\iprod{u}{v}^2}{\norm{v}^2} 
    \norm{u}\norm{v} \geq |\iprod{u}{v}|.$
  \end{proof}
\end{frame}

\section{Projections}

\begin{frame}
  \frametitle{Projections}
  \begin{definition}
    Let $V$ be an inner product space and $x,y \in V$. The
    \textbf{projection} of $y$ onto $x$ is 
    \[ P_x y = \frac{\iprod{y}{x}}{\norm{x}^2} x. \]
    More generally, the projection of $y$ onto a finite set $\{x_1, x_2,
    ... , x_k\}$ is 
    \[ P_{\{x_j\}_j=1^k} y = \sum_{j=1}^k P_{x_j - \sum_{i=1}^{j-1}
      P_{x_i} x_j} y. \]
  \end{definition}
\end{frame}

\begin{frame}
  \frametitle{Projections}
  \begin{definition}
    More generally still, if $X \subseteq V$ is a linear subspace, then
    the projection of $y$ onto $X$ is
    \[ P_{X} y = P_{\{b_j\}_{j=1}^k} y \]    
    where $b_j \in X$ and $b_1, ..., b_k$ span $X$. 
    
    Finally, if $Y \subseteq V$ the projection of $Y$ onto $X$ is just
    the set consisting of the projection of each element of $y$ onto
    $X$, i.e.
    \[ P_{X} Y = \{ P_{X} y : y \in Y \}. \]
  \end{definition}
\end{frame}

\begin{frame}
  \frametitle{Projections}
  \begin{lemma}
    Any projection is an idempotent linear transformation. 
  \end{lemma}
  \begin{proof}
    \begin{itemize}
    \item Verify that projections have the two properties required
      for them to be linear transformations.
    \item Show that projections are idempotent. 
      \begin{align*}
        P_x (P_x y) = & P_x\left(\frac{\iprod{x}{y}}{\norm{x}^2} x \right)
        \\
        = & \frac{\iprod{x}{\frac{\iprod{x}{y}}{\norm{x}^2}
            x}}{\norm{x}^2} x \\ 
        = & \frac{\iprod{x}{y}}{\norm{x}^2}
        \frac{\iprod{x}{x}}{\norm{x}^2} x \\
        = & P_x y.
      \end{align*}
    \end{itemize}
  \end{proof}
\end{frame}

\section{Row, column, and null space}

\subsection{Row space}
\begin{frame}[shrink]
  \frametitle{Row space}
  \begin{definition}
    Let $A$ be an $m$ by $n$ matrix. The \textbf{row space} of $A$,
    denoted $\row (A)$, is the space spanned by the row vectors of $A$.
  \end{definition}
  \begin{itemize}
  \item $\row(A) \subseteq \R^n$
  \end{itemize}
  \begin{lemma}
    Performing Gaussian elimination does not change the row space of a
    matrix. 
  \end{lemma}
  \begin{proof}
    Let $a_{1},...,a_m$ be the row vectors of $A$. Each step of Gaussian
    elimination transforms some $a_j$ into $a_j + g a_k$ with $k \neq j$
    or $g \neq -1$. Can show that
    \[ \spn(a_1,...,a_m) = \spn(a_1 + g a_k , ..., a_m). \]
  \end{proof}
  \begin{corollary}\label{cor:rankdimrow}
    The dimension of the row space of a matrix is equal to its rank.
  \end{corollary}
\end{frame}

\subsection{Column space}

\begin{frame}
  \frametitle{Column space}
  \begin{definition}
    Let $A$ be an $m$ by $n$ matrix. The \textbf{column space} of $A$,
    denoted $\col(A)$, is the space spanned by the column vectors of
    $A$.
  \end{definition}
  \begin{itemize}
  \item $\col{A} \subseteq \R^m$
  \end{itemize}
  \begin{lemma}
    Let $A$ be an $m$ by $n$ matrix. Then $A x = b$ has a solution iff $b
    \in \col(A)$.
  \end{lemma}
\end{frame}

\begin{frame}
  \begin{definition}
    A column of a matrix, $A$, is \textbf{basic} if the corresponding
    column of the row echelon form, $A_r$, contains a pivot.
  \end{definition}
  \begin{theorem}
    The basic columns of $A$ form a basis for $\col(A)$. 
  \end{theorem}
  \begin{proof}
    Let $A$ be $m \times n$ and denote its columns as $v_1,...,v_n$. Let
    $A_r$ be the row echelon form of $A$ and denotes its columns as
    $w_1,...,w_n$. Let $w_{i_1}, ..., w_{i_k}$ be the basic columns of
    $A_r$. Each has more zeros, so $w_{i_1},...,w_{i_k}$
    are linearly independent. By definition of row echelon form, the
    final $m-k$ rows of $A_r$ are all zero. Therefore $\dim \col(A_r) \leq
    k$, and $w_{i_1},...,w_{i_k}$ must be a basis for $\col(A_r)$.
  \end{proof}    
\end{frame}
\begin{frame}
  \begin{proof}[Continued]
    Now we show that $v_{i_1}, ..., v_{i_k}$ are a basis for
    $\col(A)$. Suppose
    \[ c_1 v_{i_1} + ... + c_k v_{i_k} = 0. \]
    Then we could do Gaussian elimination to convert this system to
    \[ c_1 w_{i_1} + ... + c_k w_{i_k} = 0. \]
    $w_{i_1},... , w_{i_k}$ are linearly independent so $c_1 = 0
    ,... c_k = 0$.
    
    Add any other $v_j$, $j \not\in \{i_1, ..., i_k\}$,
    then by the same argument there must exist a non-zero $c$ than
    solves 
    \[ c_1 v_{i_1} + ... + c_k v_{i_k} + c_j v_j = 0. \]
    Thus, $v_{i_1} , ..., v_{i_k}$ is a basis for $\col(A)$. 
  \end{proof}
\end{frame}

\begin{frame}
  \begin{corollary}
    The dimensions of the row and column spaces of any matrix are
    equal. 
  \end{corollary}
  \begin{corollary}
    $\rank A = \rank A^T$.
  \end{corollary}
\end{frame}

\subsection{Null space}
\begin{frame}
  \frametitle{Null space}
  \begin{definition}
    Let $A$ be $m$ by $n$. The set of solutions to the homogeneous
    equation $Ax = 0$ is the \textbf{null space} (or kernel) of
    $A$, denoted by $\mathcal{N}(A)$ (or $\mathrm{Null}(A)$).
  \end{definition}
  
  \begin{definition}
    Let $V \subseteq \R^n$ be a linear subspace, and let $c \in \R^n$ be
    a fixed vector. The set
    \[ \{ x \in \R^n: x = v + c \text{ for some } v \in V \} \]
    is called the set of \textbf{translates} of $V$ by $c$, and is
    denoted $c + V$. Any set of
    translates of a linear subspace is called an \textbf{affine} space. 
  \end{definition}
\end{frame}

\begin{frame}
  \begin{lemma}
    Let $A x = b$ be an $m$ by $n$ system of linear equations. Let $x_0$
    be any particular solution. Then the set of solutions is $x_0 +
    \mathcal{N}(A)$. 
  \end{lemma}
  \begin{proof}
    Let $w \in x_0 + \mathcal{N}(A)$. Then 
    \begin{align*}
      Aw = & A(x_0) + A(\underbrace{w-x_0}_{\in \mathcal{N}(A)}) \\
      = & b + 0.
    \end{align*}
    Let $w$ be a solution to $Ax = b$. Then
    \begin{align*}
      A(w - x_0) = Aw - Ax_0 = 0
    \end{align*}
    so $w - x_0 \in \mathcal{N}(A)$ and $w \in x_0 + \mathcal{N}(A)$. 
  \end{proof}
\end{frame}

\begin{frame}
  \begin{theorem}
    Let $A$ be an $m$ by $n$ matrix. Then 
    $\dim \mathcal{N}(A) = n - \rank A$
  \end{theorem}
  \begin{proof}
    \begin{itemize}
    \item Let $u_1, ..., u_k$ be a basis for $\mathcal{N}(A)$. We can add
      $u_{k+1},...,u_n$ to $u_1, ..., u_k$ to form a basis for
      $\R^n$.
    \item Show that  $A u_{k+1}, ..., A u_n$ are a basis for the
      column space
      \begin{itemize}
      \item linearly independent 
      \item span $\col A$.
      \end{itemize}
    \end{itemize}
  \end{proof}
\end{frame}

\begin{frame}
  \begin{theorem}[Rouch\'{e}-Capelli] \label{thm:rc} A system of linear
    equations with $n$ variables has a solution if and only if the rank
    of its coefficient matrix, $A$, is equal to the rank of its
    augmented matrix, $\hat{A}$. Equivalently, a solution exists if and
    only if $b \in \col(A)$.
    
    If a solution exists and $\rank A$ is equal to its number of
    columns, the solution is unique. If a solution exists and $\rank A$
    is less than its number of columns, there are infinite solutions. In
    this case the set of solutions forms is $x_0 + \mathcal{N}(A)$,
    where $x_0$ is any particular solution to $A x = b$. This set of
    solutions is an affine subspace of dimension $n - \rank A$.
  \end{theorem}
\end{frame}

\begin{frame}\frametitle{Relationship among row, column, and null
    spaces}
  \begin{itemize}
  \item $\col(A) = \row(A^T) \subseteq \R^m$
  \item $\row(A) = \col(A^T) \subseteq \R^n$
  \item $\mathcal{N}(A) \subseteq \R^n$ and $\mathcal{N}(A^T)
    \subseteq \R^m$
  \item Let $x \in \mathcal{N}(A)$, $y \in \row(A)$, what is
    $\iprod{x}{y}$? 
  \end{itemize}
\end{frame}

\begin{frame}[shrink]
  \frametitle{Relationship among row, column, and null
    spaces}
  \begin{itemize}
  \item If $x \in \mathcal{N}(A)$, $y \in \row(A) = \col(A^T)$, then
    $\iprod{x}{y} = 0$. 
  \item $\mathcal{N}(A)$ and $\row(A)$ are orthogonal subspaces of $\R^n$
  \item If $x \in \mathcal{N}(A^T)$, $y \in \row(A^T) = \col(A)$, then
    $\iprod{x}{y} = 0$. 
  \item $\mathcal{N}(A^T)$ and $\col(A)$ are orthogonal subspaces of
    $\R^m$
  \item $\forall x \in \R^n$, 
    \begin{align*}
      A x = & A (P_{\row{A}}x + P_{\mathcal{N}(A)} x)  \\
      = &  A (P_{\row{A}}x) + A (P_{\mathcal{N}(A)} x)  \\
      = & A (P_{\row{A}})x \in \col(A) = \row(A^T)
    \end{align*}
  \item $\forall w \in \R^m$,
    \begin{align*}
      A^T w = & A^T (P_{\col{A}}w + P_{\mathcal{N}(A^T)} w)  \\
      = &  A^T (P_{\col{A}}w) + A^T (P_{\mathcal{N}(A^T)} w) \\
      = & A^T (P_{\col{A}})w \in \row(A) = \col(A^T)
    \end{align*}    
  \end{itemize}
\end{frame}


\section{Applications}

\subsection{Portfolio analysis}

\begin{frame}
  \frametitle{Portfolio analysis}
  \begin{itemize}
  \item Model of portfolio analysis taken from Simon and Blume
  \item Good way of reviewing results about matrices and systems of
    linear equations 
  \end{itemize}
\end{frame}

\begin{frame}
  \frametitle{Setup}
  \begin{itemize}
  \item $A$ assets
  \item $S$ states of nature
  \item Initial value of $i = v_i$
  \item After state revealed, value $y_{si}$
  \item Realized return $R_{si} = \frac{y_{si}}{v_i}$
  \item Matrix of returns $\underbrace{\mathcal{R}}_{S \times A}$
  \end{itemize}
\end{frame}

\begin{frame}
  \frametitle{Setup}
  \begin{itemize}
  \item Choosing budget shares $x_i$, $\sum_{i=1}^A x_i = 1$ (but
    individual shares could be positive or negative), call $x =
    (x_1,...,x_A)$ a portfolio
  \end{itemize}
\end{frame}

\begin{frame} \frametitle{Riskless portfolios}
  \begin{definition}
    A portfolio is \textbf{riskless} if 
    \begin{align*}
      \sum_{i=1}^A R_{1i} x_i  =  \cdots   =\sum_{i=1}^A R_{Si} x_i 
    \end{align*}
  \end{definition}
  \begin{itemize}
  \item Solution to $\mathbf{c} = (c, ..., c)$
    \begin{align*}
      \underbrace{\mathcal{R}}_{S \times A} \underbrace{x}_{A \times 1} =
      \underbrace{\mathbf{c}}_{S \times 1}
    \end{align*}
  \item Exists if $\mathbf{c} \in \col(\mathcal{R})$
  \end{itemize}
\end{frame}

\begin{frame}\frametitle{Insurable states}
  \begin{definition}
    A state $s^*$ is \textbf{insurable} if $\exists$ portfolio $x$ such
    that $\sum_{a=1}^A R_{s^*a} x_a > 0$ and $\sum_{a=1}^A R_{sa} x_a =
    0$ for all $s \neq s^*$. 
  \end{definition}
  \begin{itemize}
  \item $s^*$ insurable iff $\exists$ solution to
    \[ \mathcal{R}x = e_{s^*} \] 
  \item So if $e_{s^*} \in \col(\mathcal{R})$
  \item All states insurable iff $\col(\mathcal{R}) = \R^S$
  \end{itemize}
\end{frame}

\begin{frame}
  \frametitle{Duplicable portfolios}
  \begin{definition}
    A portfolio $x$ is \textbf{duplicable} if there is
    another portfolio $w$ such that $\sum_{i=1}^A x_i = \sum_{i=1}^A w_i
    $ and 
    \[ \mathcal{R} x = \mathcal{R} w. \]
  \end{definition}
  \begin{itemize}
  \item i.e.\  multiple solutions to
    \begin{align*}
      \begin{pmatrix} \mathcal{R} \\ 1 \cdots 1 \end{pmatrix} x 
      = & \begin{pmatrix} \mathcal{R} \\ 1 \cdots 1 \end{pmatrix} w
      \\
      \widetilde{\mathcal{R}} x = \widetilde{\mathcal{R}} w
    \end{align*}
    if and only if $\dim \mathcal{N}(\widetilde{\mathcal{R}})> 0$ if and
    only if $\rank \widetilde{\mathcal{R}} < A$
  \end{itemize}
\end{frame}

\subsection{First and second welfare theorems}

\begin{frame}
  \frametitle{Welfare theorems}
  \begin{itemize}
  \item[1st:] Every competitive equilibrium is Pareto efficient (under
    some conditions)
  \item[2nd:] Every Pareto efficient allocation is a competitive
    equilibrium (under some stronger conditions)
  \item We will
    \begin{itemize}
    \item Carefully define competitive equilibrium and Pareto
      efficiency
    \item State and prove theorems
    \end{itemize}
  \end{itemize}
\end{frame}

\begin{frame}
  \frametitle{Welfare theorems}
  \begin{itemize}
  \item Proofs will use:
    \begin{itemize}
    \item Vector spaces, linear transformations
      \begin{itemize}
      \item Should mostly understand
      \end{itemize}
    \item Convexity, separating hyperplane theorem
    \item Continuity --- $\forall \epsilon > 0 \exists \delta > 0$
      s.t. ...
      \begin{itemize}
      \item Maybe difficult to follow, but we will see a lot more of
        this in calculus
      \end{itemize}
    \end{itemize}
  \item Economic issues to think about:
    \begin{itemize}
    \item The result of the two theorems, their assumptions, and how
      they relate to the real world
    \item General role of assumptions in proofs, so can conjecture
      what happens when assumptions are false 
    \end{itemize}
  \end{itemize}
\end{frame}

\subsubsection{Lines, planes, and hyperplanes }

\begin{frame}
  \frametitle{Lines, planes, and hyperplanes}
  \begin{definition}
    A \textbf{hyperplane} in $\R^n$ is an $n-1$ dimensional affine
    subspace. Equivalently, a hyperplane is the set of solutions to a
    single equation with $n$ variables.
  \end{definition}
  Any hyperplane can be written in the form:
  \[ H_{\xi,c} = \{x: \iprod{\xi}{x} = c \} \] where $c \in \R$, $\xi \in
  \R^n$, and $\norm{\xi} = 1$.
\end{frame}

\begin{frame}
  \frametitle{Separating hyperplanes}
  \begin{definition}
    A set $S \subseteq \R^n$ is convex if $\forall x_1, x_2 \in S$ and
    $\lambda \in (0,1)$, we have $x_1 \lambda + x_2(1-\lambda) \in S$.  
  \end{definition}
  \begin{theorem}[Separating hyperplane theorem]
    If $S_1$ and $S_2 \subseteq \R^n$ are convex. If $S_1 \cap S_2 =
    \emptyset$ then there exists a hyperplane, $H_{\xi c} = \{ x:
    \xi'x = c \}$ such that
    \[ \iprod{s_1}{\xi} \leq c \leq \iprod{s_2}{\xi} \]
    for all $s_1 \in S_1$ and $s_2 \in S_2$. We say that $H_{\xi,c}$
    separates $S_1$ and $S_2$. 
  \end{theorem}
  \begin{itemize}
  \item Draw picture
  \end{itemize}
\end{frame}

\begin{frame}
  \begin{centering}
    \includegraphics[width=\linewidth]{separatingHyperplane}
  \end{centering}
\end{frame}
%%%%%%%%%%%%%%%%%%%%%%%%%%%%%%%%%%%%%%%%%%%%%%%%%%%%%%%%%%%%%%%%%%%%%%%%% 


%\subsubsection*{Setup}

\begin{frame}
  \frametitle{Setup}
  \begin{itemize}
  \item Commodities $s \in S$, a normed vector space (e.g. $S$ linear
    subspace of $\R^n$)
  \item Consumers $i=1,..., I$ with consumption possibility set $X_i
    \subseteq S$
    (think of as physical constraint, not budget constraint)
  \item Preference relation $\prefeq_i$ s.t.
    \begin{enumerate}
    \item (total) $\forall x, z \in X_i$, either $x \prefeq_i z$ or $z
      \prefeq_i x$ or both,
    \item (transitive) $\forall x, w, z \in X_i$, if $x \prefeq_i w$ and $w
      \prefeq_i z$ then $x \prefeq_i z$,
    \item (reflexive) $\forall x \in X_i$, $x \prefeq_i x$.
    \end{enumerate}
    Denote strict preference $\pref_i$, and indifference $\simeq_i$
    \begin{itemize}
    \item Could come from  utility function i.e.\ $x \prefeq_i z$ iff
      $u_i(x) \geq u_i(z)$, but does not have to
    \end{itemize}    
  \end{itemize}
\end{frame}

\begin{frame}
  \frametitle{Setup}
  \begin{itemize}
  \item Firms $j=1,...,J$ produce $y_j$ from within production
    possibility sets $Y_j \subseteq S$
    \begin{itemize}
    \item Example: 3 commodities, food $f$, clothing $c$, and labor
      $l$. Production function $f = F^f(l)$, and $c = F^c(l)$,
      \[ Y_j = \{ (f,c,l) \in S: l \leq 0 \wedge f \leq F^f(\alpha |l|)
      \wedge c \leq F^c((1-\alpha) |l|) \text{ for some } \alpha \in
      [0,1]\}. \] 
    \end{itemize}
  \item $I+J$ tuple $((x_i),(y_j))$ is an \textbf{allocation} 
  \item Allocation is feasible if $x_i \in X_i \forall i$, $y_j \in
    Y_j \forall j$, and the market clears, 
    $\sum_{i=1}^I x_i = \sum_{j=1}^J y_j$.
  \end{itemize}
\end{frame}  

\begin{frame}
  \begin{definition}
    An allocation, $((x_i^0),(y_j^0))$, is \textbf{Pareto efficient}
    if it is a feasible and there is no other feasible allocation,
    $((x_i),(y_j))$, such that $x_i \prefeq_i x_i^0$ for all $i$ and
    $x_i \pref_i x_i^0$ for some $i$.
  \end{definition}

  \begin{definition}
    A price system is a continuous linear transformation $p:S
    \rightarrow \R$. (If $S$ is $n$-dimensional then $p$ is any $1
    \times n$ matrix)
  \end{definition}
\end{frame}

\begin{frame}
  \begin{definition}
    An allocation, $((x_i^0),(y_j^0))$, along with a price system, $p$,
    is a \textbf{competitive equilibrium} if 
    \renewcommand{\theenumi}{C\arabic{enumi}}
    \begin{enumerate}
    \item\label{c1} The allocation is feasible
    \item\label{c2} (Utility maximization) For each $i$ and $x \in X_i$
      if $px \leq px_i^0$ 
      then $x_i^0 \prefeq_i x$, 
    \item\label{c3} (Profit maximization) For each $j$ if $y \in Y_j$
      then $p y \leq p y_j^0$ 
    \end{enumerate}
    \renewcommand{\theenumi}{\roman{enumi}}
  \end{definition}
\end{frame}

%\subsubsection*{First welfare theorem}

\begin{frame}
  \frametitle{First welfare theorem}
  \begin{definition}
    Preference relation $\pref_i$ has the \textbf{local non-satiation
      condition} if for each $x \in X_i$ and $\epsilon > 0$ $\exists x'
    \in X_i$ such that $\norm{x - x'} \leq \epsilon $ and $x' \pref_i x$.
  \end{definition}
  
  \begin{theorem}[First welfare theorem]
    If $((x_i^0),(y^0_j))$ and $p$ is a competitive equilibrium and all
    consumers' preferences have the local non-satiation condition, then
    $((x_i^0),(y_i^0))$ is Pareto efficient.
  \end{theorem}
\end{frame}  

\begin{frame}
  \frametitle{Proof of first welfare theorem}
  \begin{itemize}
  \item Will show contradiction
  \item Suppose competitive equilibrium not Pareto efficient, then
    $\exists$ a Pareto improvement $((x_i),(y_j))$ another feasible
    allocation s.t.
    \begin{itemize}
    \item $\exists $ at least one $i^*$ with $x_{i^*} \pref_i x_i^0$
      \begin{itemize}
      \item The contrapositive of \ref{c2} implies $p x_{i^*} > P
        x_i^0$
      \end{itemize}
    \item All other $x_i \prefeq_i x_i^0$ 
      \begin{itemize}
      \item Local non-satiation and continuity of $p$ imply $p x_i
        \geq p x_i^0$  (next slide)
      \end{itemize}
    \end{itemize}
  \end{itemize}
\end{frame}

\begin{frame}
  \frametitle{Proof of first welfare theorem}
  \begin{itemize}
  \item Local non-satiation and continuity of $p$ imply $p x_i
    \geq p x_i^0$  
    \begin{itemize}
    \item If $p x_i < p x_i^0$, then local non-satiation implies
      $\exists$ $x_i'$ very close to $x_i$ such that $x_i' \pref_i x_i
      \prefeq_i x_i^0$. 
    \item $p$ continuous means that if $x_i'$ is close enough to
      $x_i$, then $p x_i'$ will be close enough to $p x_i$ that 
      $p x_i' < p x_i^0$ as well. 
    \item But $x_i' \pref_i x_i$ and $p x_i' < p x_i^0$ contradicts
      the definition of competitive equilibrium.
    \end{itemize}
  \item So in this Pareto improvement, we have
    \begin{align}
     \sum_{i=1}^I p x_i > \sum_{i=1}^I p x_i^0
    \end{align}
    next, we will show this contradicts profit maximization
  \end{itemize}
\end{frame}

\begin{frame}
  \frametitle{Proof of first welfare theorem}
  \begin{itemize}
  \item Start with market clearing, apply $p$, rearrange:
    \begin{align}
      \sum_{i=1}^I x_i & = \sum_{j=1}^J y_j  \\
      p(\sum_{i=1}^I x_i) =  \sum_{i=1}^I p x_i & 
      = \sum_{j=1}^J p y_j  = p(\sum_{j=1}^J y_j)  \\ 
    \end{align}
  \item Same thing for $x_i^0$ and $y_j^0$ with inequality from last
    slide gives
    \begin{align}
      \sum_{j=1}^J p y_j > \sum_{j=1}^J p y_j^0 
    \end{align}
    But then there is at least on $j$ with $p y_j > p y_j^0$,
    contradicting \ref{c3}
  \end{itemize}  
\end{frame}

%\subsubsection*{Second welfare theorem}
\begin{frame}
  \frametitle{Second welfare theorem}
  \begin{definition}
    A preference relation, $\prefeq_i$, is \textbf{convex} if whenever $x
    \prefeq_i z$ and $y \prefeq_i z$, then $\lambda x + (1-\lambda) y
    \prefeq_i z$ for all $\lambda \in [0,1]$. 
  \end{definition}
  
  \begin{definition}
    A preference relation, $\prefeq_i$, is \textbf{continuous} if
    for any  $x \pref_i z$ there exists a $\delta >
    0$ such that for all $x'$ with $\norm{x - x'}<\delta$ we have $x'
    \pref_i z$.
  \end{definition}
\end{frame}

\begin{frame}
  \frametitle{Second welfare theorem}
  \begin{theorem}[Second welfare theorem]
    Assume the preferences of each consumer are convex, locally
    non-satiated, and continuous, and that $X_i$ is convex and
    non-empty.  
    Also assume that $Y_j$ is convex and non-empty for each
    firm $j$. 
    
    Suppose $((x_i^e), (y_j^e))$ is a Pareto efficient allocation such
    that for any price system, $p$, there is always a cheaper bundle of
    goods, i.e.\ $\exists x_i \in X_i$ s.t. $p x_i < p x_i^e$ for each
    $i$. Then there exists a price system, $p^e$ such that $((x_i^e),
    (y_j^e))$ and $p^e$ is a competitive equilibrium.
  \end{theorem}  
\end{frame}

\begin{frame}
  Compared to first welfare theorem three more assumptions:
  \begin{enumerate}
  \item Convexity of preferences, consumption possibility sets, and
    production possibility sets
  \item Continuity of preferences
  \item Cheaper good condition
  \end{enumerate}
\end{frame}

\begin{frame}
  \frametitle{Proof of second welfare theorem}
  \begin{itemize}
  \item Price system will come from separating hyperplane theorem,
    first need two disjoint, convex sets
  \item Let $V_i = \{ x \in X_i : x \pref_i
    x_i^e \}$, set sum of $V_i$ be
    \begin{align*}
      V = \{ \chi \in S: \chi = \sum_{i=1}^I x_i \text{ for some } x_i
      \in V_i \}  
    \end{align*}
    $V_i$ convex, so $V$ convex
  \item Let 
    \begin{align*}
      Y = \{ \psi \in S: \psi = \sum_{i=j}^J y_j \text{
        for some } y_j \in Y_j \}  
    \end{align*}
    also convex.
  \end{itemize}
\end{frame}

\begin{frame}
  \frametitle{Proof of second welfare theorem}
  \begin{itemize}
  \item Show $Y$ and $V$ disjoint
    \begin{itemize}
    \item Suppose $\chi \in Y \cap V$. Then $\exists x_i \in V_i$ and
      $y_j \in Y_j$ such that $\chi = \sum_{i=1}^I x_i =
      \sum_{j=1}^J$, is feasible and $x_i \pref_i x_i^e$
    \item Contradicts $((x_i^e),(y_i^e))$ being Pareto efficient
    \end{itemize}
  \item Separating hyperplane theorem $\Rightarrow$ $\exists p$ s.t.
    \begin{align}
      p \chi \geq p \psi \label{ieq:p}
    \end{align}
  \item Next, show $((x_i^e), (y_j^e))$ with $p$ is a competitive
    equilibrium 
  \end{itemize}
\end{frame}

\begin{frame}
  \frametitle{Proof of second welfare theorem}
  \begin{itemize}
  \item Show $p \chi\geq p \chi^e = p \psi^e \geq p
    \psi$ for all $\chi \in V$, $\psi \in Y$
    \begin{itemize}
    \item $\psi^e \in Y$, so $p \chi \geq p \psi^e$
      $\forall \chi \in V$.
    \item $\chi^e = \sum x_i^e = \sum y_j^e = \psi^e$, so 
      \begin{align} p \chi \geq p \chi^e = p \psi^e \end{align} 
    \item Local non-satiation implies we can find $x_i' \pref_i x_i^e$
      very close to $x_i^e$. Continuity of $p$ means that for $x_i'$
      close enough we have $|p \chi^e - p\chi'| < \epsilon$ for any
      $\epsilon >0$. Therefore $\forall \psi \in Y$
      \begin{align}
        p\chi^e + \epsilon > p\chi' \geq p \psi
      \end{align}
      so $p\chi^e \geq p \psi$
    \end{itemize}  
  \end{itemize}
\end{frame}

\begin{frame}
  \frametitle{Proof of second welfare theorem}
  \begin{itemize}
  \item Show $p x_i \geq p x_i^e $ for each $i$ and
    all $x_i \in V_i$ 
    \begin{itemize}
    \item If not there $p x_i + \epsilon < p x_i^e$  with $x_i \in
      V_i$. 
    \item Local non-satiation and continuity of $p$ means we can
      choose other $x_k \in V_k$ such that $\left\vert \sum_{k\neq i}
        (p x_k - p x_k^e) \right\vert < \epsilon / 2$
    \item But then $p \chi < p \chi^e$ contradicting the previous
      slide 
    \end{itemize}
  \item Same reasoning but using convexity of $Y_j$ instead of local
    non-satiation means $p y_j \leq y_j^e$ for each $j$ and all $y_j
    \in Y_j$ i.e. firms are profit maximizing, \ref{c3} holds
  \end{itemize}
\end{frame}

\begin{frame}
  \frametitle{Proof of second welfare theorem}
  \begin{itemize}
  \item Have shown: if $x_i \pref_i x_i^e$ then  $p x_i \geq p x_i^e$
  \item Need: if $x_i \pref_i x_i^e$ then  $p x_i > p x_i^e$
  \item Use cheaper good condition
    \begin{itemize}
    \item If $p x_i = p x_i^e$ and $\exists x_i' \in X_i$ s.t. $p x_i'
      < p x_i^e$, then for all $\lambda \in (0,1)$,
      \begin{align}
        p(\lambda x_i +(1-\lambda) x_i') < p x_i^e
      \end{align}
    \item Continuous preferences implies for $\lambda$ close to $0$, 
      \begin{align}
        \lambda x_i + (1-\lambda)x_i' \pref_i x_i^e      
      \end{align}
      so $\lambda x_i + (1-\lambda) x_i' \in V_i$, but that
      contradicts what we have shown.
    \end{itemize}
  \end{itemize}
\end{frame}

\end{document}
