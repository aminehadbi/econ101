\documentclass[12pt,reqno]{amsart}
\usepackage{amsmath}
\usepackage{amsfonts}
\usepackage{graphicx}
%\usepackage{epstopdf}
\usepackage{hyperref}
\usepackage[left=1in,right=1in,top=0.9in,bottom=0.9in]{geometry}
\usepackage{multirow}
\usepackage{verbatim}
\usepackage{fancyhdr}
%\usepackage[small,compact]{titlesec} 

%\usepackage{pxfonts}
%\usepackage{isomath}
\usepackage{mathpazo}
%\usepackage{arev} %     (Arev/Vera Sans)
%\usepackage{eulervm} %_   (Euler Math)
%\usepackage{fixmath} %  (Computer Modern)
%\usepackage{hvmath} %_   (HV-Math/Helvetica)
%\usepackage{tmmath} %_   (TM-Math/Times)
%\usepackage{cmbright}
%\usepackage{ccfonts} \usepackage[T1]{fontenc}
%\usepackage[garamond]{mathdesign}
\usepackage{color}
\usepackage{ulem}

\newcommand{\argmax}{\operatornamewithlimits{arg\,max}}
\newcommand{\argmin}{\operatornamewithlimits{arg\,min}}
\def\inprobLOW{\rightarrow_p}
\def\inprobHIGH{\,{\buildrel p \over \rightarrow}\,} 
\def\inprob{\,{\inprobHIGH}\,} 
\def\indist{\,{\buildrel d \over \rightarrow}\,} 
\def\F{\mathbb{F}}
\def\R{\mathbb{R}}
\newcommand{\gmatrix}[1]{\begin{pmatrix} {#1}_{11} & \cdots &
    {#1}_{1n} \\ \vdots & \ddots & \vdots \\ {#1}_{m1} & \cdots &
    {#1}_{mn} \end{pmatrix}}
\newcommand{\iprod}[2]{\left\langle {#1} , {#2} \right\rangle}
\renewcommand{\det}{\mathrm{det}}
\newcommand{\rank}{\mathrm{rank}}

\newtheorem{theorem}{Theorem}[section]
\newtheorem{conjecture}{Conjecture}[section]
\newtheorem{corollary}{Corollary}[section]
\newtheorem{lemma}{Lemma}[section]
\newtheorem{proposition}{Proposition}[section]
\theoremstyle{definition}
\newtheorem{assumption}{}[section]
%\renewcommand{\theassumption}{C\arabic{assumption}}
\newtheorem{definition}{Definition}[section]
\newtheorem{step}{Step}[section]
\newtheorem{remark}{Comment}[section]
\newtheorem{example}{Example}[section]
\newtheorem*{example*}{Example}

\linespread{1.1}

\pagestyle{fancy}
%\renewcommand{\sectionmark}[1]{\markright{#1}{}}
\fancyhead{}
\fancyfoot{} 
%\fancyhead[LE,LO]{\tiny{\thepage}}
\fancyhead[CE,CO]{\tiny{\rightmark}}
\fancyfoot[C]{\small{\thepage}}
\renewcommand{\headrulewidth}{0pt}
\renewcommand{\footrulewidth}{0pt}

\fancypagestyle{plain}{%
\fancyhf{} % clear all header and footer fields
\fancyfoot[C]{\small{\thepage}} % except the center
\renewcommand{\headrulewidth}{0pt}
\renewcommand{\footrulewidth}{0pt}}

\makeatletter
\renewcommand{\@maketitle}{
  \null 
  \begin{center}%
    \rule{\linewidth}{1pt} 
    {\Large \textbf{\textsc{\@title}}} \par
    {\normalsize \textsc{Paul Schrimpf}} \par
    {\normalsize \textsc{\@date}} \par
    {\small \textsc{University of British Columbia}} \par
    {\small \textsc{Economics 526}} \par
    \rule{\linewidth}{1pt} 
  \end{center}%
  \par \vskip 0.9em
}
\makeatother

\title{Matrix algebra and introduction to vector spaces}
\date{\today}

\begin{document}

\maketitle

We have already used matrices in our study of systems of linear
equations. Matrices are useful through economics. You will see a lot
of matrices in econometrics. Understanding matrices is also essential
for understanding systems of nonlinear equations, which appear in all
fields of economics. This lecture will go over some fundamental
properties of matrices. It is based on a combination of Simon and
Blume Chapters 8-9, Appendix A of \textit{Econometric Analysis} by
William Green, and Chapter 9 of \textit{Principles of Mathematical
  Analysis} by Rudin. Carter sections 1.4.1 and 1.4.2 covers much the
same material as section 1.2 of these notes. The start of chapter 3 of
Carter (up to 3.1.1) covers much the same material as section 1.2 and
the start of section 2 of these notes. 

% more sources: 
%  http://joshua.smcvt.edu/linearalgebra/
%  http://linear.ups.edu/

\section{Vector spaces and linear transformations}

Matrices are often introduced as just arrays of numbers.  That sort of
definition works out alright. However, it makes certain subsequent
definitions somewhat mysterious. You might wonder why matrix
multiplication is defined the way it is, or why transposing is
important.  Well, there is another, more abstract, but also more
fundamental way of describing matrices. This way of looking at
matrices will reveal exactly why matrix multiplication is the way it
is, and illuminate many other properties of matrices. 

\subsection{Vector spaces}

To fully understand matrices, we must first under vector
spaces. Loosely, a vector space is a set whose elements can be added
and scaled. Vector spaces appear quite often in economics because many
economic quantities can be added and scaled. For example, if firm $A$
produces quantities $y_1^A$ and $y_2^A$ of goods  $1$ and $2$, while
firm $B$ produces $(y_1^B,y_2^B)$, then total production is
$(y_1^A+y_1^B, y_2^A+y_2^B)$. If firm $A$ becomes 10\% more
productive, then it will produce $(1.1 y_1^A, 1.1 y_2^A)$. 

You are likely already familiar with $\R^n$, especially $\R^2$ and
$\R^3$, which are the most common vector spaces. There are three ways of
approaching vector spaces. The first is geometrically --- introduce
vectors as directed arrows arrows. This works well in $\R^2$ and
$\R^3$ but is difficult in higher dimensions.  The second is
analytically --- by treating vectors as n-tuples of numbers $(x_1,
..., x_n)$. The third approach is axiomatically --- vectors are
elements of a set that has some special properties. You likely already
have some familiarity with the first two approaches. Here, we are
going to take the third approach. This approach is more abstract, but
this abstraction will allow us to generalize what we might know about
$\R^n$ to other more exotic vector spaces. 

\begin{definition}
  A \textbf{vector space} is a set $V$ and a field $\mathbb{F}$ with
  two operations, addition $+$, which takes two elements of $V$ and
  produces another element in $V$, and scalar multiplication $\cdot$,
  which takes an element in $V$ and an element in $\mathbb{F}$ and
  produces an element in $V$, such that
  \begin{enumerate}
  \item $(V, +)$ is a commutative group, i.e.
    \begin{enumerate}
    \item Closure: $\forall v_1 \in V$ and $v_2 \in V$ we have $v_1
      + v_2 \in V$. 
    \item Associativity: $\forall v_1, v_2, v_3 \in V$ we have $v_1
      + (v_2 + v_3 ) = (v_1 + v_2) + v_3 $. 
    \item Identity exists: $\exists 0 \in V$ such that $\forall v \in
      V$, we have $v + 0 = v$
    \item Invertibility: $\forall v \in V$ $\exists -v \in V$ such
      that $v + (-v) = 0$
    \item Commutativity: $\forall v_1, v_2 \in V$ we have $v_1+v_2 =
      v_2 + v_1$
    \end{enumerate}
  \item Scalar multiplication has the following properties:
    \begin{enumerate}
    \item Closure: $\forall v \in V$ and $f \in \F$ we have $vf \in V$
    \item Distributivity: $\forall v_1 , v_2 \in V$ and $f_1, f_2 \in
      \F$
      \begin{align*}
        f_1 (v_1 + v_2) = f_1 v_1 + f_1 v_2 
      \end{align*}
      and 
      \begin{align*}
        (f_1 + f_2)v_1 = f_1 v_1 + f_2 v_1
      \end{align*}
    \item Consistent with field multiplication: $\forall v \in V$ and
      $f_1, f_2 \in V$ we have
      \begin{align*}
        1 v = v
      \end{align*}
      and 
      \begin{align*}
        (f_1 f_2) v =f_1 (f_2 v)
      \end{align*}
    \end{enumerate}
  \end{enumerate}
\end{definition}

\subsubsection{Examples}
We now give some examples of vector spaces. 
\begin{example} \label{ex:Rn}
  $\R^n$ with the field $\R$ is a vector space. You are likely already
  familiar with this space. Vector addition and multiplication are
  defined in the usual way. If $\mathbf{x}_1 = (x_{11}, ..., x_{n1})$
  and $\mathbf{x}_2 = (x_{12}, ..., x_{n2})$, then vector addition is
  defined as
  \[ \mathbf{x}_1 + \mathbf{x}_2 = (x_{11}+x_{12}, ... , x_{n1} +
  x_{n2}). \]
  The fact that $(\R^n,+)$ is a commutative group follows from the
  fact that $(\R,+)$ is a commutative group. Scalar multiplication is
  defined as
  \[ a \mathbf{x} = (a x_1, ..., ax_n) \] for $a \in \R$ and
  $\mathbf{x} \in R^n$. You should verify that the three properties in
  the definition of vector space hold.  The vector space $(\R^n, \R,
  +, \cdot)$ is so common that it is called \textbf{Euclidean
    space}\footnote{To be more accurate, Euclidean space refers to
    $\R^n$ as an inner product space, which is a special kind of
    vector space that will be defined below.} We will often just refer
  to this space as $\R^n$, and it will be clear from context that we
  mean the vector space $(\R^n, \R, + , \cdot)$. In fact, we will
  often just write $V$ instead of $(V,\F,+,\cdot)$ when referring to a
  vector space.
\end{example}
One way of looking at vector spaces is that they are a way of trying
to generalize the things that we know about two and three dimensional
space to other contexts. 
\begin{example}
  Any linear subspace of $\R^n$ along with the field
  $\R$ and the usual vector addition and scalar multiplication is a
  vector space. Linear subspaces are closed under $+$ and $\cdot$ by
  definition. Linear subspaces inherit all the other properties
  required by the definition of a vector space from $\R^n$.
\end{example}

\begin{example}
  $(\mathbb{Q}^n, \mathbb{Q}, +, \cdot)$ is a vector space
  where $+$ and $\cdot$ defined as in \ref{ex:Rn}.
\end{example}

\begin{example}
  $(\mathbb{C}^n, \mathbb{C}, +, \cdot)$ where $+$ and $\cdot$ defined
  as in \ref{ex:Rn} except with complex addition and multiplication
  taking the place of real addition and multiplication. 
\end{example}
Except for the preceding two examples, all the vector spaces in this
class will be real vector spaces with the field $\mathbb{F} =
\mathbb{R}$. Two more examples of vector spaces that we have already
encountered include:
\begin{example}
  The set of all solutions to a system of linear equation with $b=0$,
  i.e., $(x_1, ..., x_n) \in \R^n$ such that 
  \begin{align*}
    a_{11} x_1 + a_{12} x_2 + ... + a_{1n} x_n = & 0 \\
    a_{21} x_1 + a_{22} x_2 + ... + a_{2n} x_n = & 0 \\
    \vdots & \vdots \\
    a_{m1} x_1 + a_{m2} x_2 + ... + a_{mn} x_n = & 0 ,
  \end{align*}
\end{example}  
and 
\begin{example}
  Any linear subspace of a vector space.
\end{example}
Most of the time, the two operations on a vector space are the usual
addition and multiplication. However, they can be different, as the
following example illustrates.
\begin{example}
  Take $V = \R^+$. Define ``addition'' as $x \oplus y = xy$ and define
  ``scalar multiplication'' as $\alpha \odot x = x^\alpha$. Then
  $(\R^+,\R, \oplus, \odot)$ is a vector space with identity element
  $1$.   
\end{example}

Spaces of functions are often vector spaces. In economic
theory, we might want to work with a set of functions because we want
to prove something for all functions in the set. That is, we prove
something for all utility functions or for all production
functions. In non-parametric econometrics, we try to estimate an
unknown function instead of an unknown finite dimensional
parameter. For example, instead of linear regression $y = x\beta +
\epsilon$ where want to estimate the unknown vector $\beta$, we might
say $y = f(x) + \epsilon$ and try to estimate the unknown function
$f$. 

Here are some examples of vector spaces of functions. It would be a good
exercise to verify that these examples have all the properties listed
in the definition of a vector space. 
\begin{example} \label{ex:funcSpace}
  Let $V = $ all functions from $[0,1]$ to $\R$. For $f, g \in V$,
  define $f + g$ by $(f+g)(x) = f(x) + g(x)$. Define scalar
  multiplication as $(\alpha f)(x) = \alpha f(x)$. Then this is a
  vector space. 
\end{example}
Sets of functions with certain properties also form vector spaces. 
\begin{example}
  The set of all continuous functions with addition and scalar
  multiplication defined as in \ref{ex:funcSpace}.
\end{example}
\begin{example}
  The set of all $k$ times continuously differentiable functions with
  addition and scalar multiplication defined as in \ref{ex:funcSpace}.
\end{example}
\begin{example}
  The set of all polynomials with addition and scalar
  multiplication defined as in \ref{ex:funcSpace}.
\end{example}
\begin{example} 
  The set of all polynomials of degree at most $d$ with addition and scalar
  multiplication defined as in \ref{ex:funcSpace}.
\end{example}
\begin{example}
  The set of all functions from $\R \to \R$ such that $f(29481763) =
  0$ with addition and scalar multiplication defined as in
  \ref{ex:funcSpace}.
\end{example}
\begin{example}\label{ex:LP}
  Let $1 \leq p < \infty$ and let $\mathcal{L}^p(0,1)$ be the set of
  functions from $(0,1)$ to $\R$ such that $\int_0^1 |f(x)|^p dx$ is
  finite.
  \footnote{For now, you can just think of $\mathcal{L}^p(0,1)$ as
    consisting of all functions from $(0,1) \to \R$. The $p$ and finite
    integral part only become important when we think of $\mathcal{L}^p$
    as normed vector spaces, which we will do in the next set of notes.}
  Then $\mathcal{L}^p (0,1)$ with the field $\R$ and addition
  and scalar multiplication defined as
  \begin{align*}
    (f + g)(x) = & f(x) + g(x) \\
    (\alpha f)(x) = & \alpha f(x)
  \end{align*} 
  is a vector space.  The only difficult part of the definition of
  vector spaces to verify is closure under addition. To prove closure
  under addition, we can use Jensen's inequality. This is a
  surprisingly useful inequality that we may or may not prove
  later. Anyway the simplest form of Jensen's inequality says that if
  $h(x)$ is convex\footnote{A real valued function $h(x)$ is (weakly)
    convex if for all $x_1,x_2$ and $\lambda \in (0,1)$ $h(\lambda x_1
    + (1-\lambda) x_2) \leq \lambda h(x_1) + (1-\lambda) h(x_2)$.},
  then for any $a_1>0$ and $a_2>0$,
  \[ h\left(\frac{a_1 x_1 + a_2 x_2}{a_1 + a_2} \right) \leq \frac{a_1
    h(x_1) + a_2 h(x_2) }{a_1 + a_2}. \]
  If $p>1$, $h(x) = |x|^p$ is convex. From Jensen's inequality,
  \begin{align*}
    | \frac{f(x) + g(x)}{2} |^p \leq & \frac{|f(x)|^p + |g(x)|^p}{2}
    \\
    |f(x) + g(x) |^p \leq 2^{p-1} |f(x)|^p + |g(x)|^p. 
  \end{align*}
  Integrating,
  \begin{align*}
    \int |f(x) + g(x) |^p dx \leq 2^{p-1} \left(\int|f(x)|^p dx + \int
      |g(x)|^p dx \right)
  \end{align*}
  The right side is finite for any $f, g \in \mathcal{L}^p(0,1)$, so
  the left side is also finite and $f+g \in \mathcal{L}^p(0,1)$. 
  
  To a get a better feel for $\mathcal{L}^p(0,1)$, you should try to
  come with some examples of familiar functions are or are not
  included. You could consider polynomials, rational functions (ratios
  of polynomials), exponential, and logarithm.
\end{example}

\subsubsection{Linear combinations}

We briefly discusses linear independence and dimension last lecture
when we introduced linear subspaces of $\R^n$. These concepts apply to
any vector space. 
\begin{definition}
  Let $V$ be a vector space and $v_1,..., v_k \in V$. A \textbf{linear
    combination} of $v_1,..., v_k$ is any vector 
  \[c_1 v_1 + ... + c_k v_k \]
  where $c_1, ..., c_k \in \F$. 
\end{definition}
Note that by the definition of a vector space (in particular the
requirement that vector spaces are closed under addition and
multiplication), it must be that $c_1 v_1 + ... + c_k v_k \in V$.

Last lecture we argued that if a system of equations has two distinct
solutions, then we can take certain linear combinations of them of the
form $\lambda x_1 + (1-\lambda) x_2$ to trace out an affine set
of solutions. Recall that $\lambda x_1 + (1-\lambda) x_2 : \lambda
\in \R\}$ is an affine set and not necessarily a linear subspace because it may
not contain $0$. If instead of using $\lambda$ and $1-\lambda$ as
weights, we take all possible linear combinations, $\{c_1 x_1 + c_2
x_2 : c_1 \in \R, c_2 \in R\}$, then the set will contain $0$, and it
will be a linear subspace. This motivates the following definition.
\begin{definition}
  Let $V$ be a vector space and $W \subseteq V$. The
  \textbf{span} of $W$ is the set
  of all finite linear combinations of elements of $W$
\end{definition}
When $W$ is finite, say $W = \{v_1, ..., v_k\}$, the span of $W$ is
the set 
\[ \{ c_1 v_1 + ... + c_k v_k : c_1, ..., c_k \in \F \}. \] When $W$
is infinite, the span of $W$ is the set of all finite weighted sums of
elements of $W$.
\begin{lemma}
  The \textbf{span} of any  $W \subseteq V$ is a linear subspace.
\end{lemma}
\begin{proof}
  Left as an exercise.
\end{proof}

\begin{example}
  Let $V$ be the vector space of all functions from $[0,1]$ to $\R$ as
  in example \ref{ex:funcSpace}. The span of $\{1, x, ..., x^n\}$ is
  the set of all polynomials of degree less than or equal $n$.
\end{example}

You might remember the next two definitions from the previous lecture.
\begin{definition}
  A set of vectors $W \in V$, is \textbf{linearly
    independent} if the only solution to
  \begin{align*}
    \sum_{j=1}^k c_j v_j = 0 
  \end{align*}
  is $c_1 = c_2 = ... = c_k = 0$ for any $k$ and $v_1, ..., v_k \in W$.
\end{definition}

\paragraph{Checking linear independence}
Given a set of vectors, $\mathbf{v}_1, ..., \mathbf{v}_n \in \R^m$ (or
any $n$-dimensional vector space), how do check whether they are
linearly independent? Well, by definition, they are linearly
independent if $c_1 = c_2 = ... = c_n = 0$ is the only solution to
\begin{align*}
  \sum_{j=1}^n c_j \mathbf{v}_j = 0 
\end{align*}
If we write this condition as a system of linear equations we have
\begin{align*}
  v_{11} c_1 + v_{12} c_2 + ... + v_{1n} c_n & = 0 \\
  \vdots & = \vdots \\
  v_{m1} c_1 + v_{m2} c_2 + ... + v_{mn} c_n & = 0 
\end{align*}
or in matrix form,
\begin{align*}
  \gmatrix{v} \begin{pmatrix} c_1 \\ \vdots \\ c_n \end{pmatrix} = &
  0 \\
  V \mathbf{c} = & 0 
\end{align*}
We call any system with $0$ on the right hand side a
\textbf{homogeneous} system. Any homogeneous system always has
$\mathbf{c}=0$ as a solution. We know from lecture 2 that it will have
other solutions if the rank of $V$ is less than $n$. This proves the
following lemma.
\begin{lemma}\label{lem:rankli}
  Vectors $\mathbf{v}_1, ..., \mathbf{v}_n \in \R^m$ are linearly
  independent if and only if
  \[ \rank(\mathbf{v}_1, ..., \mathbf{v}_n) = n. \]
\end{lemma}
\begin{corollary}\label{cor:kmli}
  Any set of $k>m$ vectors in a $\R^m$ space are linearly dependent.
\end{corollary}

\subsubsection{Dimension and basis}
Recall the definition of dimension from last class. 
\begin{definition}
  The \textbf{dimension} of a vector space, $V$, is the cardinality of
  the largest set of linearly independent elements in $V$.
\end{definition} 

\begin{definition}
  A \textbf{basis} of a vector space $V$ is any set of linearly
  independent vectors $B$ such that the span of $B$ is $V$.
\end{definition}
If $V$ has a basis with $k$ elements, then the dimension of $V$ must
be at least $k$. In fact, we will soon see that dimension of $V$ must
be exactly $k$.
\begin{example}
  A basis for $\R^n$ is $e_1 = (1, 0, ..., 0 )$, $e_2 = (0, 1, 0, ...,
  0)$, $...$, $e_n = (0, ... , 0 , 1)$. This basis is called the
  standard basis of $\R^n$. 

  The standard basis is not the only basis for $\R^n$. In fact, there
  are infinite different bases. Can you give some examples?
\end{example}
The elements of a vector space can always be written in terms of a
basis. 
\begin{lemma} \label{lem:uniqueRep}
  Let $B$ be a basis for a vector space $V$. Then
  $\forall v \in V$ there exists a unique $v_1, ..., v_k \in \F$ and
  $b_1, ..., b_k \in B$
  such that $ v = \sum_{i=1}^k v_i b_i$  
\end{lemma}
\begin{proof}
  By the definition of a basis, $B$ spans $V$, so such
  $(v_1, ..., v_k)$ must exist. Now suppose there exists another such
  $(v_1', ..., v_j')$ and associated $b_i'$. The $\{b_1, ..., b_k\}$ and
  $\{b_1', ..., b_j'\}$ might not be the same collection of elements
  of $B$. Let $\{\tilde{b}_1, ..., \tilde{b}_n \} =  \{b_1, ...,
  b_k\} U \{b_1', ..., b_j'\}$. Define $\tilde{v}_i = v_j$ if
  $\tilde{b}_i = b_j$, else $0$. Similarly define $\tilde{v}_i'$. With
  this new notation we have
  \begin{align*}
    v = \sum_{i=1}^n \tilde{v}_i \tilde{b}_i = & \sum_{i=1} \tilde{v}_i' \tilde{b}_i \\
    \sum (\tilde{v}_i - \tilde{v}_i')\tilde{b}_i = & 0 \\
  \end{align*}
  However, if $B$ is a basis, its elements must be linearly
  independent so $\tilde{v}_i = \tilde{v}_i'$ for all $i$, so the
  original $v_1, ..., v_k$ must be unique.
\end{proof}


Let us show that dimension and basis are sensible definitions by
showing that any two bases have the same size. 
\begin{lemma}
  If $B$ is a basis for a vector space $V$ and $I \subseteq V$ is a
  set of linearly independent elements then $|I| \leq |B|$. 
\end{lemma}
For vector spaces of finite dimension this lemma can be stated as
follows. If $b_1,..., b_m$ is a basis for a vector space $V$ and $v_1,
..., v_n\in V$ are linearly independent, then $n \leq m$. We will only
prove this finite dimensional version, but it is true for infinite
dimensional spaces as well.
\begin{proof}
  Write $v_i \in I$ in terms of the basis:
  \[ v_i = \sum_{i=1}^m a_{ij} b_i \]
  Let us check for linear independence as described above. 
  \begin{align*}
    0 = \underbrace{A}_{m \times n} \underbrace{c}_{n \times 1}    
  \end{align*}
  This is a system of $m$ equations with $n$ unknowns. If $n>m$, then
  it must have multiple solutions, and $v_1,...,v_n$ cannot be
  linearly independent.
\end{proof}
An immediate consequence of this lemma is that any two bases must have
the same cardinality. 
\begin{corollary}
  Any two bases for a vector space have the same cardinality.
\end{corollary}
\begin{proof}
  If not, one would have more elements than the other. That would
  contradict the previous lemma.
\end{proof}

\begin{example} 
  What is the dimension of each of the examples of vector spaces
  above? Can you find a basis for them? 
\end{example}

We have now shown that any basis of a vector space has cardinality
equal to the dimension of the space. A related useful fact is that if
we have $n$ elements that span an $n$ dimensional space, then those
elements must be linearly independent, and hence a basis. 
\begin{lemma}
  If $v_1, ..., v_n$ span an $n$ dimensional vector space, then $v_1,
  ... , v_n$ are linearly independent.
\end{lemma}
\begin{proof}
  If $v_1, ..., v_n$ span $\R^n$, then we can write each standard
  basis vector as
  \[ e_i = \sum_{j=1}^n a_{ij} v_j. \]
  Suppose we have $c_j \neq 0$ such that
  \begin{align*} 
    0 = & \sum_{j=1}^n c_j v_j \\
    v_n = & \sum_{j=1}^{n-1} -\frac{c_j}{c_n} v_j.
  \end{align*}
  Substituting,
  \begin{align*}
    e_i = \sum_{j=1}^{n-1} (a_{ij} - \frac{c_j}{c_n} a_{in} ) v_j.
  \end{align*}
  Writing these $n$ systems of linear equations in matrix form we have
  \begin{align*}
    I_n = \underbrace{A}_{n\times(n-1)} \underbrace{V}_{(n-1) \times n}.
  \end{align*}
  However, we know that $\rank I_n = n$ and $\rank (A V) \leq
  \max\{\rank A, \rank V\} \leq n-1$, so this equation cannot be
  true. The steps to get to this equation only relied on some $c_j
  \neq 0$ so that we could divide by it. Therefore, it must be that
  $c_j = 0$ for all $j$, and $v_1,...,v_n$ are linearly independent.
\end{proof}

Collecting the results above and specializing them to $\R^n$ we have
four equivalent ways of describing a basis. 
\begin{theorem}\label{thm:lind}
  Let $v_1, ..., v_n \in \R^n$ and let $V = (v_1,...,v_n)$ be the $n$
  by $n$ matrix with columns $v_i$. Then the following are equivalent:
  \begin{enumerate}
  \item $v_1,...,v_n$ are linearly independent,
  \item $v_1,...,v_n$ span $\R^n$,
  \item $v_1,...,v_n$ are a basis for $\R^n$,
  \item $\rank V = n$
  \end{enumerate}
\end{theorem}

Suppose $V$ is an $n$-dimension real\footnote{Meaning the field of
  scalars is $\R$.} vector space. By the definition of dimension,
there must be a set of $n$ linearly independent elements that span
$V$. These elements form a basis. Call them $b_1, ..., b_n$. For each
$v \in V$, there are unique $v_1, ..., v_n \in \R$ such that 
\[ v = \sum_{i=1}^n v_i b_i. \]
Thus we can construct a function, say $I: V \to \R^n$ defined by 
\[ I(v) = (v_1, ..., v_n). \]
By lemma \ref{lem:uniqueRep}, $I$ must be one-to-one. $I$ must also be
onto since by definition of a vector space, for any $(v_1, ..., v_n)
\in \R^n$, the linear combination, $\sum_{i=1}^n v_i b_i$ is in
$V$. Moreover, $I$ preserves addition in that for
any $v^1, v^2 \in V$,
\begin{align*}
  I(v^1 + v^2) = & (v_1^1 + v_1^2, ..., v_n^1 + v_n^2) \\
  = & (v_1^1, ..., v_n^1) + (v_1^2 + ... + v_n^2) \\
  = & I(v^1) + I(v^2).
\end{align*}
Similarly, $I$ preserves scalar multiplication in that for all $v \in
V$, $\alpha \in \R$
\[ I(\alpha v) = \alpha I(v). \]
Thus, $V$ and $\R^n$ are essentially the same in that there is a
one-to-one and onto mapping between them that preserves all the
properties that make them vector spaces. 
\begin{definition}
  Let $V$ and $W$ be vector spaces over the field $\F$. $V$ and $W$ are
  \textbf{isomorphic} if there exists a one-to-one and onto function,
  $I:V \to W$ such that 
  \[ I(v^1 + v^2) = I(v^1) + I(v^2) \]
  for all $v^1, v^2 \in V$,
  and 
  \[ I(\alpha v) = \alpha I(v) \]
  for all $v \in V$, $\alpha \in \F$.
  Such an $I$ is called an \textbf{isomorphism}.
\end{definition}
The discussion preceeding this definition showed that all
$n$-dimensional real vector spaces are isomorphic to $\R^n$. 
\begin{theorem}
  Let $V$ be an $n$-dimensional vector space over the field $\F$. Then
  $V$ is isomorphic to $\F^n$. 
\end{theorem}
\begin{proof}
  We proved this for $\F=\R$. The same argument works for any field.
\end{proof}



\section{Linear transformations}

An isomorphism is a one-to-one and onto (bijective) functions that
preserves addition and scalar multiplication. Can a function between
vector spaces preserve addition and multiplication without being
bijective?  Let's try to construct an example. We know from above that
all finite dimensional vector spaces are isomorphic to $\R^n$, so we
might as well work with $\R^n$. To keep everything as simple as
possible, let's just work with $\R^1$. Consider $f:\R \to \R$ defined
by $f(x) = 0$ for all $x \in \R$. Clearly, $f$ is not bijective. $f$
preserves addition since
\[ f(x) + f(y) = 0 + 0 = 0 = f(x+y). \]
$f$ also preserves multiplication because 
\[ \alpha f(x) = \alpha 0 = 0 = f(\alpha x). \]
Thus, we know there are functions that preserve addition and scalar
multiplication but are not necessarily isomorphisms. Let's give such
functions a name. 
\begin{definition}
  A \textbf{linear transformation} (aka linear function) is a
  function, $A$, from a vector space $(V,\F,+,\cdot)$ to a vector
  space $(W,\F,+,\cdot)$ such that $\forall v_1, v_2 \in V$,
  \begin{align*}
    A (v_1 + v_2) = A v_1 + A v_2 
  \end{align*}
  and 
  \begin{align*}
    A (f v_1) = f A v_1
  \end{align*}
  for all scalars $f \in \F$.   

  A linear transformation from $V$ to $V$ is called a \textbf{linear
    operator} on $V$. A linear transformation from $V$ to $\R$ is
  called a \textbf{linear functional} on $V$.
\end{definition}
Any isomorphism between vector spaces is a linear
transformation. 
\begin{example}
  Define $f: \R^2 \to \R$ by $f( (x_1, x_2) ) = x_1$, that is $f(x)$ is
  the first coordinate of $x$. Then,
  \[ f(\alpha x + y) = \alpha x_1 + y_1 = \alpha f(x) + f(y) \]
  so $f$ is a linear transformation.
\end{example}

In general we can construct linear transformations between finite
dimensional vector spaces as follows. Let 
\[ A = \gmatrix{a} \]
be a matrix. As when we were working with systems of linear equations,
let 
\[ A\mathbf{x} = 
\begin{pmatrix} 
  \sum_{j=1}^n a_{1j} x_j \\
  \vdots \\
  \sum_{j=1}^n a_{mj} x_j 
\end{pmatrix}, \] for $\mathbf{x} = (x_1, ..., x_n) \in \R^n$. Then
$A$ is a linear transformation from $\R^n$ to $\R^m$. You may want to
verify that $A(f \mathbf{x}_1 + \mathbf{x}_2 ) = f A
\mathbf{x}_1 + \mathbf{x}_2$ for scalars $f \in \R$ and
vectors $\mathbf{x}_1, \mathbf{x}_2 \in \R^n$. 

Conversely let $A$ be a linear transformation from $V$ to $W$ (if it
is helpful, you can let $V=\R^n$ and $W=\R^m$), and let $b_1, b_2,
..., b_n$ be a basis for $V$. By the definition of a basis, any $v \in
V$ can be written $v = \sum_{j=1}^n v_j b_j$ for some $v_j
\in \F$. By the definition of a linear transformation, we have
\begin{align*}
  A v = \sum_{j=1}^n v_j A b_j. 
\end{align*}
Thus, a linear transformation is completely determined by its action
on a basis. Also, if $d_1, ..., d_m$ is a basis for $W$ then for each
$A b_j$ we must be able to write $A b_j$ as a sum of the basis
elements $d_1, ..., d_m$, i.e.\
\begin{align*}
  A b_j = \sum_{i=1}^m a_{ij} d_i.
\end{align*}
Substituting this equation into the previous one, we can write $Av$ as
\begin{align*}
  Av = & \sum_{j=1}^n v_j A b_j \\
  = & \sum_{j=1}^n v_j \sum_{i=1}^m a_{ij} d_i \\
  = & \sum_{i=1}^m d_i \left( \sum_{j=1}^n a_{ij} v_j \right) 
\end{align*}
Thus, associated with a linear transformation there is an array of
$a_{ij} \in \F$ determined by the linear transformation (and choice of
basis for $V$ and $W$). In the previous paragraph, we saw that
conversely, if we have an array of $a_{ij} \in \F$ we can construct a
linear transformation. This leads us to the following result.
\begin{theorem}
  For any linear transformation, $A$, from $\R^n$ to $\R^m$ there is an
  associated $m$ by $n$ matrix,
  \[ 
  \gmatrix{a}
  \]
  where $a_{ij}$ is defined by $A e_j = \sum_{i=1}^m a_{ij}
  e_i$. Conversely, for any $m$ by $n$ matrix, there is an associated
  linear transformation from $\R^n$ to $\R^m$ defined by $A e_j =
  \sum_{i=1}^n a_{ij} e_i$.
\end{theorem}
Thus, we see that matrices and linear transformations from $\R^m$ to
$\R^n$ are the same thing. This fact will help us make sense of many
of the properties of matrices that we will go through in the next
section. Also, it will turn out that most of the properties of
matrices are properties of linear transformations. There are linear
transformations that cannot be represented by matrices, yet many of
the results and definitions that are typically stated for matrices
will apply to these sorts of linear transformations as well.

Two examples of linear transformations that cannot be represented by
matrices are integral and differential operators,
\begin{example}[Integral operator]
  Let $k(x,y)$ be a function from $(0,1)$ to $(0,1)$ such that
  $\int_0^1 \int_0^1 k(x,y)^2 dx dy$ is finite.  Define
  $K:\mathcal{L}^2(0,1) \rightarrow \mathcal{L}^2(0,1)$ by
  \begin{align*}
    (K f) (x) = \int_0^1 k(x,y) f(y) dy
  \end{align*}
  Then $K$ is a linear transformation because
  \begin{align*}
    (K ( \alpha f + g) ) (x) = & \int_0^1 k(x,y) (\alpha f(y) + g(y))dy
    \\
    = & \alpha \int_0^1 k(x,y) f(y) dy + \int_0^1 k(x,y) g(y)dy \\
    = & \alpha (K f) (x) + (K g)(x) 
  \end{align*}      
\end{example}

\begin{example}[Conditional expectation]
  One special type of an integral operator that appears often in
  economics is the conditional expectation operator. Suppose $X$ and
  $Y$ are real valued random variables with joint pdf $f_{xy}(x,y)$
  and marginal pdfs $f_x(x) = \int_\R f(x,y) dy$ and $f_y(y) = \int_\R
  f(x,y) dx$. Consider the vector spaces 
  \[ 
  V = \mathcal{L}^2(\R,f_y) = \{g:
  \R \to \R \text{ such that } \int_{\R} f_y(y) g(y)^2 dy <
  \infty \} \]
  and 
  \[ 
  W = \mathcal{L}^2(\R,f_x) = \{g:
  \R \to \R \text{ such that } \int_{\R} f_x(x) g(x)^2 dx <
  \infty \} 
  \]  
  $V$ is the space of all functions of $Y$ such that the variance of
  $g(Y)$ is finite. Similarly, $W$ is the space of all functions of
  $X$ such that the variance of $g(X)$ is finite. 
  The conditional expectation operator is $\mathcal{E}: V \to W$
  defined by 
  \begin{align*}
    (\mathcal{E} g)(x) = E[g(Y) | X = x] = & \int_\R
    \frac{f_{xy}(x,y)}{f_x(x) f_y(y)} g(y) f_y(y) dy.
  \end{align*}
  The conditional expectation operator is an integral operator, so it
  is a linear transformation. 
\end{example}

\begin{example}[Differential operator]
  Let $C^\infty(0,1)$ be the set of all infinitely differentiable
  functions from $(0,1)$ to $\R$. $C^\infty(0,1)$ is a vector space.
  Let $D:C^\infty(0,1) \rightarrow C^\infty(0,1)$ be defined by
  \[ (D f) (x) = \frac{d f}{dx}(x) \]
  Then $D$ is a linear transformation.
\end{example}
Integral and differential operators are very important when studying
differential equations.  They are also useful in many areas of
econometrics and in dynamic programming. We will encounter some linear
transformations on infinite dimensional spaces when we study optimal
control. However, for the rest of this lecture, our main focus will be
an linear transformations between finite dimensional spaces,
i.e.\ matrices. 

\section{Matrix operations and properties}

Let $A$ and $B$ be linear transformations from $\R^m$ to $\R^n$ and
let $\gmatrix{a}$ and $\gmatrix{b}$ be the associated matrices.  Since
the linear transformation $A$ and the matrix $\gmatrix{a}$ represent
the same object, we will use $A$ to denote both.  From the previous
section, we know that $A$ and $B$ are characterized by their action on
the standard basis vectors in $\R^n$. In particular, $A e_j =
\sum_{i=1}^m a_{ij} e_i$ and $B e_j = \sum_{i=1}^m b_{ij} e_i$. 


\subsection{Addition} 
To define matrix addition, it makes sense to
require $(A+B)x = Ax + Bx$. Then,
\begin{align*}
  (A + B) e_j = & A e_i + B e_j \\
  = & \sum_{j=1}^m a_{ij} e_i + \sum_{j=1}^m b_{ij} e_i\\
  = & \sum_{j=1}^m (a_{ij} + b_{ij}) e_i,
\end{align*}
so the only way sensible way to define matrix addition is 
\begin{align*}
  A + B = \begin{pmatrix} a_{11} + b_{11} & \cdots &
    a_{1n} + b_{1n}  \\ \vdots & \ddots & \vdots \\ a_{m1} + b_{m1} & \cdots &
    a_{mn}+b_{mn} \end{pmatrix}
\end{align*}
As an exercise, you might want to verify that matrix addition has the
following properties:
\begin{enumerate}
\item Associative: $A+(B + C) = (A+B) + C$,
\item Commutative: $A + B = B + A$ ,
\item Identity: $A + \mathbf{0} = A$, where $\mathbf{0}$ is an $m$ by
  $n$ matrix of zeros, and
\item Invertible $A + (-A) = \mathbf{0}$ where $-A = \gmatrix{-a}$.
\end{enumerate}

\subsection{Scalar multiplication}
The definition of linear transformations requires that $A \alpha x =
\alpha A x$ where $\alpha \in \mathbb{F}$ and $x \in V$. To be
consistent with this, for matrices we must define
\begin{align*}
  \alpha A = \begin{pmatrix} \alpha a_{11} & \cdots &
    \alpha a_{1n} \\ \vdots & \ddots & \vdots \\ \alpha a_{m1} & \cdots &
    \alpha a_{mn} \end{pmatrix}
\end{align*}
We have now defined addition and scalar multiplication for
matrices. It should be no surprise that the set of all $m$ by $n$
matrices along with these two operations and the field $\R$ forms a
vector space. 
\begin{example}
  The set of all $m$ by $n$ matrices is a vector space. 
\end{example}
In fact, the above is not only true of the set of all
$m$ by $n$ matrices, but of any set of all linear transformations
between two vector spaces.
\begin{example}
  Let $L(V,W)$ be the set of all linear transformations from $V$ to
  $W$. Define addition and scalar multiplication as above. Then 
  $L(V,W)$ is a vector space.
\end{example}
$L(\R^n, \R^m)$ is the set of all linear transformations from $\R^n
\to \R^m$, i.e.\ all $m$ by $n$ matrices. 

\subsection{Matrix multiplication}
Matrix multiplication is really the composition of two linear
transformations. Let $A$ be a linear transformation from $\R^n$ to
$\R^m$ and $B$ be a linear transformation from $\R^p$ to $\R^n$. Now,
we defined matrices by looking at how a linear tranformation acts on a
basis vectors, so to define multiplication, we should look at $A(B
e_k)$
\begin{align*}
  A(B e_k) = & A (\sum_{j=1}^n b_{jk} e_j) & \text{definition of $Be_k$}\\
  = & \sum_{j=1}^n b_{jk} A e_j & \text{Definition of linear
    transformtion} \\
  = & \sum_{j=1}^n b_{jk} \left(\sum_{i=1}^m a_{ij} e_i\right)  &
  \text{definition of $Ae_j$} \\
  = & \sum_{i=1}^m \left(\sum_{j=1}^n a_{ij} b_{jk} \right) e_i \\
  = & \begin{pmatrix} 
    \sum_{j=1}^n a_{1j} b_{j1} & \cdots & \sum_{j=1}^n a_{1j} b_{jp} \\
    \vdots & \ddots & \vdots \\
    \sum_{j=1}^n a_{mj} b_{j1} & \cdots & \sum_{j=1}^n a_{mj} b_{jp}
  \end{pmatrix} e_k  \\
  = & (AB)e_k .
\end{align*}
The indexing in the above equations is unpleasant and could be
confusing. The important thing to remember is that matrix
multiplication is the composition of linear transformations. It then
makes sense that if $A$ is $m$ by $n$ (a transformation from $\R^n$ to
$\R^m$)and $B$ is $k$ by $l$ (a transformation from $R^l$ to $\R^k$),
we can only multiply $A$ times $B$ if $k = m$. Matrix multiplication
has the following properties:
\begin{enumerate}
\item Associative: $A(BC) = (AB) C$
\item Distributive: $A(B+C) = AB + AC$ and $(A+B)C = AC + BC$. 
\item Identity: $AI_n = A$ where $A$ is $m$ by $n$ and $I_n$ is the linear
  transformation from $\R^n$ to $\R^n$ such that $I_nx = x \forall x \in
  \R^n$.
\end{enumerate}
As we saw on the review, matrix multiplication is not commutative. 

\subsection{Transpose}

As you likely know, the transpose of an $m \times n$ matrix $A$ is an
$n \times m$, $A^T$ whose rows are the columns of $A$. As a linear
transformation, if $A: V \to W$, then $A^T: W \to V$, but how can we
define $A^T$ when $A$ cannot be represented as a matrix (i.e. when $V$
or $W$ is infinite dimensional). There are two ways to define the
transpose of a linear function. They each require introducing a new
concept. We will first define the transpose in terms of the inner
product. We will then define the transpose in terms of dual
spaces. Both dual spaces and inner products will appear later in the
course as well, so its worth introducing them now. However, it is not
essential to completely understand either of these abstract
definitions of the transpose. You will only really need to understand
the transposing of matrices.

\subsubsection{Transpose and inner products}

One way to define the transpose in terms of linear
transformations, is to use another property of Euclidean
space ($\R^n$). 
\begin{definition}
  A real \textbf{inner product space} is a vector space over the field
  $\R$ with an additional operation called the inner product that is
  function from $V \times V$ to $\mathbb{R}$. We denote the inner
  product of $v_1, v_2 \in V$ by $\iprod{v_1}{v_2}$. It has the
  following properties:
  \begin{enumerate}
  \item Symmetry: $\iprod{v_1}{v_2} = \iprod{v_2}{v_1}$
  \item (Bi)linear: $\iprod{a v_1 + b v_2}{v_3} = a \iprod{v_1}{v_3} + b
    \iprod{v_2}{v_3}$ for $a, b \in \R$
  \item Positive definite: $\iprod{v}{v} \geq 0$ and equals $0$ iff
    $v=0$. 
  \end{enumerate}  
\end{definition}
Although it is not obvious from this definition, an inner product
space is a vector space where we can measure the angle between two
vectors. We will come back to this point next lecture.
\begin{example}
  $\R^n$ with the \textbf{dot product}, $x \cdot y = \sum_{i=1}^n x_i
  y_i$, is an inner product space. 
\end{example}

\begin{example}
  As it is usually defined, $\mathcal{L}^2(0,1)$ with $\iprod{f}{g}
  \equiv \int_0^1 f(x) g(x) dx$ is an inner product space. However, we
  have not defined $\mathcal{L}^2(0,1)$ carefully enough for this to
  be an inner product space. For example, consider $h(x)
  = \begin{cases} 0 \text{ if } x \neq 1/2 \\
    1 \text{ if } x = 1/2
  \end{cases}$. Then 
  \[ 
  \iprod{h}{h} = \int_0^1 h(x)^2 dx = 0,
  \]
  but $h(x)$ is not zero, violating the positive definite requirement
  of inner products. We can get around this problem by defining
  $\mathcal{L}^2(0,1)$ as not all functions, but as equivalence
  classes of functions. We say $f \simeq g$ if $\int_0^1 |f(x) -
  g(x)|^2 dx = 0$, and then define $\mathcal{L}^2(0,1)$ as the set of
  equivalence classes of functions. With this definition, the
  problematic $h$ above and $0$ represent the same element of
  $\mathcal{L}^2(0,1)$, and it becomes an inner product space. 

  As an exercise, you could try to verify some of these
  assertions. For example, show that $\simeq$ is really an equivalence
  relation, or show that $\mathcal{L}^2(0,1)$ remains a vector space
  when it is the set of equivalence classes instead of set of
  functions. Fully verifying these claims requires knowledge of
  measure theory, which is outside the scope of this course, but you
  could make some progress.
\end{example}

We can now define the transpose in terms of the inner product.
\begin{definition}
  Given a linear transformation, $A$, from a real inner product space
  $V$ to a real inner product space $W$, the
  \textbf{transpose} of $A$, denoted $A^T$ (or often $A'$) is a
  linear transformation from $W$ to $V$ such that $\forall v \in V, w
  \in W$
  \begin{align*}
    \iprod{A v}{w} = \iprod{v}{A^T w}.
  \end{align*}
\end{definition}
Now, let's take $V=\R^n$ and $W = \R^m$ and see what this definition
means for the transpose of a matrix. First observe that $A^T$ is a
mapping from $\R^m$ to $\R^n$, so as a matrix is must be $n$ by
$m$. Let $a^T_{ji}$ denote the entries of $A^T$. Rewriting the
inner product in terms of the entries of the matrix, we get
\begin{align*}
  \iprod{Av}{w} = & \sum_{i=1}^m \left(\sum_{j=1}^n a_{ij} v_j \right)
  w_i  \\
  = & \sum_{i=1}^m \sum_{j=1}^n a_{ij} w_i v_j
\end{align*}
and 
\begin{align*}
  \iprod{v}{A^T w} = & \sum_{j=1}^n v_j \left(\sum_{i=1}^m a_{ji}^T
    w_i\right) \\
  = & \sum_{i=1}^m \sum_{j=1}^n a_{ji}^T w_i v_j
\end{align*}
If $\iprod{A v}{w} = \iprod{v}{A^T w} $, for any $v$ and $w$ we must
have $a_{ji}^T = a_{ij}$. 

The transpose of a matrix simply swaps rows for columns. The transpose
has the following properties:
\begin{enumerate}
\item $(A+B)^T = A^T + B^T$
\item $(A^T)^T = A$
\item $(\alpha A)^T = \alpha A^T$
\item $(AB)^T = B^T A^T$.
\item $\rank A = \rank A^T$
\end{enumerate}

\subsubsection{Transpose and dual spaces}

Another way of defining the transpose of a linear transformation is in
terms of dual spaces. Not all vector spaces are inner product spaces,
but all vector spaces have a dual space, so defining the transpose
using dual spaces will be more general. 
Dual spaces are needed to define differentiation
on vector spaces, which we will want to do later.
\begin{definition}
  Let $V$ be a vector space. The \textbf{dual space} of $V$, denote
  $V^\ast$ is the set of all (continuous)\footnote{We have not yet
    defined continuity, so do not worry about this requirement. All
    linear functionals on finite dimensional spaces are
    continuous. Some linear functionals on infinite dimensional spaces
    are not continuous. The definition of dual space does not always
    require continuity. Often the dual space is defined as the set of
    all linear functionals, and the topological dual space is the set
    of all continuous linear functionals. We will ignore this
    distinction.}  linear functionals, $v^\ast: V \to \R$.
\end{definition}

\begin{example}
  The dual space of $\R^n$ is the set of $1 \times n$ matrices. In
  fact, for any finite dimensional vector space, the dual space is the
  set of row vectors from that space. 
\end{example}

\begin{example}
  The space $\ell_p$ for $1 \leq p \leq \infty$ is the set of
  sequences of real numbers $\mathbf{x}=(x_1, x_2, ...)$ such that
  $\sum_{i=1}^\infty |x_i|^p < \infty$. (When $p = \infty$, $\ell_\infty = \{ 
  (x_1, x_2, ...) : \max_{i \in \mathbb{N}} |x_i| < \infty \}$). Such
  spaces appear in economics in discrete time, infinite horizon
  optimization problems. 

  Let's consider the dual space of $\ell_\infty$. In macro models, we
  rule out everlasting bubbles and ponzi schemes by requiring
  consumption divided by productivity to be in $\ell_\infty$. Every
  sequence, $\mathbf{p} = (p_1, p_2, ...) \in \ell_1$ gives rise to a linear
  functional on $\ell_\infty$ defined by
  \begin{align*}
    \mathbf{p}^\ast \mathbf{x} = \sum_{i=1}^\infty p_i x_i \leq
    \left(\sum_{i=1}^\infty |p_i| \right) \left(\max_{i \in
        \mathbb{N}} |x_i| < \infty\right). 
  \end{align*}
  We can conclude that $\ell_1 \subseteq \ell_\infty^\ast$. 

  As a (difficult) exercise, you could try to show whether or not
  $\ell_1 = \ell_\infty^\ast$. Exercise 3.46 of Carter is very
  related. 
\end{example}

\begin{example}
  What is the dual space of $V = \mathcal{L}^2(\R,f_x) = \{g:
  \R \to \R \text{ such that } \int_{\R} f_x(x) g(x)^2 dx <
  \infty\}$? Let $h \in \mathcal{L}^2(\R,f_x)$. Define 
  \[ h^\ast(g) = \int_\R f_x(x) g(x) h(x) dx. \] 
  Assuming $h^\ast(g)$
  exists, $h^\ast$ is an integral operator from $V$ to $\R$, so it is
  linear. To show that $h^\ast \in V^\ast$ all we need to do is
  establish that $h^\ast(g)$ exists (is finite) for all $g \in V$. 
  H\:{o}lder's inequality\footnote{See
    e.g.
    \href{http://en.wikipedia.org/wiki/H\%C3\%B6lder\%27s_inequality}
      {Wikipedia for more information and a proof and more
        information.}} , which we have not studied but is good to be
    aware of, says that
  \[ \int_\R f_x(x) |g(x) h(x)| dx \leq \sqrt{\int f_x(x) g(x)^2 dx}
  \sqrt{\int f_x(x) h(x)^2 dx}. \]
  Since $h$ and $g \in V$, the right hand side must be finite, so
  $h^\ast(g)$ is finite as well. Thus all such $h^\ast$ is a subset of
  $V^\ast$.   In fact, all such $h^\ast$ is equal to
  $V^\ast$.\footnote{This is a consequence of the Radon-Nikodym
    theorem, which is beyond the scope of this course.}

  We will have to work with $V^\ast$ and similar dual spaces when we
  study optimal control. 
\end{example}

\begin{definition}
  If $A: V \to W$ is a linear transformation, then the
  \textbf{transpose} (or dual) of $A$ is $A^T: W^\ast \to V^\ast$
  defined by $(A^Tw^\ast)v = w^\ast(Av)$.
\end{definition}
To parse this definition, note that $A^T w^\ast$ is an element of
$V^\ast$, so it is a linear transformation from $V$ to $\R$. Thus,
$(A^T w^\ast) v \in \R$. Similarly, $Av \in W$, and $w^\ast: W \to
\R$, so $w^\ast (A v) \in \R$. 

Let us verify that this definition of the transpose coincides with the
one using inner products.\footnote{The two definitions coincide for
  $V=\R^n$ and $W = \R^m$. There are some situations when the
  definitions are not the same, but we will not encounter any. When
  you care about the difference the previous definition of transpose
  using inner products is usually called the (Hermitian) adjoint, and
  the definition using dual spaces is called the transpose.} Let $V$
and $W$ be innerproduct spaces and $A: V \to W$ be linear, and suppose
$A^T: W \to V$ satisfies the earlier definition of transpose, that is
$\iprod{A v}{w} = \iprod{v}{A^T w}$. We want to show that $A^T$ also
satisfies the later definition transpose. First, we must establish
that for inner product spaces $W^\ast = W$ and $V^\ast = V$. For each
$v \in V$ we can define a linear operator $v^\ast( y) =
\iprod{v}{y}$. Thus, $V \subseteq V^\ast$. When $V = \R^n$, it is
clear that the only linear operators on $V$ are of the form $a_1 x_1 +
... + a_n x_n$, so in that case $V^\ast = V$. In an example we argued
that ${\mathcal{L}^2}^\ast = \mathcal{L}^2$. We will take as given for
the remainder of the course $V^\ast = V$ whenever $V$ is an inner
product space\footnote{The requirement of continuity in our definition
  of dual space is essential for this to be true of infinite
  dimensional $V$.}. Given this, we can conclude that
\begin{align*}
  w^\ast (A v) = \iprod{A v}{w} =  \iprod{v}{A^T w} = (A^T w^\ast) (v).
\end{align*}

As an exercise, you should try to show that this definition of the
transpose is the same as the familiar flipping rows and columns
definition for matrices.


\subsection{Types of matrices}

There are some special types of matrices that will be useful.
\begin{definition}
  A \textbf{column} matrix is any $m$ by $1$ matrix.
\end{definition}

\begin{definition}
  A \textbf{row} matrix is any $1$ by $n$ matrix.
\end{definition}

\begin{definition}
  A \textbf{square} matrix has the same number of rows and columns.
\end{definition}

\begin{definition}
  A \textbf{diagonal} matrix is a square matrix with non-zero entries
  only along its diagonal, i.e.\ $a_{ij} = 0$ for all $i \neq j$. 
\end{definition}

\begin{definition}
  An \textbf{upper triangular} matrix is a square matrix that has
  non-zero entries only on or above its diagonal, i.e.\ $a_{ij} = 0$
  for all $j>i$. A \textbf{lower triangular} matrix is the transpose
  of an upper triangular matrix.
\end{definition}

\begin{definition}
  A matrix is \textbf{symmetric} if $A = A^T$.
\end{definition}

\begin{definition}
  A matrix is \textbf{idempotent} if $AA = A$.
\end{definition}

\begin{definition}
  A \textbf{permutation} matrix is a square matrix of $1$'s and $0$'s
  with exactly one $1$ in each row or column.  
\end{definition}

\begin{definition}
  A \textbf{nonsingular} matrix is a square matrix whose rank equals
  its number of columns.
\end{definition}

\begin{definition}
  An \textbf{orthogonal} matrix is a square matrix such that $A^TA =
  I$.
\end{definition}

\subsection{Invertibility}

We will now define the inverse of a matrix and linear
transformation. Since multiplication of linear transformations is not
commutative, it could be that right and left inverses differ. To allow
this possibility, we make the following definition. 
\begin{definition}
  Let $A$ be a linear transformation from $V$ to $W$. Let $B$ be a
  linear transfromation from $W$ to $V$. $B$ is a \textbf{right
    inverse} of $A$ if $AB = I_V$. Let $C$ be a linear transformation
  from $V$ to $W$. $C$ is a \textbf{left inverse} of $A$ if $CA = I_W$. 
\end{definition}  
We will now show that right and left inverses do not differ for square
matrices. 
\begin{lemma}
  If $A$ is a linear transformation from $V$ to $V$ and $B$ is a right
  inverse, and $C$ a left inverse, then $B = C$. 
\end{lemma}
\begin{proof}
  Left multiply $AB = I$ by $C$ to get
  \begin{align*}
    CAB & = CI \\
    (CA) B & = C \\
    B & = C.
  \end{align*}
\end{proof}
Now that we have established that right and left inverses are the
same, it makes sense to define just the inverse.
\begin{definition} 
  Let $A$ be a linear transformation from $V$ to $V$. The
  \textbf{inverse} of $A$ is written $A^{-1}$ and satisfies $A A^{-1}
  = I = A^{-1} A$.
\end{definition}
The previous lemma shows that if $A^{-1}$ exists, it is unique. 

\begin{lemma}
  Let $A$ be a linear tranformation from $V$ to $V$, and suppose $A$
  is invertible. Then $A$ is nonsingular and the unique solution to
  $Ax = b$ is $x = A^{-1} b$. 
\end{lemma}
Recall that $A$ is nonsingular if  $Ax = b$ has a unique solution for
all $b$.
\begin{proof}
  Multiply both sides of the equation by $A^{-1}$ to get
  \begin{align*}
    A^{-1} A x =  & A^{-1} b \\
    x = & A^{-1} b
  \end{align*}
  Therefore a solution exists. This solution is also unique because
  any other solution, $x'$ must also satisfy 
  \begin{align*}
    A x' =  &  b \\
    A^{-1} A x'= & A^{-1} b \\
    x' = & A^{-1} b,
  \end{align*}
  and $A^{-1}$ is unique. 
\end{proof}
Conversely we can show that if $A$ is nonsingular, then $A$ must be invertible.
\begin{lemma}
  If $A$ is nonsingular, then $A^{-1}$ exists.
\end{lemma}
\begin{proof}
  Let $e_1, e_2, ... $ be a basis for $V$.  By the definition of
  nonsingular for any $e_j$ there exists a solution to 
  \[ A x = e_j. \]
  Call this solution $x_j$.  Define a linear transformation $C$ by 
  \[ C v = \sum v_j x_j, \]
  where $v = \sum v_j e_j $. We can write any $v \in V$ in this way
  since $e_j$ is a basis. Now, consider $A(Cv)$.
  \begin{align*}
    A (C v) = & A \sum v_j x_j \\
    = & \sum v_j A x_j \\
    = & \sum v_j e_j = v.
  \end{align*}
  Thus, $AC = I$ and $C = A^{-1}$.
\end{proof}
\begin{corollary}
  A square matrix $A$ is invertible if and only if $\rank A$ is equal
  to its number of columns.
\end{corollary}
\begin{proof}
  In the previous lecture, we showed that $\rank A = $ number of
  columns if and only if $A$ is nonsingular. That result along with
  the two previous lemmas implies the corollary.
\end{proof}

Let $A$ and $B$ be invertible square matrices. The inverse has the
following properties:
\begin{enumerate}
\item $(AB)^{-1} = B^{-1} A^{-1}$
\item $(A^T)^{-1} = (A^{-1})^T$
\item $(A^{-1})^{-1} = A$
\end{enumerate}

\section{Determinants}

I am not a fan of determinants because they do not generalize well to
infinite dimensions. Nonetheless, you must at least know that the
determinant of a matrix is non-zero if and only if the matrix is
invertible. 

Given a $1$ by $1$ matrix, say, $A = (a)$, we known that $A^{-1}$
exists if and only if $a \neq 0$. Also, viewed as a linear
transformation from $\R$ to $\R$, $A$ transforms $1$ into $a$. It
would be useful to have a number with similar properties for larger
matrices. That is, for any matrix $A$, we want a number that is not
zero if and only if $A^{-1}$ exists, and that describes something
about how $A$ acts as a linear transformation. Well, the determinant
of a matrix, written $\det A$ is exactly such a number. 

Let us start by looking at a 2 by 2 matrix, $A = \begin{pmatrix} a & b
  \\ c & d \end{pmatrix}$. The inverse of $A^{-1}$ must satisfy 
\begin{align}
  A A^{-1} = & I
\end{align}
If $x_1$ is the first column of $A^{-1}$ and $x_2$ is the second
column, then we can rewrite this as two systems of linear equations, 
\[ A x_1 = \begin{pmatrix} 1 \\ 0 \end{pmatrix} \]
and 
\[ A x_2 = \begin{pmatrix} 0 \\ 1 \end{pmatrix}. \]
If you think about how we would solve these two systems using
Gauss-Jordan elimination, you will see that the steps involved depend
only on $A$ and not on the right side of the equation. Therefore, we
can solve both systems together by performing Gauss-Jordan elmination
on
\begin{align*}
  \begin{pmatrix} a & b & 1 & 0 \\
    c & d & 0 & 1 
  \end{pmatrix} \simeq & 
  \begin{pmatrix} a & b & 1 & 0 \\
    0 & \frac{ad-bc}{a} & -\frac{c}{a} & 1 
  \end{pmatrix} \\
  \simeq & 
  \begin{pmatrix} a & b & 1 & 0 \\
    0 & 1 & -\frac{c}{ad-bc} & \frac{a}{ad-bc}
  \end{pmatrix} \\
  \simeq & 
  \begin{pmatrix} a & 0 & \frac{ad}{ad-bc} & \frac{-ba}{ad-bc} \\
    0 & 1 & -\frac{c}{ad-bc} & \frac{a}{ad-bc}
  \end{pmatrix} \\
  \simeq & 
  \begin{pmatrix} 1 & 0 & \frac{d}{ad-bc} & \frac{-b}{ad-bc} \\
    0 & 1 & -\frac{c}{ad-bc} & \frac{a}{ad-bc}.
  \end{pmatrix}
\end{align*}
So, $A^{-1} = \begin{pmatrix} \frac{d}{ad-bc} & \frac{-b}{ad-bc} \\
  -\frac{c}{ad-bc} & \frac{a}{ad-bc} \end{pmatrix}$. For this to
exist, we must have $ad - bc \neq 0$. (The steps to get the formula
for $A^{-1}$ divided by $a$, so we also assumed $a \neq 0$. However,
if $a = 0$, we could start Gaussian elimination by swapping the first
and second rows, divide by $c$, and end up with the same formula. If
$c=0$ as well, then $ad - bc = 0$ too, so $ad - bc \neq 0$ is really
the only condition needed). 

This calculation suggests that $ad - bc $ is a good candidate for the
determinant of $2$ by $2$ matrices. You may remember from the review
problem set that $|ad-bc|$ is the area of the parallelogram
constructed from the columns of $A$. A nice interpretation of this
parallelogram is that it is the image of the unit square under the
linear transformation represented by $A$. Additionally, if $\det A >
0$, the transformation only stretches and shrinks the unit squares. If
$\det A<0$, the transformation also reflects the unit square over the
line $x = -y$. Similarly, in higher dimensions you can consider the
image of the unit cube (in 3d) or unit hypercube (i.e.\ higher
dimensional cube). This image will be a higher-dimension analog of a
parallelogram, and you could calculate its volume. It will turn out
that $|\det A|$ is the volume of the hyper-parallelogram.\footnote{Not
  a standard name.}
\begin{figure} \caption{Transformed unit square and unit cube}
  \begin{tabular}{cc}
    \includegraphics[width=0.45\linewidth]{det2} &
    \includegraphics[width=0.45\linewidth]{det3} 
  \end{tabular}
\end{figure}

\begin{definition}\label{deta}
  Let $A$ be an $n$ by $n$ matrix consisting of column vectors $a_1,
  ..., a_n$. The determinant of $A$ is the unique function such that
  \begin{enumerate}
  \item\label{d1} $\det I_n = 1$.
  \item\label{d2} As a function of the columnes, $\det$ is an
    alternating form: 
    $\det (A) = 0$ iff $a_1, ..., a_n$ are linearly dependent.
  \item\label{d3} As a function of the columnes, $\det$ is multi-linear:
    \[
    \det(a_1, ..., b a_j + c v, ..., a_n) = b\det(A) +
    c\det(a_1,...,v,...a_n) 
    \]
  \end{enumerate}
\end{definition}
Condition (\ref{d1}) seems very natural. Also, if we want $\det A$ to
be the volume of the transformed unit cube, we must have $\det I_n =
1$. (\ref{d2}) ensures that $\det A = 0$ whenever $A$ is not
invertible. 
\begin{lemma}
  Let $A$ be an $n$ by $n$ matrix. The $A$ is singular if and only if
  the columns of $A$ are linearly dependent.
\end{lemma}
\begin{proof}
  If $A$ is singular, then $\rank A < n$. Also $\rank A^T = \rank A$. 
  Therefore there must exist row operations such that the last row of
  $A^T$ is all $0$'s. Row operations are interchanging rows, rescaling
  by a scalar, and adding multiples of one row to another. Therefore,
  there must be $c_1, ..., c_n \in \R$ such that
  \[ c_1 a_1 + ... + c_n a_n = 0 \]
  where $a_i$ are the rows of $A^T$, which are also the columns of
  $A$.

  Similarly if the columns of $A$ are linearly dependent, then there
  are $c_1, ..., c_n \in \R$ such that
  \[ c_1 a_1 + ... + c_n a_n = 0. \]
  These are all row operations, so we can perform them to make the
  last row of $A^T$ all $0$'s. We can then perform Gaussian
  elimination on the first $n-1$ rows of $A^T$ to put it into row
  echelon form with the last row all $0$'s. Therefore, $\rank A<n$ and
  $A$ is singular.
\end{proof}
\begin{corollary}\label{cor:dns}
  $A$ is nonsingular if and only if $\det A \neq 0$.
\end{corollary}

The third condition on the determinant, \ref{d3}, guarantee that $\det
A$ can be interpreted as the volume of the transformed unit cube. It
is easiest to see this by thinking about a diagonal matrix. If $A$ is
diagonal with elements $a_{11}, a_{22}, ..., a_{nn}$, then volume of
the transformed unit cube will be a higher dimensional rectangle and
its volume will be $|\det A| = |\prod_{i=1}^n a_{ii} |$. If we
multiply any of the columns of $A$ by $c$, we will scale one of the
corners of the rectangle by $c$, so the volume should be multiplied by
$c$. Similarly, if we add $v$ to the $j$ diagonal element, the
volume will be 
\begin{align*}
  |a_{11}\times ...\times (a_{jj}+v)\times ... \times a_{nn} | = &
  |\prod_{i=1}^n a_{ii} | +  |a_{11}\times ...\times v \times
  ... \times a_{nn} | 
\end{align*}
So, at least for diagonal matrices, if $|\det A| $ is to be the volume
of the transformed unit cube, it must be multilinear. It turns out
that this is true in general as well.

Okay, so we have defined the determinant in a way that has a nice
interpretation, but how can we calculate it? There is an explicit
formula for the determinant, but it is not so simple.  
\begin{definition}\label{detc}
  The \textbf{determinant} of a square matrix $A$ is defined
  recursively as
  \begin{enumerate}
  \item For $1$ by $1$ matrices, $\det A = a_{11}$
  \item For $n$ by $n$ matrices, 
    \[ \det A = \sum_{j=1}^n (-1)^{1+j} a_{1j} \det A_{-1,-j} \]
    where $A_{-i,-j}$ is the $n-1$ by $n-1$ matrix obtained by
    deleting the $i$th row and $j$th column of $A$.
  \end{enumerate}
\end{definition}
$\det A_{-i,-j}$ is called a \textbf{minor} of $A$. $(-1)^{i+j}\det
A_{-i,-j}$ is called a \textbf{cofactor} of $A$. The above definition
is sometimes called the expansion by cofactors of the determinant. If
you are more comfortable with this definition than with the abstract
definition (\ref{deta}) above, you can consider this formula to be the
definition of the determinant. The two defitions are the same, as
stated in the following theorem, which we will not prove.
\begin{theorem}
  The two definitions of the determinant, (\ref{deta}) and
  (\ref{detc}), are equivalent.
\end{theorem}
In some situations, the abstract definition (\ref{deta}) is more
useful. Proving corollary \ref{cor:dns} from the abstract definition
was not difficult. Proving it using the recursive definition
(\ref{detc}) is a bit tedious. On the other hand, it will be easier to
prove part of lemma \ref{lem:detprop} using the recursive definition
than the abstract definition.

Some useful properties of determinants are stated in the following lemma.
\begin{lemma}\label{lem:detprop}
  Let $A$ and $B$ be square matrices then
  \begin{enumerate}
  \item $\det A^T = \det A$
  \item $\det (AB) = (\det A) (\det B)$
  \item $\det A^{-1} = (\det A)^{-1}$
  \item Usually, $\det(A + B) \neq \det A + \det B$
  \item If $A$ is diagonal, $\det A = \prod_{i=1}^n a_{ii}$
  \item\label{dpt} If $A$ is upper or lower triangular $\det A = \prod_{i=1}^n
    a_{ii}$.  
  \end{enumerate}
\end{lemma}
\begin{proof}
  The proof of all these properties would be quite long, so we will
  only prove \ref{dpt}. We can just prove it for lower triangular
  matrices, since $\det A^T = \det A$ and the transpose of upper
  triangular matrices is lower triangular. We will use induction on
  the size of $A$. If $A$ is $1$ by $1$, the claim is true. If $A$ is
  $n$ by $n$, from definition \ref{detc},
  we have
  \[ \det A = \sum_{j=1}^n (-1)^{1+j} a_{1j} \det A_{-1,-j}. \]
  $A$ is lower triangular, so only the first term in the sum is
  nonzero.
  \[ \det A = a_{11} \det A_{-1,-1} \]
  $A_{-1,-1}$ is also be lower triangular and is size $n-1$ by $n-1$,
  so by induction, $\det A = \prod_{i=1}^n a_{ii}$. 
\end{proof}

We can use the determinant to solve systems of linear equations and to
calculate $A^{-1}$. 
\begin{theorem}
  Let $A$ be nonsingular. Then,
  \begin{enumerate}
  \item $A^{-1} = \frac{1}{\det A} \begin{pmatrix} 
      \det A_{-1,-1} & \cdots & (-1)^{1+n} \det A_{-n,-1} \\
      \vdots & \ddots & \vdots \\
      (-1)^{1+n} \det A_{-1,-n} & \cdots & (-1)^{n+n} \det A_{-n,-n} 
    \end{pmatrix}$
  \item (\textbf{Cramer's rule}) The unique solution to $A x = b$ is 
    \[ x_i = \frac{\det B_i}{\det A} \]
    where $B_i$ is the matrix $A$ with the $i$th column replaced by
    $b$. 
  \end{enumerate}
\end{theorem}
This theorem can sometimes be useful for analyzing small systems of
linear equations and doing comparative statics. For an example, see
Simon and Blume 9.3. However, it is a very inefficient way to compute
$A^{-1}$ or solve a system of equations. 

\section{Computational efficiency}
In a past year, someone asked about whether there was a more efficient
way to calculate the rank of a matrix than Gaussian elimination. The
way we usually compare the efficiency of algorithms is by looking at
the rate at which number of steps required grows as the size of the
problem increases. As an example, we will compare solving a system of
equations by Gaussian elimination to using Cramer's rule.

Let's start by considering computing the determinant
recursively as in definition \ref{detc}. Let $d(n)$ be the number of
steps required to compute the determinant for an $n$ by $n$
matrix. Given the recursive definition, $d(n)$ must satisfy
\begin{align*}
  d(n) = n d(n-1) + 2n 
\end{align*}
since $\det A_{-1,-i}$ must be computed $n$ times, and there are $n$
additions and multiplications. We could show by induction that
\begin{align*}
  d(n) = 2n! \sum_{k=1}^n \frac{1}{(n-k)!} .
\end{align*}
When comparing algorithms, we often mainly care about the number of
steps for large $n$. We say $d(n)$ is on the order of $f(n)$ or $d(n)$
is big O of $f(n)$ and write $d(n) = O(f(n))$ if $\exists n_0$ such
that
\[
  d(n) \leq M f(n) 
\]
for some constant $M$ and all $n \geq n_0$. From the above, $d(n) =
O(n!)$. Factorial grows very fast, so the determinant is quite
expensive to compute using the recursive definition for large
matrices.\footnote{Any reasonable computer program will not be
  computing determinants this way. It will compute some matrix
  decomposition (LU, QR, or Cholesky, or less likely, SVD) and then
  use that to compute the determinant.} To actually solve a system of
equations using Cramer's rule, we would have to compute $n+1$
determinants, so Cramer's rule is $O((n+1)!)$.

Now lets look at Gaussian elimination. Let $g(n)$ be the maximal
number of steps Gaussian elimination takes on an $n$ by $n$ matrix. In
the worst case, Gaussian elimination on the first row involves multiplying
the first row by $n-1$ numbers, so there are $n(n-1)$ multiplications,
and adding it to each of the $n-1$ rows, so there are $n(n-1)$
additions. The total number of steps to get to the next $n-1$ by $n-1$
submatrix is then $2n(n-1)$. Hence, $g(n)$ is given by
\[ g(n) = 2n(n-1) + g(n-1) \]
and,
\begin{align*}
  g(n) = & 2 \sum_{k=1}^n k(k-1) \\
  = & \frac{2}{3} (n^3 - n)
\end{align*}
Therefore, $g(n) = O(n^3)$. To solve a system of equations, we also
have to back substitute. The first substitution involes one division,
and then $n-1$ additions. The second involves one division and $n-2$
additions. In total there will be $\sum_{k=1}^n k = \frac{1}{2}
n(n-1)$ steps. So solving a system by Gaussian elimination takes 
$\frac{2}{3} (n^3 - n) + \frac{1}{2}(n^2 - n) = O(n^3)$ steps. It is
much faster than Cramer's rule. 

\section{Matrix decompositions}

There are number of matrix decompositions that are useful for
computing solutions to systems of linear equations, inverting
matrices, and computing determinants. 

\begin{definition}
  The \textbf{LU decomposition} of a square matrix is a decomposition
  of the form
  \[ P A = L U \]
  where $P$ is a permutation matrix, $L$ is lower triangular, and $U$
  is upper triangular. 
\end{definition}
We already know how to compute an LU decomposition by performing
Gaussian elimination. All row interchange operations can be
represented by the permutation matrix $P$. Adding multiples of one row to
another are represented by $L$, and the resulting row echelon form is
$U$. The LU decomposition exists for any nonsingular matrix. Sometimes
it is also required that the diagonal of $L$ by all $1$'s. In that
case, the LU decomposition is unique. The LU decomposition is used to
solve linear equations, compute $A^{-1}$, and compute $\det A$.  

\begin{definition}
  The \textbf{Cholesky decomposition} of a square, symmetric, and
  positive definite ($\det A > 0$) matrix $A$ is
  \[ A = L L^T \] 
  where $L$ is lower triangular with positive entries on the
  diagonal. 
\end{definition}
The Cholesky decomposition takes half as many steps to compute than
the LU decomposition and can be used for all the same
purposes. Symmetric positive definite matrices are quite common in
economics. Examples include covariance matrices and the matrix of
second derivatives of a convex minimization problem. 

\begin{definition}
  The \textbf{QR decomposition} of an $m$ by $n$ matrix is 
  \[ A = QR \]
  where $R$ is $m$ by $n$ and upper triangular and $Q$ is $m$ by $m$
  and orthogonal (i.e.\ $Q^T = Q^{-1}$).
\end{definition}
The QR decomposition can be used for all the things the LU
decomposition can be used for. The QR decomposition takes twice as
many steps to compute, but can be computed more accurately (which can
matter for very large matrices, but usually isn't important for small
ones).  The QR decomposition is also used for solving least square
problems. We will (likely) learn how to calculate the QR decomposition
and learn an interpretation of $Q$ next lecture. 

\begin{definition}
  The \textbf{singular value decomposition} of a matrix $m$ by $n$
  matrix $A$ is
  \[ A = U \Sigma V \]
  where $U$ is $m$ by $m$ and orthogonal, $\Sigma$ is $m$ by $n$ and
  diagonal, and $V$ is $n$ by $n$ and orthogonal.
\end{definition}
We will study the singular value decomposition in more detail next
week. 


\end{document}
