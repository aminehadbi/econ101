\documentclass[compress]{beamer} 
\usepackage{amsmath}
\usepackage{amsfonts}
\usepackage{graphicx}
%\usepackage{epstopdf}
\usepackage{hyperref}
\usepackage{multirow}
\usepackage{verbatim}
%\usepackage[small,compact]{titlesec} 
%\usecolortheme{beaver}
\usetheme{Hannover}
%\usecolortheme{whale}

%\usepackage{pxfonts}
%\usepackage{isomath}
%\usepackage{mathpazo}
%\usepackage{arev} %     (Arev/Vera Sans)
%\usepackage{eulervm} %_   (Euler Math)
%\usepackage{fixmath} %  (Computer Modern)
%\usepackage{hvmath} %_   (HV-Math/Helvetica)
%\usepackage{tmmath} %_   (TM-Math/Times)
%\usepackage{tgheros}
%\usepackage{cmbright}
%\usepackage{ccfonts} \usepackage[T1]{fontenc}
%\usepackage[garamond]{mathdesign}
\usepackage{color}
\usepackage{ulem}

\setbeamertemplate{navigation symbols}{}
\AtBeginSection[] % Do nothing for \section*
{ \frame{\sectionpage} }

\newcommand{\argmax}{\operatornamewithlimits{arg\,max}}
\newcommand{\argmin}{\operatornamewithlimits{arg\,min}}
\def\inprobLOW{\rightarrow_p}
\def\inprobHIGH{\,{\buildrel p \over \rightarrow}\,} 
\def\inprob{\,{\inprobHIGH}\,} 
\def\indist{\,{\buildrel d \over \rightarrow}\,} 
\def\F{\mathbb{F}}
\def\R{\mathbb{R}}
\newcommand{\gmatrix}[1]{\begin{pmatrix} {#1}_{11} & \cdots &
    {#1}_{1n} \\ \vdots & \ddots & \vdots \\ {#1}_{m1} & \cdots &
    {#1}_{mn} \end{pmatrix}}
\newcommand{\iprod}[2]{\left\langle {#1} , {#2} \right\rangle}
\newcommand{\norm}[1]{\left\Vert {#1} \right\Vert}
\newcommand{\abs}[1]{\left\vert {#1} \right\vert}
\renewcommand{\det}{\mathrm{det}}
\newcommand{\rank}{\mathrm{rank}}
\newcommand{\spn}{\mathrm{span}}
\newcommand{\row}{\mathrm{Row}}
\newcommand{\col}{\mathrm{Col}}
\renewcommand{\dim}{\mathrm{dim}}
\newcommand{\prefeq}{\succeq}
\newcommand{\pref}{\succ}
\newcommand{\seq}[1]{\{{#1}_n \}_{n=1}^\infty }
\renewcommand{\to}{{\rightarrow}}
\renewcommand{\L}{{\mathcal{L}}}


\title{Optimal control and dynamic programming}
\author{Paul Schrimpf}
\institute{UBC \\ Economics 526}
\date{\today}

\begin{document}

\frame{\titlepage}
%\setcounter{tocdepth}{2}

\begin{frame}
  \tableofcontents  
\end{frame}

%%%%%%%%%%%%%%%%%%%%%%%%%%%%%%%%%%%%%%%%%%%%%%%%%%%%%%%%%%%%%%%%%%%%%%%%
\section{Introduction} 

\begin{frame} \frametitle{Example: consumption and savings}
  \begin{itemize}
  \item Infinite horizon consumption-savings problem, 
    \[ \max_{\{c_t\}_{t=0}^\infty,\{s_t\}_{t=1}^\infty} \sum_{t=0}^\infty
    \beta^t u(c_t) \text{ s.t. } s_{t+1} = (1+r_t)(s_t - c_t), \]  
  \item Countably infinite $c_t$ and $s_t$
  \end{itemize} 
\end{frame}

\begin{frame}
  \frametitle{Example: contracting with continuum of types}
  \begin{itemize}
  \item Type $\theta$ gets $0$ utility from not buying the good, and
    $\theta \nu(q) - T$ from buying $q$ units of the good at cost
    $T$
  \item $\theta \in [\theta_l,\theta_h]$, density $f_\theta$
  \item Menu of contracts $(q(\theta),T(\theta))$ such that type $\theta$ will
    choose contract $(q(\theta),T(\theta))$
    \begin{align}
      \max_{q(\theta),T(\theta)} & \int_{\theta_l}^{\theta_h} 
      \left[T(\theta) - cq(\theta)\right]
      f_\theta(\theta) d\theta \notag \\
      & \text{s.t.} \notag \\
      &\theta \nu\left(q(\theta)\right) - T(\theta) \geq 0  \forall
      \theta \label{pc} \\
      &\theta \nu\left(q(\theta)\right) - T(\theta) \geq
      \max_{\tilde{\theta}} \theta \nu\left(q(\tilde{\theta}) \right) -
      T(\tilde{\theta}) \forall \theta \label{ic} 
    \end{align}
  \end{itemize}
\end{frame}

\begin{frame} \frametitle{}
  \begin{itemize}
  \item These problems can be solved using same techniques as before
    \begin{itemize}
    \item First and second order conditions
    \item Involves differentiation in infinite dimensional vector
      spaces
    \item This approach sometimes called ``calculus of variations''
    \end{itemize}
  \item Optimal control and dynamic programming are special tricks to
    make solving these problems easier
    \begin{itemize}
    \item Can be derived from calculus of variations
    \end{itemize}
  \end{itemize}
\end{frame}

%%%%%%%%%%%%%%%%%%%%%%%%%%%%%%%%%%%%%%%%%%%%%%%%%%%%%%%%%%%%%%%%%%%%%%%%
\section{Differentiation in vector spaces}

\begin{frame}\frametitle{Differentiation in vector spaces}
  \begin{align*}
    \max_{x \in U} f(x) \text{ s.t } & h(x) = c \\
    & g(x) \leq b 
  \end{align*}
  \begin{itemize}
  \item $x \in U \subseteq V$, a Banach space
    \begin{itemize}
    \item Countably infinite problems, usually 
      \[ V = \ell^p = \{\{x_t\}_{t=1}^\infty:
      \left(\sum_{t=1}^\infty |x_t|^p\right)^{1/p} < \infty\} \]      
    \item Uncountably infinite problems, usually $V = \mathcal{L}^p$
      or $V = C^k$
    \end{itemize}
  \item $f: V \to \R$, derivative $Df_x: V \to \R$ is continuous and
    linear such that
    \[ \lim_{\norm{h} \to 0} \frac{\norm{f(x+h) - f(x) - Df_x h}}{\norm{h}}
    = 0 \]    
  \end{itemize}
\end{frame}

\begin{frame}
  \begin{definition}
    Let $V$ and $W$ be Banach spaces and $f:V \to W$. The
    \textbf{Fr\'{e}chet derivative} of $f$ at $x \in V$ is a continuous linear
    transformation from $V$ to $W$, denoted $Df_x$ such that 
    \[ \lim_{\norm{h} \to 0} \frac{\norm{f(x+h) - f(x) - Df_x h
      }}{\norm{h}} = 0. \]
  \end{definition}
  
  \begin{definition}
    The \textbf{directional (G\^{a}teaux) derivative} in direction $v$
    at $x$ is
    \begin{align*}
      df(x;v) = \lim_{\alpha \to 0} \frac{f(x + \alpha v) - f(x)}{\alpha}.
    \end{align*}  
    where $\alpha \in \R$ is a scalar.
  \end{definition}

\end{frame}

\begin{frame}
  \begin{lemma}\label{lem:fregat}
    If $f: V \to W$ is Fr\'{e}chet differentiable at $x$, then the
    G\^{a}teaux derivative, $df(x;v)$, exists for all $v \in V$, and
    \[ df(x;v) = Df_x v. \]
  \end{lemma}

  \begin{lemma}\label{lem:gatfre}
    If $f: V \to W$ has G\^{a}teaux derivatives that are linear in $v$
    and ``continuous'' in $x$ in the sense that $\forall \epsilon>0$
    $\exists \delta > 0$ such that if $\norm{x_1 - x} < \delta$, then
    \begin{align*}
      \sup_{v \in V} \frac{\norm{df(x_1;v) - df(x;v)}}{\norm{v}} < \epsilon
    \end{align*}
    then $f$ is Fr\'{e}chet differentiable with $Df_{x_0} v = df(x;v)$.
  \end{lemma}
\end{frame}

\begin{frame}\frametitle{Dual spaces}
  Lagrange multipliers are elements of dual spaces
  \begin{definition}
    Let $V$ be a vector space. The \textbf{dual space} of $V$, denote $V^\ast$
    is the set of all (continuous) linear functionals, $v^\ast: V \to \R$.
  \end{definition}
  Examples
  \begin{itemize}
  \item $(\R^n)^\ast = \R^n$
  \item $(\ell^p)^\ast = \ell^q$ where $\frac{1}{p} + \frac{1}{q} = 1$
  \item $(\L^p)^\ast = \L^q$ where $\frac{1}{p} + \frac{1}{q} = 1$
  \item $(\ell^1)^\ast = \ell^\infty$, but $(\ell^\infty)^\ast \supset
    \ell^1$, not equal
  \end{itemize}
\end{frame}

%%%%%%%%%%%%%%%%%%%%%%%%%%%%%%%%%%%%%%%%%%%%%%%%%%%%%%%%%%%%%%%%%%%%%%%%
\section{Optimization in vector spaces}

\begin{frame} 
  \begin{theorem}[First order condition for maximization with equality
    constraints] \label{thm:econv} Let $f:U \to \R$ and $h:U \to W$ be
    continuously differentiable on $U \subseteq V$, where $V$ and $W$
    are Banach spaces.  Suppose $x^* \in \mathrm{interior}(U)$ is a
    local maximizer of $f$ on $U$ subject to $h(x) = 0$. Suppose that
    $Dh_{x^*}:V \to W$ is onto. Then, there exists $\mu^* \in W^\ast$
    such that for
    \[ L(x,\mu) = f(x) - \mu h(x). \]
    we have
    \begin{align*}
      D_xL(x^*,\mu^*) = & Df_{x*} - \mu^* Dh_{x^*} = 0_{} \\
      D_\mu L(x^*,\mu^*) = & h(x^*) = 0_{W}
    \end{align*}
  \end{theorem}
\end{frame}

%%%%%%%%%%%%%%%%%%%%%%%%%%%%%%%%%%%%%%%%%%%%%%%%%%%%%%%%%%%%%%%%%%%%%%%%
\section{Optimal control}

%%%%%%%%%%%%%%%%%%%%%%%%%%%%%%%%%%%%%%%%%%%%%%%%%%%%%%%%%%%%%%%%%%%%%%%%
\subsection{Continous time optimal control}
\begin{frame}[shrink] 
  \frametitle{Optimal control}
  Classic continuous time optimal control problem:
  \begin{align*}
    \max_{x(t),y(t)} & \int_0^T F(x(t),y(t),t) dt \\
    & \text{ s.t.} \\
    & \frac{d y}{dt} = g(x(t),y(t),t) \forall t \in [0,T] \\ 
    & y(0) = y_0
  \end{align*}
  \begin{itemize}
  \item $x$ is called a control variable
  \item $y$ is called a state variable
  \item $V =$ space of pairs of functions $(x(t),y(t))$ from
    $[0,T]$ to $\R$
  \item Objective $f(x,y) = \int_0^T F(x(t),y(t),t) dt$, $f:V \to \R$
  \item Constraint $h(x,y): V \to W = $ functions from $[0,T]$ to $\R$
    \[ h(x,y)(t) = \frac{dy}{dt}(t) - g(x(t),y(t),t) \]
  \item Multipliers $\mu \in W^\ast$, $\mu_0 \in \R^\ast = \R$
  \end{itemize}
\end{frame}

\begin{frame}
  \begin{align*}
    \max_{x(t),y(t)} \int_0^T & F(x(t),y(t),t) dt \\
    \text{ s.t.} & \frac{d y}{dt} = g(x(t),y(t),t) \forall t \in [0,T] \\ 
    & y(0) = y_0
  \end{align*}
  \begin{itemize}
  \item Lagrangian: 
    \begin{align*}
      L(x,y,\mu,\mu_0) = & \int_0^T F(x(t),y(t),t) dt  -
      \mu\left(\frac{dy}{dt} - g(x,y,\cdot) \right) -\\ 
      & - \mu_0(y(0) - y_0) 
    \end{align*}
  \item Assume that $\mu(w) = \int_{0}^T w(t) \lambda(t) dt$
  \item First order conditions are
    \begin{align}
      [x]: && D_xf_{x^*,y^*} - \mu D_xh_{x^*,y^*} = & 0 \\
      [y]: && D_yf_{x^*,y^*} - \mu D_yh_{x^*,y^*} - \mu_0 = & 0 \\
      [\mu]: && h(x^*,y*) = & 0 
    \end{align}
  \end{itemize}
\end{frame}

\begin{frame}[shrink]
  \begin{itemize}
  \item Rewrite first order conditions as 
    {\small
    \begin{align*}
      [x]:\, 0 =& \int_0^T v(t) \left(\frac{\partial F}{\partial x}(x(t),y(t),t) +
        \frac{\partial g}{\partial x}(x(t),y(t),t) \lambda(t)\right)
      dt  \\
      [y]:\, 0 = & \int_0^T v(t) \left(\frac{\partial F}{\partial
          y} + \frac{\partial g}{\partial y}
        \lambda(t)  \right)dt  + \\
      & + \int_0^T \frac{d\lambda}{dt}(t)v(t)dt -
      \lambda(T)v(T) + \lambda(0)v(0)  - \mu_0v(0) \\
      [\mu]:\, \frac{dy}{dt} = & g(x(t),y(t),t)  
    \end{align*}    
  }
  \item Equivalently, $\forall t \in [0,T]$
    {\small
    \begin{align*}
      [x]:&& 0 = &\frac{\partial F}{\partial x}(x(t),y(t),t) +
      \frac{\partial g}{\partial x}(x(t),y(t),t) \lambda(t) \\
      [y]:&& -\frac{d\lambda}{dt}(t) = & \frac{\partial F}{\partial
        y}(x(t),y(t),t) + \frac{\partial g}{\partial y}(x(t),y(t),t)
      v(t) \lambda(t) \\
      [\mu]:&& \frac{dy}{dt} = & g(x(t),y(t),t) 
    \end{align*}
    }
    and $\lambda(T) = 0$ and $\mu_0 = \lambda(0)$
  \end{itemize}
\end{frame}

\begin{frame}
  \begin{theorem}[Pontryagin's maximum principle]\label{thm:optcon}
    Consider 
    \begin{align}
      \max_{x,y \in U \subseteq X \times Y} & \int_0^T F(x(t),y(t),t) dt \notag \\
      & \text{ s.t.} \\
      &  \frac{d y}{dt} = g(x(t),y(t),t) \forall t \in
      [0,T] \label{e:maxp} \\ 
      & y(0) = y_0. \notag
    \end{align}
    where $X$ and $Y$ are some Banach spaces of differentiable 
    functions from $[0,T]$ to $\R$, and $F,g:\R^2 \times [0,T] \to \R$ are
    continuously differentiable. Define the Hamiltonian as 
    \[ H(x,y,\lambda,t) = F(x(t),y(t),t) + \lambda(t) g(x(t),y(t),t). \]
    If $x^*$ and $y^*$ are a local constrained maximum of
    (\ref{e:maxp}) in the interior of $U$, then there exists
    $\lambda^*(t)$ such that 
    \begin{align*}
      [x]: && 0 = & \frac{\partial H}{\partial x}(x^*,y^*,\lambda^*,t)
      \\
      [y]: && -\frac{d\lambda}{dt}(t) = & \frac{\partial H}{\partial y}(x^*,y^*,\lambda^*,t) \\
      [\lambda]: && \frac{dy}{dt}(t) = & \frac{\partial H}{\partial
        \lambda}(x^*,y^*,\lambda^*,t)
    \end{align*}
  \end{theorem}
\end{frame}

%%%%%%%%%%%%%%%%%%%%%%%%%%%%%%%%%%%%%%%%%%%%%%%%%%%%%%%%%%%%%%%%%%%%%%%% 
\subsection{Application: optimal contracting with a continuum of
  types} 

\begin{frame}\frametitle{Application: optimal contracting with a continuum of
    types}
  \begin{align}
    \max_{q(\theta),T(\theta)} & \int_{\theta_l}^{\theta_h} 
    \left[T(\theta) - cq(\theta)\right]
    f_\theta(\theta) d\theta \notag \\
    & \text{s.t.} \notag \\
    &\theta \nu\left(q(\theta)\right) - T(\theta) \geq 0  \forall
    \theta \label{pc2} \\
    &\theta \nu\left(q(\theta)\right) - T(\theta) \geq
    \max_{\tilde{\theta}} \theta \nu\left(q(\tilde{\theta}) \right) -
    T(\tilde{\theta}) \forall \theta \label{ic2} 
  \end{align}
\end{frame}

\begin{frame}
  \begin{itemize}
  \item Can show equivalent to
    \begin{align}
      \max_{q(\theta),T(\theta)} & \int_{\theta_l}^{\theta_h} 
      \left[T(\theta) - cq(\theta)\right]
      f_\theta(\theta) d\theta \notag \\
      & \text{s.t.} \notag \\
      &\theta_l \nu\left(q(\theta_l)\right) - T(\theta_l) \geq 0
      \label{pcl} \\
      & \theta \nu'(q(\theta))q'(\theta) - T'(\theta) =  0 \label{lic2} \\
      & \frac{dq(\theta)}{d\theta} \geq  0 \label{mon2}
    \end{align}
  \item Manipulate first order conditions to eventually show
    \begin{align*}
      \left(\theta - \frac{1-F_\theta(\theta)}{f_\theta(\theta)}
      \right)\nu'(q(\theta)) = & c 
    \end{align*}
  \end{itemize}
\end{frame}
 
%%%%%%%%%%%%%%%%%%%%%%%%%%%%%%%%%%%%%%%%%%%%%%%%%%%%%%%%%%%%%%%%%%%%%%%%
\subsection{Discrete time optimal control}

\begin{frame}
  \frametitle{Discrete time optimal control}
  \begin{align*}
    \max_{x_t,y_y} & \sum_{t=0}^\infty F(x_t,y_t,t) \\
    & \text{ s.t. } \\
    & y_{t+1} - y_t = g(x_t,y_t,t) 
  \end{align*}
  First order conditions can be written as:
  \begin{align*}
    [x]: && \frac{\partial F}{\partial x}(x_t,y_t,t)  + \mu_t \frac{\partial
      g}{\partial x}(x_t,y_t,t) = &  0 \\
    [y]: && \mu_{t+1}-\mu_t = \frac{\partial F}{\partial
      y}(x_t,y_t,t)  + \mu_t \frac{\partial 
      g}{\partial y}(x_t,y_t,t) = &  0 \\
    [\mu]:&& y_{t+1} - y_t = & g(x_t,y_t,t) 
  \end{align*}
  and $y_0 = 0$
  \begin{itemize}
  \item Same as continuous time, but with $\mu_{t+1} - \mu_t$ instead
    of $\frac{d\lambda}{dt}$ and $y_{t+1}-y_t$ instead of
    $\frac{dy}{dt}$
  \end{itemize}
\end{frame}

%%%%%%%%%%%%%%%%%%%%%%%%%%%%%%%%%%%%%%%%%%%%%%%%%%%%%%%%%%%%%%%%%%%%%%%% 
\section{Dynamic programming}

%\begin{frame}\frametitle{Dynamic programming}  
%\end{frame}

\end{document}
