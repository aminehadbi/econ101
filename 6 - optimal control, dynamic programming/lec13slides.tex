\documentclass[compress]{beamer} 
\usepackage{amsmath}
\usepackage{amsfonts}
\usepackage{graphicx}
%\usepackage{epstopdf}
\usepackage{hyperref}
\usepackage{multirow}
\usepackage{verbatim}
%\usepackage[small,compact]{titlesec} 
%\usecolortheme{beaver}
\usetheme{Hannover}
%\usecolortheme{whale}

%\usepackage{pxfonts}
%\usepackage{isomath}
%\usepackage{mathpazo}
%\usepackage{arev} %     (Arev/Vera Sans)
%\usepackage{eulervm} %_   (Euler Math)
%\usepackage{fixmath} %  (Computer Modern)
%\usepackage{hvmath} %_   (HV-Math/Helvetica)
%\usepackage{tmmath} %_   (TM-Math/Times)
%\usepackage{tgheros}
%\usepackage{cmbright}
%\usepackage{ccfonts} \usepackage[T1]{fontenc}
%\usepackage[garamond]{mathdesign}
\usepackage{color}
\usepackage{ulem}

\setbeamertemplate{navigation symbols}{}
\AtBeginSection[] % Do nothing for \section*
{ \frame{\sectionpage} }

\newcommand{\argmax}{\operatornamewithlimits{arg\,max}}
\newcommand{\argmin}{\operatornamewithlimits{arg\,min}}
\def\inprobLOW{\rightarrow_p}
\def\inprobHIGH{\,{\buildrel p \over \rightarrow}\,} 
\def\inprob{\,{\inprobHIGH}\,} 
\def\indist{\,{\buildrel d \over \rightarrow}\,} 
\def\F{\mathbb{F}}
\def\R{\mathbb{R}}
\newcommand{\gmatrix}[1]{\begin{pmatrix} {#1}_{11} & \cdots &
    {#1}_{1n} \\ \vdots & \ddots & \vdots \\ {#1}_{m1} & \cdots &
    {#1}_{mn} \end{pmatrix}}
\newcommand{\iprod}[2]{\left\langle {#1} , {#2} \right\rangle}
\newcommand{\norm}[1]{\left\Vert {#1} \right\Vert}
\newcommand{\abs}[1]{\left\vert {#1} \right\vert}
\renewcommand{\det}{\mathrm{det}}
\newcommand{\rank}{\mathrm{rank}}
\newcommand{\spn}{\mathrm{span}}
\newcommand{\row}{\mathrm{Row}}
\newcommand{\col}{\mathrm{Col}}
\renewcommand{\dim}{\mathrm{dim}}
\newcommand{\prefeq}{\succeq}
\newcommand{\pref}{\succ}
\newcommand{\seq}[1]{\{{#1}_n \}_{n=1}^\infty }
\renewcommand{\to}{{\rightarrow}}
\renewcommand{\L}{{\mathcal{L}}}
\newcommand{\Er}{\mathrm{E}}

\title{Optimal control and dynamic programming}
\author{Paul Schrimpf}
\institute{UBC \\ Economics 526}
\date{\today}

\begin{document}

\frame{\titlepage}
%\setcounter{tocdepth}{2}

\begin{frame}
  \tableofcontents  
\end{frame}

%%%%%%%%%%%%%%%%%%%%%%%%%%%%%%%%%%%%%%%%%%%%%%%%%%%%%%%%%%%%%%%%%%%%%%%%
\section{Introduction} 

\begin{frame} \frametitle{Example: consumption and savings}
  \begin{itemize}
  \item Infinite horizon consumption-savings problem, 
    \[ \max_{\{c_t\}_{t=0}^\infty,\{s_t\}_{t=1}^\infty} \sum_{t=0}^\infty
    \beta^t u(c_t) \text{ s.t. } s_{t+1} = (1+r_t)(s_t - c_t), \]  
  \item Countably infinite $c_t$ and $s_t$
  \end{itemize} 
\end{frame}

\begin{frame}
  \frametitle{Example: contracting with continuum of types}
  \begin{itemize}
  \item Type $\theta$ gets $0$ utility from not buying the good, and
    $\theta \nu(q) - T$ from buying $q$ units of the good at cost
    $T$
  \item $\theta \in [\theta_l,\theta_h]$, density $f_\theta$
  \item Menu of contracts $(q(\theta),T(\theta))$ such that type $\theta$ will
    choose contract $(q(\theta),T(\theta))$
    \begin{align}
      \max_{q(\theta),T(\theta)} & \int_{\theta_l}^{\theta_h} 
      \left[T(\theta) - cq(\theta)\right]
      f_\theta(\theta) d\theta \notag \\
      & \text{s.t.} \notag \\
      &\theta \nu\left(q(\theta)\right) - T(\theta) \geq 0  \forall
      \theta \label{pc} \\
      &\theta \nu\left(q(\theta)\right) - T(\theta) \geq
      \max_{\tilde{\theta}} \theta \nu\left(q(\tilde{\theta}) \right) -
      T(\tilde{\theta}) \forall \theta \label{ic} 
    \end{align}
  \end{itemize}
\end{frame}

\begin{frame} \frametitle{}
  \begin{itemize}
  \item These problems can be solved using same techniques as before
    \begin{itemize}
    \item First and second order conditions
    \item Involves differentiation in infinite dimensional vector
      spaces
    \item This approach sometimes called ``calculus of variations''
    \end{itemize}
  \item Optimal control and dynamic programming are special tricks to
    make solving these problems easier
    \begin{itemize}
    \item Can be derived from calculus of variations
    \end{itemize}
  \end{itemize}
\end{frame}

%%%%%%%%%%%%%%%%%%%%%%%%%%%%%%%%%%%%%%%%%%%%%%%%%%%%%%%%%%%%%%%%%%%%%%%%
\section{Differentiation in vector spaces}

\begin{frame}\frametitle{Differentiation in vector spaces}
  \begin{align*}
    \max_{x \in U} f(x) \text{ s.t } & h(x) = c \\
    & g(x) \leq b 
  \end{align*}
  \begin{itemize}
  \item $x \in U \subseteq V$, a Banach space
    \begin{itemize}
    \item Countably infinite problems, usually 
      \[ V = \ell^p = \{\{x_t\}_{t=1}^\infty:
      \left(\sum_{t=1}^\infty |x_t|^p\right)^{1/p} < \infty\} \]      
    \item Uncountably infinite problems, usually $V = \mathcal{L}^p$
      or $V = C^k$
    \end{itemize}
  \item $f: V \to \R$, derivative $Df_x: V \to \R$ is continuous and
    linear such that
    \[ \lim_{\norm{h} \to 0} \frac{\norm{f(x+h) - f(x) - Df_x h}}{\norm{h}}
    = 0 \]    
  \end{itemize}
\end{frame}

\begin{frame}
  \begin{definition}
    Let $V$ and $W$ be Banach spaces and $f:V \to W$. The
    \textbf{Fr\'{e}chet derivative} of $f$ at $x \in V$ is a continuous linear
    transformation from $V$ to $W$, denoted $Df_x$ such that 
    \[ \lim_{\norm{h} \to 0} \frac{\norm{f(x+h) - f(x) - Df_x h
      }}{\norm{h}} = 0. \]
  \end{definition}
  
  \begin{definition}
    The \textbf{directional (G\^{a}teaux) derivative} in direction $v$
    at $x$ is
    \begin{align*}
      df(x;v) = \lim_{\alpha \to 0} \frac{f(x + \alpha v) - f(x)}{\alpha}.
    \end{align*}  
    where $\alpha \in \R$ is a scalar.
  \end{definition}

\end{frame}

\begin{frame}
  \begin{lemma}\label{lem:fregat}
    If $f: V \to W$ is Fr\'{e}chet differentiable at $x$, then the
    G\^{a}teaux derivative, $df(x;v)$, exists for all $v \in V$, and
    \[ df(x;v) = Df_x v. \]
  \end{lemma}

  \begin{lemma}\label{lem:gatfre}
    If $f: V \to W$ has G\^{a}teaux derivatives that are linear in $v$
    and ``continuous'' in $x$ in the sense that $\forall \epsilon>0$
    $\exists \delta > 0$ such that if $\norm{x_1 - x} < \delta$, then
    \begin{align*}
      \sup_{v \in V} \frac{\norm{df(x_1;v) - df(x;v)}}{\norm{v}} < \epsilon
    \end{align*}
    then $f$ is Fr\'{e}chet differentiable with $Df_{x_0} v = df(x;v)$.
  \end{lemma}
\end{frame}

\begin{frame}\frametitle{Dual spaces}
  Lagrange multipliers are elements of dual spaces
  \begin{definition}
    Let $V$ be a vector space. The \textbf{dual space} of $V$, denote $V^\ast$
    is the set of all (continuous) linear functionals, $v^\ast: V \to \R$.
  \end{definition}
  Examples
  \begin{itemize}
  \item $(\R^n)^\ast = \R^n$
  \item $(\ell^p)^\ast = \ell^q$ where $\frac{1}{p} + \frac{1}{q} = 1$
  \item $(\L^p)^\ast = \L^q$ where $\frac{1}{p} + \frac{1}{q} = 1$
  \item $(\ell^1)^\ast = \ell^\infty$, but $(\ell^\infty)^\ast \supset
    \ell^1$, not equal
  \end{itemize}
\end{frame}

%%%%%%%%%%%%%%%%%%%%%%%%%%%%%%%%%%%%%%%%%%%%%%%%%%%%%%%%%%%%%%%%%%%%%%%%
\section{Optimization in vector spaces}

\begin{frame} 
  \begin{theorem}[First order condition for maximization with equality
    constraints] \label{thm:econv} Let $f:U \to \R$ and $h:U \to W$ be
    continuously differentiable on $U \subseteq V$, where $V$ and $W$
    are Banach spaces.  Suppose $x^* \in \mathrm{interior}(U)$ is a
    local maximizer of $f$ on $U$ subject to $h(x) = 0$. Suppose that
    $Dh_{x^*}:V \to W$ is onto. Then, there exists $\mu^* \in W^\ast$
    such that for
    \[ L(x,\mu) = f(x) - \mu h(x). \]
    we have
    \begin{align*}
      D_xL(x^*,\mu^*) = & Df_{x*} - \mu^* Dh_{x^*} = 0_{} \\
      D_\mu L(x^*,\mu^*) = & h(x^*) = 0_{W}
    \end{align*}
  \end{theorem}
\end{frame}

%%%%%%%%%%%%%%%%%%%%%%%%%%%%%%%%%%%%%%%%%%%%%%%%%%%%%%%%%%%%%%%%%%%%%%%%
\section{Optimal control}

%%%%%%%%%%%%%%%%%%%%%%%%%%%%%%%%%%%%%%%%%%%%%%%%%%%%%%%%%%%%%%%%%%%%%%%%
\subsection{Continous time optimal control}
\begin{frame}[shrink] 
  \frametitle{Optimal control}
  Classic continuous time optimal control problem:
  \begin{align*}
    \max_{x(t),y(t)} & \int_0^T F(x(t),y(t),t) dt \\
    & \text{ s.t.} \\
    & \frac{d y}{dt} = g(x(t),y(t),t) \forall t \in [0,T] \\ 
    & y(0) = y_0
  \end{align*}
  \begin{itemize}
  \item $x$ is called a control variable
  \item $y$ is called a state variable
  \item $V =$ space of pairs of functions $(x(t),y(t))$ from
    $[0,T]$ to $\R$
  \item Objective $f(x,y) = \int_0^T F(x(t),y(t),t) dt$, $f:V \to \R$
  \item Constraint $h(x,y): V \to W = $ functions from $[0,T]$ to $\R$
    \[ h(x,y)(t) = \frac{dy}{dt}(t) - g(x(t),y(t),t) \]
  \item Multipliers $\mu \in W^\ast$, $\mu_0 \in \R^\ast = \R$
  \end{itemize}
\end{frame}

\begin{frame}
  \begin{align*}
    \max_{x(t),y(t)} \int_0^T & F(x(t),y(t),t) dt \\
    \text{ s.t.} & \frac{d y}{dt} = g(x(t),y(t),t) \forall t \in [0,T] \\ 
    & y(0) = y_0
  \end{align*}
  \begin{itemize}
  \item Lagrangian: 
    \begin{align*}
      L(x,y,\mu,\mu_0) = & \int_0^T F(x(t),y(t),t) dt  -
      \mu\left(\frac{dy}{dt} - g(x,y,\cdot) \right) -\\ 
      & - \mu_0(y(0) - y_0) 
    \end{align*}
  \item Assume that $\mu(w) = \int_{0}^T w(t) \lambda(t) dt$
  \item First order conditions are
    \begin{align}
      [x]: && D_xf_{x^*,y^*} - \mu D_xh_{x^*,y^*} = & 0 \\
      [y]: && D_yf_{x^*,y^*} - \mu D_yh_{x^*,y^*} - \mu_0 = & 0 \\
      [\mu]: && h(x^*,y*) = & 0 
    \end{align}
  \end{itemize}
\end{frame}

\begin{frame}[shrink]
  \begin{itemize}
  \item Rewrite first order conditions as 
    {\small
    \begin{align*}
      [x]:\, 0 =& \int_0^T v(t) \left(\frac{\partial F}{\partial x}(x(t),y(t),t) +
        \frac{\partial g}{\partial x}(x(t),y(t),t) \lambda(t)\right)
      dt  \\
      [y]:\, 0 = & \int_0^T v(t) \left(\frac{\partial F}{\partial
          y} + \frac{\partial g}{\partial y}
        \lambda(t)  \right)dt  + \\
      & + \int_0^T \frac{d\lambda}{dt}(t)v(t)dt -
      \lambda(T)v(T) + \lambda(0)v(0)  - \mu_0v(0) \\
      [\mu]:\, \frac{dy}{dt} = & g(x(t),y(t),t)  
    \end{align*}    
  }
  \item Equivalently, $\forall t \in [0,T]$
    {\small
    \begin{align*}
      [x]:&& 0 = &\frac{\partial F}{\partial x}(x(t),y(t),t) +
      \frac{\partial g}{\partial x}(x(t),y(t),t) \lambda(t) \\
      [y]:&& -\frac{d\lambda}{dt}(t) = & \frac{\partial F}{\partial
        y}(x(t),y(t),t) + \frac{\partial g}{\partial y}(x(t),y(t),t)
      v(t) \lambda(t) \\
      [\mu]:&& \frac{dy}{dt} = & g(x(t),y(t),t) 
    \end{align*}
    }
    and $\lambda(T) = 0$ and $\mu_0 = \lambda(0)$
  \end{itemize}
\end{frame}

\begin{frame}
  \begin{theorem}[Pontryagin's maximum principle]\label{thm:optcon}
    Consider 
    \begin{align}
      \max_{x,y \in U \subseteq X \times Y} & \int_0^T F(x(t),y(t),t) dt \notag \\
      & \text{ s.t.} \\
      &  \frac{d y}{dt} = g(x(t),y(t),t) \forall t \in
      [0,T] \label{e:maxp} \\ 
      & y(0) = y_0. \notag
    \end{align}
    where $X$ and $Y$ are some Banach spaces of differentiable 
    functions from $[0,T]$ to $\R$, and $F,g:\R^2 \times [0,T] \to \R$ are
    continuously differentiable. Define the Hamiltonian as 
    \[ H(x,y,\lambda,t) = F(x(t),y(t),t) + \lambda(t) g(x(t),y(t),t). \]
    If $x^*$ and $y^*$ are a local constrained maximum of
    (\ref{e:maxp}) in the interior of $U$, then there exists
    $\lambda^*(t)$ such that 
    \begin{align*}
      [x]: && 0 = & \frac{\partial H}{\partial x}(x^*,y^*,\lambda^*,t)
      \\
      [y]: && -\frac{d\lambda}{dt}(t) = & \frac{\partial H}{\partial y}(x^*,y^*,\lambda^*,t) \\
      [\lambda]: && \frac{dy}{dt}(t) = & \frac{\partial H}{\partial
        \lambda}(x^*,y^*,\lambda^*,t)
    \end{align*}
  \end{theorem}
\end{frame}

%%%%%%%%%%%%%%%%%%%%%%%%%%%%%%%%%%%%%%%%%%%%%%%%%%%%%%%%%%%%%%%%%%%%%%%% 
\subsection{Application: optimal contracting with a continuum of
  types} 

\begin{frame}\frametitle{Application: optimal contracting with a continuum of
    types}
  \begin{align}
    \max_{q(\theta),T(\theta)} & \int_{\theta_l}^{\theta_h} 
    \left[T(\theta) - cq(\theta)\right]
    f_\theta(\theta) d\theta \notag \\
    & \text{s.t.} \notag \\
    &\theta \nu\left(q(\theta)\right) - T(\theta) \geq 0  \forall
    \theta \label{pc2} \\
    &\theta \nu\left(q(\theta)\right) - T(\theta) \geq
    \max_{\tilde{\theta}} \theta \nu\left(q(\tilde{\theta}) \right) -
    T(\tilde{\theta}) \forall \theta \label{ic2} 
  \end{align}
\end{frame}

\begin{frame}
  \begin{itemize}
  \item Can show equivalent to
    \begin{align}
      \max_{q(\theta),T(\theta)} & \int_{\theta_l}^{\theta_h} 
      \left[T(\theta) - cq(\theta)\right]
      f_\theta(\theta) d\theta \notag \\
      & \text{s.t.} \notag \\
      &\theta_l \nu\left(q(\theta_l)\right) - T(\theta_l) \geq 0
      \label{pcl} \\
      & \theta \nu'(q(\theta))q'(\theta) - T'(\theta) =  0 \label{lic2} \\
      & \frac{dq(\theta)}{d\theta} \geq  0 \label{mon2}
    \end{align}
  \item Manipulate first order conditions to eventually show
    \begin{align*}
      \left(\theta - \frac{1-F_\theta(\theta)}{f_\theta(\theta)}
      \right)\nu'(q(\theta)) = & c 
    \end{align*}
  \end{itemize}
\end{frame}
 
%%%%%%%%%%%%%%%%%%%%%%%%%%%%%%%%%%%%%%%%%%%%%%%%%%%%%%%%%%%%%%%%%%%%%%%%
\subsection{Discrete time optimal control}

\begin{frame}
  \frametitle{Discrete time optimal control}
  \begin{align*}
    \max_{x_t,y_y} & \sum_{t=0}^\infty F(x_t,y_t,t) \\
    & \text{ s.t. } \\
    & y_{t+1} - y_t = g(x_t,y_t,t) 
  \end{align*}
  First order conditions can be written as:
  \begin{align*}
    [x]: && \frac{\partial F}{\partial x}(x_t,y_t,t)  + \mu_t \frac{\partial
      g}{\partial x}(x_t,y_t,t) = &  0 \\
    [y]: && \mu_{t+1}-\mu_t = \frac{\partial F}{\partial
      y}(x_t,y_t,t)  + \mu_t \frac{\partial 
      g}{\partial y}(x_t,y_t,t) = &  0 \\
    [\mu]:&& y_{t+1} - y_t = & g(x_t,y_t,t) 
  \end{align*}
  and $y_0 = 0$
  \begin{itemize}
  \item Same as continuous time, but with $\mu_{t+1} - \mu_t$ instead
    of $\frac{d\lambda}{dt}$ and $y_{t+1}-y_t$ instead of
    $\frac{dy}{dt}$
  \end{itemize}
\end{frame}

%%%%%%%%%%%%%%%%%%%%%%%%%%%%%%%%%%%%%%%%%%%%%%%%%%%%%%%%%%%%%%%%%%%%%%%% 
\section{Dynamic programming}

\begin{frame}\frametitle{Dynamic programming}
  \begin{itemize}
  \item Optimal control: characterize maximizer $x^\ast$
  \item Dynamic programming: characterize maximal function value
  \end{itemize}
\end{frame}

\begin{frame}\frametitle{Finite horizon}
\begin{itemize}
\item Example: finite horizon consumption and savings
  \[ \max_{c_t,s_t} \sum_{t=0}^T \beta^t u(c_t) \text{ s.t. }
  s_{t+1} = (1+r_t) (s_t - c_t) \]
  $s_0$ given and $s_{T+1} = 0$
\item Backward induction:
  \begin{itemize}
  \item Time $T$
    \[ V_T(s_T) \equiv \max_{c_T} u(c_T) \text{ s.t. } s_{T+1} = (1+r_T)(s_T
    -c_T) = 0 \] 
  \item $T-1$
    \begin{align*}
      V_{T-1}(s_{T-1}) \equiv & \max_{c_{T-1},s'} u(c_{T-1}) + \beta
      V_T(s') \\
      \text{ s.t.} & s' =
    (1+r_{T-1})(c_{T-1}-s_{T-1}) 
  \end{align*}
\item In general, Bellman equation
    \begin{align} 
      V_t(s) = \max_{c_t,s'} u(c_{t}) + \beta V_{t+1}(s') \text{ s.t.} s' =
      (1+r_{t})(c_{t}-s)
    \end{align}
  \end{itemize}
\end{itemize}
\end{frame}

\begin{frame}
  \textbf{Principle of optimality}: 
  ``An optimal policy has the property that whatever the initial state
  and initial decision are, the remaining decisions must constitute an
  optimal policy with regard to the state resulting from the first
  decision.''  (Bellman and Dreyfus 1962) 
\end{frame}

\begin{frame}\frametitle{Example: exercising a stock option}
  \begin{itemize}
  \item Option contract: any time in next $T$ periods, can buy a share
    at price $p$
  \item Market price at time $t$ is $x_t$, $x_t = x_{t-1} +
    \epsilon_t$, $\epsilon_t$ i.i.d. with $\Er|\epsilon_t|$ finite
  \item Problem: choose selling time $s$ to maximize expected $x_s -
    p$
  \item Backward induction:
    \begin{itemize}
    \item $V_T(x_T) = \max\{ x_T - p, 0 \}$
    \item $V_{T-1}(x_{T-1}) = \max\{ x_{T-1} - p, \Er[V_T(x_T) |
      x_{T-1}]\} = \max\{ x_{T-1} - p, \Er[V_T(x_{T-1} +
      \epsilon_{T-1}) ]\}$
    \item $V_t(x) = \max\{x - p, \Er[V_{t+1}(x + \epsilon) ] \}$    
    \end{itemize}
  \end{itemize}
\end{frame}

\begin{frame}\frametitle{Example: exercising a stock option}
  \begin{itemize}
  \item Use induction to show:
    \begin{enumerate}
    \item\label{vt} If $t < \tau$ then $V_t(x) \geq V_\tau(x)$
    \item\label{vx} If $x < y$ then $V_t(x) - x \geq V_t(y) - y$
    \item\label{vc} $V_t(x)$ is continuous
    \end{enumerate}
  \item \ref{vx}, \ref{vc}, and $F(x) \geq x-p$ imply should use
    option at time $t$ if $x \geq a_t$ 
  \item \ref{vt} implies $a_1 \geq a_2 \geq ... \geq a_T$
  \end{itemize}
\end{frame}

\begin{frame}\frametitle{Infinite horizon}
  \begin{itemize}
  \item Example:
    \[ \max_{c_t,s_t} \sum_{t=0}^\infty \beta^t u(c_t) \text{ s.t. } s_{t+1} =
    (1+r_t) (s_t - c_t) \]
  \item Cannot use backward induction
  \item If problem does not depend on $t$, 
    \[ V(s) = \max_{c,s'} u(c) + \beta V(s') \text{ s.t. } s' =
    (1+r)(s-c) \] 
  \item When will $V$ not depend on $t$? Or will $\exists V$ that
    solves the above equation? And when does that $V$ match the
    original problem?
  \end{itemize}
\end{frame}

\begin{frame}\frametitle{Existence of value function}
  \begin{itemize}
  \item Example:
    \begin{align*}
      \max_{c_t,s_t} & \sum_{t=0}^\infty \beta^t u(c_t,s_t) \\
      & s_{t+1} = g(c_t,s_t),
    \end{align*}
    where $c \in \R$, $s \in \R$, $0<\beta<1$, and $u,g:\R^2 \to \R$.
  \item Finding value function:
    \begin{itemize}
    \item Guess $v_0: \R\to \R$
    \item Set 
      \[ v_1(s) = \max_{c,s'} u(c,s) + \beta v_0(s') \text{ s.t. }
      s'=g(c,s) \]
    \item Repeat: $v_j(s) = T(v_{j-1})(s)$ with 
      \[ T(v)(s) = \max_{c,s'} u(c,s) + \beta v(s') \text{ s.t. }
      s'=g(c,s) \]
    \item Recall: contraction mappings  have unique fixed points 
    \item If $T$ is a contraction, then $V$ exists and $v_j \to V$
    \end{itemize} 
  \end{itemize}
\end{frame}

\begin{frame}
  \frametitle{Existence of value function}
  \begin{itemize}
  \item Want to show $T$ is contraction mapping, i.e.\
    \[ \norm{T(v) - T(v')} \leq c \norm{v-v'} \]
    for $c < 1$
  \item Sufficient conditions:
    \begin{itemize}
    \item Define space for $v$: $u$ bounded, $u(c,s) \leq M$, then
      $v(s) \leq \frac{M}{1-\beta}$, i.e. $v \in
      \mathcal{L}^\infty(\R)$, and $\norm{v} = \sup_{x \in \R}
      \abs{v(s)}$
    \item Assure solution to 
      \begin{align*}
        \max_{c_i,s_i'} u(c_i,s) + \beta v_i(s_i') \text{ s.t. }
        s_i'=g(c_i,s).
      \end{align*}
      exists: assume $c,s'$ in compact set
    \end{itemize}
  \item Other more general conditions known
  \end{itemize}
\end{frame}

\begin{frame}[shrink]  \frametitle{Existence of value function}
  \begin{itemize}
  \item Show $T$ contraction: 
    \begin{itemize}
    \item Note:
      \begin{align*}
        T(v_0)(s)-T(v_1)(s) =  & \left(u(c_0,s) + \beta v_0(s_0')
        \right) - \\ & - 
        \left(u(c_1,s) + \beta v_1(s_1')  \right)
      \end{align*} 
      where $c_i,s_i'$ are maximizer to Bellman equation with $v_i$
    \item By maximization: 
      \[ T v_0 (s) = u(c_0,s) + \beta v_0(s_0')  \geq u(c_1,s) + \beta
      v_0(s_1') \] 
      and vice versa
    \item 
      Therefore, 
      \[ T(v_0)(s) - T(v_1)(s) \geq \beta (v_0(s_1') - v_1(s_1')) \]
      and
      \[ T(v_0)(s)-T(v_1)(s) \leq \beta (v_0(s_0') - v_1(s_0'))  \]
    \item Conclude:
      \begin{align*}
        \norm{T(v_0)-T(v_1)} = & \sup_{s} \left\vert T(v_0)(s)-T(v_1)(s) \right\vert \\
        \leq & \sup_s \left\vert \beta(v_0(s) -v_1(s)) \right\vert =
        \beta \norm{v_0 - v_1}
      \end{align*}
    \end{itemize}
  \end{itemize}
\end{frame}

\begin{frame} \frametitle{Example: linear-quadratic}
  \[ \min_{x_t,u_t} \sum_{t=0}^\infty \beta^t (x_t - a)^2 \text{
    s.t. } x_{t+1} = b x_t + c u_t \]
  \begin{itemize}
  \item Strategy: guess and check functional form of $V(x)$ 
    \begin{itemize}
    \item Does not always work, but when it does $V$ usually has same
      form as the objective function
    \item Many problems have no closed form expression for $V$
    \end{itemize}
  \item Guess $V(x) = v_0 + v_1 x + v_2 x^2$
  \item Solve $\min_u (x-a)^2 + \beta V(bx + c u)$, get $u^\ast(x) =
    \frac{v_1}{v_2} - \frac{b}{c} x$
  \item Substitute:
    \begin{align*}
      V(x) = & (x-a)^2 + \beta V(bx+cu^\ast(x)) \\
      v_0 + v_1 x + v_2 x^2 = & (x-a)^2 + \beta\left[ v_0 + \frac{c
          v_1^2}{v_2} + \frac{v_1^2 c^2}{v_2} \right]
    \end{align*}
    solve for $v_0, v_1, v_2$ (will not be possible if guess incorrect)
  \end{itemize}
\end{frame}

\begin{frame}
  \frametitle{Example: optimal growth}
  \begin{itemize}
  \item $\log$ preferences, Cobb-Douglas production
    \begin{align*}
      \max_{\{c_t\}_{t=0}^\infty} & \sum_{t=0}^\infty \delta^t
      \log(c_t) \\
      \text{s.t. } & c_t + k_{t+1} = k_t^\alpha.
    \end{align*}
  \item Bellman equation:
    \begin{align*}
      v(k) = & \max_{k' \in [0,k^\alpha]} \log(k^\alpha - k') + \delta
      v(k')
    \end{align*}
  \item Could guess and verify $v(k) = c_0 + c_1 \log(k)$
  \item If cannot guess, can try iterating the Bellman operator
  \end{itemize}
\end{frame}

\begin{frame}[allowframebreaks]
  \frametitle{Example: optimal growth}
  \begin{itemize}
  \item Set $v_0(k) = 0$ (or anything)
  \item Make sequence $ v_i = T(v_{i-1})$
  \item $v_1$:
    \begin{align*}
      v_1(k) = & \max_{k' \in [0,k^\alpha]} \log(k^\alpha - k') + \delta v_0(k)
      \\
      = & \max_{k' \in [0,k^\alpha]} \log(k^\alpha - k') \\
      = & \alpha \log k.
    \end{align*}
  \item $v_2$: 
    \begin{align*}
      v_2(k) = & \max_{k' \in [0,k^\alpha]} \log(k^\alpha - k') + \delta v_1(k)
      \\
      = & \max_{k' \in [0,k^\alpha]} \log(k^\alpha - k') + \delta
      \alpha \log(k')  \\
      = & c_2 + (\alpha + \delta \alpha^2) \log k,
    \end{align*}
  \item Etc.
    \begin{align*}
      v_3(k) = & c_3 + (\alpha + \delta \alpha^2 + \delta^2 \alpha^3)
      \log k \\
      v_4(k) = & c_4 + (\alpha + \delta \alpha^2 + \delta^2 \alpha^3 +
      \delta^3 \alpha^4) \log k \\
      \vdots &  \\
      v_i(k) = & c_i + \alpha \sum_{j = 0}^i (\alpha \delta)^j \log(k)
    \end{align*}
  \item Take, limit $i \to \infty$, 
    \[
      v(k) = 
      C + \frac{\alpha}{1-\alpha \delta} \log k 
      \]      
  \end{itemize}  
\end{frame}



\end{document}
