\documentclass[12pt,reqno]{amsart}

\usepackage{amsmath}
\usepackage{amsfonts}
\usepackage{graphicx}
%\usepackage{epstopdf}
\usepackage{hyperref}
\usepackage[left=1in,right=1in,top=0.9in,bottom=0.9in]{geometry}
\usepackage{multirow}
\usepackage{verbatim}
\usepackage{fancyhdr}
%\usepackage[small,compact]{titlesec} 

%\usepackage{pxfonts}
%\usepackage{isomath}
\usepackage{mathpazo}
%\usepackage{arev} %     (Arev/Vera Sans)
%\usepackage{eulervm} %_   (Euler Math)
%\usepackage{fixmath} %  (Computer Modern)
%\usepackage{hvmath} %_   (HV-Math/Helvetica)
%\usepackage{tmmath} %_   (TM-Math/Times)
%\usepackage{cmbright}
%\usepackage{ccfonts} \usepackage[T1]{fontenc}
%\usepackage[garamond]{mathdesign}
\usepackage{color}
\usepackage[normalem]{ulem}

\newtheorem{theorem}{Theorem}[section]
\newtheorem{conjecture}{Conjecture}[section]
\newtheorem{corollary}{Corollary}[section]
\newtheorem{lemma}{Lemma}[section]
\newtheorem{proposition}{Proposition}[section]
\theoremstyle{definition}
\newtheorem{assumption}{}[section]
%\renewcommand{\theassumption}{C\arabic{assumption}}
\newtheorem{definition}{Definition}[section]
\newtheorem{step}{Step}[section]
\newtheorem{remark}{Comment}[section]
\newtheorem{example}{Example}[section]
\newtheorem*{example*}{Example}

\linespread{1.1}

\pagestyle{fancy}
%\renewcommand{\sectionmark}[1]{\markright{#1}{}}
\fancyhead{}
\fancyfoot{} 
%\fancyhead[LE,LO]{\tiny{\thepage}}
\fancyhead[CE,CO]{\tiny{\rightmark}}
\fancyfoot[C]{\small{\thepage}}
\renewcommand{\headrulewidth}{0pt}
\renewcommand{\footrulewidth}{0pt}

\fancypagestyle{plain}{%
\fancyhf{} % clear all header and footer fields
\fancyfoot[C]{\small{\thepage}} % except the center
\renewcommand{\headrulewidth}{0pt}
\renewcommand{\footrulewidth}{0pt}}

\makeatletter
\renewcommand{\@maketitle}{
  \null 
  \begin{center}%
    \rule{\linewidth}{1pt} 
    {\Large \textbf{\textsc{\@title}}} \par
    {\normalsize \textsc{Paul Schrimpf}} \par
    {\normalsize \textsc{\@date}} \par
    {\small \textsc{University of British Columbia}} \par
    {\small \textsc{Economics 526}} \par
    \rule{\linewidth}{1pt} 
  \end{center}%
  \par \vskip 0.9em
}
\makeatother

\newcommand{\argmax}{\operatornamewithlimits{arg\,max}}
\newcommand{\argmin}{\operatornamewithlimits{arg\,min}}
\def\inprobLOW{\rightarrow_p}
\def\inprobHIGH{\,{\buildrel p \over \rightarrow}\,} 
\def\inprob{\,{\inprobHIGH}\,} 
\def\indist{\,{\buildrel d \over \rightarrow}\,} 
\def\F{\mathbb{F}}
\def\R{\mathbb{R}}
\newcommand{\gmatrix}[1]{\begin{pmatrix} {#1}_{11} & \cdots &
    {#1}_{1n} \\ \vdots & \ddots & \vdots \\ {#1}_{m1} & \cdots &
    {#1}_{mn} \end{pmatrix}}
\newcommand{\iprod}[2]{\left\langle {#1} , {#2} \right\rangle}
\newcommand{\norm}[1]{\left\Vert {#1} \right\Vert}
\newcommand{\abs}[1]{\left\vert {#1} \right\vert}
\renewcommand{\det}{\mathrm{det}}
\newcommand{\rank}{\mathrm{rank}}
\newcommand{\spn}{\mathrm{span}}
\newcommand{\row}{\mathrm{Row}}
\newcommand{\col}{\mathrm{Col}}
\renewcommand{\dim}{\mathrm{dim}}
\newcommand{\prefeq}{\succeq}
\newcommand{\pref}{\succ}
\newcommand{\seq}[1]{\{{#1}_n \}_{n=1}^\infty }
\renewcommand{\to}{{\rightarrow}}
\newcommand{\corres}{\overrightarrow{\rightarrow}}

\title{Functions}
\date{\today}

\begin{document}

\maketitle

\section{Definition and examples}

 We have already used functions in this course, so perhaps we should
have defined them earlier. Anyway, a \textbf{function} from a set $A$
to a set $B$ is a rule that assigns to each $a \in A$ one and only one
$b \in B$. If we want to call this function $f$, we denote this by
$f:A \to B$, which is read as ``$f$ is a function from $A$ to $B$'' or
simply ``$f$ from $A$ to $B$.'' The set $A$ is called the
\textbf{domain} of $f$. $B$ is called the \textbf{target} of $f$. The
set 
\[ \{y  \in B:  f(x) = y \text{ for some } x \in A \} \]
is called the \textbf{image} of $f$. 

\subsection{Examples}

\begin{enumerate}
\item Production functions: $f:\R^2 \to \R$ 
  \begin{itemize}
  \item Linear $ f(x_1,x_2) = a_1 x_1 + a_2 x_2 $
  \item Cobb-Douglas: $f(x_1,x_2) = K x_1^{\alpha_1} x_2^{\alpha_2}$
  \item Constant elasticity of substitution: $f(x_1,x_2) = K (c_1
    x_1^{-a} + c_2 x_2^{-a})^{-b/a}$
  \end{itemize}
\item Utility functions: $u: \R^T \to \R$
  \begin{itemize}
  \item Constant relative risk aversion: $u(c_1,...,c_T) =
    \sum_{t=1}^T \beta^t \frac{c_t^{1-\gamma}}{1-\gamma}$
  \item Constant absolute risk aversion: $u(c_1,...,c_T) =
    \sum_{t=1}^T \beta^t (-e^{-\alpha c_t})$
  \end{itemize}
\item Demand function with constant elasticity,  $D:\R^3 \to \R^2$
  \[ D(p_1,p_2,y) = \begin{pmatrix} M p_1^{\alpha_{11}}
    p_2^{\alpha_{12}} y^{\beta_1} \\
    M p_1^{\alpha_{21}} p_2^{\alpha_{22}} y^{\beta_2}
  \end{pmatrix}
  \]  
  where $p_1$ and $p_2$ are the prices of two goods and $y$ is income.
\end{enumerate}
I do not expect you to remember the names of these functions, but it
is very likely that you will repeatedly encounter them this year. 

\subsection{Visualizing functions}
It is often useful to visualize a function. Simon and Blume have a
whole section (13.2) about how to draw graphs of functions. That may
have made sense in 1994, but it seems excessive now. To graph a
function, use a computer. \href{http://wolframalpha.com}{Wolfram
  alpha} is a pretty good website for creating a quick graph.  I'm
sure you can find many other websites and cell phone apps with similar
plotting capabilities. You can create nicer graphs using something
like Matlab (or probably excel, or python, or whatever).  You are
probably familiar with indifference curves and isoquants from another
economics course. Indifference curves and isoquants are examples of
level sets.
\begin{definition}
  The \textbf{level sets} of a function $f:X\to Y$ are sets of the
  form 
  \[ \{ x \in X: f(x) = y \} \]
  for some fixed $y \in Y$.
\end{definition}
When you draw indifference curves, you are drawing level sets of a
utility function. When you draw isoquants, you are drawing level sets
of a production function. Figures 1-3 show isoquants and indifference
curves for some of the examples of functions above. 

\begin{figure}
  \caption{CES, $a = 2$, $b = \frac{4}{5}$}
  \begin{tabular}{cc}
    \includegraphics[width=0.5\linewidth]{cesc} &
    \includegraphics[width=0.5\linewidth]{ces3d} 
  \end{tabular}
\end{figure}

\begin{figure}
  \caption{CRRA, $\gamma = 2$, $\beta = 0.95$}
  \begin{tabular}{cc}
    \includegraphics[width=0.5\linewidth]{crrac} &
    \includegraphics[width=0.5\linewidth]{crra3d} 
  \end{tabular}
\end{figure}

\begin{figure}
  \caption{CARA, $\alpha = 1$, $\beta = 0.95$}
  \begin{tabular}{cc}
    \includegraphics[width=0.5\linewidth]{carac} &
    \includegraphics[width=0.5\linewidth]{cara3d} 
  \end{tabular}
\end{figure}

\newpage

\section{Special types of functions}

There are some special types of functions that you are probably
familiar with on $\R^1$ that we will generalize to $\R^n$. We have
already covered linear functions in detail, although we called them
linear transformations.
\begin{definition}
  A function $f:V \to W$ where $V$ and $W$ are vector spaces is
  \textbf{linear} if $f$ preserves addition and scalar multiplication, ie
  \begin{itemize}
  \item $f(x+y) = f(x) + f(y)$
  \item $f(\alpha x) = \alpha f(x)$
  \end{itemize}
\end{definition}
As we have already seen, linear functions from $\R^n$ to $\R^m$ can be
represented by $m$ by $n$ matrices. We went over a, perhaps very
confusing, proof of this fact. You can find the same result in theorem
13.2 of Simon and Blume. Perhaps that proof will be clearer if you
still find the relationship between matrices and linear functions
confusing. 

You probably are familiar with quadratic functions from $\R$ to
$\R$. They look like
\[ a_0 + a_1 x + a_2 x^2. \]
We can generalize this to functions from $\R^n$ to $\R$ as follows.
\begin{definition}
  $q:\R^n \to R$ is a \textbf{quadratic} if 
  \[ q(x_1, ..., x_n) = a_0 + \sum_{i=1, j \geq i}^n a_{ij} x_i x_j \]
\end{definition}
A quadratic function can be written using matrix notation as
\[ q(x_1,..., x_n) = a_0 + x^T A x \]
where $A = \begin{pmatrix} a_{11} & \frac{1}{2} a_{12} & \cdots &
  \frac{1}{2} a_{1n} \\
  \frac{1}{2} a_{12} & a_{22} \cdots & \frac{1}{2} a_{2n} \\
  \vdots & \ddots & \ddots & \vdots \\
  \frac{1}{2} a_{1n} & \cdots & \cdots & a_{nn}
\end{pmatrix}$. Note that this choice of $A$ is not unique. The
$\frac{1}{2}$'s below the diagonal and the $\frac{1}{2}$'s above the
diagonal can be replaced by any two numbers whose sum is $1$. If you
want to practice matrix multiplication, you could verify this. It
might be easiest to start with a $2$ by $2$ or $3$ by $3$ example. 

Next, we generalize polynomials to $\R^n$. 
\begin{definition}
  A \textbf{monomial} $f:\R^n \to R$ is any function of the form
  \[ f(x_1, ..., x_n) = c x_1^{a_1} x_2^{a_2} \cdots x_n^{a_n} \]
  where $a_i$ are nonnegative integers. 

  $\sum_{i=1}^n a_i$ is the \textbf{degree} of the monomial. 

  A \textbf{polynomial} $f:\R^n \to \R$ is the sum of finitely many
  monomials, i.e.
  \[ f(x_1, ..., x_n) = \sum_{k=1}^k c_k x_1^{a_{1k}}\cdots
  x_n^{a_{nk}} \]
  The maximum degree of the monomials making up a polynomial is the
  degree of the polynomial.
\end{definition}

\section{Continuous functions}

A continuous function is a function without any jumps or
holes. Formally,
\begin{definition}
  A function $f:X \to Y$ where $X$ and $Y$ are metric spaces is
  \textbf{continuous} at $x$ if whenever $\seq{x}$ converges to $x$ in
  $X$, then $f(x_n) \to f(x)$ in $Y$.  
\end{definition}
We simply say that $f$ is continuous if it is continuous at every $x
\in X$.  There are some equivalent definitions of continuity that are
also useful. You may have seen continuity defined as the result of the
following lemma.
\begin{lemma}\label{lem:ced}
  $f: X \to Y$ is continuous at $x$ if and only if for every
  $\epsilon>0$ $\exists$ $\delta >0$ such that $d(x,x') < \delta $
  implies $d(f(x),f(x')) < \epsilon$.
\end{lemma}
\begin{proof}
  This was going to be a question on problem set 3, but it has been
  unassigned. 
  
  Suppose $f$ is continuous and there is an $\epsilon>0$ such that
  such that for any $\delta>0$, $d(x,x') < \delta$ does not imply
  $d(f(x),f(x'))< \epsilon$. Then by letting $\delta = 1/n$ we can
  construct a convergent sequence by choosing $x_n$ such that
  $d(x,x_n) < 1/n$ and $d(f(x),f(x_n)) \geq \epsilon$. $x_n \to x$,
  but $f(x_n) \not\to f(x)$. Therefore, if $f$ is continuous it must
  be impossible to construct such a sequence. This means that there
  must be some $\delta>0$ such that $d(x,x') < \delta$ implies
  $d(f(x),f(x')) < \epsilon$.

  Now suppose $\forall \epsilon>0$ $\exists \delta>0$ such that
  $d(x,x')<\delta$ implies $d(f(x),f(x')) <\epsilon$. Let $x_n \to
  x$. Then $\exists N$ s.t. $n\geq N$ implies $d(x,x_n) < \delta$. But
  this implies $d(f(x),f(x_n)) < \epsilon$, so $f$ is continuous.
\end{proof}

Another useful way of defining continuity is in terms of open
sets. First, another definition.
\begin{definition}
  Let $f: X \to Y$. The \textbf{preimage} of $V \subseteq Y$ is the
  set in $X$, $f^{-1}(V)$ defined by
  \[ f^{-1} (V) = \{ x \in X: f(x) \in V \} \]
\end{definition}

\begin{lemma}\label{lem:copen}
  $f:X \to Y$ is continuous if and only if $f^{-1}(V)$ is open for all
  open $V \subseteq Y$. 
\end{lemma}
\begin{proof}
  On problem set.
\end{proof}
\begin{corollary}
  $f:X \to Y$ is continuous if and only if $f^{-1}(V)$ is closed for all
  closed $V \subseteq Y$. 
\end{corollary}

Continuity is preserved by arithmetic. 
\begin{theorem}
  Let $f:X \to Y$ and $g:X \to Y$ be continuous and $X$ and $Y$ be
  vector spaces. Then $(f+g)(x) = f(x) + g(x)$ is continuous.
\end{theorem}
\begin{proof}
  If $f$ and $g$ are continuous, then by definition $f(x_n) \to f(x)$
  and $g(x_n) \to g(x)$ whenever $x_n \to x$. From the previous
  lecture the limit of a (finite) sum is the sum of limits, so 
  $f(x_n) + g(x_n) \to f(x) + g(x)$, and $f+g$ is continuous.
\end{proof}
Similar results can be shown for subtraction, multiplication, etc,
whenever they are well defined.

Continuity is also preserved by composition.
\begin{theorem}
  Let $f:X \to Y$ and $g:Y \to Z$ be continuous where $X$, $Y$, and
  $Z$ are metric spaces. Then $f \circ g$ is continuous, where
  \[ (f \circ g)(x) = f(g(x)). \]
\end{theorem}  
\begin{proof}
  Let $x_n \to x$. $g$ is continuous, so $g(x_n) \to g(x)$. $f$ is
  also continuous, so $f(g(x_n)) \to f(g(x))$.
\end{proof}
$f\circ g$ is called the composition of $f$ and $g$.

\subsection{Onto, one-to-one, and inverses}
We have already used the concepts of onto, one-to-one, and inverses.
We restates the definitions here for completeness.
\begin{definition}
  $f:X \to Y$ is \textbf{one-to-one} or \textbf{injective} if for all
  $x_1, x_2 \in X$, 
  \[ f(x_1) = f(x_2) \]
  if and only if $x_1 = x_2$.
\end{definition}
Equivalently, $f$ is injective if for each $y \in Y$, the set $\{x:
f(x) = y\}$ is either a singleton or empty.  In terms of a nonlinear
equation, if $f$ is one-to-one, then $f(x) = b$ has at most one
solution. 
\begin{definition}
  $f:X \to Y$ is \textbf{onto} or \textbf{surjective} if $\forall y
  \in Y$, $\exists x \in X$ such that $f(x) = y$.
\end{definition}
In terms of a nonlinear equation, if $f$ is onto, then $f(x) = b$ has
at least one solution. When $f$ is one-to-one and onto, we say that
$f$ is \textbf{bijective}. A bijective function has an inverse.
\begin{definition}
  If $f:X \to Y$ is bijective, then the \textbf{inverse} of $f$,
  written $f^{-1}$ satisfies
  \[ f(f^{-1} (y)) = y  \]
  and 
  \[ f^{-1} ( f(x) ) = x. \]
\end{definition}

\begin{remark}
  While writing these notes, I briefly tried to prove that if $f:X \to
  Y$ is bijective and continuous, then $f^{-1}$ is continuous. I could
  not do this, which is good, because that statement is false. You have
  to be a little creative in defining $X$ and $Y$ to come up with a
  counterexample. Let $X = [0,2\pi)$ and $Y = \{(x,y) \in \R^2: x^2 +
  y^2 = 1 \}$. Then $f(x) = (cos(x), sin(x))$ is bijective and
  continuous, but $f^{-1}$ is not continuous at $(1,0)$. 
  
  This counterexample is actually related to a fundamental fact in
  topology. You may remember from last lecture that topology is about
  studying spaces with open and closed sets that do not necessarily
  have a metric. The main thing that people are interested when
  studying such spaces is finding a continuous (in both directions)
  bijections between them. Loosely speaking, two topological spaces
  will have a continuous bijection between them if one can be bent and
  stretched into the form of another. You cannot bend a circle into an
  interval without breaking the circle, so there is no continuous
  bijection between the circle and an interval. When there is a
  continuous bijection between two spaces, they have the same
  collection of open sets, so to a topologist, they are the same. We
  then call the spaces homeomorphic (or topologically
  isomorphic). Loosely speaking, spaces will be homeomorphic if they
  are the same dimension and their shapes have the same number of
  holes.  The circle has one hole, an interval has none, so they are
  not topologically isomorphic. I'd be remiss not to make a joke now,
  so here goes: Why did the topologist eat her/his coffee mug and
  drink from his/her donut?  Because they're topologically
  isomorphic. Hahaha.
\end{remark}

%%%%%%%%%%%%%%%%%%%%%%%%%%%%%%%%%%%%%%%%%%%%%%%%%%%%%%%%%%%%%%%%%%%%%%%%
\section{Correspondences \label{sec:corr}} 

\footnote{This section is largely based on section 2.1.5 of Carter.}A
function, $f:X \to Y$ associates exactly one element of $Y$, $f(x)$,
with each $x \in X$. Often we encounter things that are like
functions, but for each $x \in X$, there are multiple elements of
$Y$. We call this generalization of a function as correspondence. 
\begin{definition}
  A \textbf{correspondence} from a set $X$ to a set $Y$, is a rule
  that assigns to each a $x \in X$ a subset of $Y$. We denote a
  correspondence by $\phi: X \corres Y$.
\end{definition}
An equivalent definition is that $\phi: X \corres Y$ is a function
from $X$ to the power set of $Y$. Correspondences appear often in
economics, especially as constraint sets in optimization problems.
\begin{example}[Budget correspondence]
  Suppose there are $n$ goods with prices $p \in \R^n$. Then given
  income of $m$, a consumer can afford $\chi(p,m)=\{ x \in X \subseteq
  \R^n: p'x \leq m\}$, which defines a correspondence $\chi: \R^{n+1}
  \corres X$. We can write the consumer's problem of maximizing
  utility subject to the budget constraint as
  \begin{align*}
    \max_{x \in \chi(p,m)} u(x) 
  \end{align*}
  If this problem has a solution, then the indirect utility function
  is the maximized utility,
  \[ v(p,m) = \max_{x \in \chi(p,m)} u(x). \]
  The demand correspondence (usually function) is
  \[ x^*(p,m) = \argmax_{x \in \chi(p,m)} u(x). \] 
  Such maximization problems are central to economics. To derive
  properties of the indirect utility and demand functions it is often
  useful to treat the budge set as a correspondence.
\end{example}
Correspondences also appear in economics in any model where we
multiple equilibria, such as many games. 

Defining continuity is a bit more complicated for correspondences than
for functions. A function can either be continuous or it can jump. A
correspondence can also expand or contract. For example, consider
$\xi:\R \corres \R$ defined by 
\[ \xi(x) = \begin{cases} 
  [0,1] & \text{ if } x > 0 \\
  [1/4,3/4] & \text{ if } x \leq 0 
\end{cases} \]
and $\psi: \R \corres \R$ defined by
\[ \psi(x) = \begin{cases} 
  [0,1] & \text{ if } x \geq 0 \\
  [1/4,3/4] & \text{ if } x < 0 
\end{cases} \]
Both these correspondences are somewhat continuous because they
contain a continuous function, e.g. $f(x) = 1/2$,  for all
$x$. However, they are also somewhat discontinuous because the
corresponding set changes suddenly at $0$. Motivated by this
observation we define the following:
\begin{definition}
  A correspondence, $\phi: X \corres Y$ is upper hemicontinuous at $x$
  if for all sequences $x_n \to x$ and $y_n \in \phi(x_n)$ with $y_n
  \to y$, then $y \in \phi(x)$. 
\end{definition}
In the previous example, $\psi$ is upper hemicontinuous at  $0$, but $\xi$ is
not. To see this consider $x_n = 1/n$ and $y_n = 1$. 
\begin{definition}
  A correspondence, $\phi: X \corres Y$ is lower hemicontinuous at $x$
  if for all sequences $x_n \to x$ and $y \in \phi(x)$, there exists a
  subsequence, $x_{nk}$ and $y_k \in x_{nk}$ with $y_k \to y$.  
\end{definition}
In the previous example, $\xi$ is lower hemicontinuous at $0$, but
$\psi$ is not. To see this consider $x_n = -1/n$ and $y = 1$. 

We say that a correspondence is continuous if it is both upper and
lower hemicontinuous. At all $x \neq 0$, $\xi$ and $\psi$ are
continuous. 

\end{document}