\documentclass[12pt,reqno]{amsart}

\usepackage{amsmath}
\usepackage{amsfonts}
\usepackage{graphicx}
%\usepackage{epstopdf}
\usepackage{hyperref}
\usepackage[left=1in,right=1in,top=0.9in,bottom=0.9in]{geometry}
\usepackage{multirow}
\usepackage{verbatim}
\usepackage{fancyhdr}
%\usepackage[small,compact]{titlesec} 

%\usepackage{pxfonts}
%\usepackage{isomath}
\usepackage{mathpazo}
%\usepackage{arev} %     (Arev/Vera Sans)
%\usepackage{eulervm} %_   (Euler Math)
%\usepackage{fixmath} %  (Computer Modern)
%\usepackage{hvmath} %_   (HV-Math/Helvetica)
%\usepackage{tmmath} %_   (TM-Math/Times)
%\usepackage{cmbright}
%\usepackage{ccfonts} \usepackage[T1]{fontenc}
%\usepackage[garamond]{mathdesign}
\usepackage{color}
\usepackage[normalem]{ulem}

\newtheorem{theorem}{Theorem}[section]
\newtheorem{conjecture}{Conjecture}[section]
\newtheorem{corollary}{Corollary}[section]
\newtheorem{lemma}{Lemma}[section]
\newtheorem{proposition}{Proposition}[section]
\theoremstyle{definition}
\newtheorem{assumption}{}[section]
%\renewcommand{\theassumption}{C\arabic{assumption}}
\newtheorem{definition}{Definition}[section]
\newtheorem{step}{Step}[section]
\newtheorem{remark}{Comment}[section]
\newtheorem{example}{Example}[section]
\newtheorem*{example*}{Example}

\linespread{1.1}

\pagestyle{fancy}
%\renewcommand{\sectionmark}[1]{\markright{#1}{}}
\fancyhead{}
\fancyfoot{} 
%\fancyhead[LE,LO]{\tiny{\thepage}}
\fancyhead[CE,CO]{\tiny{\rightmark}}
\fancyfoot[C]{\small{\thepage}}
\renewcommand{\headrulewidth}{0pt}
\renewcommand{\footrulewidth}{0pt}

\fancypagestyle{plain}{%
\fancyhf{} % clear all header and footer fields
\fancyfoot[C]{\small{\thepage}} % except the center
\renewcommand{\headrulewidth}{0pt}
\renewcommand{\footrulewidth}{0pt}}

\makeatletter
\renewcommand{\@maketitle}{
  \null 
  \begin{center}%
    \rule{\linewidth}{1pt} 
    {\Large \textbf{\textsc{\@title}}} \par
    {\normalsize \textsc{Paul Schrimpf}} \par
    {\normalsize \textsc{\@date}} \par
    {\small \textsc{University of British Columbia}} \par
    {\small \textsc{Economics 526}} \par
    \rule{\linewidth}{1pt} 
  \end{center}%
  \par \vskip 0.9em
}
\makeatother

\newcommand{\argmax}{\operatornamewithlimits{arg\,max}}
\newcommand{\argmin}{\operatornamewithlimits{arg\,min}}
\def\inprobLOW{\rightarrow_p}
\def\inprobHIGH{\,{\buildrel p \over \rightarrow}\,} 
\def\inprob{\,{\inprobHIGH}\,} 
\def\indist{\,{\buildrel d \over \rightarrow}\,} 
\def\F{\mathbb{F}}
\def\R{\mathbb{R}}
\newcommand{\gmatrix}[1]{\begin{pmatrix} {#1}_{11} & \cdots &
    {#1}_{1n} \\ \vdots & \ddots & \vdots \\ {#1}_{m1} & \cdots &
    {#1}_{mn} \end{pmatrix}}
\newcommand{\iprod}[2]{\left\langle {#1} , {#2} \right\rangle}
\newcommand{\norm}[1]{\left\Vert {#1} \right\Vert}
\newcommand{\abs}[1]{\left\vert {#1} \right\vert}
\renewcommand{\det}{\mathrm{det}}
\newcommand{\rank}{\mathrm{rank}}
\newcommand{\spn}{\mathrm{span}}
\newcommand{\row}{\mathrm{Row}}
\newcommand{\col}{\mathrm{Col}}
\renewcommand{\dim}{\mathrm{dim}}
\newcommand{\prefeq}{\succeq}
\newcommand{\pref}{\succ}
\newcommand{\seq}[1]{\{{#1}_n \}_{n=1}^\infty }
%\newcommand{\to}{{\rightarrow}}


\title{Limits and topology of metric spaces}
\date{\today}

\begin{document}

\maketitle

This lecture focuses on sequences, limits, and topology. Similar
material is covered in chapters 12 and 29 of Simon and Blume, or 1.3
of Carter. 

\section{Sequences and limits}

A \textbf{sequence} is a list of elements, $\{x_1, x_2, ... \}$ or
$\seq{x}$ or sometimes just $\{x_n\}$. Although the notation for a
sequence is similar to the notation for a set, they should not be
confused. Sequences are different from sets in that the order of
elements in a sequence matters, and the same element can appear many
times in a sequence.  Some examples of sequences with $x_i \in \R$
include
\begin{enumerate}
\item\label{sfib} $\{ 1, 1, 2, 3, 5, 8, ... \}$
\item\label{sc} $\{ 1, \frac{1}{2}, \frac{1}{3}, \frac{1}{4}, ... \}$
\item\label{sd} $\{\frac{1}{2}, \frac{-2}{3}, \frac{3}{4}, \frac{-4}{5},
  \frac{5}{6}, ... \}$
\end{enumerate}
Some sequences, like \ref{sc}, have elements that all get closer and
closer to some fixed point. We say that these types of sequences
converge. A sequence that does not converge diverges. Some divergent
sequences like \ref{sfib}, increase without bound. Other divergent
sequences, like \ref{sd}, are bounded, but they do not converge to any
single point. 

To analyze sequences with elements that are not necessarily real
numbers, we need to be able to say how far apart the entries in the
sequence are. 
\begin{definition}
  A \textbf{metric space} is a set, $X$, and function $d:X\times X
  \rightarrow \R$ called a \textbf{metric} (or distance) such that
  $\forall x, y, z \in X$
  \begin{enumerate}
  \item $d(x,y) > 0$ unless $x=y$ and then $d(x,x) = 0$
  \item (symmetry) $d(x,y) = d(y,x)$
  \item (triangle inequality) $d(x,y) \leq d(x,z) + d(z,y)$.
  \end{enumerate}
\end{definition}
\begin{example}
  $\R$ is a metric space with $d(x,y) = |x - y|$. 
\end{example}
\begin{example}
  Any normed vector space is a metric space with $d(x,y) =
  \norm{x-y}$. 
\end{example}
The most common metric space that we will encounter will be $\R^n$
with the Euclidean metric, $d(x,y) = \norm{x-y} = \sqrt{\sum_{i=1}^n
  (x_i - y_i)^2}$.  
\begin{definition}
  A sequence $\{x_n\}_{n=1}^\infty$ in a metric space \textbf{converges} to $x$
  if $\forall \epsilon > 0$ $\exists N$ such that 
  \[ d(x_n, x) < \epsilon \]
  for all $n \geq N$. We call $x$ the \textbf{limit} of $\seq{x}$ and
  write $\lim_{n \rightarrow \infty} x_n = x$ or $x_n \rightarrow x$. 
\end{definition}
\begin{example}
  The sequence $\{ 1, \frac{1}{2}, \frac{1}{3}, \frac{1}{4}, ... \} =
  \{1/n\}_{n=1}^\infty$ converges. To see this, take any
  $\epsilon>0$. Then $\exists$ $N$ such that $1/N < \epsilon$. For all
  $n \geq N$, $d(1/n,0) = 1/n < \epsilon$. 
\end{example}
If a sequence does not converge, it diverges. 
\begin{example}
  The Fibonacci sequence, $\{ 1, 1, 2, 3, 5, 8, ... \}$ diverges.
\end{example}
\begin{definition}
  $a$ is an \textbf{accumulation point} of $\seq{x}$ if $\forall
  \epsilon > 0$ $\exists$ infinitely many $x_i$ such that 
  \[ d(a, x_i ) < \epsilon. \]
\end{definition}
\begin{example}
  The sequence $\{\frac{1}{2}, \frac{-2}{3}, \frac{3}{4},
  \frac{-4}{5}, \frac{5}{6}, ... \}$ has two accumulation points, $1$ 
  and $-1$.
\end{example}
The limit of any convergent sequence is an accumulation point of the
sequence. In fact, it is the only accumulation point.
\begin{lemma}
  If $x_n \rightarrow x$, then $x$ is the only accumulation point of
  $\seq{x}$. 
\end{lemma}
\begin{proof}
  Let $\epsilon>0$ be given. By the definition of convergence,
  $\exists N$ such that 
  \[ d(x_n,x) < \epsilon \]
  for all $n \geq N$. $\{n\in \mathbb{N}: n\geq N\}$ is infinite, so
  $x$ is an accumulation point. 

  Suppose $x'$ is another accumulation point. Then $\forall
  \epsilon>0$ $\exists N$ and $N'$ such that if $n \geq N$ and $n \geq
  N'$, then $d(x_n, x) < \epsilon/2$ and $d(x_n,x') < \epsilon/2)$. By
  the triangle inequality, $d(x,x') \leq d(x_n,x') + d(x_n,x) <
  \epsilon$. Since this inequality holds for any $\epsilon$, it must
  be that $d(x,x') = 0$. $d$ is a metric, so then $x = x'$, and the
  limit of sequence is the sequence's unique accumulation point. 
\end{proof}
The third example of a sequence at the start of this section,
\ref{sd}, shows that the converse of this lemma is false. Not every
accumulation point is a limit. 
\begin{definition}
  Given $\seq{x}$ and any sequence of positive integers, $\{n_k\}$
  such that $n_1 < n_2 < ... $ we call $\{x_{n_k}\}$ a
  \textbf{subsequence} of $\seq{x}$. 
\end{definition}
In example \ref{sd}, there are two accumulation points, $-1$ and $1$,
and you can find subsequences that converge to these points. 
\begin{lemma}
  Let $a$ be an accumulation point of $\{x_n\}$. Then $\exists$
  a subsequence that converges to $a$. 
\end{lemma}
\begin{proof}
  We can construct a subsequence as follows. Let $\{\epsilon_k\}$ be a
  sequence that converges to zero with $\epsilon_k >0 \forall k$, (for
  example, $\epsilon_k = 1/k$). By the definition of accumulation
  point, for each $\epsilon_k$ $\exists$ infinitely many $x_n$ such
  that 
  \begin{align}
    d(x_n, a) < \epsilon_k \label{ieq:a}
  \end{align}
  Pick any $x_{n_1}$ such that (\ref{ieq:a}) holds for
  $\epsilon_1$. For $k>1$, pick $n_k \neq n_{j}$ for all $j<k$ and
  such that (\ref{ieq:a}) holds for $\epsilon_k$. Such an $n_k$ always
  exists because there are infinite $x_n$ that satisfy
  (\ref{ieq:a}). By construction, $\lim_{k\rightarrow \infty} x_{n_k}
  = a$ (you should verify this using the definition of limit).
\end{proof}

Convergence of sequences is often preserved by arithmetic
operations, as in the following two theorems.
\begin{theorem}
  Let $\{x_n\}$ and $\{y_n\}$ be sequences in a normed vector space
  $V$. If $x_n \to x$ and $y_n \to y$, then
  \[ x_n + y_n \to x + y. \]
\end{theorem}
\begin{proof}
  Let $\epsilon > 0$ be given. Then $\exists$ $N_x$ such that for all
  $n \geq N_x$, 
  \[ d(x_n,x) < \epsilon/2,\] and $\exists N_y$ such that
  for all $n \geq N_y$,  
  \[ d(y_n,y) < \epsilon/2. \]
  Let $N =\max\{N_x,N_y\}$. Then for all $n \geq N$, 
  \begin{align*}
    d(x_n + y_n,x+y) = \norm{(x_n + y_n) - (x+y)} \leq & \norm{x_n -
      x} + \norm{y_n - y} \\
    < & \epsilon/2 + \epsilon /2  = \epsilon.
  \end{align*}  
\end{proof}
\begin{theorem}
  Let $\{x_n \}$ be a sequence in a normed vector space with scalar
  field $\R$ and let $\{c_n\}$ be a sequence in $\R$. If
  $x_n \to x$ and $c_n \to c$ then 
  \[ x_n c_n \to x c. \]
\end{theorem}
\begin{proof}
  On problem set. 
\end{proof}
In fact, in the next lecture we will see that if $f(\cdot,\cdot)$ is
continuous, then $\lim f(x_n, y_n) = f(x,y)$. The previous two
theorems are examples of this with $f(x,y) = x+y$ and $f(c,x) = c x$,
respectively.

\subsection{Series}

Infinite sums or series are formally defined as the limit of the
sequence of partial sums. 
\begin{definition}
  Let $\seq{x}$ be a sequence in a normed vector space. Let $s_n =
  \sum_{i=1}^n x_i$ denote the sum of the first $n$ elements of the
  sequence. We call $s_n$ the $n$th partial sum. We define the sum of
  all the $x_i$s as
  \[ \sum_{i=1}^\infty x_i \equiv \lim_{n \to \infty} s_n \]
  This is called a(n infinite) \textbf{series}. 
\end{definition}

\begin{example}
  Let $\beta \in \R$. $\sum_{i=0}^\infty \beta^i$ is called a
  geometric series. Geometric series appear often in
  economics, where $\beta$ will be the subjective discount factor or
  perhaps $1/(1+r)$.  Notice  that
  \begin{align*}
    s_n = & 1 + \beta + \beta^2 + \cdots + \beta^n \\
    = & 1 + \beta ( 1 + \beta + \cdots + \beta^{n-1} ) \\
    = & 1 + \beta ( 1 + \beta + \cdots + \beta^{n-1} + \beta^n) -
    \beta^{n+1} \\
    s_n(1 - \beta) = & 1 - \beta^{n+1} \\
    s_n = & \frac{1 - \beta^{n+1}}{1-\beta} ,
  \end{align*}
  so, 
  \begin{align*}
    \sum_{i=0}^\infty \beta^i = & \lim s_n \\
    = & \lim \frac{1 - \beta^{n+1}}{1-\beta} \\
    = & \frac{1}{1-\beta} \text{ if } |\beta|<1.
  \end{align*}
\end{example}


%%%%%%%%%%%%%%%%%%%%%%%%%%%%%%%%%%%%%%%%%%%%%%%%%%%%%%%%%%%%%%%%%%%%%%
\subsection{Cauchy sequences}

We have defined convergent sequences as ones whose entries all get
close to a fixed limit point. This means that all the entries of the
sequence are also getting closer together. You might imagine a
sequence where the entries get close together without necessarily
reaching a fixed limit.
\begin{definition}
  A sequence $\seq{x}$ is a \textbf{Cauchy} sequence if for any
  $\epsilon > 0$ $\exists N$ such that for all $i,j\geq N$,
  $d(x_i,x_j) < \epsilon$.
\end{definition}
It turns that in $\R^n$ Cauchy sequences and convergent sequences are
the same. This is a consequence of $\R$ having the least upper bound
property. 
\begin{theorem}
  A sequence in $\R^n$ converges if and only if it is a Cauchy
  sequence. 
\end{theorem}
There is a proof of this in Chapter 29.1 of Simon and Blume. If you
want more practice with the sort of proofs in this lecture, it would
be good to read that section. The convergence of Cauchy sequences in
the real numbers is a consequence of the least upper bound property
that we discussed in lecture 1. Cauchy sequences do not converge in
all metric spaces. For example, the rational numbers are a metric
space, and any sequence of rationals that converges to an irrational
number in $\R$ is a Cauchy sequence in $\mathbb{Q}$ but has no limit
in $\mathbb{Q}$. Having Cauchy sequences converge is necessary for
proving many theorems, so we have a special name for metric spaces
where Cauchy sequences converge.
\begin{definition}
  A metric space, $X$, is \textbf{complete} if every Cauchy sequence of
  points in $X$ converges in $X$.
\end{definition}
Completeness is important for so many results that complete version of
vector spaces are named after the mathematicians who first studied
them extensively. A \textbf{Banach space} is a complete normed vector
space. A \textbf{Hilbert space} is a complete inner product
space. Since any inner product space is a normed vector space with
norm $\norm{x} = \sqrt{\iprod{x}{x}}$, any Hilbert space is also a
Banach space. 
\begin{example}
  $\R^n$ is a Hilbert space.
\end{example}
\begin{example}
  $\ell^p = \{ (x_1, x_2, ...)\, s.t.\, x_i \in \R, \sum_{i=1}^\infty |x_i|^p <
  \infty \}$ with norm 
  \[ \norm{x} = \left( \sum_{i=1}^\infty |x_i|^p \right)^{1/p} \]
  is a Banach space. 
  
  $\ell^2$ with 
  \[ \iprod{x}{y} = \sum_{i=1}^\infty x_i y_i \]
  is a Hilbert space. 
  
  Showing that $\ell^p$ is complete is slightly tricky because you
  have deal with a sequence of $\mathbf{x}_i \in \ell^p$, each element
  of which is itself an infinite sequence. You should not worry if you
  have difficulty following the rest of this example.
  
  To show that $\ell^p$ is complete, let $\seq{\mathbf{x}}$ be a
  Cauchy sequence. Denote the elements of $\mathbf{x}_i$ by $x_{i1},
  x_{i2}, ... $. First, let's show that for any $n$, $x_{1n}, x_{2n},
  ... $ is a Cauchy sequence in $\R$. Let $\epsilon > 0$. Since
  $\seq{\mathbf{x}}$ is Cauchy, $\exists N_\epsilon$ such that for all $i,j
  \geq N_\epsilon$, 
  \[  \norm{\mathbf{x}_i - \mathbf{x}_j } < \epsilon. \]
  Since 
  \[ \norm{\mathbf{x}_i - \mathbf{x}_j }^p = \left( \sum_{m=1}^\infty
    \abs{x_{i m} - x_{j m}}^p \right) \]
  All terms in the sum on the right are non-negative and the sum
  includes $\abs{x_{i n} - x_{jn}}$, so 
  \begin{align*} \abs{x_{i n} - x_{j n} }^p \leq & \norm{\mathbf{x}_i -
      \mathbf{x}_j }^p \\
    \abs{x_{i n} - x_{j n} } \leq & 
    \norm{\mathbf{x}_i - \mathbf{x}_j }  
  \end{align*}
  Therefore, $\abs{x_{i n} - x_{j n}} < \epsilon$ for all $i,j \geq
  N_\epsilon$, i.e. $x_{1n}, x_{2n}, ...$ is a Cauchy sequence in
  $\R$. $\R$ is complete, so it has some limit. Denote the limit by
  $x^\ast_n$.  

  Now we will show that $\mathbf{x}^\ast = (x_{1}^\ast, x_{2}^\ast,
  ... )$ is the limit of $\seq{\mathbf{x}}$.  First, we should show
  that $\mathbf{x}^\ast \in \ell^p$. Let 
  \begin{align*}
    s_m^\ast = \sum_{n=1}^m |x^\ast_n|^p.
  \end{align*}
  We need to show that $\lim s_m^\ast $ exists.  Since
  $\seq{\mathbf{x}}$ is Cauchy, $\exists j$ such that if $i \geq j$,
  $\norm{\mathbf{x}_i - \mathbf{x}_j} < 1$. Using the triangle
  inequality, 
  \[ \norm{\mathbf{x}_i} \leq \norm{\mathbf{x}_i - \mathbf{x}_j} +
  \norm{\mathbf{x}_j} = 1 + \norm{\mathbf{x}_j} \equiv M \]
  for all $i \geq j$ and some fixed $j$. Thus, $\norm{\mathbf{x}_i}
  \leq M$ for some constant $M$ and all $i \geq j$. Then, 
  \[
  s_m^\ast = \lim_{i \to \infty} \sum_{n=1}^m |x_{in}|^p \leq
  M^{p} \]
  for all $m$. $s_m^\ast$ is a bounded weakly increasing sequence in
  $\R$, so it must converge.\footnote{Let $\seq{x} \in \R$ and suppose $x_1
    \leq x_2 \leq x_3 \leq ...$ and $\seq{x}$ is bounded, then we will
    show $\seq{x}$ converges. Suppose not. Then the sequence has no
    accumulation points. In particular, $x_i$ is not an accumulation
    point of the sequence for any $i$ i.e. there is an $\epsilon>0$
    such that for all $i$ there are finitely many $j$ with $d(x_i,
    x_j) < \epsilon$. Then we can construct a subsequence
    by choosing $j_k$ such that $j_k> j_{k-1}$ and $|x_{j_k} -
    x_{j_{k-1}} | > \epsilon$. But then 
    \begin{align*}
      x_{j_k} = & x_{j_1} + (x_{j_2} - x_{j_1}) + (x_{j_3} - x_{j_2}) +
      ... + (x_{j_k} - x_{j_{k-1}} ) \\
      \geq & x_{j_1} + (k-1) \epsilon 
    \end{align*}
    which is not bounded.}
  
  Finally, we should show that $\seq{\mathbf{x}}$ converges to
  $\mathbf{x}^\ast$. Let $\epsilon>0$. 
  Since the original sequence is Cauchy, there is a $N$ such that if
  $i,j > N$, then
  \begin{align*}
    \sum_{m=1}^M \abs{x_{im} - x_{jm}}^p \leq \norm{\mathbf{x}_i -
      \mathbf{x}_j}^p < \epsilon
  \end{align*}
  for all $M$. Therefore, 
  \begin{align*}
    \lim_{j \to \infty} \sum_{m=1}^M \abs{x_{im} - x_{jm}}^p =
    \sum_{m=1}^M\abs{x_{im} - x_m^\ast} < \epsilon 
  \end{align*}
  for all $i \geq N$ and all $M$. Thus, 
  \[
  \norm{\mathbf{x}_i - \mathbf{x}^\ast} = \lim_{M \to \infty}
  \sum_{m=1}^M\abs{x_{im} - x_m^\ast} < \epsilon 
  \]
  for all $i \geq N$, so the sequence converges.
 \end{example}

%%%%%%%%%%%%%%%%%%%%%%%%%%%%%%%%%%%%%%%%%%%%%%%%%%%%%%%%%%%%%%%%%%%%%%
\section{Open sets}

\begin{definition}
  Let $X$ be a metric space and $x \in X$. A \textbf{neighborhood} of
  $x$ is the set 
  \[ N_\epsilon (x) = \{y \in X: d(x,y) < \epsilon. \]
\end{definition}
A neighborhood is also called an open $\epsilon$\textbf{-ball} of $x$
and written $B_{\epsilon}(x)$. 
\begin{definition}
  A set, $S \subseteq X$ is \textbf{open} if $\forall x \in S$,
  $\exists$ $\epsilon>0$ such that 
  \[ N_\epsilon(x) \subset S. \]
\end{definition}
For every point in an open set, you can find a small neighborhood
around that point such that the neighborhood lies entirely within the
set. 
\begin{example}
  Any open interval, $(a,b) = \{x \in \R: a<x<b\}$, is an open set. 
\end{example}
\begin{example}
  Any linear subspace of dimension $k < n$ in $\R^n$ is not open. 
\end{example}

\begin{theorem} $\,$
  \begin{enumerate} \label{thm:uio}
  \item Any union of open sets is open. (finite or infinite)
  \item The \emph{finite} intersection of open sets is open.
  \end{enumerate}
\end{theorem}
\begin{proof}
  Let $S_j$, $j \in J$ be a collection of open sets. Pick any $j_0 \in
  J$. If $x \in \cup_{j \in J} S_j$, then there must be $\epsilon_{j_0} >
  0$ such that $N_{\epsilon_{j_0}} (x) \subset S_{j_0}$. It is
  immediate that $N_{\epsilon_{j_0}} (x) \subset \cup_{j \in J} S_j$
  as well. 

  Let $S_1, .., S_k$ be a finite collection of open sets. For each $i$
  $\exists \epsilon_i > 0$ such that $N_{\epsilon_i}(x) \subset
  S_i$. Let $\underline{\epsilon} = \min_{i \in \{1,..., k\}}
  \epsilon_i$. Then $\underline{\epsilon}>0$ since it is the minimum
  of a finite set of positive numbers. Also,
  $N_{\underline{\epsilon}}(x) \subset S_i$ for each $i$, so
  $N_{\underline{\epsilon}}(x) \subset \cap_{i=1}^k S_i$. 
\end{proof}
\begin{definition}
  The \textbf{interior} of a set $A$ is the union of all open sets
  contained in $A$. It is denoted as $\mathrm{int}(A)$.
\end{definition}
From the previous, theorem, we know that the interior of any set is
open. 
\begin{example}
  Here some examples of the interior of sets in $\R$.
  \begin{enumerate}
  \item $A = (a,b)$, $\mathrm{int}(A) = (a,b)$.
  \item $A = [a,b]$, $\mathrm{int}(A) = (a,b)$.
  \item $A = \{1, 2, 3, 4, ... \}$, $\mathrm{A} = \emptyset$
  \end{enumerate}  
\end{example}

\section{Closed sets}

A closed set is the opposite of an open set. 
\begin{definition}
  A set $S \subseteq X$ is closed if its complement, $X^c$, is open. 
\end{definition}
\begin{theorem} $\;$\label{thm:uic}
  \begin{enumerate}
  \item The intersection of any collection of closed sets is closed.
  \item\label{uic2} The union of any finite collection of closed sets is closed. 
  \end{enumerate}
\end{theorem}
\begin{proof}
  Let $C_j$, $j\in J$ be a collection of closed sets. Then 
  $ \left(\cap_{j \in J} C_j\right)^c = \cup_{j \in J}
  C^c_j $. 
  $C^c_j$ are open, so by theorem \ref{thm:uio},  $\cup_{j \in J}
  C^c_j = \left(\cap_{j \in J} C_j\right)^c =$ is open.  
  
  The proof of part \ref{uic2} is similar. 
\end{proof}

\begin{example}[Closed sets]
  Some examples of closed sets include
  \begin{enumerate}
  \item $[a,b] \subseteq \R$
  \item Any linear subspace of $\R^n$
  \item $\{(x,y) \in \R^2: x^2 + y^2 \leq 1\}$
  \end{enumerate}
\end{example}

Closed sets can also be defined as sets that contain the limit of any
convergent sequence in the set. Simon and Blume use this
definition. The next theorem shows that their definition is equivalent
to ours.
\begin{theorem}\label{thm:clim}
  Let $\{x_n\}$ be any convergent sequence with each element contained
  in a set $C$. Then $\lim x_n = x \in C$ for all such $\{x_n\}$ if
  and only if $C$ is closed.
\end{theorem}
\begin{proof}
  First, we will show that any set that contains the limit points of
  all its sequences is closed. Let $x \in C^c$. Consider
  $N_{1/n}(x)$. If for any $n$, $N_{1/n}(x) \subset C^c$, then $C^c$
  is open, and $C$ is closed as desired. If for all $n$, $N_{1/n}(x)
  \not \subset C^c$, then $\exists y_n \in N_{1/n}(x) \cap C$. The
  sequence $\{y_n\}$ is in $C$ and $y_n \to x$. However, by assumption
  $C$ contains the limit of any sequence within it. Therefore, there
  can be no such $x$, and $C^c$ must be open and $C$ is closed.

  Suppose $C$ is closed. Then $C^c$ is open. Let $\{x_n\}$ be in $C$
  and $x_n \to x$. Then $d(x_n, x) \to 0$, and for any $\epsilon > 0$,
  $\exists x_n \in N_\epsilon(x)$. Hence, there can be no $\epsilon$
  neighborhood of $x$ contained in $C^c$. $C^c$ is open by assumption,
  so $x \not\in C^c$ and it must be that $x \in C$. 
\end{proof}

\begin{definition}
  The \textbf{closure} of a set $S$, denoted by $\overline{S}$ (or
  $\mathrm{cl}(S)$), is the intersection of all closed sets containing $S$.
\end{definition}

%\begin{definition}
%  A \textbf{limit point} of a set $S$ is any $x$ such that for any
%  $\epsilon>0$, $N_\epsilon(x) \cap S \neq \emptyset$.
%\end{definition}
\begin{example}
  If $S$ is closed, $\overline{S} = S$. 
\end{example}
\begin{example}
  $\overline{(0,1]} = [0,1]$
\end{example}
\begin{lemma}\label{lem:closure}
  $\overline{S}$ is the set of limits of convergent sequences in $S$.
\end{lemma}
\begin{proof}
  Let $\{x_n\}$ be a convergent sequence in $S$ with limit $x$. If $C$
  is any closed set containing $S$, then $\{x_n\}$ is in $C$ and by
  theorem \ref{thm:clim}, $x \in C$. Therefore, $x \in \overline{S}$. 
  
  Let $x \in \overline{S}$. For any $\epsilon>0$, $N_\epsilon(x) \cap
  S \neq \emptyset$ because otherwise $N_\epsilon(x)^c$ is a closed
  set containing $S$, but not $x$. Therefore, we can construct a
  sequence $x_n \in S \cap N_{1/n}(x)$ that converges to $x$ and is in
  $S$. 
\end{proof}
\begin{example}
  $\overline{\{1/n\}_{n \in \mathbb{N}}} = \{0, 1, 1/2, 1/3, ... \}$
\end{example}
\begin{definition}
  The \textbf{boundary} of a set $S$ is $\overline{S} \cap
  \overline{S^c}$. 
\end{definition}
\begin{example}
  The boundary of $[0,1]$ is $\{0,1\}$.
\end{example}
\begin{example}
  The boundary of the unit ball, $\{x \in \R^2: \norm{x} < 1\}$ is the
  unit circle, $\{x \in \R^2: \norm{x} = 1\}$.
\end{example}
\begin{lemma}
  If $x$ is in the boundary of $S$ then $\forall \epsilon>0$,
  $N_\epsilon(x) \cap S \neq \emptyset$ and $N_\epsilon(x) \cap S^c
  \neq \emptyset$. 
\end{lemma}
\begin{proof}
  As in the proof of lemma \ref{lem:closure}, all
  $\epsilon$-neighborhoods of $x \in \overline{S}$ must intersect with
  $S$. The same applies to $S^c$. 
\end{proof}

\section{Compact sets}

\begin{definition}
  An \textbf{open cover} of a set $S$ is a collection of open sets,
  $\{G_\alpha\}$  $\alpha \in \mathcal{A}$ such that $S \subset
  \cup_{\alpha \in \mathcal{A}} G_\alpha$. 
\end{definition}
\begin{example}
  Some open covers of $\R$ are:
  \begin{itemize}
  \item $\{\R\}$ 
  \item $\{(-\infty, 1), (-1, \infty)$
  \item $\{..., (-3, -1), (-2, 0),(-1,1), (0,2), (1,3), ... \}$
  \item $\{(x,y) : x<y \}$
  \end{itemize}
  The first two are finite open covers since they consist of finitely
  many open sets. The third is a countably infinite open cover. The
  fourth is an uncountably infinite open cover. 
\end{example}
\begin{example}
  Let $X$ be a metric space and $A \subseteq X$. The set of open balls
  of radius $\epsilon$ centered at all points in $A$ is an open cover
  of $A$. If $A$ is finite / countable / uncountable, then this open
  cover will also be finite / countable / uncountable. 
\end{example}
Open covers of the form in the previous example are often used to
prove some property applies to all of $A$ by verifying the property in
each small $N_\epsilon(x)$. Unfortunately, this often involves taking
a maximum or sum of something for each set in the open cover. When the
open cover is infinite, it can be hard to ensure that the infinite sum
or maximum stays finite. When we have a finite open cover, we know
that things will remain finite. 
\begin{definition}
  A set $K$ is \textbf{compact} if every open cover of $K$ has a
  finite subcover. 
\end{definition}
By a finite subcover, we mean that there is finite set $G_{\alpha_1},
... G_{\alpha_k}$ such that $S \subset \cup_{j=1}^k
G_{\alpha_j}$. Compact sets are a generalization of finite sets. Many
facts that are obviously true of finite sets are also true for compact
sets, but not true for infinite sets that are not compact. Suppose we
want to show a set has some property. If the set is compact, we can
cover it with a finite number of small $\epsilon$ balls and then we
just need to show that each small ball has the property we want. We
will see many concrete examples of this technique in the next few
weeks.
\begin{example}
  $\R$ is not compact. $\{..., (-3, -1), (-2, 0),(-1,1), (0,2), (1,3),
  ... \}$ is an infinite cover, but if we leave out any single
  interval (the one beginning with $n$) we will fail to cover
  some number ($n+1$).
\end{example}
\begin{example}
  Let $K = \{x\}$, a set of a single point. Then $K$ is compact. Let
  $\{G_\alpha\}_{\alpha \in \mathcal{A}}$ be an open cover of
  $K$. Then $\exists$ $\alpha$ such that $x \in G_\alpha$. This single
  set is a finite subcover.
\end{example}
\begin{example}
  Let $K = \{x_1, ..., x_n\}$ be a finite set. Then $K$ is
  compact.  Let
  $\{G_\alpha\}_{\alpha \in \mathcal{A}}$ be an open cover of
  $K$. Then for each $i$, $\exists$ $\alpha_i$ such that $x_i \in
  G_{\alpha_i}$. 
  The collection $\{G_{\alpha_1}, ... G_{\alpha_n} \}$ is a finite
  subcover.
\end{example}
\begin{example}
  $(0,1) \subseteq \R$ is not compact. $\{(1/n,1)\}_{n=2}^\infty$ is
  an open cover, but there can be no finite subcover. Any finite
  subcover would have a largest $n$ and could not contain,
  e.g. $1/(n+1)$. 
\end{example}
\begin{example}
  Let $x \in V$, a normed vector space. Let $K = \{x\frac{1}{2},
  x\frac{2}{3}, x\frac{3}{4}, ... \}$. Then $K$ is not
  compact. Consider the open cover $N_{\norm{x}\frac{1}{3(n+2)^2}}(x
  \frac{n}{n+1})$ for $n=1,2, ...$. Assuming $x \neq 0$, each of these
  neighborhoods contains exactly one point of $K$, so there is no
  finite subcover. 
\end{example}

Before using compactness, let's investigate how being compact relates
to other properties of sets, such as closed/open. 
\begin{lemma}\label{lem:compactClosed}
  Let $X$ be a metric space and $K \subseteq X$. If $K$ is compact,
  then $K$ is closed.
\end{lemma}
\begin{proof}
  Let $x \in K^c$. The collection $\{N_{d(x.y)/3}(y)\}$, $y\in K$ is an
  open cover of $K$. $K$ is compact, so there is a finite subcover,
  $N_{d(x,y_1)/3}(y_1), ... , N_{d(x,y_n)/3}(y_n)$.  For each $i$,
  $N_{d(x,y_i)/3}(y_i) \cap N_{d(x,y_i)/3}(x) = \emptyset$, so 
  \[ \cap_{i=1}^n N_{d(x,y_i)/3}(x) \]
  is an open neighborhood of $x$ that is contained in $K^c$. $K^c$ is
  open, so $K$ is closed.
\end{proof}
\begin{lemma}
  Let $X$ be a metric space, $C \subseteq K \subseteq X$. If $K$ is
  compact and $C$ is closed. Then $C$ is also compact.
\end{lemma}
\begin{proof}
  Let $\{G_\alpha\}_{\alpha \in \mathcal{A}}$ be an open cover for
  $C$. Then $\{G_\alpha\}_{\alpha \in \mathcal{A}}$ plus $C^c$ is an
  open cover for $K$. Since $K$ is compact there is a finite
  subcover. Since $C \subseteq K$, the finite subcover also covers
  $C$. Therefore, $C$ is compact. 
\end{proof}

The definition of compactness is somewhat abstract. We just saw that
compact sets are always closed. Another property of compact sets is
that they are bounded. 
\begin{definition}
  Let $X$ be a metric space and $S \subseteq X$. $S$ is
  \textbf{bounded} if $\exists x_0 \in S$ and $r \in \R$ such that 
  \[ d(x,x_0) < r \]
  for all $x \in S$.
\end{definition}
A bounded set is one that fits inside an open ball of finite
radius. For subsets of $\R$ this definition is equivalent to there
being a lower and upper bound for the set. For subsets of a normed
vector space, if $S$ is bounded then there exists some $M$ such that
$\norm{x} < M$ for all $x \in S$. 
\begin{lemma}
  Let $K \subseteq X$ be compact. Then $K$ is bounded. 
\end{lemma}
\begin{proof}
  Pick $x_0 \in K$. $\{N_{r}(x_0) \}_{r \in \R}$ is an open cover of
  $K$, so there must be a finite subcover. The finite subcover has
  some maximum $r^*$. Then $K \subseteq N_{r^*}(x_0)$, so $K$ is bounded.
\end{proof}
This lemma along with lemma \ref{lem:compactClosed} show that if a set
is compact then it is also closed and bounded. In $\R^n$, the converse
is also true.
\begin{theorem}[Heine-Borel]\label{thm:hb}
  A set $S \subseteq \R^n$ is compact if and only if it is closed and
  bounded. 
\end{theorem}
\begin{proof}
  We already showed that if $S$ is compact, then it is closed and
  bounded. 

  Now suppose $S$ is closed and bounded. Since $S$ is bounded, it is a
  subset of some $n$-dimensional cube, say $[-a,a]^n$ (i.e.\ the set
  of all vectors $x = (x_1,..,x_n)$ with $-a \leq x_i \leq a$).  We
  will show $[-a,a]^n$ is compact, and then use the fact that a
  closed subset of a compact set is compact.

  Let's just show $[-a,a]^n$ is compact for $n=1$. The argument for
  larger $n$ is similar, but the notation is more cumbersome. If
  $[-a,a]$ is not compact, then there is an infinite open cover with
  no finite subcover, say $\{G_\alpha\}_{\alpha \in \mathcal{A}}$. If
  we cut the interval into two halves, $[-a,0]$ and $[0,a]$, at least
  one of them must have no finite subcover. We can repeat this
  argument many times to get nested closed intervals of length
  $a/(2^k)$ for any $k$. Call the $k$th interval $I_k$. We claim that
  $\cap_{k=1}^\infty I_k \neq \emptyset$. To show this take the
  sequence of lower endpoints of the intervals, call it
  $\seq{x}$. This is a Cauchy sequence, so it converges to some limit,
  $x_0$. Also, for any $k$, $\{x_n\}_{n=k}^\infty$ is a sequence in
  $I_k$. $I_k$ is closed so $x_0 \in I_k$. Thus $x_0 \in
  \cap_{k=1}^\infty I_k$. On the other hand, $I_k \subset \cup_{\alpha
    \in \mathcal{A}} G_\alpha$ for all $k$. Therefore, $x_0$ must be
  in some open $G_\alpha$ as part of this cover. Then $\exists
  \epsilon>0$ such that $N_\epsilon(x_0) \subset G_\alpha$. However,
  for $k>1/\epsilon$, $I_k \subset N_\epsilon(x_0) \subset G_\alpha$,
  and then $I_k$ has a finite subcover. Therefore, $[-a,a]$ must be
  compact. The argument for $n>1$ is very similar. For $n=2$, we would
  divide the square $[-a,a]^2$ into four smaller squares. For $n=3$,
  we would divide the cube into eight smaller cubes. In general we
  would divide the hypercube $[-a,a]^n$ into $2^n$ hypercubes with
  half the side length. 
\end{proof}
You may wonder whether closed and bounded sets are always compact. We
know that all finite dimensional real vector spaces are isomorphic to
$\R^n$. In any such space, sets are compact iff they are closed and
bounded. However, in infinite dimensional spaces, there are closed and
bounded sets that are not compact. The argument in the previous proof
does not apply to infinite-dimensional spaces because an infinite
dimensional hypercube can only be divided into infinitely many
hypercubes with half the side length. 
\begin{example}
  $\ell^\infty = \{ (x_1, x_2, ...) : \sup_{i} |x_i| < \infty$ with
  norm $\norm{x} = \sup_{i} |x_i|$ is a normed vector space. Let $e_i$
  be the element of all $0$s except for the $i$th position, which is
  $1$. Then $E = \{e_i\}_{i=1}^\infty$ is closed and bounded. However,
  $E$ is not compact because $\{N_{1/2}(e_i)\}_{i=1}^\infty$ is an
  open cover with no finite subcover. 
\end{example}

We saw that closed sets contain the limit points of all their
convergent sequences. There is also a relationship between compactness
and sequences. 
\begin{definition}
  Let $X$ be a metric space and $K \subseteq X$. $K$ is
  \textbf{sequentially compact} if every sequence in $K$ has an
  accumulation point in $K$.
\end{definition}
Sometimes this definition is written as: $K$ is sequentially compact
if every sequence in $K$ has a subsequence that converges in
$K$. Compactness implies sequential compactness. 
\begin{lemma}\label{lem:compactSeqCompact}
    Let $X$ be a metric space and $K \subseteq X$ be compact. Then $K$
    is sequentially compact. 
\end{lemma}
\begin{proof}
  Let $\seq{x}$ be a sequence in
  $K$. Pick any $\epsilon>0$, $N_\epsilon(x)$, $x \in K$ is an open
  cover of $K$, so there is a finite subcover. Therefore, one of the
  $\epsilon$ neighborhoods must contain an infinite number of the
  elements from the sequence. Call this neighborhood
  $N_\epsilon(x_1^\ast)$. Pick the smallest $n$ such that $x_n \in
  N_\epsilon(x_1^\ast)$ and call it $n_1$. $\overline{N_{\epsilon}(x_1^\ast)}
  \cap K$ is a closed subset of the compact set $K$, so is itself
  compact. Repeat the above argument with $\epsilon/2$ in place of
  $\epsilon$ and $\overline{N_{\epsilon}(x_1^\ast)} \cap K$ in place of $K$
  to find an $n_2$, $n_3$, etc. Then the subsequence $x_{n_1},
  x_{n_2}, ...$ is a Cauchy sequence, so it converges. Its limit is an
  accumulation point. Its limit must be in $K$ because $K$ is compact,
  and so, closed. Therefore, $K$ is sequentially compact.
\end{proof}

In $\R^n$, a set is sequentially compact iff it is compact iff it is
closed and bounded.
\begin{theorem}[Bolzano-Weierstrass] \label{thm:bw}
  A set $S \subseteq \R^n$ is closed and bounded if and only if it is
  sequentially compact. 
\end{theorem}
\begin{proof}
  Let $S$ be closed and bounded. By the Heine-Borel theorem
  (\ref{thm:hb}), $S$ is compact. By lemma
  \ref{lem:compactSeqCompact}, $S$ is sequentially compact. 

  Let $S$ be sequentially compact. Let $\{x_n\}$ be a convergent
  sequence in $S$. Its limit is an accumulation point, so it must be
  in $S$. Therefore, $S$ is closed. To show $S$ is bounded, pick $x_0
  \in S$. Suppose $\exists x_1 \in S$ such that $d(x_1, x_0) \geq 1$,
  and $x_2 \in S$ such that $d(x_2, x_0) \geq 2$ etc. This sequence is
  not Cauchy because of the reverse triangle inequality,  
  \begin{align*}
    d(x_i,x_j) \geq & \abs{d(x_i,x_0) - d(x_j,x_0) } = |i-j|
  \end{align*}  
  Therefore, this would be a sequence in $S$ with no accumulation
  points. Therefore, it must not always be possible to find such
  $x_n$. In other words, $S$ must be bounded.
\end{proof}

\begin{remark}
  This theorem is sometimes stated as ``each bounded sequence in $\R^n$
  has a convergent subsequence.'' As an exercise, you may want to
  verify that this statement is equivalent to the one above.
\end{remark}
Simon and Blume also prove this theorem in chapter 29.2. They do not
prove the Heine-Borel theorem first though, so their proof is of the
Bolzano-Weierstrass theorem is longer. Perhaps unsurprisingly, the
details of their proof are somewhat similar to our proof of the
Heine-Borel theorem.

In $\R^n$, compactness, sequential compactness, and closed and bounded
are all the same. In general metric spaces, this need not be true. We
saw above that in infinite dimensional normed vector spaces, there are
closed and bounded sets which are not compact.  However it is always
true that sequential compactness and compactness are the same for
metric spaces. We already showed that compactness implies sequential
compactness.  The proof that sequential compactness implies
compactness is a somewhat long and difficult, and you may want to skip
it unless you are especially interested.
\begin{theorem}
  Let $X$ be a metric space and $K \subseteq X$. $K$ is compact if and
  only if $K$ is sequentially compact. 
\end{theorem}
\begin{proof}
  Lemma \ref{lem:compactSeqCompact} shows that if $K$ is compact, then
  $K$ is sequentially compact.
  
  Suppose every sequence in $K$ has a convergent subsequence with a
  limit point in $K$. Let $G_{\alpha}$, $\alpha \in \mathcal{A}$ be an
  open cover of $K$. $\mathcal{A}$ could be uncountable, so we will
  begin by showing that there must be a countable subcover.  Let
  $n=1$. Pick $x_1 \in K$. If possible choose $x_2 \in K$ such that
  $d(x_1,x_2) \geq 1/n$. Repeat this process, choosing $x_j$ in $K$
  such that $d(x_j,x_i) \geq 1/n$ for each $i<j$. Eventually this will
  no longer be possible because otherwise we could construct a
  sequence with no convergent subsequence. When it is no longer
  possible, set $n = n+1$. This gives a countable collection of open
  neighborhoods $N_{1/n}(x_i)$ that cover $K$ for each $n$ and get
  arbitrarily small as $n$ increases. Call these neighborhoods
  $\eta_j$ for $j = 1,2,..$. Let $J$ be set of all $\eta_j$ such that
  $\eta_j \subseteq G_\alpha$ for some $\alpha$. $J$ is a subset of a
  countable set, so $J$ is countable. Note that $\cup_{j \in J} \eta_j
  \supset K$ because if $x \in K$, then $x \in G_\alpha$ for some
  $\alpha$, and then $\exists \epsilon$ such that $N_{\epsilon}(x) \in
  G_{\alpha}$ and $\exists j$ s.t. $\eta_j \subset N_{\epsilon}(x)$.
  Finally, for each $j \in J$ choose $G_{\alpha_j}$ such that $\eta_j
  \subseteq G_{\alpha_j}$. Such $\alpha_j$ exist by construction. Also
  $\cup_{j \in J} G_{\alpha_j} \supset \cup_{j \in J} \eta_j \supset
  K$. So $G_{\alpha_j}$ is a countable subcover.

  If $G_{\alpha_j}$ has no finite subcover, then for each $n$, 
  \[ F_n = \left(\cup_{i=1}^n G_{\alpha_i}\right)^c \cap K \] is not
  empty (if it were empty, then $\cup_{i=1}^n G_{\alpha_i}$ would be a
  finite subcover). Choose $x_n \in F_n$. Then $\{x_n\}$ is a sequence in $K$,
  and it must have a convergent subsequence with a limit, $x_0$, in
  $K$. However, each $F_{i+1} \subset F_i$ and $F_i$ are all
  closed. Therefore, the sequence $\{x_j\}_{j=i}^\infty$ is also in
  $F_i$ and so is its limit. Then $x_0 \in \cap_{i=1}^\infty
  F_i$. However, 
  \[ \cap_{i=1}^\infty F_i = \left(\cup_{i=1}^\infty
    G_{\alpha_i}\right)^c \cap K, \]  
  but $G_{\alpha_i}$ is a countable cover of $K$, which implies 
  \[ \cap_{i=1}^\infty F_i = \left(\cup_{i=1}^\infty
    G_{\alpha_i}\right)^c \cap K = \emptyset\] 
  and we have a contradiction. Therefore, $G_{\alpha_j}$ must have a
  finite subcover, and $K$ is compact.
\end{proof}

\begin{remark}
  There are non-metric spaces where sequential compactness and
  compactness are not equivalent. One can define open sets on a space
  without a metric by simply specifying which sets are open and making
  it such that theorem \ref{thm:uio} holds. Such a space is called a
  topological space. You can then define closed sets, compact sets,
  and sequential compactness in terms of open sets. On the problem
  set, you will see that you can define continuity of functions in
  terms of open and closed sets. Topology is the branch of mathematics
  that studies topological spaces. One interesting observation is that
  on $\R^n$, if a set is open with respect to some $p$-norm, then it
  is also open with respect to any other $p$-norm. Thus, we say that
  $\R^n$ with the $p$-norms are topologically equivalent or
  homeomorphic. Properties like continuity and compactness are the
  same regardless of what $p$-norm we use. 
  
  As far as I know, topological spaces that are not metric spaces do
  not come up very often in economics, so we will not be studying
  them.
\end{remark}

To review, in $\R^n$ a set is compact if any of the following three
things hold:
\begin{enumerate}
\item For every open cover there exists a finite subcover,
\item Every sequence in the set has a convergent subsequence, or
\item The set is closed and bounded.
\end{enumerate}
In infinite dimensional spaces, closed and bounded sets need not be
compact, but compact sets are always closed and bounded. In any metric
space, a set is compact iff it is sequentially compact.

\end{document}
