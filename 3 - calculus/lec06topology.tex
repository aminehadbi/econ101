\documentclass[12pt,reqno]{amsart}

\usepackage{amsmath}
\usepackage{amsfonts}
\usepackage{graphicx}
%\usepackage{epstopdf}
\usepackage{hyperref}
\usepackage[left=1in,right=1in,top=0.9in,bottom=0.9in]{geometry}
\usepackage{multirow}
\usepackage{verbatim}
\usepackage{fancyhdr}
%\usepackage[small,compact]{titlesec} 

%\usepackage{pxfonts}
%\usepackage{isomath}
\usepackage{mathpazo}
%\usepackage{arev} %     (Arev/Vera Sans)
%\usepackage{eulervm} %_   (Euler Math)
%\usepackage{fixmath} %  (Computer Modern)
%\usepackage{hvmath} %_   (HV-Math/Helvetica)
%\usepackage{tmmath} %_   (TM-Math/Times)
%\usepackage{cmbright}
%\usepackage{ccfonts} \usepackage[T1]{fontenc}
%\usepackage[garamond]{mathdesign}
\usepackage{color}
\usepackage[normalem]{ulem}

\newtheorem{theorem}{Theorem}[section]
\newtheorem{conjecture}{Conjecture}[section]
\newtheorem{corollary}{Corollary}[section]
\newtheorem{lemma}{Lemma}[section]
\newtheorem{proposition}{Proposition}[section]
\theoremstyle{definition}
\newtheorem{assumption}{}[section]
%\renewcommand{\theassumption}{C\arabic{assumption}}
\newtheorem{definition}{Definition}[section]
\newtheorem{step}{Step}[section]
\newtheorem{remark}{Comment}[section]
\newtheorem{example}{Example}[section]
\newtheorem*{example*}{Example}

\linespread{1.1}

\pagestyle{fancy}
%\renewcommand{\sectionmark}[1]{\markright{#1}{}}
\fancyhead{}
\fancyfoot{} 
%\fancyhead[LE,LO]{\tiny{\thepage}}
\fancyhead[CE,CO]{\tiny{\rightmark}}
\fancyfoot[C]{\small{\thepage}}
\renewcommand{\headrulewidth}{0pt}
\renewcommand{\footrulewidth}{0pt}

\fancypagestyle{plain}{%
\fancyhf{} % clear all header and footer fields
\fancyfoot[C]{\small{\thepage}} % except the center
\renewcommand{\headrulewidth}{0pt}
\renewcommand{\footrulewidth}{0pt}}

\makeatletter
\renewcommand{\@maketitle}{
  \null 
  \begin{center}%
    \rule{\linewidth}{1pt} 
    {\Large \textbf{\textsc{\@title}}} \par
    {\normalsize \textsc{Paul Schrimpf}} \par
    {\normalsize \textsc{\@date}} \par
    {\small \textsc{University of British Columbia}} \par
    {\small \textsc{Economics 526}} \par
    \rule{\linewidth}{1pt} 
  \end{center}%
  \par \vskip 0.9em
}
\makeatother

\newcommand{\argmax}{\operatornamewithlimits{arg\,max}}
\newcommand{\argmin}{\operatornamewithlimits{arg\,min}}
\def\inprobLOW{\rightarrow_p}
\def\inprobHIGH{\,{\buildrel p \over \rightarrow}\,} 
\def\inprob{\,{\inprobHIGH}\,} 
\def\indist{\,{\buildrel d \over \rightarrow}\,} 
\def\F{\mathbb{F}}
\def\R{\mathbb{R}}
\newcommand{\gmatrix}[1]{\begin{pmatrix} {#1}_{11} & \cdots &
    {#1}_{1n} \\ \vdots & \ddots & \vdots \\ {#1}_{m1} & \cdots &
    {#1}_{mn} \end{pmatrix}}
\newcommand{\iprod}[2]{\left\langle {#1} , {#2} \right\rangle}
\newcommand{\norm}[1]{\left\Vert {#1} \right\Vert}
\newcommand{\abs}[1]{\left\vert {#1} \right\vert}
\renewcommand{\det}{\mathrm{det}}
\newcommand{\rank}{\mathrm{rank}}
\newcommand{\spn}{\mathrm{span}}
\newcommand{\row}{\mathrm{Row}}
\newcommand{\col}{\mathrm{Col}}
\renewcommand{\dim}{\mathrm{dim}}
\newcommand{\prefeq}{\succeq}
\newcommand{\pref}{\succ}
\newcommand{\seq}[1]{\{{#1}_n \}_{n=1}^\infty }
%\newcommand{\to}{{\rightarrow}}


\title{Limits and topology of metric spaces}
\date{\today}

\begin{document}

\maketitle

This lecture focuses on sequences, limits, and topology. Similar
material is covered in chapters 12 and 29 of Simon and Blume, or 1.3
of Carter. 

\section{Sequences and limits}

A \textbf{sequence} is a list of elements, $\{x_1, x_2, ... \}$ or
$\seq{x}$ or sometimes just $\{x_n\}$. Although the notation for a
sequence is similar to the notation for a set, they should not be
confused. Sequences are different from sets in that the order of
elements in a sequence matters, and the same element can appear many
times in a sequence.  Some examples of sequences with $x_i \in \R$
include
\begin{enumerate}
\item\label{sfib} $\{ 1, 1, 2, 3, 5, 8, ... \}$
\item\label{sc} $\{ 1, \frac{1}{2}, \frac{1}{3}, \frac{1}{4}, ... \}$
\item\label{sd} $\{\frac{1}{2}, \frac{-2}{3}, \frac{3}{4}, \frac{-4}{5},
  \frac{5}{6}, ... \}$
\end{enumerate}
Some sequences, like \ref{sc}, have elements that all get closer and
closer to some fixed point. We say that these types of sequences
converge. A sequence that does not converge diverges. Some divergent
sequences like \ref{sfib}, increase without bound. Other divergent
sequences, like \ref{sd}, are bounded, but they do not converge to any
single point. 

To analyze sequences with elements that are not necessarily real
numbers, we need to be able to say how far apart the entries in the
sequence are. 
\begin{definition}
  A \textbf{metric space} is a set, $X$, and function $d:X\times X
  \rightarrow \R$ called a \textbf{metric} (or distance) such that
  $\forall x, y, z \in X$
  \begin{enumerate}
  \item $d(x,y) > 0$ unless $x=y$ and then $d(x,x) = 0$
  \item (symmetry) $d(x,y) = d(y,x)$
  \item (triangle inequality) $d(x,y) \leq d(x,z) + d(z,y)$.
  \end{enumerate}
\end{definition}
\begin{example}
  $\R$ is a metric space with $d(x,y) = |x - y|$. 
\end{example}
\begin{example}
  Any normed vector space is a metric space with $d(x,y) =
  \norm{x-y}$. 
\end{example}
The most common metric space that we will encounter will be $\R^n$
with the Euclidean metric, $d(x,y) = \norm{x-y} = \sqrt{\sum_{i=1}^n
  (x_i - y_i)^2}$.  
\begin{definition}
  A sequence $\{x_n\}_{n=1}^\infty$ in a metric space \textbf{converges} to $x$
  if $\forall \epsilon > 0$ $\exists N$ such that 
  \[ d(x_n, x) < \epsilon \]
  for all $n \geq N$. We call $x$ the \textbf{limit} of $\seq{x}$ and
  write $\lim_{n \rightarrow \infty} x_n = x$ or $x_n \rightarrow x$. 
\end{definition}
\begin{definition}
  $a$ is an \textbf{accumulation point} of $\seq{x}$ if $\forall
  \epsilon > 0$ $\exists$ infinitely many $x_i$ such that 
  \[ d(a, x_i ) < \epsilon. \]
\end{definition}

\begin{lemma}
  If $x_n \rightarrow x$, then $x$ is an accumulation point of
  $\seq{x}$. 
\end{lemma}
\begin{proof}
  Let $\epsilon>0$ be given. By the definition of convergences,
  $\exists N$ such that 
  \[ d(x_n,x) < \epsilon \]
  for all $n \geq N$. $\{n\in \mathbb{N}: n\geq N\}$ is infinite, so
  $x$ is an accumulation point.
\end{proof}
The third example of a sequence at the start of this section,
\ref{sd}, shows that the converse of this lemma is false. Not every
accumulation point is a limit. 
\begin{definition}
  Given $\seq{x}$ and any sequence of positive integers, $\{n_k\}$
  such that $n_1 < n_2 < ... $ we call $\{x_{n_k}\}$ a
  \textbf{subsequence} of $\seq{x}$. 
\end{definition}
In example \ref{sd}, there are two accumulation points, $-1$ and $1$,
and you can find subsequences that converge to these points. 
\begin{lemma}
  Let $a$ be an accumulation point of $\{x_n\}$. Then $\exists$
  a subsequence that converges to $a$. 
\end{lemma}
\begin{proof}
  We can construct a subsequence as follows. Let $\{\epsilon_k\}$ be a
  sequence that converges to zero with $\epsilon_k >0 \forall k$, (for
  example, $\epsilon_k = 1/k$). By the definition of accumulation
  point, for each $\epsilon_k$ $\exists$ infinitely many $x_n$ such
  that 
  \begin{align}
    d(x_n, a) < \epsilon_k \label{ieq:a}
  \end{align}
  Pick any $x_{n_1}$ such that (\ref{ieq:a}) holds for
  $\epsilon_1$. For $k>1$, pick $n_k \neq n_{j}$ for all $j<k$ and
  such that (\ref{ieq:a}) holds for $\epsilon_k$. Such an $n_k$ always
  exists because there are infinite $x_n$ that satisfy
  (\ref{ieq:a}). By construction, $\lim_{k\rightarrow \infty} x_{n_k}
  = a$ (you should verify this using the definition of limit).
\end{proof}

\begin{theorem}
  Let $\{x_n\}$ and $\{y_n\}$ be sequences in a normed vector space
  $V$. If $x_n \to x$ and $y_n \to y$, then
  \[ x_n + y_n \to x + y. \]
\end{theorem}
\begin{proof}
  Let $\epsilon > 0$ be given. Then $\exists$ $N_x$ such that for all
  $n \geq N_x$, 
  \[ d(x_n,x) < \epsilon/2,\] and $\exists N_y$ such that
  for all $n \geq N_y$,  
  \[ d(y_n,y) < \epsilon/2. \]
  Let $N =\max\{N_x,N_y\}$. Then for all $n \geq N$, 
  \begin{align*}
    d(x_n + y_n,x+y) = \norm{(x_n + y_n) - (x+y)} \leq & \norm{x_n -
      x} + \norm{y_n - y} \\
    < & \epsilon/2 + \epsilon /2  = \epsilon.
  \end{align*}  
\end{proof}

\begin{theorem}
  Let $\{x_n \}$ be a sequence in a normed vector space with scalar
  field $\R$ and let $\{c_n\}$ be a sequence in $\R$. If
  $x_n \to x$ and $c_n \to c$ then 
  \[ x_n c_n \to x c. \]
\end{theorem}
\begin{proof}
  On problem set. 
\end{proof}

%%%%%%%%%%%%%%%%%%%%%%%%%%%%%%%%%%%%%%%%%%%%%%%%%%%%%%%%%%%%%%%%%%%%%%
\subsection{Cauchy sequences}

We have defined convergent sequences that as ones whose entries all
get close to a fixed limit point. This means that all the entries of
the sequence are also getting closer together. You might imagine a
sequence where the entries get close together without necessarily
reaching a fixed limit. 
\begin{definition}
  A sequence $\seq{x}$ is a \textbf{Cauchy} sequence if for any
  $\epsilon > 0$ $\exists N$ such that for all $i,j\geq N$,
  $d(x_i,x_j) < \epsilon$.
\end{definition}
It turns that in $\R^n$ Cauchy sequences and convergent sequences are
the same.
\begin{theorem}
  A sequence in $\R^n$ converges if and only if it is a Cauchy
  sequence. 
\end{theorem}
There is a proof of this in Chapter 29.1 of Simon and Blume. If you
want more practice with the sort of proofs in this lecture, it would
be good to read that section. The convergence of Cauchy sequences in
the real numbers is a consequence of the least upper bound property
that we discussed in lecture 1. Cauchy sequences do not converge in
all metric spaces. For example, the rational numbers are a metric
space, and any sequence of rationals that converges to an irrational
number in $\R$ is a Cauchy sequence in $\mathbb{Q}$ but has no limit
in $\mathbb{Q}$. Having Cauchy sequences converge is necessary for
proving many theorems, so we have a special name for metric spaces
where Cauchy sequences converge.
\begin{definition}
  A metric space, $X$, is \textbf{complete} if every Cauchy sequence of
  points in $X$ converges in $X$.
\end{definition}
Completeness is important for so many results that complete version of
vector spaces are named after the mathematicians who first studied
them. A \textbf{Banach space} is a complete normed vector space. A
\textbf{Hilbert space} is a complete inner product space.

%%%%%%%%%%%%%%%%%%%%%%%%%%%%%%%%%%%%%%%%%%%%%%%%%%%%%%%%%%%%%%%%%%%%%%
\section{Open sets}

\begin{definition}
  Let $X$ be a metric space and $x \in X$. A \textbf{neighborhood} of
  $x$ is the set 
  \[ N_\epsilon (x) = \{y \in X: d(x,y) < \epsilon. \]
\end{definition}
A neighborhood is also called an open $\epsilon$\textbf{-ball} of $x$
and written $B_{\epsilon}(x)$. 
\begin{definition}
  A set, $S \subseteq X$ is \textbf{open} if $\forall x \in S$,
  $\exists$ $\epsilon>0$ such that 
  \[ N_\epsilon(x) \subset S. \]
\end{definition}
For every point in an open set, you can find a small neighborhood
around that point such that the neighborhood lies entirely within the
set. 
\begin{example}
  Any open interval, $(a,b) = \{x \in \R: a<x<b\}$, is an open set. 
\end{example}
\begin{example}
  Any linear subspace of dimension $k < n$ in $\R^n$ is not open. 
\end{example}

\begin{theorem}
  \begin{enumerate} \label{thm:uio}
  \item Any union of open sets is open. (finite or infinite)
  \item The \emph{finite} intersection of open sets is open.
  \end{enumerate}
\end{theorem}
\begin{proof}
  Let $S_j$, $j \in J$ be a collection of open sets. Pick any $j_0 \in
  J$. If $x \in \cup_{j \in J} S_j$, then there must be $\epsilon_{j_0} >
  0$ such that $N_{\epsilon_{j_0}} (x) \subset S_{j_0}$. It is
  immediate that $N_{\epsilon_{j_0}} (x) \subset \cup_{j \in J} S_j$
  as well. 

  Let $S_1, .., S_k$ be a finite collection of open sets. For each $i$
  $\exists \epsilon_i > 0$ such that $N_{\epsilon_i}(x) \subset
  S_i$. Let $\underline{\epsilon} = \min_{i \in \{1,..., k\}}
  \epsilon_i$. Then $\underline{\epsilon}>0$ since it is the minimum
  of a finite set of positive numbers. Also,
  $N_{\underline{\epsilon}}(x) \subset S_i$ for each $i$, so
  $N_{\underline{\epsilon}}(x) \subset \cap_{i=1}^k S_i$. 
\end{proof}
\begin{definition}
  The \textbf{interior} of a set $A$ is the union of all open sets
  contained in $A$. It is denoted as $\mathrm{int}(A)$.
\end{definition}
From the previous, theorem, we know that the interior of any set is
open. 
\begin{example}
  Here some examples of the interior of sets in $\R$.
  \begin{enumerate}
  \item $A = (a,b)$, $\mathrm{int}(A) = (a,b)$.
  \item $A = [a,b]$, $\mathrm{int}(A) = (a,b)$.
  \item $A = \{1, 2, 3, 4, ... \}$, $\mathrm{A} = \emptyset$
  \end{enumerate}  
\end{example}

\section{Closed sets}

A closed set is the opposite of an open set. 
\begin{definition}
  A set $S \subseteq X$ is closed if its complement, $X^c$, is open. 
\end{definition}
\begin{theorem} \label{thm:uic}
  \begin{enumerate}
  \item The intersection of any collection of closed sets is closed.
  \item\label{uic2} The union of any finite collection of closed sets is closed. 
  \end{enumerate}
\end{theorem}
\begin{proof}
  Let $C_j$, $j\in J$ be a collection of closed sets. Then 
  $ \left(\cap_{j \in J} C_j\right)^c = \cup_{j \in J}
  C^c_j $. 
  $C^c_j$ are open, so by theorem \ref{thm:uio},  $\cup_{j \in J}
  C^c_j = \left(\cap_{j \in J} C_j\right)^c =$ is open.  
  
  The proof of part \ref{uic2} is similar. 
\end{proof}

\begin{example}[Closed sets]
  Some examples of closed sets include
  \begin{enumerate}
  \item $[a,b] \subseteq \R$
  \item Any linear subspace of $\R^n$
  \item $\{(x,y) \in \R^2: x^2 + y^2 \leq 1\}$
  \end{enumerate}
\end{example}

Simon and Blume define closed sets differently, but the next theorem
shows that their definition is equivalent to ours. 
\begin{theorem}\label{thm:clim}
  Let $\{x_n\}$ be any convergent sequence with each element contained
  in a set $C$. Then $\lim x_n = x \in C$ for all such $\{x_n\}$ if
  and only if $C$ is closed.
\end{theorem}
\begin{proof}
  First, we will show that any set that contains the limit points of
  all its sequences is closed. Let $x \in C^c$. Consider
  $N_{1/n}(x)$. If for any $n$, $N_{1/n}(x) \subset C^c$, then $C^c$
  could be open. If for all $n$, $N_{1/n}(x) \not \subset C^c$, then $\exists y_n
  \in N_{1/n}(x) \cap C$. The sequence $\{y_n\}$ is in $C$ and $y_n
  \to x$. However, by assumption $C$ contains the limit of any
  sequence within it. Therefore, there can be no such $x$, and $C^c$
  must be open and $C$ is closed. 

  Suppose $C$ is closed. Then $C^c$ is open. Let $\{x_n\}$ be in $C$
  and $x_n \to x$. Then $d(x_n, x) \to 0$, and for any $\epsilon > 0$,
  $\exists x_n \in N_\epsilon(x)$. Hence, there can be no $\epsilon$
  neighborhood of $x$ contained in $C^c$. $C^c$ is open by assumption,
  so $x \not\in C^c$ and it must be that $x \in C$. 
\end{proof}

\begin{definition}
  The \textbf{closure} of a set $S$, denoted by $\overline{S}$ (or
  $\mathrm{cl}(S)$), is the intersection of all closed sets containing $S$.
\end{definition}

%\begin{definition}
%  A \textbf{limit point} of a set $S$ is any $x$ such that for any
%  $\epsilon>0$, $N_\epsilon(x) \cap S \neq \emptyset$.
%\end{definition}

\begin{lemma}\label{lem:closure}
  $\overline{S}$ is the set of limits of convergent sequences in $S$.
\end{lemma}
\begin{proof}
  Let $\{x_n\}$ be a convergent sequence in $S$ with limit $x$. If $C$
  is any closed set containing $S$, then $\{x_n\}$ is in $C$ and by
  theorem \ref{thm:clim}, $x \in C$. Therefore, $x \in \overline{S}$. 
  
  Let $x \in \overline{S}$. For any $\epsilon>0$, $N_\epsilon(x) \cap
  S \neq \emptyset$ because otherwise $N_\epsilon(x)^c$ is a closed
  set containing $S$, but not $x$. Therefore, we can construct a
  sequence $x_n \in S \cap N_{1/n}(x)$ that converges to $x$ and is in
  $S$. 
\end{proof}

\begin{definition}
  The \textbf{boundary} of a set $S$ is $\overline{S} \cap
  \overline{S^c}$. 
\end{definition}
\begin{lemma}
  If $x$ is in the boundary of $S$ then $\forall \epsilon>0$,
  $N_\epsilon(x) \cap S \neq \emptyset$ and $N_\epsilon(x) \cap S^c
  \neq \emptyset$. 
\end{lemma}
\begin{proof}
  As in the proof of lemma \ref{lem:closure}, all
  $\epsilon$-neighborhoods of $x \in \overline{S}$ must intersect with
  $S$. The same applies to $S^c$. 
\end{proof}

\section{Compact sets}

\begin{definition}
  An \textbf{open cover} of a set $S$ is a collection of open sets,
  $\{G_\alpha\}$  $\alpha \in \mathcal{A}$ such that $S \subset
  \cup_{\alpha \in \mathcal{A}} G_\alpha$. 
\end{definition}

\begin{definition}
  A set $K$ is \textbf{compact} if every open cover of $K$ has a
  finite subcover. 
\end{definition}
By a finite subcover, we mean that there is finite set $G_{\alpha_1},
... G_{\alpha_k}$ such that $S \subset \cup_{j=1}^k
G_{\alpha_j}$. Compact sets are a generalization of finite sets. Many
facts that are obviously true of finite sets are also true for compact
sets, but not true for infinite sets that are not compact. Suppose we
to show a set has some property. If the set is compact, we can cover
it with a finite number of small $\epsilon$ balls and then we just
need to show that each small ball has the property we want. We will
see many concrete examples of this technique in the next few weeks.
\begin{lemma}
  Let $X$ be a metric space and $K \subseteq X$. If $K$ is compact,
  then $K$ is closed.
\end{lemma}
\begin{proof}
  Let $x \in K^c$. The collection $\{N_{d(x.y)/3}(y)\}$, $y\in K$ is an
  open cover of $K$. $K$ is compact, so there is a finite subcover,
  $N_{d(x,y_1)/3}(y_1), ... , N_{d(x,y_n)/3}(y_n)$.  For each $i$,
  $N_{d(x,y_i)/3}(y_i) \cap N_{d(x,y_i)/3}(x) = \emptyset$, so 
  \[ \cap_{i=1}^n N_{d(x,y_i)/3}(x) \]
  is an open neighborhood of $x$ that is contained in $K^c$. $K^c$ is
  open, so $K$ is closed.
\end{proof}
The definition of compactness is somewhat abstract. Fortunately, in
$\R^n$, compactness is equivalent to being closed and bounded, two
properties that are easier to understand.
\begin{definition}
  Let $X$ be a metric space and $S \subseteq X$. $S$ is
  \textbf{bounded} if $\exists x_0 \in S$ and $r \in \R$ such that 
  \[ d(x,x_0) < r \]
  for all $x \in S$.
\end{definition}
A bounded set is one that fits inside an open ball of finite
radius. For subsets of $\R$ this definition is equivalent to there
being a lower and upper bound for the set. For subsets of a normed
vector space, if $S$ is bounded then there exists some $M$ such that
$\norm{x} < M$ for all $x \in S$. 
\begin{theorem}[Heine-Borel]\label{thm:hb}
  A set $S \subseteq \R^n$ is compact if and only if it is closed and
  bounded. 
\end{theorem}
\begin{proof}
  Suppose $S \subseteq \R^n$ is compact. Pick
  $\epsilon>0$. $N_\epsilon(x)$ for each $x\in S$ is an open cover, so
  there exists a finite subcover.  Call the finite subcover
  $N_\epsilon(x_0), N_{\epsilon}(x_1), ... ,N_{\epsilon}(x_n)$. Let $x
  \in S$. Then $x \in N_\epsilon(x_j)$ for some $j$. By the triangle
  inequality, 
  \begin{align*}
    d(x_0,x) \leq d(x_0,x_j) + d(x_j,x) < d(x_0,x_j) + \epsilon.
  \end{align*}
  For any $x \in S$, we have
  \begin{align*}
    d(x_0,x) < \max_{j\in\{1, ..., n\}} d(x_0,x_j) + \epsilon 
  \end{align*}
  so $S$ is bounded.
  
  Now suppose $S$ is closed and bounded. Since $S$ is bounded, it is a
  subset of some $n$-dimensional cube, say $[-a,a]^n$ (i.e.\ the set
  of all vectors $x = (x_1,..,x_n)$ with $-a \leq x_i \leq a$).  We
  will show (1) $[-a,a]^n$ is compact, and then (2) a closed subset of
  a compact set is compact. 

  Let's just show (1) for $n=1$. If $[-a,a]$ is not compact, then
  there is an infinite open cover with no finite subcover. If we cut
  the interval into two halves, $[-a,0]$ and $[0,a]$, at least one of
  them must have an infinite open cover with no finite subcover. We
  can repeat this argument many times to get nested closed intervals
  of length $a/(2^k)$ for any $k$. Call the $k$th interval $I_k$. We
  claim that $\cap_{k=1}^\infty I_k \neq \emptyset$. To show this take
  the sequence of lower endpoints of the intervals, call it
  $\seq{x}$. This is a Cauchy sequence, so it converges to some limit,
  $x_0$. Also, for any $k$, $\{x_n\}_{n=k}^\infty$ is a sequence in
  $I_k$. $I_k$ is closed so $x_0 \in I_k$. Thus $x_0 \in
  \cap_{k=1}^\infty I_k$. On the other hand, there is supposed to be an
  infinite open cover of $I_k$ for each $k$. $x_0$ must be in some
  open $G_\alpha$ as part of this cover. Then $\exists \epsilon>0$
  such that $N_\epsilon(x_0) \subset G_\alpha$. However, for $k$ large
  enough, $I_k \subset N_\epsilon(x_0) \subset G_\alpha$, and then we
  would have a finite subcover. Therefore, $[-a,a]$ must be
  compact. The argument for $n>1$ is very similar.

  Now, let $G_\alpha$ be an open cover of $S$. Then $G_\alpha$ and
  $S^c$ is an open cover for $[-a,a]^n$. $[-a,a]^n$ is compact so
  there must be a finite subcover. This finite subcover is also
  contains $S$, so $S$ is compact.
\end{proof}

We saw that closed sets contain the limit points of all their
convergent sequences. There is also a relationship between compactness
and sequences. 
\begin{definition}
  Let $X$ be a metric space and $K \subseteq X$. $K$ is
  \textbf{sequentially compact} if every sequence in $K$ has an
  accumulation point in $K$.
\end{definition}
Sometimes this definition is written as: $K$ is compact if every
sequence in $K$ has a subsequence that converges in $K$.
\begin{theorem}[Bolzano-Weierstrass] \label{thm:bw}
  A set $S \subseteq \R^n$ is closed and bounded if and only if it is
  sequentially compact. 
\end{theorem}
\begin{proof}
  Let $S$ be closed and bounded. By the Heine-Borel theorem
  (\ref{thm:hb}), $S$ is compact. Let $\seq{x}$ be a sequence in
  $S$. For any $\epsilon>0$, $N_\epsilon(x)$, $x \in S$ is an open
  cover of $S$, so there is a finite subcover. Therefore, one of the
  $\epsilon$ neighborhoods must contain an infinite number of the
  elements from the sequence. This subsequence is a Cauchy sequence,
  so it converges. Its limit is an accumulation point. Therefore, $S$
  is sequentially compact.

  Let $S$ be sequentially compact. Let $\{x_n\}$ be a convergent
  sequence in $S$. Its limit is an accumulation point, so it must be
  in $S$. Therefore, $S$ is closed. To show $S$ is bounded, pick $x_0
  \in S$. Suppose $\exists x_1 \in S$ such that $d(x_1, x_0) \geq 1$,
  and $x_2 \in S$ such that $d(x_2, x_0) \geq 2$, etc. This would form
  a sequence in $S$ with no accumulation points. Therefore, it must
  not always be possible to find such $x_n$. In other words, $S$ must
  be bounded.
\end{proof}

\begin{remark}
  This theorem is sometimes stated as ``each bounded sequence in $\R^n$
  has a convergent subsequence.'' As an exercise, you may want to
  verify that this statement is equivalent to the one above.
\end{remark}
Simon and Blume also prove this theorem in chapter 29.2. They do not
prove the Heine-Borel theorem first though, so their proof is of the
Bolzano-Weierstrass theorem is longer. Perhaps unsurprisingly, the
details of their proof are somewhat similar to our proof of the
Heine-Borel theorem.

So in $\R^n$, compactness, sequential compactness, and closed and
bounded are all the same. In general metric spaces, this need not be
true. However it is always true that sequential compactness and
compactness are the same for metric spaces. It is not very hard to
show that compactness implies sequential compactness. This is done in
the first paragraph of the proof below. The proof that sequential
compactness implies compactness is a somewhat long and difficult, and
you may want to skip it unless you are especially interested.
\begin{theorem}
  Let $X$ be a metric space and $K \subseteq X$. $K$ is compact if and
  only if $K$ is sequentially compact. 
\end{theorem}
\begin{proof}
  Let $K$ be compact. If $\{x_n\}$ is a sequence in $K$ with no
  accumulation point in $K$, then each $x_n$ has some neighborhood
  with $N_{\epsilon_n}(x_n) \cap K = \{x_n\}$. The collection
  $N_{\epsilon_n}(x_n)$ along with the complement of the closure the
  set of $\{x_n\}$ is an open cover for $K$, but there can be no
  finite subcover. Therefore, $\{x_n\}$ must has an accumulation point
  in $K$ when $K$ is compact.

  Suppose every sequence in $K$ has a convergent subsequence with a
  limit point in $K$. Let $G_{\alpha}$, $\alpha \in \mathcal{A}$ be an
  open cover of $K$. $\mathcal{A}$ could be uncountable, so we will
  begin by showing that there must be a countable subcover.  Let
  $n=1$. Pick $x_1 \in K$. If possible choose $x_2 \in K$ such that
  $d(x_1,x_2) \geq 1/n$. Repeat this process, choosing $x_j$ in $K$
  such that $d(x_j,x_i) \geq 1/n$ for each $i<j$. Eventually this will
  no longer be possible because otherwise we could construct a
  sequence with no convergent subsequence. When it is no longer
  possible, set $n = n+1$. This gives a countable collection of open
  neighborhoods $N_{1/n}(x_i)$ that cover $K$ for each $n$ and get
  arbitrarily small as $n$ increases. Call these neighborhoods
  $\eta_j$ for $j = 1,2,..$. Let $J$ be set of all $\eta_j$ such that
  $\eta_j \subseteq G_\alpha$ for some $\alpha$. $J$ is a subset of a
  countable set, so $J$ is countable. Note that $\cup_{j \in J} \eta_j
  \supset K$ because if $x \in K$, then $x \in G_\alpha$ for some
  $\alpha$, and then $\exists \epsilon$ such that $N_{\epsilon}(x) \in
  G_{\alpha}$ and $\exists j$ s.t. $\eta_j \subset N_{\epsilon}(x)$.
  Finally, for each $j \in J$ choose $G_{\alpha_j}$ such that $\eta_j
  \subseteq G_{\alpha_j}$. Such $\alpha_j$ exist by construction. Also
  $\cup_{j \in J} G_{\alpha_j} \supset \cup_{j \in J} \eta_j \supset
  K$. So $G_{\alpha_j}$ is a countable subcover.

  If $G_{\alpha_j}$ has no finite subcover, then for each $n$, 
  \[ F_n = \left(\cup_{i=1}^n G_{\alpha_i}\right)^c \cap K \] is not
  empty (if it were empty, then $\cup_{i=1}^n G_{\alpha_i}$ would be a
  finite subcover). Choose $x_n \in F_n$. Then $\{x_n\}$ is a sequence in $K$,
  and it must have a convergent subsequence with a limit, $x_0$, in
  $K$. However, each $F_{i+1} \subset F_i$ and $F_i$ are all
  closed. Therefore, the sequence $\{x_j\}_{j=i}^\infty$ is also in
  $F_i$ and so is its limit. Then $x_0 \in \cap_{i=1}^\infty
  F_i$. However, 
  \[ \cap_{i=1}^\infty F_i = \left(\cup_{i=1}^\infty
    G_{\alpha_i}\right)^c \cap K, \]  
  but $G_{\alpha_i}$ is a countable cover of $K$, which implies 
  \[ \cap_{i=1}^\infty F_i = \left(\cup_{i=1}^\infty
    G_{\alpha_i}\right)^c \cap K = \emptyset\] 
  and we have a contradiction. Therefore, $G_{\alpha_j}$ must have a
  finite subcover, and $K$ is compact.
\end{proof}

\begin{remark}
  There are non-metric spaces where sequential compactness and
  compactness are not equivalent. One can define open sets on a space
  without a metric by simply specifying which sets are open and making
  it such that theorem \ref{thm:uio} holds. Such a space is called a
  topological space. You can then define closed sets, compact sets,
  and sequential compactness in terms of open sets. On the problem
  set, you will see that you can define continuity of functions in
  terms of open and closed sets. Topology is the branch of mathematics
  that studies topological spaces. One interesting observation is that
  on $\R^n$, if a set is open with respect to some $p$-norm, then it
  is also open with respect to any other $p$-norm. Thus, we say that
  $\R^n$ with the $p$-norms are topologically equivalent or
  homeomorphic. Properties like continuity and compactness are the
  same regardless of what $p$-norm we use. 
  
  As far as I know, topological spaces that are not metric spaces do
  not come up very often in economics, so we will not be studying
  them.
\end{remark}

To review, in $\R^n$ a set is compact if any of the following three
things hold:
\begin{enumerate}
\item For every open cover there exists a finite subcover,
\item Every sequence in the set has a convergent subsequence, or
\item The set is closed and bounded.
\end{enumerate}

\end{document}
