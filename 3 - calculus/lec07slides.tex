\documentclass[compress]{beamer} 
\usepackage{amsmath}
\usepackage{amsfonts}
\usepackage{graphicx}
%\usepackage{epstopdf}
\usepackage{hyperref}
\usepackage{multirow}
\usepackage{verbatim}
%\usepackage[small,compact]{titlesec} 
%\usecolortheme{beaver}
\usetheme{Hannover}
%\usecolortheme{whale}

%\usepackage{pxfonts}
%\usepackage{isomath}
%\usepackage{mathpazo}
%\usepackage{arev} %     (Arev/Vera Sans)
%\usepackage{eulervm} %_   (Euler Math)
%\usepackage{fixmath} %  (Computer Modern)
%\usepackage{hvmath} %_   (HV-Math/Helvetica)
%\usepackage{tmmath} %_   (TM-Math/Times)
\usepackage{tgheros}
%\usepackage{cmbright}
%\usepackage{ccfonts} \usepackage[T1]{fontenc}
%\usepackage[garamond]{mathdesign}
\usepackage{color}
\usepackage{ulem}

\setbeamertemplate{navigation symbols}{}
\AtBeginSection[] % Do nothing for \section*
{ \frame{\sectionpage} }

\newcommand{\argmax}{\operatornamewithlimits{arg\,max}}
\newcommand{\argmin}{\operatornamewithlimits{arg\,min}}
\def\inprobLOW{\rightarrow_p}
\def\inprobHIGH{\,{\buildrel p \over \rightarrow}\,} 
\def\inprob{\,{\inprobHIGH}\,} 
\def\indist{\,{\buildrel d \over \rightarrow}\,} 
\def\F{\mathbb{F}}
\def\R{\mathbb{R}}
\newcommand{\gmatrix}[1]{\begin{pmatrix} {#1}_{11} & \cdots &
    {#1}_{1n} \\ \vdots & \ddots & \vdots \\ {#1}_{m1} & \cdots &
    {#1}_{mn} \end{pmatrix}}
\newcommand{\iprod}[2]{\left\langle {#1} , {#2} \right\rangle}
\newcommand{\norm}[1]{\left\Vert {#1} \right\Vert}
\newcommand{\abs}[1]{\left\vert {#1} \right\vert}
\renewcommand{\det}{\mathrm{det}}
\newcommand{\rank}{\mathrm{rank}}
\newcommand{\spn}{\mathrm{span}}
\newcommand{\row}{\mathrm{Row}}
\newcommand{\col}{\mathrm{Col}}
\renewcommand{\dim}{\mathrm{dim}}
\newcommand{\prefeq}{\succeq}
\newcommand{\pref}{\succ}
\newcommand{\seq}[1]{\{{#1}_n \}_{n=1}^\infty }
\renewcommand{\to}{{\rightarrow}}
\newcommand{\corres}{\overrightarrow{\rightarrow}}

\title{Functions}
\author{Paul Schrimpf}
\institute{UBC \\ Economics 526}
\date{\today}

\begin{document}

\frame{\titlepage}
%\setcounter{tocdepth}{2}

\begin{frame}
  \tableofcontents  
\end{frame}


\maketitle

\section{Definition and examples}

\begin{frame}
  \frametitle{Functions}
  \begin{itemize}
  \item  A \textbf{function} from a set $A$
    to a set $B$ is a rule that assigns to each $a \in A$ one and only one
    $b \in B$ 
  \item $f:A \to B$
  \item $A$ is the domain
  \item $B$ is the target space
  \item \textbf{Image} of $A$ under $f$
    \[ \{y  \in B:  f(x) = y \text{ for some } x \in A \} \]
  \end{itemize}
\end{frame}
\subsection{Examples}

\begin{frame}[shrink]
  \frametitle{Examples}
  \begin{enumerate}
  \item Production functions: $f:\R^2 \to \R$ 
    \begin{itemize}
    \item Linear $ f(x_1,x_2) = a_1 x_1 + a_2 x_2 $
    \item Cobb-Douglas: $f(x_1,x_2) = K x_1^{\alpha_1} x_2^{\alpha_2}$
    \item Constant elasticity of substitution: $f(x_1,x_2) = K (c_1
      x_1^{-a} + c_2 x_2^{-a})^{-b/a}$
    \end{itemize}
  \item Utility functions: $u: \R^T \to \R$
    \begin{itemize}
    \item Constant relative risk aversion: $u(c_1,...,c_T) =
      \sum_{t=1}^T \beta^t \frac{c_t^{1-\gamma}}{1-\gamma}$
    \item Constant absolute risk aversion: $u(c_1,...,c_T) =
      \sum_{t=1}^T \beta^t (-e^{-\alpha c_t})$
    \end{itemize}
  \item Demand function with constant elasticity,  $D:\R^3 \to \R^2$
    \[ D(p_1,p_2,y) = \begin{pmatrix} M p_1^{\alpha_{11}}
      p_2^{\alpha_{12}} y^{\beta_1} \\
      M p_1^{\alpha_{21}} p_2^{\alpha_{22}} y^{\beta_2}
    \end{pmatrix}
    \]
  \end{enumerate}
\end{frame}

\subsection{Visualizing functions}
\begin{frame}
  \frametitle{Visualizing function}
  \begin{definition}
    The \textbf{level sets} of a function $f:X\to Y$ are sets of the
    form 
    \[ \{ x \in X: f(x) = y \} \]
    for some fixed $y \in Y$.
  \end{definition}
  \begin{itemize}
  \item Indifference curves
  \item Isoquants
  \end{itemize}
\end{frame}

\begin{frame}
  \begin{figure}
    \caption{CES, $a = 2$, $b = \frac{4}{5}$}
    \begin{tabular}{cc}
      \includegraphics[width=0.5\linewidth]{cesc} &
      \includegraphics[width=0.5\linewidth]{ces3d} 
    \end{tabular}
  \end{figure}
\end{frame}

\begin{frame}
  \begin{figure}
    \caption{CRRA, $\gamma = 2$, $\beta = 0.95$}
    \begin{tabular}{cc}
      \includegraphics[width=0.5\linewidth]{crrac} &
      \includegraphics[width=0.5\linewidth]{crra3d} 
    \end{tabular}
  \end{figure}
\end{frame}

\begin{frame}
  \begin{figure}
    \caption{CARA, $\alpha = 1$, $\beta = 0.95$}
    \begin{tabular}{cc}
      \includegraphics[width=0.5\linewidth]{carac} &
      \includegraphics[width=0.5\linewidth]{cara3d} 
    \end{tabular}
  \end{figure}
\end{frame}

\section{Special types of functions}

\begin{frame}
  \frametitle{Types of functions}
  \begin{definition}
    A function $f:V \to W$ where $V$ and $W$ are vector spaces is
    \textbf{linear} if $f$ preserves addition and scalar multiplication, ie
    \begin{itemize}
    \item $f(x+y) = f(x) + f(y)$
    \item $f(\alpha x) = \alpha f(x)$
    \end{itemize}
  \end{definition}
\end{frame}

\begin{frame}
  \frametitle{Quadratic}
  \begin{itemize}
  \item $\R \to \R$:  $a_0 + a_1 x + a_2 x^2$
  \end{itemize}
  \begin{definition}
    $q:\R^n \to R$ is a \textbf{quadratic} if 
    \[ q(x_1, ..., x_n) = a_0 + \sum_{i=1, j \geq i}^n a_{ij} x_i x_j \]
  \end{definition}
  \begin{itemize}
  \item Written using matrix:
    \[ q(x_1,..., x_n) = a_0 + x^T A x \]
    where $A = \begin{pmatrix} a_{11} & \frac{1}{2} a_{12} & \cdots &
      \frac{1}{2} a_{1n} \\
      \frac{1}{2} a_{12} & a_{22} \cdots & \frac{1}{2} a_{2n} \\
      \vdots & \ddots & \ddots & \vdots \\
      \frac{1}{2} a_{1n} & \cdots & \cdots & a_{nn}
    \end{pmatrix}$ (not unique)
  \end{itemize}
\end{frame}

\begin{frame}
  \frametitle{Polynomials}
  \begin{definition}
    A \textbf{monomial} $f:\R^n \to R$ is any function of the form
    \[ f(x_1, ..., x_n) = c x_1^{a_1} x_2^{a_2} \cdots x_n^{a_n} \]
    where $a_i$ are nonnegative integers. 
    
    $\sum_{i=1}^n a_i$ is the \textbf{degree} of the monomial. 

    A \textbf{polynomial} $f:\R^n \to \R$ is the sum of finitely many
    monomials, i.e.
    \[ f(x_1, ..., x_n) = \sum_{k=1}^k c_k x_1^{a_{1k}}\cdots
    x_n^{a_{nk}} \]
    The maximum degree of the monomials making up a polynomial is the
    degree of the polynomial.
  \end{definition}
\end{frame}

\section{Continuous functions}

\begin{frame}
  \frametitle{Continuous functions}
  \begin{definition}
    A function $f:X \to Y$ where $X$ and $Y$ are metric spaces is
    \textbf{continuous} if whenever $\seq{x}$ converges to $x$ in $X$,
    then $f(x_n) \to f(x)$ in $Y$. 
  \end{definition}
  \begin{itemize}
  \item No jumps or holes.
  \end{itemize}
\end{frame}

\begin{frame}[shrink]
  \frametitle{$\epsilon-\delta$ definition of continuity}
  \begin{lemma}\label{lem:ced}
    $f: X \to Y$ is continuous at $x$ if and only if for every
    $\epsilon>0$ $\exists$ $\delta >0$ such that $d(x,x') < \delta $
    implies $d(f(x),f(x')) < \epsilon$.
  \end{lemma}
  \begin{proof}
    Suppose $f$ is continuous and there is an $\epsilon>0$ such that
    such that for any $\delta>0$, $d(x,x') < \delta$ does not imply
    $d(f(x),f(x'))< \epsilon$. Then by letting $\delta = 1/n$ we can
    construct a convergent sequence by choosing $x_n$ such that
    $d(x,x_n) < 1/n$ and $d(f(x),f(x_n)) \geq \epsilon$. $x_n \to x$,
    but $f(x_n) \not\to f(x)$. Therefore, if $f$ is continuous it must
    be impossible to construct such a sequence. This means that there
    must be some $\delta>0$ such that $d(x,x') < \delta$ implies
    $d(f(x),f(x')) < \epsilon$.
    
    Now suppose $\forall \epsilon>0$ $\exists \delta>0$ such that
    $d(x,x')<\delta$ implies $d(f(x),f(x')) <\epsilon$. Let $x_n \to
    x$. Then $\exists N$ s.t. $n\geq N$ implies $d(x,x_n) < \delta$. But
    this implies $d(f(x),f(x_n)) < \epsilon$, so $f$ is continuous.
  \end{proof}
\end{frame}

\begin{frame}[shrink]
  \frametitle{Topological definition of continuity}
  \begin{definition}
    Let $f: X \to Y$. The \textbf{preimage} of $V \subseteq Y$ is the
    set in $X$, $f^{-1}(V)$ defined by
    \[ f^{-1} (V) = \{ x \in X: f(x) \in V \} \]
  \end{definition}
  
  \begin{lemma}\label{lem:copen}
    $f:X \to Y$ is continuous if and only if $f^{-1}(V)$ is open for all
    open $V \subseteq Y$. 
  \end{lemma}

  \begin{corollary}
    $f:X \to Y$ is continuous if and only if $f^{-1}(V)$ is closed for all
    closed $V \subseteq Y$. 
  \end{corollary}
\end{frame}

\begin{frame}
  \frametitle{Continuity and arithmetic}
  \begin{theorem}
    Let $f:X \to Y$ and $g:X \to Y$ be continuous and $X$ and $Y$ be
    vector spaces. Then $(f+g)(x) = f(x) + g(x)$ is continuous.
  \end{theorem}
  \begin{proof}
    If $f$ and $g$ are continuous, then by definition $f(x_n) \to f(x)$
    and $g(x_n) \to g(x)$ whenever $x_n \to x$. From the previous
    lecture the limit of a (finite) sum is the sum of limits, so 
    $f(x_n) + g(x_n) \to f(x) + g(x)$, and $f+g$ is continuous.
  \end{proof}
  \begin{itemize}
  \item Same for subtraction, multiplication, etc
  \end{itemize}
\end{frame}

\begin{frame}
  \frametitle{Continuity and composition}
  \begin{definition}
    $(f \circ g)(x) = f(g(x))$ is the \textbf{composition} of $f$ and $g$.
  \end{definition}
  \begin{theorem}
    Let $f:X \to Y$ and $g:Y \to Z$ be continuous where $X$, $Y$, and
    $Z$ are metric spaces. Then $f \circ g$ is continuous.
  \end{theorem}  
  \begin{proof}
    Let $x_n \to x$. $g$ is continuous, so $g(x_n) \to g(x)$. $f$ is
    also continuous, so $f(g(x_n)) \to f(g(x))$.
  \end{proof}
\end{frame}

\subsection{Onto, one-to-one, and inverses}

\begin{frame}
  \frametitle{One-to-one}
  \begin{definition}
    $f:X \to Y$ is \textbf{one-to-one} or \textbf{injective} if for all
    $x_1, x_2 \in X$, 
    \[ f(x_1) = f(x_2) \]
    if and only if $x_1 = x_2$.
  \end{definition}
  \begin{itemize}
  \item $f$ is injective if for each $y \in Y$, the set $\{x:
    f(x) = y\}$ is either a singleton or empty
  \item If $f$ is one-to-one, then $f(x) = b$ has at most one
    solution
  \end{itemize}
\end{frame}

\begin{frame}
  \frametitle{Onto}
  \begin{definition}
    $f:X \to Y$ is \textbf{onto} or \textbf{surjective} if $\forall y
    \in Y$, $\exists x \in X$ such that $f(x) = y$.
  \end{definition}
  \begin{itemize}
  \item If $f$ is onto, then $f(x) = b$ has at least one solution
  \end{itemize}
  \begin{definition} 
    When $f$ is one-to-one and onto, $f$ is \textbf{bijective}. 
  \end{definition}
  \begin{definition}
    If $f:X \to Y$ is bijective, then the \textbf{inverse} of $f$,
    written $f^{-1}$ satisfies
    \[ f(f^{-1} (y)) = y  \]
    and 
    \[ f^{-1} ( f(x) ) = x. \]
  \end{definition}
\end{frame}

\begin{frame}
  \begin{itemize}
  \item The inverse of a continuous function need not be continuous.
    \begin{example}
      \begin{itemize}
      \item $X = [0,2\pi)$, $Y = \{(y_1,y_2) \in \R^2 : y_1^2 + y_2^2
        = 1 \}$.  
      \item $f:X \to Y$, $f(x) = (cos(x), sin(x))$
      \item $f$ is bijective and continuous
      \item $f^{-1}(y_1,y_2) = cos^{-1}(y_1) + \pi 1\{y_2<0\}$
      \item $f^{-1}$ is not continuous at $(1,0)$ 
      \item There does not exist any continuous function with a
        continuous inverse from $[0,2\pi)$ to the unit circle
      \end{itemize}
    \end{example}
  \end{itemize}
\end{frame}

\section{Correspondences}

\begin{frame}\frametitle{Correspondence}
  \begin{definition}
    A \textbf{correspondence} from a set $X$ to a set $Y$, is a rule
    that assigns to each a $x \in X$ a subset of $Y$. We denote a
    correspondence by $\phi: X \corres Y$.
  \end{definition}
\end{frame}

\begin{frame}
  \begin{example}[Budget correspondence]
    \begin{itemize}
    \item $n$ goods with prices $p \in \R^n$. 
    \item Income of $m$, a consumer can afford $\chi(p,m)=\{ x \in X \subseteq
      \R^n: p'x \leq m\}$
    \item Consumer's problem
      \begin{align*}
        \max_{x \in \chi(p,m)} u(x) 
      \end{align*}
    \item Indirect utility function
      \[ v(p,m) = \max_{x \in \chi(p,m)} u(x). \]
    \item The demand correspondence (usually function) is
      \[ x^*(p,m) = \argmax_{x \in \chi(p,m)} u(x). \] 
    \end{itemize}
  \end{example}
\end{frame}

\subsection{Continuity}
\begin{frame}\frametitle{Continuity}
  INSERT PICTURE OF upper and lower hemicontinuous  
\end{frame}

\begin{frame}\frametitle{Continuity}
  \begin{definition}
    A correspondence, $\phi: X \corres Y$ is \textbf{upper hemicontinuous} at $x$
    if for all sequences $x_n \to x$ and $y_n \in \phi(x_n)$ with $y_n
    \to y$, then $y \in \phi(x)$. 
  \end{definition}
  \begin{definition}
    A correspondence, $\phi: X \corres Y$ is \textbf{lower hemicontinuous} at $x$
    if for all sequences $x_n \to x$ and $y \in \phi(x)$, there exists a
    subsequence, $x_{nk}$ and $y_k \in \phi(x_{nk})$ with $y_k \to y$.  
  \end{definition}
  \begin{definition}
    A correspondence is \textbf{continuous} at $x$ if it is both upper
    and lower hemicontinuous at $x$
  \end{definition}
\end{frame}

\end{document}
