

\section{SB 15.4}
Let's use the implicit function theorem to get some comparative
statics results. We will analyze the model of a simple exchange
economy from chapter 15.4 in Simon and Blume. There are two consumers
and two goods. Consumer 1 has endowment $(e^1_1,0)$ and consumer 2 has
$(0,e^2_2)$. The utility function of consumer $i$ is 
\[ U_i(x_i,y_i) = \alpha u_i(x_i) + (1-\alpha) u_i(y_i) \]
Let $p$ be the price of good $1$ and $q$ the price of good $2$. Each
consumer maximizes utility subject to a budget constraint treating
prices as given.
\begin{align}
  \max_{x_i,y_i} U_i(x_i,y_i) \text{ s.t. } p x_i + q y_i \leq p
  e_1^i + q e_2^i  \label{e:max}
\end{align}
You are likely familiar with how to solve constrained optimization
problems. If not, do not worry because we will study it in detail
soon. Anyway, the solution to the maximization problem satisfies the
first order conditions:
\begin{align*}
  [x_i]: & & \frac{\partial U_i}{\partial x_i} = & \lambda_i p \\
  [y_i]: & & \frac{\partial U_i}{\partial y_i} = & \lambda_i q \\
  [\lambda_i]: & &  p x_i + q y_i = &  p e_1^i + q e_2^i  \text{ or
    less than and }
  \lambda_i = 0\\
\end{align*}
where $\lambda_i$ is called a Lagrange multiplier. Under some
assumptions, any solution to (\ref{e:max}) must also solve the first
order conditions. However, not all solutions to the first order
conditions are solutions to the maximization problem (\ref{e:max}). To
ensure that a solution to the first order conditions also solves,
(\ref{e:max}), we should check the second order condition as well. We
will cover that later, and for now just assume that a solution to the
first order condition also solves (\ref{e:max}). We can rewrite the
first conditions as:
\begin{align*}
  \frac{ (\partial U_i)/(\partial x_i) }{ (\partial U_i)/(\partial
    y_i)} = & \frac{p}{q} \\
  p x_i + q y_i = &  p e_1^i + q e_2^i
\end{align*}
or in terms of $u$,
\begin{align}
  \frac{ \alpha u_i'(x_i) }{ (1-\alpha) u_i'(y_i) } = &
  \frac{p}{q} \label{ir} \\ 
  p x_i + q y_i = &  p e_1^i + q e_2^i \label{bc}
\end{align}
Also total consumption must equal the total endowment, so we have the
following two market clearing conditions.
\begin{align}
  x_1 + x_2 = & e_1^1 \label{mc1} \\ 
  y_1 + y_2 = & e_2^2 \label{mc2}
\end{align}
Thus, we have six unknowns, $x_1, x_2, y_1, y_2, p, q$ and six
equations, (\ref{ir}) and (\ref{bc}) for $i=1,2$ and (\ref{mc1}) and
(\ref{mc2}). Let's use the implicit function theorem to get some comparative
statics results. We will analyze the model of a simple exchange
economy from chapter 15.4 in Simon and Blume. There are two consumers
and two goods. Consumer 1 has endowment $(e^1_1,0)$ and consumer 2 has
$(0,e^2_2)$. The utility function of consumer $i$ is 
\[ U_i(x_i,y_i) = \alpha u_i(x_i) + (1-\alpha) u_i(y_i) \]
Let $p$ be the price of good $1$ and $q$ the price of good $2$. Each
consumer maximizes utility subject to a budget constraint treating
prices as given.
\begin{align}
  \max_{x_i,y_i} U_i(x_i,y_i) \text{ s.t. } p x_i + q y_i \leq p
  e_1^i + q e_2^i  \label{e:max}
\end{align}
You are likely familiar with how to solve constrained optimization
problems. If not, do not worry because we will study it in detail
soon. Anyway, the solution to the maximization problem satisfies the
first order conditions:
\begin{align*}
  [x_i]: & & \frac{\partial U_i}{\partial x_i} = & \lambda_i p \\
  [y_i]: & & \frac{\partial U_i}{\partial y_i} = & \lambda_i q \\
  [\lambda_i]: & &  p x_i + q y_i = &  p e_1^i + q e_2^i  \text{ or
    less than and }
  \lambda_i = 0\\
\end{align*}
where $\lambda_i$ is called a Lagrange multiplier. Under some
assumptions, any solution to (\ref{e:max}) must also solve the first
order conditions. However, not all solutions to the first order
conditions are solutions to the maximization problem (\ref{e:max}). To
ensure that a solution to the first order conditions also solves,
(\ref{e:max}), we should check the second order condition as well. We
will cover that later, and for now just assume that a solution to the
first order condition also solves (\ref{e:max}). We can rewrite the
first conditions as:
\begin{align*}
  \frac{ (\partial U_i)/(\partial x_i) }{ (\partial U_i)/(\partial
    y_i)} = & \frac{p}{q} \\
  p x_i + q y_i = &  p e_1^i + q e_2^i
\end{align*}
or in terms of $u$,
\begin{align}
  \frac{ \alpha u_i'(x_i) }{ (1-\alpha) u_i'(y_i) } = &
  \frac{p}{q} \label{ir} \\ 
  p x_i + q y_i = &  p e_1^i + q e_2^i \label{bc}
\end{align}
Also total consumption must equal the total endowment, so we have the
following two market clearing conditions.
\begin{align}
  x_1 + x_2 = & e_1^1 \label{mc1} \\ 
  y_1 + y_2 = & e_2^2 \label{mc2}
\end{align}
Thus, we have six unknowns, $x_1, x_2, y_1, y_2, p, q$ and six
equations, (\ref{ir}) and (\ref{bc}) for $i=1,2$ and (\ref{mc1}) and
(\ref{mc2}). 