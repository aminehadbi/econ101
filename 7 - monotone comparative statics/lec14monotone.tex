
\usepackage{amsmath}
\usepackage{amsfonts}
\usepackage{graphicx}
%\usepackage{epstopdf}
\usepackage{hyperref}
\usepackage{multirow}
\usepackage{verbatim}
%\usepackage[small,compact]{titlesec} 
%\usecolortheme{beaver}
\usetheme{Hannover}
%\usecolortheme{whale}

%\usepackage{pxfonts}
%\usepackage{isomath}
%\usepackage{mathpazo}
%\usepackage{arev} %     (Arev/Vera Sans)
%\usepackage{eulervm} %_   (Euler Math)
%\usepackage{fixmath} %  (Computer Modern)
%\usepackage{hvmath} %_   (HV-Math/Helvetica)
%\usepackage{tmmath} %_   (TM-Math/Times)
%\usepackage{tgheros}
%\usepackage{cmbright}
%\usepackage{ccfonts} \usepackage[T1]{fontenc}
%\usepackage[garamond]{mathdesign}
\usepackage{color}
\usepackage{ulem}

\setbeamertemplate{navigation symbols}{}
\AtBeginSection[] % Do nothing for \section*
{ \frame{\sectionpage} }

\newcommand{\argmax}{\operatornamewithlimits{arg\,max}}
\newcommand{\argmin}{\operatornamewithlimits{arg\,min}}
\def\inprobLOW{\rightarrow_p}
\def\inprobHIGH{\,{\buildrel p \over \rightarrow}\,} 
\def\inprob{\,{\inprobHIGH}\,} 
\def\indist{\,{\buildrel d \over \rightarrow}\,} 
\def\F{\mathbb{F}}
\def\R{\mathbb{R}}
\newcommand{\gmatrix}[1]{\begin{pmatrix} {#1}_{11} & \cdots &
    {#1}_{1n} \\ \vdots & \ddots & \vdots \\ {#1}_{m1} & \cdots &
    {#1}_{mn} \end{pmatrix}}
\newcommand{\iprod}[2]{\left\langle {#1} , {#2} \right\rangle}
\newcommand{\norm}[1]{\left\Vert {#1} \right\Vert}
\newcommand{\abs}[1]{\left\vert {#1} \right\vert}
\renewcommand{\det}{\mathrm{det}}
\newcommand{\rank}{\mathrm{rank}}
\newcommand{\spn}{\mathrm{span}}
\newcommand{\row}{\mathrm{Row}}
\newcommand{\col}{\mathrm{Col}}
\renewcommand{\dim}{\mathrm{dim}}
\newcommand{\prefeq}{\succeq}
\newcommand{\pref}{\succ}
\newcommand{\seq}[1]{\{{#1}_n \}_{n=1}^\infty }
\renewcommand{\to}{{\rightarrow}}
\renewcommand{\L}{{\mathcal{L}}}
\newcommand{\Er}{\mathrm{E}}

\title{Monotone comparative statics}
\author{Paul Schrimpf}
\institute{UBC \\ Economics 526}
\date{\today}

\begin{document}

\frame{\titlepage}
%\setcounter{tocdepth}{2}

\begin{frame}
  \tableofcontents  
\end{frame}

%%%%%%%%%%%%%%%%%%%%%%%%%%%%%%%%%%%%%%%%%%%%%%%%%%%%%%%%%%%%%%%%%%%%%%%%
\section{Introduction} 

\begin{frame}
  \frametitle{References}
  \begin{itemize}
  \item Amir (2005) ``Supermodularity and
    Complementarity in Economics: An Elementary Survey,''
    \textit{Southern Economic Journal}
    \url{http://www.jstor.org/stable/20062066}
  \item Carter sections 2.2.3, 6.2.3
  \item Milgrom and Shannon (1994) ``Monotone Comparative Statics''
    \textit{Econometrica} \url{http://www.jstor.org/stable/2951479}
  \item Athey, Milgrom, and Roberts \textit{Robust Comparative
      Statics} 
  \end{itemize}
\end{frame}

\begin{frame}
  \frametitle{Introduction}
  \begin{itemize}
  \item Problem:
    \[ x^*(\alpha) \in \argmax_x f(x,\alpha) \text{ s.t. }
    g(x,\alpha) \leq 0 \]
    How does $x^*(\alpha)$ vary with $\alpha$
  \item Approaches:
    \begin{enumerate}
    \item Explicitly solve for $x^*(\alpha)$
      \begin{itemize}
      \item Often not possible
      \end{itemize}
    \item Implicit function theorem
      \begin{itemize}
      \item Apply to first order condition, if constraint does not
        bind,  
        \begin{align*}
          0 = & \frac{\partial f}{\partial x} (x^\ast, \alpha) \\
          \frac{d x^\ast}{d \alpha} = & -\left(\frac{\partial^2 f}{\partial
              x \partial \alpha}\right)^{-1} \frac{\partial^2 f}{\partial
            \alpha^2} 
        \end{align*}
      \item Requires $f$ to be smooth, $\frac{\partial^2 f}{\partial
          x \partial \alpha} \neq 0$, constraint not binding,
        $x^\ast(\alpha)$ unique maximizer
      \end{itemize}
    \end{enumerate}
  \end{itemize}
\end{frame}

\begin{frame}\frametitle{Introduction}
  \begin{itemize}
  \item Original problem:
    \[ \max_{x \in \mathcal{X}} f(x,\alpha) \text{ s.t. }
    g(x,\alpha) \leq 0 \]
    \begin{itemize}
    \item Same as
      \[ \max_{x \in \mathcal{X}} h(f(x,\alpha)) \text{ s.t. } g(x,\alpha) \leq 0 \]  
      for any strictly increasing $h:\R \to \R$ including non differentiable
      and non continuous $h$
    \item \[ \max_{y \in \mathcal{Y}} f(k(y),\alpha)) \text{ s.t. }
      g(k(y),\alpha) \leq 0 \]   
      for any surjective $k:\mathcal{Y} \to \mathcal{X}$
      including non differentiable and non continuous $k$
    \end{itemize}
  \item $\Rightarrow$ Conclusions of comparative statics should not
    depend on smoothness, uniqueness of $x^*$, etc
  \end{itemize}
\end{frame}

\section{Monotone maximum theorem}

\begin{frame} \frametitle{Simple setup}
  \[ x^*(\alpha) = \argmax_x f(x,\alpha) \text{ s.t. } x \in
  [g(\alpha),h(\alpha)] \] 
  \begin{itemize}
  \item $x, \alpha \in \R$, $f:\R^2 \to \R$, $g, h:\R \to \R$
  \item $x^*(\alpha)$ is set valued
  \item Define $\bar{x}(\alpha) = \sup_{x \in x^*(\alpha)} x$,
    $\underline{x}(\alpha) = \inf_{x \in x^*(\alpha)} x$ 
  \item Goal: find conditions for $\bar{x}(\alpha)$ to be increasing
    in $\alpha$
  \end{itemize}
\end{frame}

\begin{frame}
  \[ x^*(\alpha) = \argmax_x f(x,\alpha) \text{ s.t. } x \in
  [g(\alpha),h(\alpha)] \] 
  \begin{itemize}
  \item $f(x,\alpha) = x$ or $-x$ implies $x^*(\alpha) = \{h(\alpha)\}$ or
    $\{g(\alpha)\}$, so need $h(\alpha)$ and $g(\alpha)$ to both be
    increasing in $\alpha$
  \item Let $\alpha' > \alpha$ and suppose $\bar{x}(\alpha')<
    \bar{x}(\alpha)$ we want to find flexible assumptions that make
    this impossible
  \item $h(\alpha) \leq h(\alpha') \leq \bar{x}(\alpha') <
    \bar{x}(\alpha) \leq g(\alpha) \leq g(\alpha')$
    \note{First inequality from $h$ increasing. Second from
      $\bar{x}(\alpha') \in [h(\alpha'),g(\alpha')$]. Third by
      assumption. Fourth and fifth same as second and first.}
  \item Maximization implies
    \[ f(\bar{x}(\alpha),\alpha) - f(\bar{x}(\alpha'),\alpha) \geq 0\]
    and 
    \[ f(\bar{x}(\alpha),\alpha') - f(\bar{x}(\alpha'),\alpha') \leq
    0\]    
  \end{itemize}
\end{frame}

\begin{frame}
  \begin{itemize}
  \item Define: $f$ has \textbf{increasing differences} in $x,\alpha$
    if for all $x'>x$, $\alpha'>\alpha$ 
    \[ f(x',\alpha') - f(x,\alpha') \geq f(x',\alpha) - f(x,\alpha) \]
  \item Increasing differences implies:
    \[ 0 \geq f(\bar{x}(\alpha),\alpha') - f(\bar{x}(\alpha'),\alpha')
    \geq f(\bar{x}(\alpha),\alpha) - f(\bar{x}(\alpha'),\alpha) \geq
    0 \]
    i.e. $f(\bar{x}(\alpha),\alpha')=f(\bar{x}(\alpha'),\alpha')$
  \item But then $\bar{x}(\alpha) \in x^*(\alpha')$ and
    $\bar{x}(\alpha)>\bar{x}(\alpha')$, contradicting
    $\bar{x}(\alpha') = \sup \{x^*(\alpha')\}$
  \end{itemize}
\end{frame}

\begin{frame}
  \frametitle{Monotone maximum theorem on $\R$}
  \begin{theorem}
    Let $f: \R^2 \to \R$, $g,h: \R \to \R$. Define
    \[ x^*(\alpha) = \argmax_{x \in [g(\alpha), h(\alpha)]}
    f(x,\alpha). \]
    If 
    \begin{enumerate}
    \item $f$ has increasing differences
    \item $g$ and $h$ are increasing
    \end{enumerate}
    then $\bar{x}(\alpha) = \sup\{x^*(\alpha)\}$ and
    $\underline{x}(\alpha) = \inf\{x^*(\alpha)\}$ are increasing in
    $\alpha$ 
  \end{theorem}
\end{frame}

\begin{frame}\frametitle{Increasing differences}
  \begin{itemize}
  \item $f$ has \textbf{increasing differences} in $x,\alpha$
    if for all $x'>x$, $\alpha'>\alpha$ 
    \[ f(x',\alpha') - f(x,\alpha') \geq f(x',\alpha) - f(x,\alpha) \]
  \item Equivalently
    \[ f(x',\alpha') - f(x',\alpha) \geq f(x,\alpha') - f(x,\alpha) \]
  \item If $f$ is twice continuously differentiable, then $f$ has
    increasing differences iff $\frac{\partial^2 f}{\partial
      x \partial \alpha} \geq 0$ $\forall x, \alpha$
  \end{itemize}
\end{frame}

\begin{frame}\frametitle{Example: normal good}
  \[ \max_{x_1\geq 0, x_2\geq 0} u(x_1,x_2) \text{ s.t. } p_1 x_1 + p_2 x_2 \leq m \]
  \begin{itemize}
  \item $u$ increasing in $x_2$ implies constraint binds
    \[ \max_{x_1} u\left(x_1,\frac{m-p_1 x_1}{p_2} \right) \text{
      s.t. } x_1 \in [0, m/p_1] \]
  \item Question: when is $x_1$ a normal good, i.e. $x_1^*(m)$ is
    increasing?
  \item Monotone maximum theorem says if $u\left(x_1,\frac{m-p_1
        x_1}{p_2} \right)$ has increasing difference in $(x_1, m)$
    \begin{itemize}
    \item If $u$ smooth, has increasing differences iff
      \[ \frac{1}{p_2}\frac{\partial^2 u}{\partial x_1 \partial x_2}
      - \frac{\partial^2 u}{\partial x_2^2} \frac{p_1}{p_2^2} \geq
      0 \] 
    \end{itemize}
  \item Conclusion holds even if 
    \begin{itemize}
    \item Demand integer constrained
    \item Constraint binds sometimes but not always
    \item $u$ not concave
    \end{itemize}
  \end{itemize}
\end{frame}



\end{document}
